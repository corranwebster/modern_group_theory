\documentclass[10pt]{book}

\newif\ifispdf
\ifx\pdfoutput\undefined
\ispdffalse % we are not running PDFLaTeX
\else
\pdfoutput=1 % we are running PDFLaTeX
\ispdftrue
\fi

\ifispdf
\usepackage[pdftex]{graphicx}
\else
\usepackage{graphicx}
\fi


\usepackage{amsmath}
\usepackage{amsfonts}
\usepackage{theorem}
\usepackage{latexsym}
\usepackage{epsfig}
\usepackage{ifthen}
\usepackage{makeidx}
\usepackage{hyperref}

\newcommand{\version}{0.1}

%theorems and similar
\theoremstyle{break}
\newtheorem{theorem}{Theorem}[section]
\newtheorem{proposition}[theorem]{Proposition}
\newtheorem{lemma}[theorem]{Lemma}
\newtheorem{corollary}[theorem]{Corollary}

\theorembodyfont{\slshape}
\newtheorem{definition}{Definition}[section]

\theorembodyfont{\rmfamily}
%\newtheorem{example}{Example}[section]

\newcounter{example}[section]

\renewcommand{\theexample}{\thesection.\arabic{example}}

\newenvironment{example}[1][]%
{\par\vspace{\theorempreskipamount}\refstepcounter{example}\noindent\textbf{Example
\theexample\ifthenelse{\equal{#1}{}}{}{
(#1)}}\\\hspace*{\parindent}}%
{\hfill$\Diamond$\par\vspace{\theorempostskipamount}}

\newenvironment{proof}[1][]%
{\par\noindent\textit{Proof\ifthenelse{\equal{#1}{}}{}{
(#1)}:}\\\hspace*{\parindent}}%
{\hfill\rule{2ex}{2ex}\par\vspace{\theorempostskipamount}}

\newcounter{exercise}[section]
\renewcommand{\theexercise}{\thesection.\arabic{exercise}}

\newenvironment{exercises}%
{\begin{list}{\theexercise.}{\usecounter{exercise}}}%
{\end{list}}

\newcounter{theoremenum}
\renewcommand{\thetheoremenum}{\roman{theoremenum}}

\newenvironment{theoremenum}%
{\begin{list}{(\thetheoremenum)}{\usecounter{theoremenum}}}%
{\end{list}}

\newcommand{\defn}[2]{\textsl{\textbf{#1\index{#2|emph}}}}

\newcommand{\sidebar}[2]{\mbox{}\marginpar{\footnotesize\raggedright\hspace{0pt}\textbf{#1:}
\textsl{#2}}}

\newcommand{\opsidebar}[2]{\reversemarginpar\sidebar{#1}{#2}\normalmarginpar}

\newcommand{\naturals}{\ensuremath{\mathbb{N}}}
\newcommand{\integers}{\ensuremath{\mathbb{Z}}}
\newcommand{\rationals}{\ensuremath{\mathbb{Q}}}
\newcommand{\reals}{\ensuremath{\mathbb{R}}}
\newcommand{\complex}{\ensuremath{\mathbb{C}}}
\newcommand{\torus}{\ensuremath{\mathbb{T}}}
\newcommand{\field}{\ensuremath{\mathbb{F}}}
\newcommand{\ring}{\ensuremath{\mathbb{K}}}

\newcommand{\powerset}{\mathcal{P}}
\newcommand{\divides}{\mid}
\newcommand{\intersect}{\cap}
\newcommand{\bigintersect}{\bigcap}
\newcommand{\union}{\cup}
\newcommand{\bigunion}{\bigcup}
\newcommand{\symdiff}{\bigtriangleup}
\newcommand{\cross}{\times}
\newcommand{\isom}{\cong}

\newcommand{\id}{\text{id}}
\newcommand{\tr}{\text{tr}}
\newcommand{\diag}{\text{diag}}
\newcommand{\diam}{\text{diam}}

\newcommand{\seq}[1]{(#1_{n})_{n=1}^{\infty}}
\newcommand{\net}[3]{(#1_{#2})_{#2 \in #3}}

\renewcommand{\Re}{\operatorname{Re}}
\renewcommand{\Im}{\operatorname{Im}}

\newcommand{\Sym}{\operatorname{Sym}}

\newcommand{\sign}{\operatorname{sign}}
\newcommand{\dom}{\operatorname{dom}}
\newcommand{\cod}{\operatorname{cod}}
\newcommand{\parity}{\operatorname{parity}}

\renewcommand{\emptyset}{\mbox{\O}}

\makeatletter
\newcommand{\ps@draft}{%
\renewcommand{\@evenhead}{\textrm{\thepage}\hfil\textsl{\leftmark}}%
\renewcommand{\@oddhead}{\textsl{\rightmark}\hfil\textrm{\thepage}}%
\renewcommand{\@evenfoot}{\small \today\hfil Version\ \version}%
\renewcommand{\@oddfoot}{\@evenfoot}}
\makeatother

\renewcommand{\chaptermark}[1]{\markboth{#1}{}}
\renewcommand{\sectionmark}[1]{\markright{\thesection. #1}}

\makeindex
\title{Group Theory}
\author{Corran Webster}

\begin{document}

\maketitle
\pagenumbering{roman}

This work is licensed under a
\href{http://creativecommons.org/licenses/by-nc-sa/3.0/}{Creative Commons
Attribution-NonCommercial-ShareAlike 3.0 Unported License}.

\tableofcontents

\setcounter{chapter}{-1}

\chapter{Preliminaries}

\pagenumbering{arabic}

\pagestyle{draft}

Abstract algebra does not require a great deal of mathematical background to
get started: we really only need the concepts of sets and functions to
present the theory.  You should have come across the formal definitions of
these concepts in previous courses, such as a typical discrete mathematics
course.

\section{Sets}

A \defn{set}{set} is a collection of mathematical objects.  We do not care about
the order that the objects are presented, nor any potential duplication of
elements.  The mathematical objects contained in a set $S$ are called
the \defn{elements}{element} or \defn{members}{member} of a set, and write $x \in S$ to say
that $x$ is an element of $S$.  We say that two sets are equal if they have
exactly the same elements.

The simplest way to present a set is as a list of all the elements of the
set enclosed in parentheses, such as the set $\{1, 2, 3\}$.  For sets with
large numbers of elements, or infinite sets, this presentation is tedious
(or impossible!), so there are two alternatives.  If there is a clear
pattern to the elements, one can use ellipses to elide the majority of the
set, leaving just enough to make the pattern of elements clear:
\[
  \{2, 4, 6, \ldots, 100\} \qquad \text{and} \qquad \{2, 3, 5, 7, 11, 13,
\ldots\}
\]
are clearly meant to represent the set of all even numbers from 2 to 100,
and the set of all prime numbers respectively.  However some sets are too
complicated for this sort of presentation, and for these we use ``set
builder'' notation.  In set builder notation we simply specify the set by
some property $P$ which defines the set:
\[
  \{x | x \text{ satisfies } P\} \qquad \text{or} \qquad \{x : x \text{ satisfies }
P\}.
\]
For example, one could write the set of all prime numbers as
\[
  \{ x | x \text{ is prime}\},
\]
or the set of all numbers greater than $2$ and less than or equal to $10$ as
\[
  \{ x : 2 < x \le 10 \}.
\]
This last example illustrates an ambiguity: which collection of numbers do
we mean? Integers? Rational numbers? Real numbers?  To resolve this
ambiguity, we usually specify the set $S$ from which we take our elements,
and use the notation
\[
  \{x \in S | x \text{ satisfies } P\} \qquad \text{or} \qquad \{x \in S : x
\text{ satisfies } P\}.
\]
Therefore the interval of all real numbers greater than 2 and less than or
equal to 10 would most clearly be represented by
\[
  \{ x \in \reals : 2 < x \le 10\}.
\]

There are several special sets that come up with sufficient frequency to
deserve their own notation.  The most important is the \defn{empty set}{set!empty}
$\emptyset = \{\}$, the set which contains no elements.  The next most
important are the various sets of numbers:

\begin{alignat*}{4}
  \naturals &= \{1, 2, 3, 4, \ldots \} & \qquad & \text{\defn{natural
numbers}{natural numbers}}\\
  \integers &= \{\ldots, -3, -2, -1, 0, 1, 2, 3, \ldots \} &&
\text{\defn{integers}{integers}}\\
  \rationals &= \{p/q : p \in \integers, q \in \naturals, \text{$p$ and $q$
    coprime}\} && \text{\defn{rational numbers}{rational numbers}}\\
  \reals &= \{ x : \text{$x$ is an infinite decimal\footnote{This is far
from the whole story: take a real analysis course for more information.}}\}
&& \text{\defn{real numbers}{real numbers}}\\
  \complex &= \{ x+iy : x, y \in \reals \} && \text{\defn{complex numbers}{complex numbers}}
\end{alignat*}

We say that a set $A$ is a \defn{subset}{subset} of another set $B$, and write $A
\subseteq B$, if every element of $A$ is an element of $B$.  For example,
$\{2, 4, 6\} \subseteq \{1, 2, 3, 4, 5, 6\}$.  Note that if $A$ is equal to
$B$, it is still a subset of $B$, and that the empty set is always a subset
of any other set.  We say that $A$ is a \defn{proper subset}{subset!proper} of $B$ if $A
\subset B$ and $A \ne B$, and we denote this by $A \subset B$.

We can combine sets using a number of different \defn{set operations}{set operations}.  The
\defn{union}{union} of two sets $A$ and $B$ is the set containing all the elements
of both sets combined, ie.
\[
  A \union B = \{ x : x \in A \text{ or } x \in B\}.
\]
The \defn{intersection}{intersection} of $A$ and $B$ is the set containing the objects that
are elements of both of the sets, ie.
\[
  A \intersect B = \{ x : x \in A \text{ and } x \in B\}.
\]
Intersection and union are both \defn{commutative}{commutative} and \defn{associative}{associative}
operations, and are \defn{distributive}{distributive} with respect to one another:
\begin{align*}
  A \union B  &= B \union A\\
  A \intersect B  &= B \intersect A\\
  A \union (B \union C) &= (A \union B) \union C = A \union B \union C\\
  A \intersect (B \intersect C) &= (A \intersect B) \intersect C = A \intersect B \intersect
C\\
  A \union (B \intersect C) &= (A \union B) \intersect (A \union C)\\
  A \intersect (B \union C) &= (A \intersect B) \union (A \intersect C)\\
  A \union \emptyset &= A\\
  A \intersect \emptyset &= \emptyset
\end{align*}

If there is some natural \defn{universal set}{universal set} $U$ of elements which we are
considering, we can define the \defn{complement}{complement} of a set $A$ as the set of all
things in $U$ not in $A$, ie.
\[
  A^{c} = \{x \in U : x \notin A \}.
\]
The complement of the complement is the original set, and complements
interact with union and intersection via \defn{DeMorgan's laws}{DeMorgan's laws}:
\begin{align*}
  (A^{c})^{c} &= A\\
  (A \union B)^{c} &= A^{c} \intersect B^{c}\\
  (A \intersect B)^{c} &= A^{c} \union B^{c}\\
  \emptyset^{c} &= U \\
  U^{c} &= \emptyset.
\end{align*}
Note that sometimes the notation $\overline{A}$ is used for complements.

Even in the absence of a universal set, we can define the \defn{set
difference}{set difference} operation: $A \setminus B$ is everything in $A$ which is not in
$B$.  That is
\[
  A \setminus B = \{ x \in A : x \notin B\}.
\]
If complements make sense, then we have $A \setminus B = A \intersect
B^{c}$.  We can also define the \defn{symmetric difference}{symmetric difference} of $A$ and $B$
as the set of all things in either $A$ or $B$, but not in both,
\[
  A \symdiff B = \{x \in A \union B : x \notin A \intersect B\}
\]
or equivalently
\[
  A \symdiff B = (A \union B) \setminus (A \intersect B) = (A \setminus B)
\union (B \setminus A).
\]
Clearly $A \symdiff B = B \symdiff A$.

Perhaps the most important set operation for our purposes, since it appears
in just about every core definition in abstract algebra, is the \defn{cartesian product}{cartesian product}.  The product of two sets, $A \cross B$ is the set
consisting of tuples $(x,y)$, where $x \in A$ and $y \in B$, ie.
\[
  A \cross B = \{(x,y) : x \in A, y \in B\}.
\]
More generally, we define a product of $n$ sets to be the set of $n$-tuples:
\[
  A_{1} \cross A_{2} \cross \cdots \cross A_{n} = \{ (a_{1}, a_{2}, \ldots,
a_{n}) : a_{k} \in A_{k}, k = 1, 2, \ldots, n\}.
\]
We also define
\[
  A^{n} = \underbrace{A \cross A \cross \cdots \cross A}_{n\text{ times}}
\]
to be the set of all $n$-tuples of elements of $A$. This notation is familiar
from calculus, where $\reals^{n}$ is the set of all $n$-tuples of real numbers.
Note that $A \times B$ is not equal to $B \times A$ in general, although they
are clearly closely related.  \sidebar{Proving Equality of Sets}{Often we have
two sets, $A$ and $B$, which we want to show are equal.  A
very common technique to show that this is in fact the case is to show that
each set is a subset of the other.  We can then conclude that they are
equal.  In summary:\\
\hspace{2em}\fbox{\parbox{1in}{$A \subseteq B$ and $B \subseteq A$
implies $A = B$}} }

\subsection*{Exercises}

\begin{enumerate}
  \item In this section many identities are stated without proof.  Pick
    8 of them and show why they hold.  Be careful not to use any identity
    or fact which is dependent on what you are proving.
\end{enumerate}

\section{Functions}

A \defn{function}{function} $f$ from $A$ to $B$ is a rule which relates every element $x$
of $A$ to some unique element $y$ of $B$.  What is key here is that the
function associates $x$ with \emph{precisely} one element of $B$.  We write $y
= f(x)$.  More formally, we denote the function with the notation
\begin{align*}
  f : A & \to B \\
      x & \mapsto y.
\end{align*}
The set $A$ is the \defn{domain}{domain}, the set $B$ the \defn{codomain}{codomain}, while the
set
\[
  f(A) = \{ f(x) : x \in A \}
\]
is the \defn{range}{range} of the function.  The \defn{graph}{graph!of a function} of the function is
the subset
\[
  \mathcal{G}_{f} = \{(x, f(x)) : x \in A \}
\]
of $A \cross B$.

From time to time, we will wish to specify an abstract function without
specifying an exact formula or rule.  In this case, we will just write $f: A
\to B$, specifying the domain and codomain, but nothing else.  We will also
write $\mathcal{F}(A,B)$ for the set of all functions from $A$ to
$B$.

Given a function $f: A \to B$, and a subset $X \subseteq A$, we define the
\defn{image}{image} of $X$ to be the subset of $B$ given by
\[
  f(X) = \{f(x) : x \in B\}.
\]
If $Y \subseteq B$, we also define the \defn{inverse image}{image!inverse}
of $Y$ to be the subset of $A$ given by
\[
  f^{-1}(Y) = \{x \in A : f(x) \in Y\}.
\]
In other words $f^{-1}(Y)$ is the set of elements of $A$ whose value lies in
the set $Y$.

Given a function $g: A \to B$ and another function $f: B \to C$, we define
the \defn{composition}{composition!of two functions} of $f$ and $g$ to be
the function $f \circ g : A \to C$ defined by $(f \circ g)(x) = f(g(x))$.

A function is \defn{one-to-one}{one-to-one function} or \defn{injective}{injective function} if it satisfies the
condition
\[
  f(x_{1}) = f(x_{2}) \text{ implies } x_{1} = x_{2}.
\]
A function is \defn{onto}{onto function} or \defn{surjective}{surjective function} if the range equals the
entire codomain, or equivalently
\[
  f(A) = B.
\]
A function which is both injective and surjective is called a
\defn{bijective}{bijective function} function.

A bijective function automatically has an \defn{inverse
function}{function!inverse} $f^{-1}: B \to A$ defined by $f^{-1}(b) = a$ if
and only if $f(a) = b$.  The fact that $f$ is onto guarantees that $f^{-1}$
is defined on all of $B$, while the fact that $f$ is injective ensures that
$f^{-1}$ is a function.  It follows from the definition that $(f \circ
f^{-1})(x) = x$ and $(f^{-1} \circ f)(x) = x$.

\begin{proposition}\label{prop:functionfacts}
  Let $A$, $B$ $C$ and $D$ be sets, and $f : A \to B$, $g : B \to C$ and
  $h: C \to D$ be functions.  Then we have:
  \begin{theoremenum}
    \item Composition of functions satisfies an associative law:
      $(h \circ g) \circ f = h \circ (g \circ f)$.
    \item If $f$ and $g$ are both one-to-one, then so is $g \circ f$.
    \item If $f$ and $g$ are both onto, then so is $g \circ f$.
    \item If $f$ and $g$ are both bijections, then so is $g \circ f$.
    \item If $f$ is a bijection, then so is $f^{-1}$.
  \end{theoremenum}
\end{proposition}
\begin{proof}
  The proof is left as an exercise.
\end{proof}

\begin{example}
We could formally write the function $f(x) = \sqrt{x}
+ 1$ as:
\begin{align*}
  f : [0,\infty) &\to \reals \\
      x &\mapsto \sqrt{x} + 1.
\end{align*}
As you would expect, the domain is $[0,\infty)$, the codomain is $\reals$,
the range is $[1, \infty)$, and the graph is the set of points
\[
  \{(x, \sqrt{x} + 1) : x \in [0,\infty)\}.
\]
The function is one-to-one, but is not surjective or bijective.
\end{example}

\subsection*{Exercises}

\begin{enumerate}
  \item Prove Proposition~\ref{prop:functionfacts}.
\end{enumerate}

\chapter{Introduction}

\pagestyle{draft}

Algebraic structures of various types occur naturally in many different areas
of mathematics.  The most straightforward examples arise in arithmetic, but
there are numerous other examples which are not as obvious.  In this chapter
we start our study of group theory by looking at a number of concrete
situations where an algebraic structure arises naturally.  We will see that all
these algebraic structures share common features, and these common features
will lead us to the definition of a group in Chapter 2.

\section{Symmetry}
\label{section:symmetry}

You are familiar, at least in an informal way, with the idea of symmetry from
Euclidean geometry and calculus.  For example, the letter ``\textsf{A}'' has
reflective symmetry in its vertical axis, ``\textsf{E}'' has reflective
symmetry in its horizontal axis, ``\textsf{N}'' has rotational symmetry of
$\pi$ radians about its centre, ``\textsf{H}'' has all three types of
symmetry, and the letter ``\textsf{F}'' has none of these symmetries.

Symmetry is also important in understanding real world phenomena.  As some
examples:
\begin{itemize}
  \item The symmetries of molecules can affect possible chemical reactions.
    For example, many proteins and amino acids (the basic building blocks of
    life) have ``left-handed'' and ``right-handed'' versions which are
    reflections of one-another.  Life on earth uses the ``left-handed''
    versions almost exclusively.
    
  \item Crystals have very strong symmetries, largely determined by the
    symmetries of the atoms or molecules of which the crystal is built.
    
  \item Most animals and plants have some sort of symmetry in their
    body-shapes, although they are never perfectly symmetrical.  Most
    animals have bilateral symmetry, while plants often have five-fold
    or six-fold rotational symmetry.
  
  \item In art and design, symmetrical patterns are often found to be more
    pleasing to the eye than asymmetrical patterns, or simply more practical.
    
  \item Waves in fluids, and the vibrations of a drumhead or string are often
    symmetrical, or built out of symmetric components.  These symmetries
    are usually inherent in the underlying equations that we use to model
    such systems, and understanding the symmetry can be crucial in finding
    solutions to these equations.
\end{itemize}

But what, precisely, do we mean by symmetry?

\begin{definition}
  Let $\Omega$ be a subset of $\reals^{n}$.  A \defn{symmetry}{symmetry} of
  $\Omega$ is a function $T: \reals^{n} \to \reals^{n}$ such that
  \begin{theoremenum}
    \item $\{ T(x) : x \in \Omega \} = \Omega$, and
    \item $T$ preserves distances between points: if $d(x,y)$ is the 
    distance between the points $x$ and $y$, then $d(T(x),T(y)) = 
    d(x,y)$.
  \end{theoremenum}
  We denote the set of all symmetries of $\Omega$ by $\Sym(\Omega)$.  Every
  set $\Omega$ has at least one symmetry, the \defn{identity
  symmetry}{symmetry!identity} $I(x) = x$.

  Functions which preserve distance are called
  \defn{isometries}{isometries}, so every symmetry is an isometry.
\end{definition}

\begin{proposition}\label{prop:symmetryfacts}
  Let $\Omega$ be a subset of $\reals^{n}$, and let $S$ and $T$ be
  symmetries of $\Omega$.  Then
  \begin{theoremenum}
    \item $T$ is one-to-one and onto.
    
    \item the inverse function $T^{-1}$ is a symmetry of $\Omega$
    
    \item the composition $T \circ S$ is a symmetry of $\Omega$

    \item the compositions $T \circ T^{-1}$ and $T^{-1} \circ T$ always
       equal the identity symmetry $I$.
  \end{theoremenum}
\end{proposition}
\begin{proof}
  (i) This follows immediately from the technical result that every
  isometry from $\reals^{n}$ to $\reals^{n}$ is one-to-one and onto
  (this is proved in Lemma~\ref{lemma:isometrybijective} at the end of
  this chapter).
  
  (ii) Since $T$ is one-to-one and onto, it has an inverse function
  $T^{-1}$.  We observe that $T^{-1}(\Omega) = T^{-1}(T(\Omega)) =
  \Omega$, and also that $d(T^{-1}(x), T^{-1}(y)) = d(T(T^{-1}(x)),
  T(T^{-1}(y))) = d(x,y)$.  Hence $T^{-1}$ is a symmetry of $\Omega$.

  Parts (iii) and (iv) are left as a simple exercise.
\end{proof}

We\sidebar{Notation}{Many algebra texts write $ST$ for $T \circ S$, because
$S$ is applied first, then $T$.  In these notes, however, we will remain
consistent with the traditional function composition order, but you must
keep this clear in your head to avoid confusion.\\
\indent Another commonly used convention in algebra is to apply functions
on the right, so $T(x)$ is written as $xT$, so that $S(T(x))$ would be
written as $xTS$.}
will usually write the composed symmetry $T \circ S$ as simply $TS$.  Remember
that because function composition works from right to left, $TS$ means that the
symmetry $S$ is applied first, followed by the symmetry $T$.

You should also recall that composition of functions is associative
(see Proposition~\ref{prop:functionfacts}) so composition of
symmetries is always associative.  In other words if $S$, $T$ and $U
\in \Sym(\Omega)$, then $S(TU) = (ST)U$.  However, composition of
functions is not usually commutative, so without additional evidence,
we cannot conclude that $ST = TS$.

\begin{figure}\label{fig:symmetryofH}
  \centering
  \begin{picture}(7,7)(-1,-1)
    \thicklines
    \put(1,1){\line(1,0){1}}
    \put(2,1){\line(0,1){1}}
    \put(2,2){\line(1,0){1}}
    \put(3,2){\line(0,-1){1}}
    \put(3,1){\line(1,0){1}}
    \put(4,1){\line(0,1){3}}
    \put(4,4){\line(-1,0){1}}
    \put(3,4){\line(0,-1){1}}
    \put(3,3){\line(-1,0){1}}
    \put(2,3){\line(0,1){1}}
    \put(2,4){\line(-1,0){1}}
    \put(1,4){\line(0,-1){3}}
    
    \thinlines
    \put(-1,2.5){\vector(1,0){7}}
    \put(2.5,-1){\vector(0,1){7}}
  \end{picture}
  \caption{The set $\Omega$ of Example~\ref{eg:symmetryofH}}
\end{figure}

\begin{example}\label{eg:symmetryofH}
  Let $\Omega \subseteq \reals^{2}$ be the H-shaped set illustrated in
  Figure~\ref{fig:symmetryofH}.  Then $\Omega$ has percisely the following
  symmetries:
  \begin{alignat*}{4}
    I(x,y) &= (x,y) &\qquad& \text{(Identity)} \\
    H(x,y) &= (x,-y) && \text{(Reflection in the $x$-axis)} \\
    V(x,y) &= (-x,y) && \text{(Reflection in the $y$-axis)} \\
    R(x,y) &= (-x,-y) && \text{(Rotation by $\pi$ radians about the origin)}
  \end{alignat*}
  
  We can confirm by direct calculation that $I^{-1} = I$, $H^{-1} = H$, $V^{-1}
  = V$ and $R^{-1} = R$.  In other words, each of these transformations is its
  own inverse.  These symmetries compose in the following ways:
  \begin{alignat*}{6}
    H \circ H &= I & \qquad & H \circ V &= R & \qquad & H \circ R &= V\\
    V \circ H &= R && V \circ V &= I && V \circ R &= H\\
    R \circ H &= V && R \circ V &= H && R \circ R &= I
  \end{alignat*}
  In fact we can summarize this using a ``multiplication table'':
  \[
    \begin{array}{c|cccc}
      \circ & I & H & V & R \\
      \hline
          I & I & H & V & R \\
          H & H & I & R & V \\
          V & V & R & I & H \\
          R & R & V & H & I 
    \end{array}
  \]
  This sort of ``multiplication table'' is called a \defn{Cayley table}{Cayley
  table} for the operation.
  
  The composition of symmetries in this example is commutative.  You can verify
  this by simply checking every possible product.  For example, from the Cayley
  table we have $HR = V$, and $RH = V$, so $HR = RH$.
\end{example}

There is nothing really special about the set $\Omega$ in the previous
example, other than the fact that composition is commutative for this set.
As the following example shows, we should not expect composition of
symmetries to be commutative in every case.

\begin{figure}\label{fig:symmtriangle}
  \centering
  \begin{picture}(7,7)(-3.5,-3.5)
    \thicklines
    \qbezier(2,0)(0,1.155)(-1,1.732)
    \qbezier(2,0)(0,-1.155)(-1,-1.732)
    \qbezier(-1,1.732)(-1,0)(-1,-1.732)

    \thinlines
    \put(-3.5,0){\vector(1,0){7}}
    \put(0,-3.5){\vector(0,1){7}}
    
    \put(2,0){\makebox(0,0)[tl]{1}}
    \put(0,2){\makebox(0,0)[l]{1}}
    \put(-2,0){\makebox(0,0)[t]{-1}}
    \put(0,-2){\makebox(0,0)[l]{-1}}
  \end{picture}
  \caption{The equilateral triangle of Example~\ref{eg:symmtriangle}}
\end{figure}

\begin{example}\label{eg:symmtriangle}
  Let $\Omega \subseteq \reals^{2}$ be an equilateral triangle with veritces
  $(1,0)$, $(-1/2, \sqrt{3}/2)$ and $(-1/2, -\sqrt{3}/2)$.  Then $\Omega$ has
  the following symmetries:

  \begin{tabular}{lp{3.5in}}
    $I$ & Identity \\
    $R_{1}$ & Rotation by $2\pi/3$ radians clockwise \\
    $R_{2}$ & Rotation by $2\pi/3$ radians anticlockwise \\
    $H_{0}$ & Reflection in the $x$-axis \\
    $H_{1}$ & Reflection in the line through $(0,0)$ and $(-1/2,\sqrt{3}/2)$ \\
    $H_{2}$ & Reflection in the line through $(0,0)$ and $(-1/2,-\sqrt{3}/2)$
  \end{tabular}
  
  The precise formulas for these symmetries are an exercise.  A little thought
  tells us that $I^{-1} = I$, $R_{1}^{-1} = R_{2}$, $R_{2}^{-1} = R_{1}$,
  $H_{1}^{-1} = H_{1}$, $H_{2}^{-1} = H_{2}$, and $H_{3}^{-1} = H_{3}$.
  The Cayley table for these symmetries is:
    
  \medskip
  \hspace{1in}\begin{tabular}{c|cccccc}
    $\circ$ & $I$ & $R_{1}$ & $R_{2}$ & $H_{0}$ & $H_{1}$ & $H_{2}$ \\
    \hline
    $I$ & $I$ & $R_{1}$ & $R_{2}$ & $H_{0}$ & $H_{1}$ & $H_{2}$ \\
    $R_{1}$ & $R_{1}$ & $R_{2}$ & $I$ & $H_{1}$ & $H_{2}$ & $H_{0}$ \\
    $R_{2}$ & $R_{2}$ & $I$ & $R_{1}$ & $H_{2}$ & $H_{0}$ & $H_{1}$ \\
    $H_{0}$ & $H_{0}$ & $H_{2}$ & $H_{1}$ & $I$ & $R_{2}$ & $R_{1}$ \\
    $H_{1}$ & $H_{1}$ & $H_{0}$ & $H_{2}$ & $R_{1}$ & $I$ & $R_{2}$ \\
    $H_{2}$ & $H_{2}$ & $H_{1}$ & $H_{0}$ & $R_{2}$ & $R_{1}$ & $I$ \\
  \end{tabular}
  
  \medskip
  
  This operation is associative, but it is clearly \emph{not} commutative:
  $H_{0} \circ H_{1} = R_{1}$, but $H_{1} \circ H_{0} = R_{2}$, for example.
\end{example}

Some sets have infinite collections of symmetries, but even in these cases
we can still understand how composition works.

\begin{example}\label{eg:circlesymmetry}
  Let $\Omega = \{(x,y) \in \reals^{2} : x^{2} + y^{2} = 1\}$ be the unit
  circle.  Then $\Omega$ has infinitely many symmetries, which fall into two
  classes:
  
  \begin{tabular}{lp{3.5in}}
    $R_{\theta}$ & Rotation by $\theta$ radians clockwise,
      $0 \le \theta < 2\pi$ \\
    $H_{\varphi}$ & Reflection in the line which makes an angle $\varphi$
      to the $x$-axis at the origin, $0 \le \varphi < \pi$.
  \end{tabular}
  
  The identity is $R_{0}$, rotation by $0$ radians.  We can also check that
  the inverse of $R_{\theta}$ is $R_{2\pi - \theta}$ for $0 < \theta < 2\pi$,
  and the inverse of $H_{\varphi}$ is $H_{\varphi}$.
  
  Because the set of symmetries is infinite, we can't write down a Cayley
  table, but we can list how the generic symmetries compose:
  \begin{alignat*}{4}
    R_{\theta} \circ R_{\omega} &= R_{\theta + \omega} &\qquad&
    R_{\theta} \circ H_{\varphi} &=  H_{\varphi - \theta/2}\\
    H_{\varphi} \circ R_{\theta} &= H_{\varphi + \theta/2}&&
    H_{\varphi} \circ H_{\psi} &= R_{2\psi - 2\varphi}
  \end{alignat*}
  where all angles are reduced to lie in the appropriate ranges.  The easiest
  way to verify this table is to note that $H_{\varphi} = H_{0} \circ
  R_{2\varphi} = R_{-2\varphi} \circ H_{0}$, which greatly simplifies
  calculations involving $H_{\varphi}$.
\end{example}

All the examples so far have used rotational and reflective symmetries, but
some sets also have translational symmetry.

\begin{example}
  Let $\Omega \{(x,y) \in \reals^{2} : y = 0\}$ be the $x$-axis in
  $\reals^{2}$. Then $\Omega$ has symmetries of the form
  \[
    T_{c}(x,y) = (x + c,y),
  \]
  ie. right translation by $c$, for any $c \in \reals$, and
  \[
    S_{c}(x,y) = (-x + c,y),
  \]
  ie. reflection about $0$, followed by right translation by $c$, for any $c
  \in \reals$ (which is equal to reflection about the point $c/2$).
  
  The identity symmetry is $T_{0}$, the inverse symmetry of $T_{c}$ is
  $T_{-c}$, and the inverse symmetry of $S_{c}$ is $S_{c}$. The
  symmetries of this set compose by the rules
  \begin{alignat*}{4}
    T_{a} \circ T_{b} &= T_{a+b} &\qquad&
    T_{a} \circ S_{b} &=  S_{a+b}\\
    S_{a} \circ T_{b} &=  S_{a-b}&&
    S_{a} \circ S_{b} &=  T_{a-b}
  \end{alignat*}
\end{example}

\subsection*{Exercises}

\begin{exercises}
  \item Find the set of symmetries for each capital letter of the alphabet
    (assume uniform, sans serif letter shapes).
  
  \item Prove Proposition~\ref{prop:symmetryfacts} (iii-iv).
  
  \item Write down formulas for each of the symmetries in
    Example~\ref{eg:symmtriangle}.
    
    Hint 1: the point $(x,y) \in \reals^{2}$ rotated clockwise by an angle
    $\theta$ about the origin is $(x\cos \theta + y\sin \theta, -x\sin \theta +
    y\cos \theta)$.
    
    Hint 2: from the Cayley table, we have $H_{1} = R_{1} \circ H_{0}$ and
    $H_{2} = R_{2} \circ H_{0}$, and it is easy to find the formula of a
    composition of functions.
  
  \item Let $\Omega \subseteq \reals^{2}$ be a square, centred at the origin,
    with side length 1.
    Find all 8 symmetries of $\Omega$, and write down the formula for each. 
    Find the inverses of each symmetry.
    Write out the Cayley table for the symmetries of a square.
  
  \item\label{ex:symtetra} (*) Let $\Omega \subseteq \reals^{3}$ be a regular tetrahedron
    centred at the origin.  Show that $\Omega$ has 24 symmetries.
  
  \item (*) Let $\Omega = \integers^{2} \subseteq \reals^{2}$ be the integer
    lattice in the plane, ie.
    \[
      \integers^{2} = \{ (m,n) \in \reals^{2} : m, n \in \integers \}.
    \]
    Classify the symmetries of $\integers^{2}$.
    Find the inverses of each symmetry.
    As in Example~\ref{eg:circlesymmetry}, find the product of typical symmetries.
\end{exercises}

\section{Review: Sets}

Group theory does not require a great deal of mathematical background to
get started: we really only need the concepts of sets and functions to
present the basics of the theory. You should have come across the formal definitions of
these concepts in previous courses, such as a typical discrete mathematics
course.  A large part of the discussion in this section and the next will be
to fix notation and terminology.

A \defn{set}{set} is a collection of mathematical objects.  We do not care about
the order that the objects are presented, nor any potential duplication of
elements.  The mathematical objects contained in a set $S$ are called
the \defn{elements}{element} or \defn{members}{member} of a set, and write $x \in S$ to say
that $x$ is an element of $S$.  We say that two sets are equal if they have
exactly the same elements.

The simplest way to present a set is as a list of all the elements of
the set enclosed in braces, such as the set $\{1, 2, 3\}$.  For sets
with large numbers of elements, or infinite sets, this presentation is
tedious (or impossible!), so there are two alternatives.  If there is
a clear pattern to the elements, one can use ellipses to elide the
majority of the set, leaving just enough to make the pattern of
elements clear:
\[
  \{2, 4, 6, \ldots, 100\} \qquad \text{and} \qquad \{2, 3, 5, 7, 11, 
  13, 17
\ldots\}
\]
are clearly meant to represent the set of all even numbers from 2 to 100,
and the set of all prime numbers respectively.  However some sets are too
complicated for this sort of presentation, and for these we use ``set
builder'' notation.  In set builder notation we simply specify the set by
some property $P$ which defines the set:
\[
  \{x | x \text{ satisfies } P\} \qquad \text{or} \qquad \{x : x \text{ satisfies }
P\}.
\]
For example, one could write the set of all prime numbers as
\[
  \{ x | x \text{ is prime}\},
\]
or the set of all numbers greater than $2$ and less than or equal to $10$ as
\[
  \{ x : 2 < x \le 10 \}.
\]
This last example illustrates an ambiguity: which collection of numbers do
we mean? Integers? Rational numbers? Real numbers?  To resolve this
ambiguity, we usually specify the set $S$ from which we take our elements,
and use the notation
\[
  \{x \in S | x \text{ satisfies } P\} \qquad \text{or} \qquad \{x \in S : x
\text{ satisfies } P\}.
\]
Therefore the interval of all real numbers greater than 2 and less than or
equal to 10 would most clearly be represented by
\[
  \{ x \in \reals : 2 < x \le 10\}.
\]

There are several special sets that come up with sufficient frequency to
deserve their own notation.  The most important is the \defn{empty set}{set!empty}
$\emptyset = \{\}$, the set which contains no elements.  The next most
important are the various sets of numbers:

\begin{alignat*}{4}
  \naturals &= \{1, 2, 3, 4, \ldots \} & \qquad & \text{\defn{natural
numbers}{natural numbers}}\\
  \integers &= \{\ldots, -3, -2, -1, 0, 1, 2, 3, \ldots \} &&
\text{\defn{integers}{integers}}\\
  \rationals &= \{p/q : p \in \integers, q \in \naturals, \text{$p$ and $q$
    coprime}\} && \text{\defn{rational numbers}{rational numbers}}\\
  \reals &= \{ x : \text{$x$ is an infinite decimal\footnote{This is far
from the whole story: take a real analysis course for more information.}}\}
&& \text{\defn{real numbers}{real numbers}}\\
  \complex &= \{ x+iy : x, y \in \reals \} && \text{\defn{complex numbers}{complex numbers}}
\end{alignat*}

We  \sidebar{Proving Equality of Sets}{Often we have
two sets, $A$ and $B$, which we want to show are equal.  A
very common technique to show that this is in fact the case is to show that
each set is a subset of the other.  We can then conclude that they are
equal.  In summary:\\
\hspace{2em}\fbox{\parbox{1in}{$A \subseteq B$ and $B \subseteq A$
implies $A = B$}} }
 say that a set $A$ is a \defn{subset}{subset} of another set $B$, and write $A
\subseteq B$, if every element of $A$ is an element of $B$.  For example,
$\{2, 4, 6\} \subseteq \{1, 2, 3, 4, 5, 6\}$.  Note that if $A$ is equal to
$B$, it is still a subset of $B$, and that the empty set is always a subset
of any other set.  We say that $A$ is a \defn{proper subset}{subset!proper} of $B$ if $A
\subset B$ and $A \ne B$, and we denote this by $A \subset B$.

We can combine sets using a number of different \defn{set operations}{set operations}.  The
\defn{union}{union} of two sets $A$ and $B$ is the set containing all the elements
of both sets combined, ie.
\[
  A \union B = \{ x : x \in A \text{ or } x \in B\}.
\]
The \defn{intersection}{intersection} of $A$ and $B$ is the set containing the objects that
are elements of both of the sets, ie.
\[
  A \intersect B = \{ x : x \in A \text{ and } x \in B\}.
\]
Intersection and union are both \defn{commutative}{commutative} and \defn{associative}{associative}
operations, and are \defn{distributive}{distributive} with respect to one another:
\begin{align*}
  A \union B  &= B \union A\\
  A \intersect B  &= B \intersect A\\
  A \union (B \union C) &= (A \union B) \union C = A \union B \union C\\
  A \intersect (B \intersect C) &= (A \intersect B) \intersect C = A \intersect B \intersect
C\\
  A \union (B \intersect C) &= (A \union B) \intersect (A \union C)\\
  A \intersect (B \union C) &= (A \intersect B) \union (A \intersect C)\\
  A \union \emptyset &= A\\
  A \intersect \emptyset &= \emptyset
\end{align*}

If there is some natural \defn{universal set}{universal set} $U$ of elements which we are
considering, we can define the \defn{complement}{complement} of a set $A$ as the set of all
things in $U$ not in $A$, ie.
\[
  A^{c} = \{x \in U : x \notin A \}.
\]
The complement of the complement is the original set, and complements
interact with union and intersection via \defn{DeMorgan's laws}{DeMorgan's laws}:
\begin{align*}
  (A^{c})^{c} &= A\\
  (A \union B)^{c} &= A^{c} \intersect B^{c}\\
  (A \intersect B)^{c} &= A^{c} \union B^{c}\\
  \emptyset^{c} &= U \\
  U^{c} &= \emptyset.
\end{align*}
Note that sometimes the notation $\overline{A}$ is used for complements.

Even in the absence of a universal set, we can define the \defn{set
difference}{set difference} operation: $A \setminus B$ is everything in $A$ which is not in
$B$.  That is
\[
  A \setminus B = \{ x \in A : x \notin B\}.
\]
If complements make sense, then we have $A \setminus B = A \intersect
B^{c}$.  We can also define the \defn{symmetric difference}{symmetric difference} of $A$ and $B$
as the set of all things in either $A$ or $B$, but not in both,
\[
  A \symdiff B = \{x \in A \union B : x \notin A \intersect B\}
\]
or equivalently
\[
  A \symdiff B = (A \union B) \setminus (A \intersect B) = (A \setminus B)
\union (B \setminus A).
\]
Clearly $A \symdiff B = B \symdiff A$.

Perhaps the most important set operation for our purposes, since it appears
in just about every core definition in abstract algebra, is the \defn{Cartesian
product}{cartesian product}.  The product of two sets, $A \cross B$ is the set
consisting of tuples $(x,y)$, where $x \in A$ and $y \in B$, ie.
\[
  A \cross B = \{(x,y) : x \in A, y \in B\}.
\]
More generally, we define a product of $n$ sets to be the set of $n$-tuples:
\[
  A_{1} \cross A_{2} \cross \cdots \cross A_{n} = \{ (a_{1}, a_{2}, \ldots,
a_{n}) : a_{k} \in A_{k}, k = 1, 2, \ldots, n\}.
\]
We also define
\[
  A^{n} = \underbrace{A \cross A \cross \cdots \cross A}_{n\text{ times}}
\]
to be the set of all $n$-tuples of elements of $A$. This notation is familiar
from calculus, where $\reals^{n}$ is the set of all $n$-tuples of real numbers.
Note that $A \times B$ is not equal to $B \times A$ in general, although they
are clearly closely related.

If $A \subseteq C$, and $B \subseteq D$ it is straightforward to see that
$A \cross B \subseteq C \cross D$.

We denote the \defn{cardinality}{cardinality} of a set $A$ by $|A|$.  For sets
with a finite number of elements, the cardinality of $A$ is simply the number
of elements in the set.  For infinite sets, cardinality is a more complicated
matter, but for the purposes of this course it really only matters whether a
set is infinite or not.  You should, however, be aware that there are
countably infinite sets (such as $\naturals^{n}$, $\integers^{n}$ and
$\rationals^{n}$) and uncountably infinite sets (such as $\reals^{n}$ and
$\complex^{n}$) and that countable and uncountable sets have different
cardinality.

For finite sets, we have the following facts from basic counting theory:
the inclusion-exclusion principle
\[
  |A \union B| = |A| + |B| - |A \intersect B|,
\]
and the multiplication principle
\[
  |A \cross B| = |A||B|.
\]
Both of these will be of importance when exploring the structure of finite
groups.

\subsection*{Exercises}

\begin{exercises}
  \item In this section many identities are stated without proof.  Pick
    8 of them and show why they hold.  Be careful not to use any identity
    or fact which is dependent on what you are proving.
  
  \item Show that $|A \setminus B| = |A| - |A \intersect B|$.
\end{exercises}

\section{Review: Functions}

A \defn{function}{function} $f$ from $A$ to $B$ is a rule which relates every element $x$
of $A$ to some unique element $y$ of $B$.  What is key here is that the
function associates $x$ with \emph{precisely} one element of $B$.  We write $y
= f(x)$.  More formally, we denote the function with the notation
\begin{align*}
  f : A & \to B \\
      x & \mapsto y.
\end{align*}
The set $A$ is the \defn{domain}{domain}, the set $B$ the \defn{codomain}{codomain}, while the
set
\[
  f(A) = \{ f(x) : x \in A \}
\]
is the \defn{range}{range} of the function.  The \defn{graph}{graph!of a function} of the function is
the subset
\[
  \mathcal{G}_{f} = \{(x, f(x)) : x \in A \}
\]
of $A \cross B$.

From time to time, we will wish to specify an abstract function without
specifying an exact formula or rule.  In this case, we will just write $f: A
\to B$, specifying the domain and codomain, but nothing else.  We will also
write $\mathcal{F}(A,B)$ for the set of all functions from $A$ to
$B$.  Some texts use $B^{A}$ instead for this set.

Given a function $f: A \to B$, and a subset $X \subseteq A$, we define the
\defn{image}{image} of $X$ to be the subset of $B$ given by
\[
  f(X) = \{f(x) : x \in B\}.
\]
If $Y \subseteq B$, we also define the \defn{inverse image}{image!inverse}
of $Y$ to be the subset of $A$ given by
\[
  f^{-1}(Y) = \{x \in A : f(x) \in Y\}.
\]
In other words $f^{-1}(Y)$ is the set of elements of $A$ whose value lies in
the set $Y$.

Given a function $g: A \to B$ and another function $f: B \to C$, we define
the \defn{composition}{composition!of two functions} of $f$ and $g$ to be
the function $f \circ g : A \to C$ defined by $(f \circ g)(x) = f(g(x))$.

A function is \defn{one-to-one}{one-to-one function} or \defn{injective}{injective function} if it satisfies the
condition
\[
  f(x_{1}) = f(x_{2}) \text{ implies } x_{1} = x_{2}.
\]
A function is \defn{onto}{onto function} or \defn{surjective}{surjective function} if the range equals the
entire codomain, or equivalently
\[
  f(A) = B.
\]
A function which is both injective and surjective is called a
\defn{bijective}{bijective function} function.

A bijective function automatically has an \defn{inverse
function}{function!inverse} $f^{-1}: B \to A$ defined by $f^{-1}(b) = a$ if
and only if $f(a) = b$.  The fact that $f$ is onto guarantees that $f^{-1}$
is defined on all of $B$, while the fact that $f$ is injective ensures that
$f^{-1}$ is a function.  It follows from the definition that $(f \circ
f^{-1})(x) = x$ and $(f^{-1} \circ f)(x) = x$.

\begin{proposition}\label{prop:functionfacts}
  Let $A$, $B$ $C$ and $D$ be sets, and $f : A \to B$, $g : B \to C$ and
  $h: C \to D$ be functions.  Then we have:
  \begin{theoremenum}
    \item Composition of functions satisfies an associative law:
      $(h \circ g) \circ f = h \circ (g \circ f)$.
    \item If $f$ and $g$ are both one-to-one, then so is $g \circ f$.
    \item If $f$ and $g$ are both onto, then so is $g \circ f$.
    \item If $f$ and $g$ are both bijections, then so is $g \circ f$.
    \item If $f$ is a bijection, then so is $f^{-1}$.
    \item If $f$ is a bijection $|A| = |B|$.
  \end{theoremenum}
\end{proposition}
\begin{proof}
  The proof is left as an exercise.
\end{proof}

\begin{example}
We could formally write the function $f(x) = \sqrt{x}
+ 1$ as:
\begin{align*}
  f : [0,\infty) &\to \reals \\
      x &\mapsto \sqrt{x} + 1.
\end{align*}
As you would expect, the domain is $[0,\infty)$, the codomain is $\reals$,
the range is $[1, \infty)$, and the graph is the set of points
\[
  \{(x, \sqrt{x} + 1) : x \in [0,\infty)\}.
\]
The function is one-to-one, but is not surjective or bijective.
\end{example}

\subsection*{Exercises}

\begin{exercises}
  \item Prove Proposition~\ref{prop:functionfacts}.
\end{exercises}

\section{Permutations}

A \defn{permutation}{permutation} of a set $X$ is simply a re-arrangement of
the elements, or more precisely a function $p$ that maps each element of $X$ to
an element of $X$ with no two distinct elements being mapped to the same
element (and for infinite sets, we also need $p(X) = X$).  Another way of
saying this is that a permutation of $X$ is simply a bijection $p: X \to X$.

Normally we are interested only in permutations of finite sets, and we
really only care how many elements there are to permute.  Hence it is
customary to consider permutations of the set $\{1, 2, 3, ..., n\}$.

Since permutations are just functions,we can define them as we would any other
function, by specifying the value that the function takes at each point in the
domain.  Unfortunately, unlike the usual functions you see in a calculus course,
you usually can't specify permutations using a formula.

\begin{example}
  The following function $p$ is a permutation of the set $\{1,2,3,4,5,6,7,8\}$:
  \begin{alignat*}{8}
    p(1) &= 2 &\qquad&
    p(2) &= 4 &\qquad&
    p(3) &= 6 &\qquad&
    p(4) &= 8 \\
    p(5) &= 7 &&
    p(6) &= 5 &&
    p(7) &= 3 &&
    p(8) &= 1  
  \end{alignat*}
\end{example}

A more compact way of writing down a permutation is to write it as an array of
numbers, with $1$, through $n$ on the top row, and the respective image of each
in the second row, like so:
\[
  p = \begin{pmatrix}
    1 & 2 & 3 & \ldots & n \\
    p(1) & p(2) & p(3) & \ldots & p(n)
  \end{pmatrix}
\]

\begin{example}
  The permutation $p$ of the previous example can be written as follows:
  \[
    p = \begin{pmatrix}
      1 & 2 & 3 & 4 & 5 & 6 & 7 & 8 \\
      2 & 4 & 6 & 8 & 7 & 5 & 3 & 1
    \end{pmatrix}
  \]
\end{example}

We denote the set of all permutations of $\{1,2,3,\ldots,n\}$ by $S_{n}$.

\begin{example}\label{eg:perm3part1}
  The set $S_{3}$ is
  \[
    \left\{\begin{pmatrix}
      1 & 2 & 3 \\
      1 & 2 & 3
    \end{pmatrix},
    \begin{pmatrix}
      1 & 2 & 3 \\
      3 & 1 & 2
    \end{pmatrix},
    \begin{pmatrix}
      1 & 2 & 3 \\
      2 & 3 & 1
    \end{pmatrix},
    \begin{pmatrix}
      1 & 2 & 3 \\
      1 & 3 & 2
    \end{pmatrix},
    \begin{pmatrix}
      1 & 2 & 3 \\
      2 & 1 & 3
    \end{pmatrix},
    \begin{pmatrix}
      1 & 2 & 3 \\
      3 & 2 & 1
    \end{pmatrix} \right\}
  \]
\end{example}

Note that, as in the above example, the \defn{identity
permutation}{permutation!identity} $p(k) = k$ is always a permutation.

Since every permutation is a one-to-one and onto function, there is an inverse
function $p^{-1}$ associated with every permutation $p$.

We can ``multiply'' two permutations by applying the first permutation, and
then using the second permutation to permute the result.  If $p$ and $q$ are
permutations of the same set, $pq(k)$ is the what you get from applying $q$
to $p(k)$, ie.\ $pq(k) = q(p(k))$, so $pq = q \circ p$ (note the reversal of
terms in the product versus the composition).

\begin{proposition}\label{prop:permgroup}
  Let $X$ be any set, and $p$ and $q$ be permutations of $X$, then
  \begin{theoremenum}
    \item $p^{-1}$ is a permutation of $X$,
    \item $pq$ is a permutation of $X$,
    \item the product satisfies an associative law: $(pq)r = p(qr)$.
  \end{theoremenum}
\end{proposition}
\begin{proof}
  These follow immediately from Proposition~\ref{prop:functionfacts}:
  the inverse
  function of a bijection is a bijection, proving (i); the composition
  of bijective functions is a bijective function, proving (ii); and
  composition of functions is associative, so 
  \[
    (pq)r = r \circ (q \circ p) = (r \circ q) \circ p = p(qr),
  \]
  proving (iii).
\end{proof}

\begin{example}
  Let
  \[
    p = \begin{pmatrix}
      1 & 2 & 3 \\
      3 & 2 & 1
    \end{pmatrix}
    \qquad \text{and} \qquad
    q = \begin{pmatrix}
      1 & 2 & 3 \\
      3 & 1 & 2
    \end{pmatrix}.
  \]
  We can find $pq$ fairly easily: for example if
  $k=1$, we know that $p(1) = 3$, and $q(3) = 2$, so $pq(1) = 2$. Repeating
  for $k = 2$ and $3$, we get So
  we have
  \[
    pq = \begin{pmatrix}
      1 & 2 & 3 \\
      2 & 1 & 3
    \end{pmatrix}.
  \]
\end{example}

\begin{example}\label{eg:perm3part2}
  We listed all the elements of $S_{3}$ in Example~\ref{eg:perm3part1}.
  To simplify notation we will give each of these a symbol to identify it:
  \begin{alignat*}{6}
    p_{0} &= \begin{pmatrix}
      1 & 2 & 3 \\
      1 & 2 & 3
    \end{pmatrix} &\qquad&
    p_{1} &= \begin{pmatrix}
      1 & 2 & 3 \\
      3 & 1 & 2
    \end{pmatrix} &\qquad&
    p_{2} &= \begin{pmatrix}
      1 & 2 & 3 \\
      2 & 3 & 1
    \end{pmatrix}\\
    p_{3} &= \begin{pmatrix}
      1 & 2 & 3 \\
      1 & 3 & 2
    \end{pmatrix} &&
    p_{4} &= \begin{pmatrix}
      1 & 2 & 3 \\
      2 & 1 & 3
    \end{pmatrix} &&
    p_{5} &= \begin{pmatrix}
      1 & 2 & 3 \\
      3 & 2 & 1
    \end{pmatrix}
  \end{alignat*}
  It is easy to verify that $p_{0}^{-1} = p_{0}$, $p_{1}^{-1} = p_{2}$,
  $p_{2}^{-1} = p_{1}$, $p_{3}^{-1} = p_{3}$, $p_{4}^{-1} = p_{4}$, 
  and $p_{5}^{-1} = p_{5}$.
  
  Just as with symmetries, we can write out a Cayley table for the products
  of these permutations:
  \[
  \begin{array}{c|cccccc}
               & p_{0} & p_{1} & p_{2} & p_{3} & p_{4} & p_{5} \\
    \hline
    p_{0} & p_{0} & p_{1} & p_{2} & p_{3} & p_{4} & p_{5} \\
    p_{1} & p_{1} & p_{2} & p_{0} & p_{4} & p_{5} & p_{3} \\
    p_{2} & p_{2} & p_{0} & p_{1} & p_{5} & p_{3} & p_{4} \\
    p_{3} & p_{3} & p_{5} & p_{4} & p_{0} & p_{2} & p_{1} \\
    p_{4} & p_{4} & p_{3} & p_{5} & p_{1} & p_{0} & p_{2} \\
    p_{5} & p_{5} & p_{4} & p_{3} & p_{2} & p_{1} & p_{0} 
  \end{array}
  \]
  This product is not commutative.
  
  It's probably not immediately obvious, but if you look closely you will see
  that the pattern of this Cayley table is exactly the same as the pattern of
  the Cayley table of Example~\ref{eg:symmtriangle}, with the correspondences
    $p_{0} \leftrightarrow I$,
    $p_{1} \leftrightarrow R_{1}$,
    $p_{2} \leftrightarrow R_{2}$,
    $p_{3} \leftrightarrow H_{0}$,
    $p_{4} \leftrightarrow H_{1}$,
    $p_{5} \leftrightarrow H_{2}$.
  Indeed, the inverses of each element have the same pattern under these same
  correspondences.
  
  In other words, if we look at these two examples abstractly, we seem to be
  getting the same underlying mathematical object.
  
  This correspondence can be made even more concrete in the following way: if
  we label the vertices of the equilateral triangle of Example~\ref{eg:symmtriangle}
  with the numbers 1, 2 and 3, starting at $(0,0)$ and working clockwise, we
  find that the symmetries of the triangle permute the vertices in exactly
  the same way that the corresponding permutations permute the corresponding
  numbers.
\end{example}

\subsection{Cycles}

Even with the current notation, expressing and working with permutations can
be cumbersome.  There is another, alternative, notation which can speed up
the process of working with permutations.  This notation works by looking at
the \defn{cycles}{cycle} withing a permutation.  If $p$ is a permutation
of the set $X$, the cycle of an element $k$ of $X$ in $p$ is the sequence
of elements $(k, p(k), p^{2}(k), \ldots, p^{m}(k))$ (where $p^{l}$ is the product
of $p$ with itself $l$ times) such that $m$ is the smallest number such that,
$p^{m+1}(k) = k$.

Note that the order of the elements in a cycle is important, but not where we
start in the cycle.  For example, we regard $(k, p(k), p^{2}(k), \ldots, p^{m}(k))$,
$(p(k),$ $p^{2}(k),\ldots, p^{m}(k), k)$, $(p^{2}(k), \ldots, p^{m}(k), k, p(k))$, etc.\ as
representing the same cycle.  If $X$ is the set $\{1, 2, \ldots n\}$, it is
usual to write a cycle starting with the smallest number in the cycle.

A cycle with $m$ elements is called an \defn{$m$-cycle}{$m$-cycle}.  A 2-cycle
is sometimes called a \defn{transposition}{transposition}, since it transposes
two elements.

\begin{example}
  In the following permutation
  \[
    \begin{pmatrix}
      1 & 2 & 3 & 4 & 5 & 6 & 7 & 8 \\
      2 & 4 & 6 & 8 & 7 & 5 & 3 & 1
    \end{pmatrix}
  \]
  we have $1 \to 2$, $2 \to 4$, $4 \to 8$ and $8 \to 1$, so $(1,2,4,8)$ is
  a cycle.  We could also write this cycle as $(2, 4, 8, 1)$, $(4,8, 1, 2)$,
  or $(8, 1, 2, 4)$.
  
  The smallest element not in this cycle is $3$, and we have
  $3 \to 6$, $6 \to 5$, $5 \to 7$ and $7 \to 3$, so $(3, 6, 5, 7)$ is another
  cycle.
  
  Since every element is in one of these two cycles, these are the only cycles
  in this permutation.
\end{example}

If we find all of the cycles of a permutation, we can represent the permutation
as a whole as a product of its cycles.  But to do that we need to understand how
to multiply cycles.

To work out how a product of cycles permutes a particular element $k$, all you
need do is work from left to right until you find the element in a cycle, and
then find the element which follows it in that cycle.  You continue from left
to right starting with the the next cycle looking for an occurrence of the new
element.  If there is, then you find the element which follows it in the cycle.
Continue on in this fashion until you run out of cycles.  The final value of
the element is the image of $k$ under the product of cycles.

\begin{example}
  Consider the permutation $p = (1, 3, 5)(2)(4, 6)$ of the set
  $\{1, 2, 3, 4, 5, 6\}$.  We can calculate $p(1)$ be looking at the first
  cycle, where we see that the element after $1$ in that cycle is $3$, and we also
  note that $3$ does not occur in any cycle after the first, so $p(1) = 3$.
  Similarly, we have $p(2) = 2$, $p(3) = 5$, $p(4) = 6$, $p(5) = 1$ and
  $p(6) = 4$.  This permutation could also be written as
  \[
    \begin{pmatrix}
      1 & 2 & 3 & 4 & 5 & 6\\
      3 & 2 & 5 & 6 & 1 & 4
    \end{pmatrix}.
  \]
\end{example}

Notice that there would be no difference in the above example if the cycle
$(2)$ was omitted.  It is common practise to leave such single-element
cycles out, particularly when the set which is being permuted is clear.

\begin{example}
  Consider the product of cycles $p = (1, 3, 5)(2, 3)(4, 6, 5)$ in the set
  $\{1, 2, 3, 4, 5,$ $6\}$.  We can calculate $p(1)$ be looking at the first
  cycle, where we see that the element after $1$ in that cycle is $3$; however
  $3$ occurs in the second cycle, and the element after it in the cycle is $2$;
  and $2$ does not occur in the remaining cycle, so $p(1) = 2$.  Similarly,
  we have $3 \to 5$ in the first cycle, and $5 \to 4$ in the last cycle, so
  $p(3) = 4$.
  Calculating everything out, we have $p(2) = 3$, $p(4) = 6$, $p(5) = 1$ and
  $p(6) = 5$.  This permutation could also be written as
  \[
    \begin{pmatrix}
      1 & 2 & 3 & 4 & 5 & 6\\
      2 & 3 & 4 & 6 & 1 & 5
    \end{pmatrix},
  \]
  or more simply in cycle notation as $(1, 2, 3, 4, 6, 5)$.
\end{example}

Any permutation can be written as a product of the cycles it contains.

\begin{theorem}
  Every permutation of $S_{n}$ can be written as a product of disjoint cycles.
  (Two cycles are disjoint if they have not elements in common.)
\end{theorem}
\begin{proof}
  Let $p$ be a permutation of $S_{n}$.  We let $c_{1}$ be the cycle which
  includes $1$,
  \[
    c_{1} = \{1, p(1), p^{2}(1), \ldots, p^{m_{1}}(1)\},
  \]
  and we let $p_{1}$ be the permutation defined by
  \[
    p_{1}(k) = \begin{cases}
      k & \text{if $k \in c_{1}$,}\\
      p(k) & \text{otherwise}.
    \end{cases}
  \]
  Then it is clear that $p = c_{1}p_{1}$.
  
  Now if we have written $p = c_{1}\ldots c_{l}p_{l}$, where $c_{1}, \ldots,\
  c_{l}$ are disjoint cycles, and $p_{l}$ is a permutation which satisfies
  has $p_{l}(k) = k$ whenever $k$ is in one of the cycles, then one of two
  things must be true: either every element of $\{1, 2, \ldots, n\}$ is an
  element of one of the cycles, or there is some smallest element $k_{l}$
  which is not in any of the cycles.
  
  In the first case, we have that $p_{l}$ must be the identity permutation,
  so $p = c_{1}\ldots c_{l}$, and we are done.
  
  In the second case, we let $c_{l+1}$ be the cycle including $k_{l}$,
  \[
    c_{l+1} = \{k_{l}, p(k_{l}), p^{2}(k_{l}), \ldots, p^{m_{l}}(k_{l})\},
  \]
  and let $p_{l+1}$ be the permutation defined by
  \[
    p_{l+1}(k) = \begin{cases}
      k & \text{if $k$ is an element of any cycle $c_{1}$, $c_{2}, \ldots, c_{l+1}$}\\
      p(k) & \text{otherwise}.
    \end{cases}
  \]
  Then $p = c_{1}\ldots c_{l+1}p_{l+1}$.
  
  Since $\{1,2,3,\ldots, n\}$ is a finite set, an induction argument using
  this construction proves the result.
\end{proof}

\begin{example}
  The permutation
  \[
    \begin{pmatrix}
      1 & 2 & 3 & 4 & 5 & 6 & 7 & 8 \\
      2 & 4 & 6 & 8 & 7 & 5 & 3 & 1
    \end{pmatrix}
  \]
  can be written as $(1, 2, 4, 8)(3, 6, 5, 7)$ or $(3, 6, 5, 7)(1, 2, 4, 8)$,
  or in many other ways.  The first is the standard form.
\end{example}

\begin{example}
  The elements of $S_{3}$ can be represented in cycle form as follows:
  \begin{alignat*}{6}
    \begin{pmatrix}
      1 & 2 & 3 \\
      1 & 2 & 3
    \end{pmatrix} &= (1)(2)(3) &\qquad&
    \begin{pmatrix}
      1 & 2 & 3 \\
      3 & 1 & 2
    \end{pmatrix} &= (1, 3, 2) &\qquad&
    \begin{pmatrix}
      1 & 2 & 3 \\
      2 & 3 & 1
    \end{pmatrix} &= (1, 2, 3)\\
    \begin{pmatrix}
      1 & 2 & 3 \\
      1 & 3 & 2
    \end{pmatrix} &= (1)(2, 3) &&
    \begin{pmatrix}
      1 & 2 & 3 \\
      2 & 1 & 3
    \end{pmatrix} &= (1, 2)(3)&&
    \begin{pmatrix}
      1 & 2 & 3 \\
      3 & 2 & 1
    \end{pmatrix} &= (1, 3)(2)
  \end{alignat*}
\end{example}

It is often convenient to simply always work with the cycle form of a
permutation. We can calculate the product of two permutations in cycle
notation by writing all long product of cycles, and then reducing to
the standard form of the cycles.

\begin{example}
  Let $p = (1, 4, 6)(3, 5)$ and $q = (1, 2, 4)(3, 6, 5)$.  Then $pq$ is given
  by the product of cycles $(1, 4, 6)(3, 5)(1, 2, 4)(3, 6, 5)$.
  
  Starting with $1$, we see that $1 \to 4 \to 1$, so the first cycle of the
  standard form is just $(1)$.
  
  Looking at $2$ next, we have $2 \to 4$, so the next cycle
  starts $(2, 4, \ldots)$.  Looking at $4$, we get $4 \to 6 \to 5$, so $5$
  is the next entry, and the cycle is starting $(2, 4, 5, \ldots)$.
  Now starting with $5$ we get $5 \to 3 \to 6$, so $6$
  is next in the cycle.  Continuing in this manner we get $6 \to 1 \to 2$,
  and $2$ is the start of the cycle, so the finished cycle is $(2, 4, 5, 6)$.
  
  The only remaining number is $3$, so $(3)$ must be the last cycle.
  
  Hence $pq = (1)(2, 4, 5, 6)(3)$, which we will usually just write as
  $pq = (2, 4, 5, 6)$.
  
  Similarly we have that $qp = (1, 2, 4)(3, 6, 5)(1, 4, 6)(3, 5)$, and
  we have $1 \to 2$, $2 \to 4 \to 6$, $6 \to 5 \to 3$ and $3 \to 6 \to 1$,
  so the first cycle is $(1, 2, 6, 3)$.  Similarly, we have $4 \to 1 \to 4$,
  so $(4)$ is a cycle.  Finally $5$ is the only element remaining, so $(5)$
  is a cycle.  Hence $qp = (1, 2, 6, 3)(4)(5) = (1, 2, 6, 3)$.
\end{example}

\subsection{Parity}

Informally, if we compare the permuations
\[
  \begin{pmatrix}
    1 & 2 & 3\\
    3 & 1 & 2
  \end{pmatrix}
  \qquad \text{and} \qquad
  \begin{pmatrix}
    1 & 2 & 3\\
    3 & 2 & 1
  \end{pmatrix},
\]
we observe that the first ``rotates'' the elements to the right, while
the second ``reflects'' the elements.  Indeed, if we consider the
correspondence between these permutations and the symmetries of a triangle
discussed in Example~\ref{eg:perm3part2}, we see that the first corresponds
to a rotation, and the second to a reflection.  In this section, we will
generalize this idea to arbitrary permutations.

The starting point of this discussion is a comparison between the following
two products: if $p \in S_{n}$ we define
\[
  D_{n} = \prod_{1 \le i < j \le n} (j - i)
\]
and
\[
  D(p) = \prod_{1 \le i < j \le n} (p(j) - p(i)).
\]
Given any pair of distinct elements $k,l \in \{1, 2, \ldots, n\}$, both of
these products contain exactly one factor which is a difference of $k$ and $l$.
This is easy to see in the product $D_{n}$, but a little thought will convince
you that it is also the case for $D(p)$.  The difference between the two
products is that in $D(p)$ it may not necessarily be the larger term minus
the smaller term.  Hence $D_{n}$ and $D(p)$ have the same magnitude, but may differ
in sign.

\begin{definition}
  Let $p \in S_{n}$.  The \defn{parity}{parity} of $p$ is
  \[
    \parity(p) = \frac{D(p)}{D_{n}}.
  \]
\end{definition}

Clearly the parity of $p$ is $1$ if $D(p) > 0$ and $-1$ if $D(p) < 0$.

\begin{example}
  Consider the permutations
  \[
    p = \begin{pmatrix}
      1 & 2 & 3\\
      3 & 1 & 2
    \end{pmatrix}
    \qquad \text{and} \qquad
    q = \begin{pmatrix}
      1 & 2 & 3\\
      3 & 2 & 1
    \end{pmatrix}.
  \]
  In both cases
  \[
    D_{3} = (3 - 1)(3 - 2)(2 - 1) = 2 \times 1 \times 1 = 2.
  \]
  Now
  \begin{align*}
    D(p) &= (p(3) - p(1))(p(3) - p(2))(p(2) - p(1)) = (2 - 3)(2 - 1)(1 - 3) \\
      &= -1 \times 1 \times -2 = 2,
  \end{align*}
  so
  \[
    \parity(p) = 2/2 = 1.
  \]
  On the other hand,
  \begin{align*}
    D(q) &= (q(3) - q(1))(q(3) - q(2))(q(2) - q(1)) = (1 - 3)(1 - 2)(2 - 3)\\
      &= -2 \times -1 \times -1 = -2,
  \end{align*}
  so
  \[
    \parity(q) = -2/2 = -1.
  \]
\end{example}

Calculating the parity in the last example was fairly straightforward,
but calculating the parity of a general permutation can be quite time
consuming: a simple counting argument tells us that if $p \in S_{n}$ we
have $n(n-1)/2$ terms in the product $D(p)$.  We need a better way to
calculate the parity.

It turns out that there is nothing particularly special about $D_{n}$ in
the definition of parity.  Let $(x_{1}, x_{2}, \ldots, x_{n})$ be a sequence of distinct numbers, and $p$
a permutation of $\{1, 2, \ldots n\}$.  We define
\begin{equation}\label{eqn:permutationaction}
  p(x_{1}, x_{2}, \ldots, x_{n}) = (x_{p(1)}, x_{p(2)}, \ldots, x_{p(n)}),
\end{equation}
and
\[
  D(x_{1}, x_{2}, \ldots, x_{n}) = \prod_{1 \le i < j \le n} (x_{j} - x_{i}).
\]
The following technical lemma shows that we can use these instead to find the
parity of $p$.

\begin{lemma}\label{lemma:paritytechnical}
  If $p$ is a permutation in $S_{n}$, then
  \[
    \parity(p) = \frac{D(x_{p(1)}, x_{p(2)}, \ldots, x_{p(n)})}{D(x_{1}, x_{2},
    \ldots, x_{n})}.
  \]
  We say that $p$ is an \defn{even permutation}{permutation!even} if
  $\parity(p) = 1$, and $p$ is an \defn{odd permutation}{permutation!odd} if
  $\parity(p) = -1$.
\end{lemma}
\begin{proof}
  See Section~\ref{subsection:parity}.
\end{proof}

Note that $D_{n} = D(1, 2, \ldots, n)$, and $D(p) = D(p(1), p(2), \ldots, p(n))$.

With this lemma in hand, we can easily prove the following important result:

\begin{theorem}\label{thm:parityproduct}
  Let $p$ and $q \in S_{n}$.  Then
  \[
    \parity(pq) = \parity(p)\parity(q).
  \]
\end{theorem}
\begin{proof}
  The key observation here is that if we have permutations $p$ and $q$,
  then
  \[
    \parity(p) = \frac{D(pq)}{D(q)}.
  \]
  Letting $a_{k} = q(k)$, so that
  \[
    D(pq) = D(q(p(1)), q(p(2)), \ldots, q(p(n))) = D(a_{p(1)}, \ldots, a_{p(n)}),
  \]
  and
  \[
    D(q) = D(a_{1}, \ldots, a_{n}),
  \]
  the lemma tells us that
  \[
    \parity(p) = \frac{D(a_{p(1)}, \ldots, a_{p(n)})}{D(a_{1}, \ldots, a_{n})}
     = \frac{D(pq)}{D(q)}.
  \]
  
  It is immediate form this that
  \[
    \parity(p) \times \parity(q) = \frac{D(pq)}{D(q)} \times \frac{D(q)}{D_{n}}
    = \frac{D(pq)}{D_{n}} = \parity(pq).
  \]
\end{proof}

Thinking in terms of cycles also helps us to calculate the parity of a
permutation, as this result shows:

\begin{theorem}
  Let $c = (k_{1}, k_{2}, \ldots, k_{m})$ be a cycle.  Then
  \[
    \parity(c) = \begin{cases}
      1 & \text{if $m$ is odd} \\
      -1 & \text{if $m$ is even}.
    \end{cases}
  \]
\end{theorem}
\begin{proof}
  First we observe that if $p = (1, 2)$, then all the factors in $D(p)$
  are positive, except for $p(2) - p(1) = 1 - 2 = -1$.  Hence $D(p)$ is
  negative, so $\parity(p) = -1$.
  
  Now if $i, j > 2$, and $i \ne j$, then simple checking shows that
  $(i,j) = (1,i)(2,j)(1,2)(1,i)(2,j)$,
  and, given this fact, the previous theorem tells us
  \begin{align*}
    \parity((i,j)) &= \parity((1,i)(2,j)) \times \parity(1,2) \times \parity((1,i)(2,j)) \\
    &= -\parity((1,i)(2,j))^{2}\\
    &= -1
  \end{align*}
  
  Finally, we observe that
  \begin{equation}\label{eqn:cycleproduct}
    c = (k_{1}, k_{2}, \ldots, k_{m}) = (k_{1}, k_{m})(k_{2}, k_{m})\cdots (k_{m-1}, k_{m}),
  \end{equation}
  and so
  \begin{align*}
    \parity(c) &= \parity((k_{1}, k_{m})) \times \parity((k_{2}, k_{m})) \times \cdots \times \parity((k_{m-1}, k_{m}))\\
    &= (-1)^{m-1} \\
    &= \begin{cases}
    1 & \text{if $m$ is odd} \\
    -1 & \text{if $m$ is even.}
    \end{cases}
  \end{align*}
\end{proof}

\begin{corollary}
  A permutation $p$ is even iff when it is expressed as a product of
  cycles there are an even number of commas in the expression.
\end{corollary}

Another interesting fact that can be squeezed out of the previous theorem
is the following:

\begin{proposition}
  Any permutation $p$ can be written as a product of $2$-cycles, and the
  number of $2$-cycles is even iff $p$ is even.
\end{proposition}
\begin{proof}
  The first fact follows from the fact that every permutation can be written
  as a product of cycles, and equation (\ref{eqn:cycleproduct}) shows that every cycle
  is a product of $2$-cycles.
  
  The second follows from the previous corollary, coupled with the fact that
  every $2$-cycle has a single comma.
\end{proof}

\begin{example}
  Let $p_{0}$, $p_{1}, \ldots, p_{5}$ be as in Example~\ref{eg:perm3part2}.
  Then $\parity(p_{0}) = \parity(p_{1}) = \parity(p_{2}) = 1$, and
  $\parity(p_{3}) = \parity(p_{4}) = \parity(p_{5}) = -1$.
  
  Observe that the ``reflections'' have parity -1, while the ``rotations''
  have parity 1.
\end{example}

\begin{example}
  The permutation
  \[
    p = \begin{pmatrix}
     1 & 2 & 3 & 4 & 5 & 6 & 7 & 8 & 9 & 10 \\
     3 & 5 & 2 & 7 & 8 & 6 & 1 & 4 & 10 & 9
    \end{pmatrix},
  \]
  can be written as $p = (1, 3, 2, 5, 8, 4, 7)(9, 10)$, and since it has $7$
  commas in the expression, it has parity $-1$.
\end{example}

\subsection{Permutation Matrices}

Another way of looking at permutations is very closely related to
Equation~\ref{eqn:permutationaction}.  

Since $x = (x_{1},x_{2},\ldots,x_{n})$ is a vector in $\reals^{n}$, the function
it implicitly defines, $T_{p}: \reals^{n} \to \reals^{n}$, where
\[
  T_{p}x = (x_{p(1)}, x_{p(2)}, \ldots, x_{p(n)})
\]
is a linear transformation:
\begin{align*}
  T_{p}(x+y) &= (x_{p(1)} + y_{p(1)}, x_{p(2)} + y_{p(2)}, \ldots, x_{p(n)} + y_{p(n)})\\
    &= (x_{p(1)}, x_{p(2)}, \ldots, x_{p(n)}) + (y_{p(1)}, y_{p(2)}, \ldots, y_{p(n)})
    = T_{p}x + T_{p}y
\end{align*}
and
\begin{align*}
  T_{p}(\lambda x) &= (\lambda x_{p(1)}, \lambda x_{p(2)}, \ldots, \lambda x_{p(n)})\\
    &= \lambda (x_{p(1)}, x_{p(2)}, \ldots, x_{p(n)})
    = \lambda T_{p}x.
\end{align*}

Let $e_{k}$ be the $k$th standard orthonormal basis vector in $\reals^{n}$,
ie.\ $e_{k}$ is the vector with $0$ in every entry except the $k$th entry,
which is $1$.  By looking at the image of each standard basis vector $e_{k}$
under the transformation $T_{p}$, we can find a corresponding $n \times n$
matrix which we will also call $T_{p}$.
We note that $T_{p}e_{k}$ has zeroes in every entry except the $p^{-1}(k)$th
entry, which is $1$.  Hence $T_{p}e_{k} = e_{p^{-1}(k)}$, so $T_{p}$ always
takes basis vectors to basis vectors.

\begin{example}
  If $p = (1,2,3) \in S_{3}$, then
  \[
    T_{p} = \begin{bmatrix}
      0 & 1 & 0\\
      0 & 0 & 1\\
      1 & 0 & 0
    \end{bmatrix}
  \]
\end{example}

\begin{proposition}
  Let $p \in S_{n}$ be a permutation.  Then
  \begin{theoremenum}
    \item $T_{p}$ is an orthogonal matrix
    \item $T_{p}$ is the matrix with $1$s in the $p^{-1}(k)$th row of the
      $k$th column, for $k = 1,2,\ldots,n$, and $0$ everywhere else.
    \item $T_{p}$ is the matrix with $1$s in the $p(k)$th column of the
      $k$th row, for $k = 1,2,\ldots,n$, and $0$ everywhere else.
    \item $T_{e} = I_{n}$.
    \item $T_{p}T_{q} = T_{pq}$.
    \item $T_{p}^{-1} = T_{p^{-1}}$.
  \end{theoremenum}
\end{proposition}
\begin{proof}
  (i) is immediate since every column of $T_{p}$ is a standard orthonormal
  basis vector, and each vector in the basis occurs exactly once.
  
  (ii) this is immediate from the fact that the $k$th column is the column
  vector $e_{p^{-1}(k)}$.
  
  (iii) using part (ii), we know that is $k = p^{-1}(j)$, then the entry in
  the $k$th row and $j$th column is $1$.  But $k = p^{-1}(j)$ if and only if
  $j = p(k)$, so the $p(k)$th column of the $k$th row is $1$, and all other
  entries in the row are $0$.
  
  (iv) the $k$th row of $T_{e}$ is $e_{k}$, so $T_{e}$ has $1$ in diagonal
  entries and $0$ everywhere else, so $T_{e} = I$.
  
  (v) Looking at the standard basis vectors, we have
  \[
    T_{pq}e_{k} = e_{(pq)^{-1}(k)} = e_{p^{-1}(q^{-1}(k))} =
    T_{p}e_{q^{-1}(k)} = T_{p}T_{q}e_{k}.
  \]
  Since any vector $v$ is a linear combination of basis vectors, and the
  transformations $T_{p}$, $T_{q}$ and $T_{pq}$ are all linear, we have that
  $T_{pq}v = T_{p}T_{q}v$ for any vector $v$, and so $T_{p}T_{q} = T_{pq}$.
  
  (vi) Since $T_{p}$ is orthogonal, it is invertible, and
  $T_{p}^{-1}e_{p^{-1}(k)} = e_{k}$.  Now $j = p^{-1}(k)$ if and only if
  $k = p(j)$, so
  \[
    T_{p}^{-1}e_{j} = e_{k} = e_{p(j)} = e_{(p^{-1})^{-1}(j)} = T_{p^{-1}}e_{j}.
  \]
  As in part (v), it is sufficient to show that this occurs for every basis
  vector to be able to conclude that $T_{p}^{-1} = T_{p^{-1}}$.
\end{proof}

\begin{example}
  We know that in $S_{3}$, if $p = (1,2,3)$ and $q = (1,2)$, then $pq = (2,3)$.
  The corresponding permutation matrices are
  \[
    T_{p} = \begin{bmatrix}
      0 & 1 & 0\\
      0 & 0 & 1\\
      1 & 0 & 0
    \end{bmatrix}
    \qquad
    T_{q} = \begin{bmatrix}
      0 & 1 & 0\\
      1 & 0 & 0\\
      0 & 0 & 1
    \end{bmatrix}
    \qquad
    T_{pq} = \begin{bmatrix}
      1 & 0 & 0\\
      0 & 0 & 1\\
      0 & 1 & 0
    \end{bmatrix}
  \]
  and matrix multiplication confirms that $T_{p}T_{q} = T_{pq}$.
\end{example}

\subsection*{Exercises}

\begin{exercises}
  \item Let
    \[
      p = \begin{pmatrix}
        1 & 2 & 3 & 4 & 5 & 6 \\
        4 & 2 & 5 & 1 & 6 & 3
      \end{pmatrix}
      \qquad \text{and} \qquad
      q = \begin{pmatrix}
        1 & 2 & 3 & 4 & 5 & 6 \\
        3 & 4 & 5 & 1 & 2 & 6
      \end{pmatrix}.
    \]
    Find $pq$ and $qp$.  Write both permutations using cycle notation.
    Write down the permutation matrices $T_{p}$ and $T_{q}$.
    Determine the parity of $p$ and $q$.
  
  \item Let $p = (1, 5, 3, 2)(4, 6, 8)$ and $q = (1, 7, 4, 3)(8, 2)(5, 6)$.
    Find $pq$ and $qp$.  Write both permutations using array notation.
    Write down the permutation matrices $T_{p}$ and $T_{q}$.
    Determine the parity of $p$ and $q$.

  \item How many distinct permutations are there of the set $\{1, 2,
    \ldots, n\}$? (Hint: they're called \emph{permutations}.)
  
  \item Let $p \in S_{3}$.  Use the Cayley table for $S_{3}$ to show that
    $p^{6}$ is always the identity permutation.
  
  \item Write down all the elements of $S_{4}$ in array, cycle and matrix form.
    Calculate the parity of each element.
    Find the inverse of each element.
    Choose 5 pairs of non-identity elements, and calculate their product.
  
  \item Let $c = (k_{1}, k_{2}, \ldots, k_{m})$ be a cycle.  What is $c^{-1}$?
    Use your answer to calculate the inverse of the permutation
    $p = (1,3,4)(2,5)$.
  
  \item Show that $D_{n} = (n-1)! (n-2)! \ldots 2! 1!$.
  
  \item Show that exactly half the permutations of $S_{n}$ are even, and half
    are odd.
  
  \item (*) Show that $S_{4}$ and the set of symmetries of a regular tetrahedron
    (see Exercise~\ref{ex:symtetra}) correspond
    in the same way as $S_{3}$ and the set of symmetries of an equilateral
    triangle.
    
    Hint: you could do this by calculating all $576$ entries in the Cayley
    table of each, and comparing the two; however it is more practical to
    find some way to classify the elements of each in a way which makes the
    correspondence clear.
  
  \item (*) Let $\Omega \subseteq \reals^{3}$ be the equilateral triangle
    with vertices $(1,0,0)$, $(0,1,0)$ and $(0,0,1)$.  Show that every
    symmetry $S \in \Sym(\Omega)$ is a linear transformation, and that there
    is a permutation $p_{S} \in S_{3}$ such that $S = T_{p_{S}}$. 
    Show that the correspondence $S \mapsto p_{S}$ preserves multiplication
    and inverses, ie.\ $T_{p_{S}p{R}} = SR$, $T_{p_{S}^{-1}} = S^{-1}$.
  
  \item (*) Let $|A|$ denote the determinant of the matrix $A$.
    Prove that $|T_{p}| = \parity(p)$.
  
  \item (**) Write a computer program that calculates and prints out the
    Cayley table for $S_{4}$.  Generalize it to print out the Cayley table
    for $S_{n}$ for any $n$.
\end{exercises}

\section{Modulo Arithmetic}\label{section:moduloarithmetic}

We say that two numbers $x$ and $y$ are \defn{equal (modulo $m$)}{modulo!equality} if $x$ and
$y$ differ by a multiple of $m$, and we write
\[
  x \equiv y \pmod m
\]
to denote this situation.  Another equivalent (and useful) way to think of
this situation is that $x$ and $y$ have the same remainder when you divide
by $m$.  Since any number greater than $m$ is equal (modulo $m$) to a number
less than $m$, it is customary when working modulo $m$ to reduce your answer
to a number in the range $[0,m)$.

For example
\[
  -1 \equiv 7 \equiv 1023 \pmod 8,
\]
and we would usually write any of these three numbers as $7 \mod 8$ if it
were the solution to a problem.

When we are working modulo $m$, we can perform the operations of addition,
multiplication and subtraction as normal, but we reduce our answers to the
range $[0,m)$.  Indeed, in complicated expressions, one can reduce at
intermediate steps to simplify calculations:
\[
  7 \times 6 + 4 \times 3 \equiv 42 + 12 \equiv 54 \equiv 6 \pmod 8
\]
could be instead calculated as
\[
  7 \times 6 + 4 \times 3 \equiv 42 + 12 \equiv 2 + 4 \equiv 6 \pmod 8.
\]

Division is a trickier topic, but since we are usually performing modulo
arithmetic with integers the na\"{i}ve way of defining modulo division does
not make sense in most cases.  Nevertheless, we will see later on that in
some cases division does make sense.

We can write out addition and multiplication tables for operations modulo
some base, and we call these Cayley tables, just as before.

\begin{example}\label{eg:mod6}
  The addition and multiplication tables, modulo 6 are as follows:
  \[
    \begin{array}{c|cccccc}
      + & 0 & 1 & 2 & 3 & 4 & 5 \\
      \hline
      0 & 0 & 1 & 2 & 3 & 4 & 5 \\
      1 & 1 & 2 & 3 & 4 & 5 & 0 \\
      2 & 2 & 3 & 4 & 5 & 0 & 1 \\
      3 & 3 & 4 & 5 & 0 & 1 & 2 \\
      4 & 4 & 5 & 0 & 1 & 2 & 3 \\
      5 & 5 & 0 & 1 & 2 & 3 & 4
    \end{array}
  \qquad
    \begin{array}{c|cccccc}
      \times & 0 & 1 & 2 & 3 & 4 & 5 \\
      \hline
      0 & 0 & 0 & 0 & 0 & 0 & 0 \\
      1 & 0 & 1 & 2 & 3 & 4 & 5 \\
      2 & 0 & 2 & 4 & 0 & 2 & 4 \\
      3 & 0 & 3 & 0 & 3 & 0 & 3 \\
      4 & 0 & 4 & 2 & 0 & 4 & 2 \\
      5 & 0 & 5 & 4 & 2 & 3 & 1
    \end{array}
  \]
\end{example}

\subsection*{Exercises}

\begin{exercises}
  \item Write down the addition and multiplication tables modulo 5 and modulo 8.

  \item\label{ex:zerodivisor} Recall that two natural numbers $p$ and $q$ are \defn{coprime}{coprime} if their
    highest common factor is $1$.  Show that if $k$ and $m$ are coprime, then
    $xk \equiv 0 \pmod{m}$ if and only if $x$ is an integer multiple of $m$.
\end{exercises}

\section{Addendum: Technical Details}

We now provide the technical proofs that were omitted from earlier discussions
in this chapter.

\subsection{Symmetry}

In Proposition~\ref{prop:symmetryfacts} we make use of the following fact.

\begin{lemma}\label{lemma:isometrybijective}
  If $T$ is a function from $\reals^{n}$ to $\reals^{n}$ that preserves
  distance, then $T$ is one-to-one and onto.
\end{lemma}
\begin{proof}
  If $T(x_{1}) = T(x_{2})$ then $d(T(x_{1}), T(x_{2})) = 0$ so the fact that
$T$ preserves distances means that $d(x_{1}, x_{2}) = 0$.  But this implies
that $x_{1} = x_{2}$, so $T$ is one-to-one.

  If $x$, $y$ and $z \in \reals^{n}$ are the vertices of a triangle, then
  $T(x)$, $T(y)$ and $T(z)$ are vertices of a triangle as well, and since
  $d(T(x), T(y)) = d(x,y)$, $d(T(x), T(z)) = d(x,z)$, and $d(T(y), T(z)) =
  d(y,z)$, the triangles are congruent.  From this observation, it follows
  that any parallelogram $x$,$y$, $z$, $w \in \reals^{n}$ is congruent to the
  corresponding parallelogram $T(x)$,$T(y)$, $T(z)$, $T(w)$.

  If $T(0) = 0$ then this means that in particular, the parallelogram
  $0$, $x$, $x+y$, $y$ is congruent to the parallelogram $0$, $T(x)$, $T(x+y)$,
  $T(y)$.  Therefore, comparing opposite sides, $T(y) = T(x+y) - T(x)$, or
  $T(x+y) = T(x) + T(y)$.  Also, the ``triangle'' with vertices $0$, $x$,
  $\lambda x$ is congruent to the triangle with vertices $0$, $T(x)$,
  $T(\lambda x)$, but since the vertices of the first triangle are collinear,
  so must the vertices of the second, so $T(\lambda x)$ is a scalar multiple
  of $T(x)$, and $d(T(\lambda x), 0) = |\lambda| d(x,0)$, so $T(\lambda x) =
  \pm \lambda x$.  However, if $T(\lambda x) = -\lambda x$, then looking
  at the corresponding sides $x$ to $\lambda x$ and $T(x)$ to $T(\lambda x)$,
  we would have $d(T(\lambda x),T(x)) = |\lambda + 1| d(x,0)$, rather than
  $|\lambda - 1| d(x,0)$.
  
  Hence if $T(0) = 0$, $T$ is a linear transformation, and a linear transformation
  from $\reals^{n}$ to $\reals^{n}$ which preserves distance is orthogonal
  and hence onto.
  
  If $T(0) = c$, then the function $S(x) = T(x) - c$ preserves distances, and
  $S(0) = 0$, so $S$ is orthogonal and hence onto.  But then $T(x) = S(x) + c$,
  so given any $y$, there is some $x$ such that $S(x) = y-c$, and so $T(x) =
  S(x) + c = (y-c) + c = y$.  Hence $T$ is onto.
\end{proof}

\subsection{Parity}\label{subsection:parity}

\begin{proof}[Lemma~\ref{lemma:paritytechnical}]
  Given any number $k \in \{1, 2, \ldots, n\}$, let $a_{k} = p^{-1}(k)$.
  Then given any $k > l$, we have that $k = p(a_{k})$ and $l = p(a_{l})$.
  
  If $a_{k} > a_{l}$, in which case the corresponding terms in each of the
  sums $D_{n}$, $D(p)$, $D(x_{1}, \ldots, x_{n})$ and $D(x_{p(1)}, \ldots,
  x_{p(n)})$ are, respectively, $k - l$, $p(a_{k}) - p(a_{l})$, $x_{k} - x_{l}$,
  and $x_{p(a_{k})} - x_{p(a_{l})}$, with the first two being equal and the
  second two being equal, and so these terms in the quotients $D(p)/D_{n}$
  and $D(x_{p(1)}, \ldots, x_{p(n)})/D(x_{1}, \ldots, x_{n})$, respectively,
  cancel each other out.

  On the other hand, if $a_{k} < a_{l}$, the corresponding terms in each of the
  sums $D_{n}$, $D(p)$, $D(x_{1}, \ldots, x_{n})$ and $D(x_{p(1)}, \ldots,
  x_{p(n)})$ are, respectively, $k - l$, $p(a_{l}) - p(a_{k})$, $x_{k} - x_{l}$,
  and $x_{p(a_{l})} - x_{p(a_{k})}$, with the first two being negatives and the
  second two being negatives, and so these terms in the quotients $D(p)/D_{n}$
  and $D(x_{p(1)}, \ldots, x_{p(n)})/D(x_{1}, \ldots, x_{n})$, respectively,
  give a factor of $-1$.
  
  Hence the number of terms giving each of the factors $1$ and $-1$ in each
  quotient are equal, so
  \[
    \parity(p) = \frac{D(p)}{D_{n}}
      = \frac{D(x_{p(1)}, x_{p(2)}, \ldots, x_{p(n)})}{D(x_{1}, x_{2}, \ldots, x_{n})}.
  \]
\end{proof}


\newpage
\section*{Assignment 1}

The following exercises are due Friday, Februrary 13.

\begin{description}
  \item[1.1] Exercises 1, 4.
  \item[1.2] Exercise 2.
  \item[1.4] Exercises 1, 2, 3, 6.
  \item[1.5] Exercise 1.
\end{description}


\chapter{Groups}

Algebra concerns the abstraction of simple arithmetic operations to
situations where the quantities involved are unknown.  In this endeavour,
we discover that there are certain rules which always apply,
such as the commutative and associative laws of addition and multiplication,
and that these laws allow us to manipulate and simplify algebraic
expressions. As we learn more mathematics, we see similar rules appear over
and over again.

In abstract algebra, instead of concentrating on specific algebraic
settings (such as algebra with numbers, vectors or, now, permutations or
symmetries) we instead look
at the \emph{rules} of algebra and ask what we can infer from reasonable
collections of such rules.  We can then apply the knowledge so gained to a
surprisingly wide collection of concrete situations which happen to satisfy
such rules.

You may have already seen such an approach in linear algebra, where one
eventually considers abstract vector spaces (as opposed to concrete ones, such
as $\reals^{n}$).  One then finds that, for example, that differentiable
functions form a vector space, and that differentiation and integration are
linear transformations, giving new (and quite important) insight into calculus.

Our starting point,
then will be the \defn{group}{group}, an object which encapsulates a reasonable set
of rules for a single algebraic operation.

\section{Binary Operations}

A \defn{binary operation}{binary operation} is a type of function that we shall be
using regularly.  A binary operation $\ast$ is simply a function
\begin{align*}
  \ast : A \cross B &\to C\\
         (x,y) &\mapsto x \ast y.
\end{align*}
The distinction lies in that instead of using ``function-style'' notation
$\ast(x,y)$, it is traditional to write binary operations ``in-line'' as
$x \ast y$.  Often $A$, $B$ and $C$ are the same set, in which case we say
that a binary operation $\ast : A \cross A \to A$ is a \defn{binary
operation on $A$}{binary operation!on a set}.

A binary operation on $A$ is \defn{commutative}{commutative} if
\[
  x \ast y = y \ast x
\]
for all $x$, $y \in A$.
It is \defn{associative}{associative} if
\[
  (x \ast y) \ast z = x \ast (y \ast z) = x \ast y \ast z.
\]
for all $x$, $y$ and $z \in A$.

\begin{lemma}
  Let $\ast : A \cross A \to A$ be an associative binary operation. Then no
  matter where you put parentheses in the product
  $x_{1} \ast x_{2} \ast \cdots \ast x_{n}$, you get the same result.
\end{lemma}
\begin{proof}
  We prove this by induction.  Since the operation is associative, it is
  true for $n \le 3$ automatically.
  
  Now assume that this is true for all $n < k$.  We need to show that for
  any choice of $i$ and $j$ with $1 \le i < j < k$, we have
  \[
    (x_{1} \ast \cdots \ast x_{i}) \ast (x_{i+1} \ast \cdots \ast x_{k})
    = (x_{1} \ast \cdots \ast x_{j}) \ast (x_{j+1} \ast \cdots \ast x_{k})
  \]
  (we do not need to worry about where the parentheses go inside each factor,
  since they are products of less than $k$ terms).  Now we know that
  \[
    (x_{1} \ast \cdots \ast x_{i}) \ast (x_{i+1} \ast \cdots \ast x_{k})
    = (x_{1} \ast \cdots \ast x_{i}) \ast
    ((x_{i+1} \ast \cdots \ast x_{j}) \ast (x_{j+1} \ast \cdots \ast x_{k}))
  \]
  since the second term is a product of less than $k$ terms.  Similarly
  \[
    (x_{1} \ast \cdots \ast x_{j}) \ast (x_{j+1} \ast \cdots \ast x_{k})
    = ((x_{1} \ast \cdots \ast x_{i}) \ast
    (x_{i+1} \ast \cdots \ast x_{j})) \ast (x_{j+1} \ast \cdots \ast x_{k}).
  \]
  But $\ast$ is associative, so
  \begin{align*}
    (x_{1} \ast \cdots \ast x_{i}) &\ast
    ((x_{i+1} \ast \cdots \ast x_{j}) \ast (x_{j+1} \ast \cdots \ast x_{k}))\\
    &= y_{1} \ast (y_{2} \ast y_{3}) \\
    &= (y_{1} \ast y_{2}) \ast y_{3} \\
    &= ((x_{1} \ast \cdots \ast x_{i}) \ast
    (x_{i+1} \ast \cdots \ast x_{j})) \ast (x_{j+1} \ast \cdots \ast x_{k}),
  \end{align*}
  and we have proven equality of the two expressions.
  
  By induction, the result follows for any $k$.
\end{proof}

An element $e$ of $A$ is an \defn{identity}{identity} for the binary operation if
\[
  e \ast x = x \qquad \text{and} \qquad x \ast e = x
\]
for every $x \in A$.  More generally, one can have a \defn{left identity}{identity!left}
$e$ which merely satisfies
\[
  e \ast x = x
\]
for every $x$.  A \defn{right identity}{identity!right} is defined analagously.

\begin{lemma}
  If $\ast: A \cross A \to A$ is a binary operation, and $e$ is an identity for
  $\ast$, then it is the only identity element.
\end{lemma}
\begin{proof}
  Assume that there is another element $e'$ so that $e' \ast x = x \ast e' =
x$.  Then in particular, if we let $x = e$, we have $e' \ast e = e$.  But by
assumption, $e$ is an identity, and so $e' \ast e = e'$.  Hence $e = e'$.
\end{proof}

Notationally, if $\ast$ behaves in a ``multiplication-like'' fashion, or it
is clear from context which binary operation we are using, we will often simply
write $xy$ for $x \ast y$.

\begin{example}
The addition operation is a binary operation in the integers
\begin{align*}
  + : \integers \cross \integers &\to \integers\\
         (x,y) &\mapsto x + y.
\end{align*}
In this case we could write $+(2,3) = 2 + 3 = 5$.  Addition is, of course,
both associative and commutative.  $0$ is an identity for addition.  In fact
addition is also an associative and commutative binary operation on any of the
standard number systems, and if $0$ is in the number system, then $0$ is an
identity.
\end{example}

\begin{example}
Multiplication is a binary operation on the set $\reals$, and it is associative
and commutative, and $1$ is an identity.  Again, like addition,
multiplication is communtative and associative on any of the
standard number systems, and if $1$ is in the number system, then $1$ is an
identity.
\end{example}

\begin{example}
If we consider the set $M_{n}(\reals)$ of $n \times n$ real-valued matrices,
then matrix addition and matrix multiplication are binary operations.  Both
operations are associative, but only matrix addition is commutative.  The
zero matrix is an identity for addition, the identity matrix $I_{n}$ (the
matrix with $1$ down the diagonal and $0$ elsewhere) is an identity for matrix
multiplication.

We also have scalar multiplication as a binary operation $\reals \cross
M_{n}(\reals) \to M_{n}(\reals)$.  This cannot be commutative or
associative, but it does have $1$ as a left identity.
\end{example}

\begin{example}
More generally, the inner or dot product on $\reals^{n}$ is a binary
operation $\cdot: \reals^{n} \cross \reals^{n} \to \reals$ which is
commutative, but cannot be associative (since the codomain is $\reals$, and
one cannot take a dot product of an element of $\reals$ and an element of
$\reals^{n}$).  Similarly, there is no identity element of any sort.
\end{example}

\begin{example}
One can define arbitrary binary products which are of little or no interest.
For example, $x \ast y = e^{x}(x + \sin(y))$ is a binary product.  But it is
neither associative, commutative, nor has an identity.  So clearly simple
binary operations are not enough to encapsulate the sorts of rules that we
expect algebraic operations to have.
\end{example}

\subsection*{Exercises}

\begin{exercises}
  \item Let $\ast: A \cross A \to A$ be a commutative binary operation.  If
$e$ is a left identity, show that it also a right identity (and hence simply
an identity).

  \item Let $X$ be any set, and let $\powerset(X)$ be the power set of
$X$ (ie. the set of all subsets of $X$).  Show that $\union : \powerset(X)
\cross \powerset(X) \to \powerset(X)$ is an associative, commutative binary
operation, and that $\emptyset$ is an identity for this operation.

  Similarly, show that $\intersect : \powerset(X) \cross \powerset(X) \to
\powerset(X)$ is an associative, commutative binary operation, and that
$X$ is an identity for this operation.

  \item Let $\ast: A \cross A \to A$ be an associative and commutative binary
    operation.  Show that for any product of $n$ elements $x_{1}$, $x_{2},
    \ldots, x_{n} \in A$, no matter what the order of elements in the product
    \[
      x_{1} \ast x_{2} \ast \cdots \ast x_{n}
    \]
    the result is the same.

\end{exercises}

\section{Groups}

If you look at the discussion of symmetries and permutations, you
will note that not only was there a binary operation, but there was an inverse.
We should have some model for this additional operation.

A \defn{group}{group}, $\mathbf{G} = (G, \ast, e)$, consists of a set $G$, a binary
operation $\ast: G \cross G \to G$, and an element $e \in G$
satisfying the following three conditions:
\begin{theoremenum}
  \item $\ast$ is associative
  \item $e$ is an identity for $\ast$
  \item every element $x \in G$ has an \defn{inverse element}{inverse element} $x^{-1} \in G$
such that $x \ast x^{-1} = x^{-1} \ast x = e$.
\end{theoremenum}
We call $\ast$ the \defn{group operation}{group operation}.

Note that there is no requirement that the group operation is commutative. 
If it does happen to be commutative, then we say that the group is an
\defn{commutative}{group!commutative} or \defn{Abelian group}{group!Abelian}.

This means that in general $x \ast y$ and $y \ast x$ are distinct elements,
but sometimes they are not.  If
\[
  x \ast y = y \ast x
\]
for a particular $x$ and $y \in G$, we say that $x$ and $y$ \defn{commute}{commuting elements}.

A number of different notations are used when working with groups elements. 
Most commonly we will omit the group operation entirely and simply write $xy$
for $x \ast y$, just as is done for multiplication.  In this case we use the
following clear notation for repeated applications of the group operations:
\[
  x^{k} = \underbrace{x x x \cdots x}_{k\text{ times}}
\]
for any natural number $k$.  To make this notation mesh nicely with the
expected behaviour of power laws, we define
\[
  x^{-k} = (x^{-1})^{k} \qquad \text{and} \qquad x^{0} = e.
\]
When then have the standard power laws
\[
  x^{m}x^{k} = x^{m+k} \qquad \text{and} \qquad (x^{m})^{k} = x^{mk},
\]
for any integers $m$ and $k$.  Its also not hard to see that $(x^{k})^{-1} =
x^{-k}$.  However, we have that
\[
  (xy)^{k} \ne x^{k}y^{k}
\]
in general.  In the case that $x$ and $y$ commute, then we do have equality.

In the case of Abelian groups, we will sometimes instead use an additive
notation.  We use $+$ for the group operation, and we customarily write the
identity element as $0$, and the inverse element of $x$ as $-x$. We then use
the notation
\[
  kx = \underbrace{x + x + \cdots + x}_{k\text{ times}}
\]
for any natural number $k$, and
\[
  -kx = k(-x) \qquad \text{and}  \qquad 0x = 0.
\]
We then have the natural rules that
\[
  kx + mx = (k + m)x, \qquad k(mx) = (km)x \qquad \text{and} \qquad kx + ky
= k(x+y)
\]
for any integers $k$ and $m$.

If the set $G$ has a finite number of elements, we say that the
\defn{order}{order!of a group} of the group is the number of elements of $G$. 
If $G$ is an infinite set, we say that the group has infinite order.  We denote
the order of the group by $|G|$.

\begin{example}[Addition and Multiplication]
  Since a principle motivation for the definition of groups are standard
  algebraic operations, it should be no surprise that the following are all
  Abelian groups:
  \begin{itemize}
    \item the additive group of real numbers $(\reals, +, 0)$
    \item the additive group of complex numbers $(\complex, +, 0)$
    \item the additive group of rational numbers $(\rationals, +, 0)$
    \item the additive group of integers $(\integers, +, 0)$
    \item the multiplicative group of real numbers $(\reals \setminus \{0\}, \times, 1)$
    \item the multiplicative group of complex numbers $(\complex \setminus \{0\}, \times, 1)$
    \item the multiplicative group of rational numbers $(\rationals \setminus \{0\}, \times, 1)$
    \item the multiplicative group of integers $(\integers \setminus \{0\}, \times, 1)$
    \item the multiplicative group of natural numbers $(\naturals, \times, 1)$
  \end{itemize}
  Note that for the multiplicative groups, we need to exclude $0$, since $0$
  has no multiplicative inverse.
  
  All of these groups have infinite order.
\end{example}

\begin{example}[Modulo Addition]
  If $m$ is any natural number, the additive group of integers modulo $m$ is
  the group $\integers_{m} = (\{0, 1, 2, \ldots, m-1\}, +, 0)$, where addition
  is performed modulo $m$.  To confirm that it is a group, we need to check
  that the axioms hold.
  
  Associativity follows from the fact that regular addition is associative and commutative.
  Given $x$, $y$ and $z$, we have $x + y = a + km$ for some $a$ and $k$, so
  $(x + y) + z \equiv a + z \pmod{m}$.  But $y = a - x + km$, so $y + z
  \equiv a - x + z \pmod{m}$, and hence $x + (y + z) \equiv x + a - x + z \equiv a
  + z \pmod{m}$.
  
  The fact that $0$ is an identity is trivial: $0 + x = x$, so $0 + x \equiv x \pmod{m}$
  follows immediately.
  
  If $x \in \{1, 2, \ldots, m-1\}$, we know that $-x \equiv m-x \pmod{m}$, and so
  $(m - x) + x \equiv 0 \pmod{m}$ and $x + (m - x) \equiv 0 \pmod{m}$.  Also
  $0$ is its own inverse. So every element has an inverse.
  
  These groups are also clearly Abelian, since regular addition is commutative.
  
  The order of $\integers_{m}$ is $m$.
\end{example}

The previous example shows that there are groups of all orders except $0$.

\begin{example}
  Multiplication modulo $m$ does not, in general, give a group structure.
  Multiplication modulo $m$ is associative, and $1$ is an identity.
  We have to exclude $0$ from the group, because it clearly does not
  have a multiplicative inverse, but even with this restriction, some other
  elements may not have multiplicative inverses.
  
  If you consider multiplication modulo $6$, as in Example~\ref{eg:mod6},
  you can see that there are no inverses for $2$, $3$, and $4$, since none
  of them have a number which you can multiply them by to give $1$.  Indeed,
  there is a somewhat deeper problem in that some products give $0$, which
  cannot be an element of the group.
  
  Multiplication modulo $m$ {\em does} sometimes give you a group, however.
  The multiplication table (omitting $0$) for multiplication modulo $5$ is
  as follows:
  \[
    \begin{array}{c|cccc}
      \times & 1 & 2 & 3 & 4 \\
      \hline
      1 & 1 & 2 & 3 & 4 \\
      2 & 2 & 4 & 1 & 3 \\
      3 & 3 & 1 & 4 & 2 \\
      4 & 4 & 3 & 2 & 1
    \end{array}
  \]
  A quick check shows that every element has an inverse.  Hence
  $(\{1, 2, 3, 4\}, \times, 1)$ is a group, where $\times$ is multiplication
  modulo $5$.
\end{example}

\begin{example}[Symmetries of a Set]
  If $\Omega \subseteq \reals^{n}$, then $(\Sym(\Omega), \circ, I)$ is a
  group.  The proof of this is the essential content of
  Proposition~\ref{prop:symmetryfacts}.
\end{example}

\begin{example}[Symmetric Group]
  The \defn{symmetric group}{group!symmetric} is the group $S_{n} = (S_{n},
  \cdot, e)$ of all permutations, with the multiplication of permutations
  being the group operation, and $e(k) = k$ being the identity
  permutation.  That this is a group is largely the content of
  Proposition~\ref{prop:permgroup}.  The only thing that needs to be checked
  is that the identity permutation is in fact a group identity, and that is
  fairly straightforward: if $p$ is any permutation in $S_{n}$,
  \[
    (pe)(k) = e(p(k)) = p(k) \qquad \text{and} \qquad (ep)(k) = p(e(k)) = p(k),
  \]
  for all $k$, so $ep = pe = p$, and $e$ is therefore the identity for this
  group operation.
\end{example}

\begin{example}[Alternating Group]
  Let $A_{n}$ be the set of all even permutations.
  The \defn{alternating group}{group!alternating} is the group $A_{n} = 
  (A_{n}, \cdot, e)$ of all even permutations, with the multiplication of
  permutations being the group operation, and $e(k) = k$ being the identity
  permutation.  We know that the product of two even permutations is 
  an even permutation, and the product is associative, and from the 
  previous example we know that $e$ is an identity.  What remains to 
  be checked is that if $p$ is an even permutation, so is $p^{-1}$.
  We note that since $\parity(p) = 1$,
  \[
    \parity(p^{-1}) = \parity(p)\parity(p^{-1}) = \parity(pp^{-1}) =
    \parity(e) = 1.
  \]
  Hence $p^{-1}$ is an even permutation.
  
  The group $A_{n}$ has order $n!/2$.
\end{example}

\begin{example}[Matrix Groups]
  Recall that a matrix $A$ is invertible if and only if $\det(A) \ne 0$.
  If we are going to find groups of matrices with matrix multiplication as
  the group operation, then they must be invertible at least.
  
  The following are all groups:
  \begin{itemize}
    \item the \defn{general linear group}{group!general linear} of $n \times n$ 
    matrices $(GL_{n}(\reals), \times, I_{n})$, where
    \[
      GL_{n}(\reals) = \{ A \in M_{n}(\reals) : \det(A) \ne 0\}.
    \]
    
    \item the \defn{orthogonal group}{group!orthogonal} of $n \times n$ 
    matrices $(O_{n}(\reals), \times, I_{n})$, where $O_{n}(\reals)$
    is the set of orthogonal matrices (ie.\ matrices whose columns form an
    orthonormal basis or, equivalently, which satisfy $A^{-1} = A^{t}$).
    
    \item the \defn{special linear group}{group!special linear} of $n \times
    n$ matrices $(SL_{n}(\reals), \times, I_{n})$, where
    \[
      SL_{n}(\reals) = \{ A \in M_{n}(\reals) : \det(A) = 1\}.
    \]
    
    \item the \defn{special orthogonal group}{group!special orthogonal} of
    $n \times n$ matrices $(SO_{n}(\reals), \times, I_{n})$, where $SO_{n}(\reals)$
    is the set of orthogonal matrices with determinant 1.
  \end{itemize}
  
  There isn't anything particularly special about $\reals$-valued matrices
  in the above.  Once can define $GL_{n}(\field)$,  $SL_{n}(\field)$, 
  $O_{n}(\field)$, and $SO_{n}(\field)$ for any field $\field$ (such as
  the complex numbers $\complex$, or the rational numbers $\rationals$).
  
  A \defn{unitary matrix}{matrix!unitary} is a complex-valued matrix which
  satisfies $A^{-1} = A^{*}$, where $A^{*}$ is the conjugate transpose matrix
  of $A$.  More precisely, if $A = [a_{i,j}]_{i,j=1}^{n}$, then
  \[
    A^{*} = [\overline{a_{i,j}}]^{t}.
  \]
  We then have two additional complex matrix groups
  \begin{itemize}
   \item the \defn{unitary group}{group!unitary} of $n \times n$ 
    matrices $(U_{n}(\complex), \times, I_{n})$, where $U_{n}$
    is the set of unitary matrices.
    
    \item the \defn{special unitary group}{group!special unitary} of
    $n \times n$ matrices $(SU_{n}(\complex), \times, I_{n})$, where $SU_{n}(\complex)$
    is the set of unitary matrices with determinant 1.
  \end{itemize}
  
  In all these cases, we know that matrix multiplication is associative, the
  identity matrix is an element of each group, and in each case there is a
  matrix inverse of each matrix.  What we need to check in each case is that
  the product of two elements is an element of the group, and that the inverse
  of an element is an element of the group.
  
  In each case it is fairly easy to verify these two facts.  The key
  identities that we use are as follows:
  \begin{theoremenum}
    \item $|AB| = |A||B|$. So if $|A|$ and $|B| \ne 0$, then $|AB| \ne 0$.
      Hence if $A$ and $B \in GL_{n}(\reals)$, then so is $AB$.
      
      Similarly if $|A|$ and $|B| = 1$, then $|AB| = |A||B| = 1$, so
      if $A$ and $B \in SL_{n}(\reals)$, then so is $AB$.
    
    \item $(A^{-1})^{-1} = A$, so if $A \in GL_{n}(\reals)$, then so is
      $A^{-1}$.
    
    \item $|A^{-1}| = |A|^{-1}$, so if $|A| = 1$, $|A^{-1}| = 1^{-1} = 1$.
      Hence if $A \in SL_{n}(\reals)$, then so is $A^{-1}$.
    
    \item if $A$ and $B \in O_{n}(\reals)$, then $(AB)^{t} = B^{t}A^{t} =
      B^{-1}A^{-1} = (AB)^{-1}$.  Also $(A^{-1})^{t} = (A^{t})^{t} = A =
      (A^{-1})^{-1}$, so $A^{-1} \in O_{n}(\reals)$.
      
      This, combined with (1) and (2) also shows that $SO_{n}(\reals)$ is a
      group.
    
    \item similarly, if $A$ and $B \in U_{n}(\complex)$, then $(AB)^{*} = B^{*}A^{*} =
      B^{-1}A^{-1} = (AB)^{-1}$.  Also $(A^{-1})^{*} = (A^{*})^{*} = A =
      (A^{-1})^{-1}$, so $A^{-1} \in U_{n}(\complex)$.
      
      This, combined with (1) and (2) also shows that $SU_{n}(\complex)$ is a
      group.
  \end{theoremenum}
\end{example}

\begin{example}
  The set of matrices
  \[
    G = \left\{
      \begin{bmatrix}
        1 & 0 \\
        0 & 1
      \end{bmatrix},
      \begin{bmatrix}
        0 & 1 \\
        1 & 0
      \end{bmatrix},
      \begin{bmatrix}
        -1 & 0 \\
        0 & -1
      \end{bmatrix},
      \begin{bmatrix}
        0 & -1 \\
        -1 & 0
      \end{bmatrix}
    \right\}
  \]
  is a group when given the standard matrix operations, and the identity is
  the identity matrix.  The easiest way to verify this is simply to show that
  the group is closed under matrix multiplication and matrix inverse.
  
  Letting
  \[
    I = \begin{bmatrix}
        1 & 0 \\
        0 & 1
      \end{bmatrix}, \qquad
    A = \begin{bmatrix}
        0 & 1 \\
        1 & 0
      \end{bmatrix}, \qquad
    B = \begin{bmatrix}
        -1 & 0 \\
        0 & -1
      \end{bmatrix},
    C = \begin{bmatrix}
        0 & -1 \\
        -1 & 0
      \end{bmatrix},
  \]
  we have that $I^{-1} = I$, $A^{-1} = A$, $B^{-1} = B$ abd $C^{-1} = C$, and
  if we draw up the Cayley table for matrix multiplication of these matrices,
  we get
  \[
    \begin{array}{c|cccc}
        & I & A & B & C \\
      \hline
      I & I & A & B & C \\
      A & A & I & C & B \\
      B & B & C & I & A \\
      C & C & B & A & I \\
    \end{array}
  \]
  
  Note that the Cayley table and inverses of this group correspond to the
  Cayley table and inverses of the symmetries of the \textsf{H}-shaped set
  of Example~\ref{eg:symmetryofH}, with $I \leftrightarrow I$, $A \leftrightarrow H$,
  $B \leftrightarrow V$, and $C \leftrightarrow R$.
\end{example}

\begin{example}[Free Groups]
  Let $a$ and $b$ two symbols, and $a^{-1}$ and $b^{-1}$ be the inverse
  of these two symbols.  A \defn{word}{word} in the \defn{letters}{letters}
  $a$, $b$, $a^{-1}$ and $b^{-1}$
  is simply a list $w = w_{1}w_{2}\cdots w_{n}$, where each $w_{k}$ is one
  of the 4 letters.  A \defn{reduced word}{word!reduced} is a word where
  we have repeatedly cancelled any adjacent occurrences of a letter and its
  inverse.
  
  For example $w = aba^{-1}ab^{-1}b^{-1}ab^{-1}ba$ is a word.  The corresponding
  reduced word can be found by cancelling: $w = aba^{-1}ab^{-1}b^{-1}ab^{-1}ba =
  abb^{-1}b^{-1}aa = ab^{-1}aa$.
  
  The empty word $e$ is the word with no letters.  The product of two words
  $v = v_{1}v_{2}\cdots v_{m}$ and $w = w_{1}w_{2}\cdots w_{n}$ is simply
  the concatenation of the two words:
  \[
    vw = v_{1}v_{2}\cdots v_{m}w_{1}w_{2}\cdots w_{n}
  \]
  
  The \defn{free group}{group!free} on $2$ symbols, $F_{2}$, is the set of
  all reduced words in $a$, $b$, $a^{-1}$ and
  $b^{-1}$, where the group operation is to multiply two words, and then reduce
  the product, and the identity is the empty word.  The inverse of a word
  $w = w_{1}w_{2}\cdots w_{n}$ is the word $w^{-1} = w_{n}^{-1}w_{n-1}^{-1}\cdots
  w_{1}^{-1}$.
  
  In a similar manner, one can construct the free group $F_{n}$ on $n$ symbols.
\end{example}

\subsection*{Exercises}

\begin{exercises}
  \item Let $(G, \ast, e)$ be a group, and let $x$ and $y$ be two elements
    of $G$ which commute.  Prove that for any $k \in \integers$, $(xy)^{k} =
    x^{k}y^{k}$.

  \item Give an example of a group and two elements of that group such that
    \[
      (xy)^{2} \ne x^{2}y^{2}.
    \]
    Provide concrete calculations to demonstrate this fact for your example.
  
  \item Show that in each of the following cases, $(G, \ast, e)$ is a group.
    \begin{theoremenum}
      \item $G = \reals^{2}$, $(x,y) \ast (x',y') = (x + x', y + y')$, $e = (0,0)$.

      \item $G = \{(x,y) : x,y \in \reals, x \ne 0\}$, $(x,y) \ast (x',y') = (xx', x'y + y')$, $e = (1,0)$.

      \item $G = \{x : x \in \reals, x \ne -1\}$, $x \ast y = x + y + xy$, $e = 0$.

      \item $\displaystyle G = SL_{2}(\integers) = \left\{
        \begin{bmatrix}
          a & b \\ c & d
        \end{bmatrix}
        : a, b, c, d \in \integers, ad - bc = 1
        \right\}$, $\ast$ is matrix multiplication, and $e$ is the identity
        matrix.

      \item $\displaystyle G = \left\{
        \begin{bmatrix}
          a & b \\ 0 & a
        \end{bmatrix}
        : a, b \in \reals, a \ne 0
        \right\}$, $\ast$ is matrix multiplication, and $e$ is the identity
        matrix.
      
      \item $G = \powerset(X)$, the power set of some set $X$, $\ast = \symdiff$,
        and $e = \emptyset$.
    \end{theoremenum}
    
    \item Let $m \ge 2$ be a natural number, and
      \[
        G = \{k \in \integers_{m}: k \ne 0, \text{ $k$ and $m$ are
        coprime}\}.
      \]
      Show that $(G, \times, 1)$
      is a group where $\times$ is performed modulo $m$.  Conclude that
      $(\integers_{p} \setminus \{0\}, \times, 1)$ is a group if and only if
      $p$ is prime.
      
      Hint: Use Exercise~\ref{ex:zerodivisor}.
      
    \item Explain why $(\naturals, \times, 1)$ is not a group.
    
    \item (*) Prove that $SL_{n}(\integers)$ is a group.
\end{exercises}

\section{Working With Abstract Groups}

A lot of the content of group theory involves proving general facts about
groups.  The point of this section is to make you familiar with the sorts of
techniques and proof methods involved.  Unfortunately, even the most ``obvious''
and basic facts need careful checking, since we have stripped away most of the
standard rules of algebra.

Consider the \defn{cancellation law}{cancellation law}:

\begin{proposition}[Cancellation Law]\label{prop:cancellation}
  Let $(G, \ast, e)$ be a group, and $x$, $y$, and $z \in G$.  If
  $x \ast z = y \ast z$, then $x = y$.  Similarly, if $z \ast x = 
  z \ast y$, then $x = y$.
\end{proposition}

Normally you would cancel like this in algebra without too much thought: the
case $z = 0$ for multiplication is really the only exceptional case in
standard algebra.  However we need to carefully justify that cancellation in
fact works for groups.

\begin{proof}
  We have
  \begin{alignat*}{4}
    x &= x \ast e &\qquad &\text{(identity axiom)} \\
      &= x \ast (z \ast z^{-1}) &&\text{(inverse axiom)} \\
      &= (x \ast z) \ast z^{-1} &&\text{(associativity)} \\
      &= (y \ast z) \ast z^{-1} &&\text{(hypothesis)} \\
      &= y \ast (z \ast z^{-1}) &&\text{(associativity)} \\
      &= y \ast e &&\text{(inverse axiom)} \\
      &= y. &&\text{(identity axiom)}
  \end{alignat*}
  
  The second part is left as an exercise.
\end{proof}

Notice how each step is justified in terms of the axioms of a group. 
Needless to say, once you get more familiar with the way that group
operations work, you will not need to justify each step, and you may
be able to skip certain trivial steps.  In the short term, however,
you should be careful that you justify each step in any calculation.

Here is another example: it should be fairly obvious that there can only be
one inverse of any particular element.  Nevertheless, we need to prove this
result.

\begin{proposition}[The Inverse is Unique]\label{prop:uniqueinverse}
  Let $(G, \ast, e)$ be a group, and $x$, $y \in G$.  Then if $x \ast y = e$,
  $y = x^{-1}$.  Similarly, if $y \ast x = e$, $y = x^{-1}$.
\end{proposition}
\begin{proof}
  We have
  \begin{alignat*}{4}
    y &= e \ast y &\qquad &\text{(identity axiom)} \\
      &= (x^{-1} \ast x) \ast y &&\text{(inverse axiom)}\\
      &= x^{-1} \ast (x \ast y) &&\text{(associativity)}\\
      &= x^{-1} \ast e &&\text{(hypothesis)}\\
      &= x^{-1}. &&\text{(identity axiom)}\\
  \end{alignat*}
  
  The second part is left as an exercise.
\end{proof}

Once we have basic facts like this, we can use them to simplify the proofs
of other facts.

\begin{proposition}
  Let $(G, \ast, e)$ be a group, and $x$, $y \in G$.  Then $(x \ast y)^{-1} =
  y^{-1} \ast x^{-1}$.
\end{proposition}
\begin{proof}
  By Proposition~\ref{prop:uniqueinverse}, we need only show that
  $(x \ast y) \ast (y^{-1} \ast x^{-1}) = e$.
  \begin{alignat*}{4}
    (x \ast y) \ast (y^{-1} \ast x^{-1})
      &= (x \ast (y \ast y^{-1})) \ast x^{-1} &\qquad &\text{(associativity)} \\
      &= (x \ast e) \ast x^{-1} &\qquad &\text{(inverse axiom)} \\
      &= x \ast x^{-1} &\qquad &\text{(identity axiom)} \\
      &= e. &\qquad &\text{(inverse axiom)}
  \end{alignat*}
  Hence $(x \ast y)^{-1} = y^{-1} \ast x^{-1}$.
\end{proof}

Notice in this example that the order of the product is reversed in the
inverse.  This is necessary if the elements do not commute.

Here is another basic fact that needs to be verified.

\begin{proposition}[Double Inverse]
  Let $(G, \ast, e)$ be a group, and $x \in G$.  Then $(x^{-1})^{-1} = x$.
\end{proposition}
\begin{proof}
  Exercise.
\end{proof}

As mentioned in the previous section, if we use power-style notation, most of
the usual power laws hold.

\begin{proposition}[Power Laws for Groups]
  Let $(G, \ast, e)$ be a group, and $x \in G$.  Then
  \begin{theoremenum}
    \item $x^{m}x^{n} = x^{m+n}$,
    \item $(x^{m})^{n} = x^{mn}$,
    \item if $y \in G$ and $x \ast y = y \ast x$, then $(x \ast y)^{n} = 
    x^{n} \ast y^{n}$.
  \end{theoremenum}
    
  Let $(G, +, 0)$ is an Abelian group, and $x$, $y \in G$.  Then
  \begin{theoremenum}
    \item $mx + nx = (m+n)x$,
    \item $m(nx) = (mn)x$,
    \item $n(x+y) = nx + ny$.
  \end{theoremenum}
\end{proposition}
\begin{proof}
  Exercise.
\end{proof}

\subsection*{Exercises}

\begin{exercises}
  \item Prove the parts of the proofs from this section that were left as
    exercises.

  \item Some texts define groups slightly differently (and slightly more
    efficiently) as follows:
    
    A \defn{group}{group}, $\mathbf{G} = (G, \ast, e)$, consists of a set $G$,
    a binary operation $\ast: G \cross G \to G$, and an element $e \in G$
    satisfying the following three conditions:
    \begin{theoremenum}
      \item $\ast$ is associative
      \item $e$ is a (left) identity for $\ast$, ie.\ $e \ast x = x$,
      \item every element $x \in G$ has an \defn{inverse element}{inverse element} $x^{-1} \in G$
        such that $x^{-1} \ast x = e$.
   \end{theoremenum}
   
   Show that if you use these axioms, you can prove that $e$ is also a right
   inverse, and $x \ast x^{-1} = e$, giving you the axioms of our definition
   of a group.
   
   This means that the two definitions are equivalent, so you can use either
   one.
   
  \item Let $(G, \ast, e)$ be a group, and $x$, $y \in G$.  Show that if
  $y^{-1}xy = x^{k}$, then $y^{-n}x^{m}y^{n} = x^{mk^{n}}$.
  
  \item Let $(G, \ast, e)$ be a group.  Show that $e^{-1} = e$.

\end{exercises}

\section{Cayley Tables}

As we have seen, there are quite a number of groups around.  We would like
to develop some way that we can present groups abstractly, without worrying
about any potential context.

For finite groups, one way of doing this is by giving a Cayley table for the
group.

\begin{definition}
  Let $(G, \ast, e)$ be a finite group of order $n$, with some particular
  ordering $x_{1}$, $x_{2}, \ldots, x_{n}$ chosen for the elements of $G$.
  Then a Cayley table of the group is an array where $x_{i} \ast x_{j}$ is
  in the $i$th row and $j$th column.
\end{definition}

We do not need to specify the inverse as a separate table, since we can find
the inverse of $x_{i}$ by looking for the $j$ such that the $j$th entry of the
$i$th row is $e$, so that $x_{i} \ast x_{j} = e$, and hence $x_{j} = x_{i}^{-1}$
by Proposition~\ref{prop:uniqueinverse}.

We have already seen a number of Cayley tables for binary operations which
turned out to be groups.  Indeed, in a number of situations, the Cayley tables
turned out to be essentially the same.

\begin{example}
  Consider the Cayley tables of $(\integers_{2}, +, 0)$ and $(\integers_{3}
  \setminus \{0\}, \times, 1)$.
  \[
    \begin{array}{c|cc}
      + & 0 & 1 \\
      \hline
      0 & 0 & 1 \\
      1 & 1 & 0
    \end{array}
    \qquad
    \begin{array}{c|cc}
      \times & 1 & 2 \\
      \hline
      1 & 1 & 2 \\
      2 & 2 & 1
    \end{array}
  \]
  
  These are the same table is you replace $+$ by $\times$, $0$ by $1$ and $1$
  by $2$.
\end{example}

In situations like this, we can agree that the two groups in question are
essentially the same.

\begin{definition}
  Two finite groups $(G, \ast, e)$ and $(H, \circ, i)$ are
  \defn{isomorphic}{isomorphic} if a Cayley table of $H$ can be obtained
  from a Cayley table of $G$ be replacing all occurrences of each symbol
  in $G$ by a corresponding symbol of $H$, and each $\ast$ by $\circ$.
  
  We write $G \isom H$ when $G$ and $H$ are isomorphic.
\end{definition}

The correspondence between any two isomorphic groups always has the two
identities corresponding, and always has the inverse of a symbol in $G$
corresponding to the inverse of the corresponding symbol in $H$.

Isomorphism is clearly a transitive relation between groups.  If $G$ and $H$
are isomorphic, and $H$ and $F$ are isomorphic, than $G$ and $F$ must also be
isomorphic.

This is not the final version of the definition of ``isomorphic'', but it will
do for now.

\begin{example}
  The groups $S_{3}$ and $\Sym(\Omega)$, where $\Omega$ is an equilateral
  triangle, are isomorphic.
\end{example}

\begin{example}\label{eg:groupsoforder4}
  Consider the Cayley tables of $\integers_{4}$ and the group of symmetries
  of the \textsf{H}-shaped set of Example~\ref{eg:symmetryofH}:
  \[
    \begin{array}{c|cccc}
      + & 0 & 1 & 2 & 3\\
      \hline
      0 & 0 & 1 & 2 & 3 \\
      1 & 1 & 2 & 3 & 0 \\
      2 & 2 & 3 & 0 & 1 \\
      3 & 3 & 0 & 1 & 2 \\
    \end{array}
    \qquad
    \begin{array}{c|cccc}
    \circ & I & H & V & R \\
    \hline
      I & I & H & V & R \\
      H & H & I & R & V \\
      V & V & R & I & H \\
      R & R & V & H & I \\
    \end{array}
  \]
  
  These two groups are not isomorphic, since $I$ must correspond to $0$, and
  one of $H$, $V$ or $R$ must correspond to $1$, and $1 + 1 = 2$, but we have
  $H \circ H = I$, $V \circ V = I$,  $R \circ R = I$, so their products cannot
  correspond to the sum, and so there is no element that can correspond to $1$.
\end{example}

It is immediate that for two groups to be isomorphic, they must have the same
order, since otherwise the Cayley tables are different sizes, and so the cannot
correspond.

Similarly if two groups are isomorphic, either both are Abelian, or 
both fail to be Abelian, since there is no way the Cayley tables can 
correspond if one is Abelian and the other not.

\begin{lemma}
  If $G$ and $H$ are finite groups, and $G \isom H$, then:
  \begin{theoremenum}
      \item $|G| = |H|$,
      
      \item $G$ is Abelian if and only if $H$ is Abelian.
  \end{theoremenum}
\end{lemma}

This leads to the following:

\begin{question}
  How many different (ie.\ non-isomorphic) classes of groups are there for
  any given order?
\end{question}

This question is one which we will spend a fair amount of time considering.

Cayley tables can be used to answer this question, at least for groups of
small order, but we need a few facts about Cayley tables first.

\begin{proposition}
  If $(G, \ast, e)$ is a finite group, then every element of $G$ occurs exactly once
  in each row and in each column of a Cayley table for $G$.
\end{proposition}
\begin{proof}
  Let $G = \{x_{1}, x_{2}, \ldots, x_{n}\}$.  Assume that $x$ occurs twice in
  the $i$th row, so that $x = x_{i} \ast x_{j}$ and $x = x_{i} \ast x_{k}$
  for some $j \ne k$.  But then we have $x_{i} \ast x_{j} = x_{i} \ast x_{k}$,
  and the cancellation law tells us that $x_{j} = x_{k}$, so $j = k$, which is
  a contradiction.  Hence $x$ can occur at most once.
  
  If $x$ does not occur at all in the $i$th row, then there must be some other
  element which occurs 2 or more times by the pidgeonhole principle, which
  is impossible.  Hence $x$ must occur exactly once.
  
  A similar argument proves the result for columns.
\end{proof}

Another way of saying this is that a Cayley table is a \defn{Latin square}{Latin
square}: a Latin square is an array of symbols in which every symbol occurs
exactly once in each row and in each column.  Latin squares are significant
in experimental design and statistics.  However, not every Latin square is a 
Cayley table for a group.

\begin{example}
  The following binary operation does not give a group:
  \[
    \begin{array}{c|ccccc}
      \ast & 1 & a & b & c & d \\
      \hline
      1 & 1 & a & b & c & d \\
      a & a & 1 & d & b & c \\
      b & b & c & 1 & d & a \\
      c & c & d & a & 1 & b \\
      d & d & b & c & a & 1 \\
    \end{array}
  \]
  The problem is that the operation it determines is not associative:
  $(a \ast b) \ast c = d \ast c = a$, while $a \ast (b \ast c) = a \ast d = c$.
  However, this table clearly has the Latin square property.
\end{example}

\begin{theorem}
  Let $n = 1$, $2$, or $3$.  Then every group of order $n$ is isomorphic to
  $(\integers_{n}, +, 0)$.
\end{theorem}
\begin{proof}
  Case $n=1$: the group has one element which must be the identity,
  so $G = \{e\}$.  The only possible Cayley table is trivial
  \[
    \begin{array}{c|c}
      \ast & e \\
       \hline
     e & e
    \end{array}
  \]
  and this clearly is isomorphic to the Cayley table of $\integers_{1}$
  \[
    \begin{array}{c|c}
      + & 0 \\
      \hline
      0 & 0
    \end{array}
  \]

  Case $n=2$: the group has two elements, one of which must be the identity,
  so $G = \{e, a\}$.  Entering in the elements which are products of the
  identity element we get
  \[
    \begin{array}{c|cc}
      \ast & e & a \\
      \hline
      e & e & a \\
      a & a &  
    \end{array}
  \]
  and clearly the only way to complete this table while keeping the Latin
  square property is to put an $e$ in the bottom right entry:
  \[
    \begin{array}{c|cc}
      \ast & e & a \\
      \hline
      e & e & a \\
      a & a & e 
    \end{array}
  \]
  Again, this is clearly isomorphic to $\integers_{2}$ when you look at the
  Cayley table
  \[
    \begin{array}{c|cc}
      + & 0 & 1 \\
      \hline
      0 & 0 & 1 \\
      1 & 1 & 0 
    \end{array}
  \]

  Case $n=3$: the group has three elements, one of which must be the identity,
  so $G = \{e, a, b\}$.  Entering in the elements which are products of the
  identity element we get
  \[
    \begin{array}{c|ccc}
      \ast & e & a & b \\
      \hline
      e & e & a & b \\
      a & a &  \\
      b & b &  
    \end{array}
  \]
  To preserve the Latin square property, the second entry of the second
  column must be $b$, otherwise the second entry of the third column would be
  $b$, which would break the Latin square property for the third column.
  This then implies that the last entries of the second row and second
  column must be $e$, and the final entry of the array must be $a$.
  \[
    \begin{array}{c|ccc}
      \ast & e & a & b \\
      \hline
      e & e & a & b \\
      a & a & b & e \\
      b & b & e & a
    \end{array}
  \]
  Again, the correspondence with the Cayley table for $\integers_{3}$ is clear:
  \[
    \begin{array}{c|ccc}
      + & 0 & 1 & 2\\
      \hline
      0 & 0 & 1 & 2\\
      1 & 1 & 2 & 0 \\
      2 & 2 & 0 & 1 
    \end{array}
  \]
\end{proof}

We know that there are at least two non-isomorphic groups of order 4.  It will
turn out that these are the only two possibilities.  We could prove this
by finding all possible Cayley tables of groups of order 4, but as we will
see, there are slicker ways to do this.

\subsection*{Exercises}

\begin{exercises}
  \item Find another example of a Latin square which is not the Cayley table
    of a group.
  
  \item Show that any group of order 4 is isomorphic to one of the two groups
    in Example~\ref{eg:groupsoforder4}.
\end{exercises}

\section{Generators}

While Cayley tables have their uses, there are clear limitations to their use
once the groups get large, and for infinite groups they at best give a tiny
snapshot of the group.  Another way of presenting groups is required, which
can deal with these larger groups.

\begin{definition}
  Let $(G, \ast, e)$ be a group.  We say that a subset $X = \{x_{1}, x_{2}, \ldots,
  x_{n}\}$ of $G$ \defn{generates}{set!generating} $G$ if every element of $G$ can be
  written as a product of powers of elements of set $X$ (possibly with
  repetition).  We say that the elements of $X$ are generators of $G$.
  
  More generally, given a subset $X = \{x_{1}, x_{2}, \ldots, x_{n}\}$ of $G$, the
  set of elements that can be written as a product of powers of elements of $X$
  (possibly with repetition) is the set \defn{generated}{generated} by $X$,
  and we denote it by $\langle x_{1}, x_{2}, \ldots, x_{n} \rangle$ or
  $\langle X \rangle$.
\end{definition}

\begin{example}
  The group $S_{3}$ is generated by the permutations $a = (1,2,3)$ and
  $b = (1,2)$.  One can easily verify that the identity permutation is $a^{0}$,
  $a^{2} = (1,3,2)$, $ab = (2,3)$ and $a^{2}b = (1,3)$.  We could write
  $S_{3} = \langle (1,2,3), (1,2) \rangle$.
  
  The group $S_{3}$ is also generated by $x = (1,2)$ and $y = (2,3)$.  This
  requires a little bit more checking, but $x^{0} = e$, $yx = (1,2,3)$,
  $xy = (1,3,2)$, and $xyx = (1,3)$, so we also have $S_{3} = \langle (1,2),
  (2,3) \rangle$.
  
  On the other hand, the permutation $a = (1,2,3)$ does not generate the
  whole group. The only elements we can get using just powers of $a$ are
  $e$, $a$, and $a^{2}$, since $a^{3} = e$.
  Hence $\langle (1,2,3) \rangle = \{ e, (1,2,3), (1,3,2) \}$.
\end{example}

\begin{example}
  The group $(\integers_{4}, +, 0)$ is generated by $1$, since $1+1 = 2$,
  $1 + 1 + 1 = 3$ and $1 + 1 + 1 + 1 = 0$.  It is also generated by $-1$.
  
  However the set generated by $2$ is simply $\{0, 2\}$.
\end{example}

We note that the identity element is always in the set generated by any
collection of elements.

Generators help us understand the structure of a group by allowing us to
represent general elements in terms of fewer symbols.

\begin{example}
  If $x = (1,2)$ and $y = (2,3)$, then $S_{3} = \{e, x, y, xy, yx, xyx\}$.
  In addition, we can see that $x^{2} = e$, $y^{2} = e$ and $yxy = xyx$.
  For example, you could calculate the product of $xyx$ and $yx$ using these
  facts as follows:
  \begin{alignat*}{4}
    (xyx)(yx) &= x(yxy)x &\qquad& \text{(associativity)}\\
              &= x(xyx)x && \text{($yxy = xyx$)}\\
              &= (xx)y(xx) && \text{(associativity)}\\
              &= eye && \text{($x^{2} = e$)}\\
              &= y && \text{(identity axiom)}
  \end{alignat*}
  We can use this information to write the Cayley table of $S_{3}$ in terms
  of $x$ and $y$ as follows:
  \[
    \begin{array}{c|cccccc}
      \cdot & e &   x &   y &  xy &  yx & xyx \\
      \hline
        e &   e &   x &   y &  xy &  yx & xyx \\
        x &   x &   e &  xy &   y & xyx &  yx \\
        y &   y &  yx &   e & xyx &   x &  xy \\
       xy &  xy & xyx &   x &  yx &   e &   y \\
       yx &  yx &   y & xyx &   e &  xy &   x \\
      xyx & xyx &  xy &  yx &   x &   y &   e
    \end{array}
  \]
\end{example}

Notice how in the above example the identities $x^{2} = e$, $y^{2} = e$ and
$yxy = xyx$ help us calculate.  Such identities are called
\defn{relations}{relations}.  In fact, given the generators $x$ and $y$
and these three relations, we can recover the Cayley table for $S_{3}$.
This leads us to another way to present a group, which we will make more
formal in a later section.  In the mean-time we can use it informally as
follows:

\begin{example}[Cyclic Groups]
  The group $C_{n}$ consists of the set $C_{n} = \{1, a, a^{2}, \ldots, a^{n-1}\}$
  and the group operation is determined by the relation $a^{n} = 1$.
  
  The group $(C_{4}, \cdot, 1)$, then has the Cayley table
  \[
    \begin{array}{c|cccc}
      \cdot &     1 &     a & a^{2} & a^{3} \\
      \hline
          1 &     1 &     a & a^{2} & a^{3} \\
          a &     a & a^{2} & a^{3} &     1 \\
      a^{2} & a^{2} & a^{3} &     1 &     a \\
      a^{3} & a^{3} &     1 &     a & a^{2} 
    \end{array}
  \]
  Clearly $C_{4}$ and $\integers_{4}$ are isomorphic.
\end{example}

\begin{example}[Dihedral Groups]
  The group $D_{2n}$ consists of the set
  \[
    D_{2n} = \{1, a, a^{2}, \ldots, a^{n-1}, b, ab, a^{2}b, \ldots, a^{n-1}b \}
  \]
  and the group operation is determined by the relations $a^{n} = 1$,
  $b^{2} = 1$ and $ba = a^{n-1}b$.  Note that $a^{n-1}a = a^{n} = 1$, so
  $a^{n-1} = a^{-1}$, and we could write the third relation as $ba = a^{-1}b$.
  
  For example, we can use these relations to show that
  \[
    ba^{k} = a^{-1}ba^{k-1} = a^{-1}a^{-1}ba^{k-2} = \cdots = a^{-k}b.
  \]
  
  The group $(D_{8}, \cdot, 1)$, then has the Cayley table
  \[
    \begin{array}{c|cccccccc}
      \cdot &      1 &      a &  a^{2} &  a^{3} &      b &     ab & a^{2}b & a^{3}b \\
      \hline
          1 &      1 &      a &  a^{2} &  a^{3} &      b &     ab & a^{2}b & a^{3}b \\
          a &      a &  a^{2} &  a^{3} &      1 &     ab & a^{2}b & a^{3}b &      b \\
      a^{2} &  a^{2} &  a^{3} &      1 &      a & a^{2}b & a^{3}b &      b &     ab \\
      a^{3} &  a^{3} &      1 &      a &  a^{2} & a^{3}b &      b &     ab & a^{2}b \\
          b &      b & a^{3}b & a^{2}b &     ab &      1 &  a^{3} &  a^{2} &      a \\
         ab &     ab &      b & a^{3}b & a^{2}b &      a &      1 &  a^{3} &  a^{2} \\
     a^{2}b & a^{2}b &     ab &      b & a^{3}b &  a^{2} &      a &      1 &  a^{3} \\
     a^{3}b & a^{3}b & a^{2}b &     ab &      b &  a^{3} &  a^{2} &      a &      1
    \end{array}
  \]
  If you were to write out the symmetry group of a square you would find that
  it is isomorphic to $D_{8}$.
  
  In fact we will prove later that the symmetry group of a regular $n$-gon
  is isomorphic to $D_{2n}$.
\end{example}

\begin{definition}
  Let $(G, \ast, e)$ be a group, and left $x \in G$.  The
  \defn{order}{order!of an element} $o(x)$ is the cardinality of the set it
  generates, $o(x) = |\langle x \rangle|$.
  
  If $G = \langle x \rangle$ for any $x \in G$, we say that $G$ is a
  \defn{cyclic group}{group!cyclic}.
\end{definition}

\begin{example}
  In $S_{3}$, $e$ has order $1$, $(1,2)$, $(2,3)$ and $(1,3)$ have order $2$,
  and the elements $(1,2,3)$ and $(1,3,2)$ have order $3$.
  
  In $\integers_{4}$, $0$ has order $1$, $2$ has order $2$, and $1$ and $-1$
  have order $4$.  Since $\integers_{4} = \langle 1 \rangle$, it is a cyclic
  group.
\end{example}

\begin{example}
  For every $n \in \naturals$, $(\integers_{n}, +, 0)$ is a cyclic group,
  since $\integers_{n} = \langle 1 \rangle$.
\end{example}

The orders of elements of a group can be used as a comparatively 
simple check to see if two groups may not be isomorphic.  It is clear 
that if $G$ and $H$ are isomorphic groups, and $x \in G$ corresponds to 
$y \in H$, then $o(x) = o(y)$.  Turning this idea around, we get the 
following theorem

\begin{theorem}
    If $G$ and $H$ are two groups, then if there is some $n$ such that
    the number of elements of order $n$ in $G$ is different from the
    number of elements of order $n$ in $H$, ie.
    \[
      |\{x \in G : o(x) = n\}| \ne |\{y \in H : o(y) = n\}|
    \]
    then $G$ and $H$ are not isomorphic.
\end{theorem}
\begin{proof}
    Without loss of generality, we may assume that $|\{x \in G : o(x) = 
    n\}| > |\{y \in H : o(y) = n\}|$.  If $G$ and $H$ are isomorphic, 
    then there must be some $x$ which corresponds to an element of $H$ 
    which does not have order $y$, because there are not enough 
    elements of order $n$ in $H$.  But this cannot happen if the 
    groups are isomorphic, giving a contradiction.
\end{proof}

We will eventually see that the converse of this theorem is not true, 
so this does not give a good test for isomorphism of groups.

The following simplified version of the theorem is often enough to 
prove that two groups are not isomorphic.

\begin{corollary}
    If $G$ and $H$ are two groups, and there is some $x \in G$ such 
    that $o(x) > o(y)$ for every $y \in H$, then $G$ and $H$ are not 
    isomorphic.
\end{corollary}

This allows us to see that a couple of groups are not isomorphic very 
quickly:

\begin{example}
    If we look at $C_{4} = \{1, a, a^{2}, a^{3}\}$, we see that 
    $o(a) = 4$.  On the other hand, $D_{4} = \{1, a, b, ab\}$ has 
    $o(1) = 1$, and $o(a) = o(b) = o(ab) = 2$.  So by the corollary, 
    $C_{4} \not\isom D_{4}$.
\end{example}

\begin{example}
    If we look at $C_{6} = \{1, a, a^{2}, a^{3}, a^{4}, a^{5}\}$, we see that 
    $o(a) = 6$.  On the other hand, $D_{6} = \{1, a, b, ab, a^{2}, 
    a^{2}b\}$ has $o(1) = 1$, $o(b) = o(ab) = o(a^{2}b) = 2$, and 
    $o(a) = o(a)^{2} = 3$.
    So by the corollary, $C_{6} \not\isom D_{6}$.
    
    One could also show this by simply observing that $C_{6}$ is 
    Abelian, but $D_{6}$ is not.
\end{example}

Related to the above discussion is the following theorem.

\begin{theorem}\label{thm:cyclicgroups}
    If $G$ is a group of order $n$ and there is some element $x \in G$ 
    of order $n$, then $G$ is a cyclic group.
\end{theorem}
\begin{proof}
    We have that $\langle x \rangle \subseteq G$, and $|\langle x 
    \rangle| = o(x) = n = |G|$.  Hence, since $G$ is finite, $\langle x 
    \rangle = G$
\end{proof}

The following theorem is fairly obvious, but needs to be stated and 
proved.

\begin{theorem}
  If $G$ and $H$ are cyclic groups of the same order, then they are
  isomorphic.
\end{theorem}
\begin{proof}
  We have that $G = \langle a \rangle = \{1 = a^{n}, a, a^{2}, \ldots a^{n-1}\}$ and
  $H = \langle b \rangle = \{1 = b^{n}, b, b^{2}, \ldots, b^{n-1}\}$, and a typical
  entry in their Cayley tables looks like
  \[
    \begin{array}{c|ccc}
            & \quad & a^{k} & \quad \\
      \hline
      \\
      a^{l} & & a^{k+l} \\
      & \\
    \end{array}
    \qquad
    \begin{array}{c|ccc}
            & \quad & b^{k} & \quad \\
      \hline
      \\
      b^{l} & & b^{k+l} \\
      &
    \end{array}
  \]
  remembering that $a^{n} = 1$ and $b^{n} = 1$.  In any case, it is clear that
  we can get from one Cayley table to the other by simply replacing powers of $a$
  with corresponding powers of $b$.
\end{proof}

\begin{corollary}
  Every group of order $1$, $2$, or $3$ is a cyclic group.
\end{corollary}

We give one last example, mainly to introduce a name.

\begin{example}
  The group $V$ consisting of the elements $\{1, a, b, ab\}$ with the
  relations $a^{2} = 1$, $b^{2} = 1$ and $ba = ab$ is called the
  \defn{four-group}{four-group} or \defn{vierergruppe}{vierergruppe} (which
  is German for ``four-group'').
  
  The Cayley table of this group is
  \[
    \begin{array}{c|cccc}
      \cdot &  1 &  a &  b & ab \\
      \hline
         1 &  1 &  a &  b & ab \\
         a &  a &  1 & ab &  b \\
         b &  b & ab &  1 &  a \\
        ab & ab &  b &  a &  1
    \end{array}
  \]
  Given the Cayley table, you can see that this is isomorphic to the group of
  symmetries of the letter \textsl{H}, via the correspondences:
  $1 \leftrightarrow I$, $a \leftrightarrow H$, $b \leftrightarrow V$, and
  $ab \leftrightarrow R$.
  
  Also, since $a^{2} = 1$ implies $a = a^{-1}$, we could have used the relation
  $ba = a^{-1}b$ instead of $ba = ab$.  Hence $V$ is the same group as $D_{4}$.
\end{example}

\subsection*{Exercises}

\begin{exercises}
  \item Show that $(1,2)$ and $(1,3)$ generate $S_{3}$.
  
  \item Let $(G, \ast, e)$ be a group, and let $x$, $y \in G$.  Show that
    \begin{theoremenum}
      \item $o(x^{-1}) = o(x)$,

      \item $o(xy) = o(yx)$,

      \item if $o(x) = 1$, then $x = e$,
      
      \item if $o(x) = n$, then $x^{m} = e$ if and only if $n$ divides $m$,
      
      \item if $o(x) = n$, then $n$ is the smallest natural number such that $x^{n} = e$.
    \end{theoremenum}
  
  \item Show that $a^{k}$ generates the cyclic group $C_{n} = \{1, a^{1},
    a^{2}, \ldots, a^{n-1}\}$ if and only if $k$ and $n$ are coprime.
    
    Show that the order of $a^{k}$ in $C_{n}$ is $k/\gcd(k,n)$.
   
  \item Let $(G, \ast, e)$ be a group, and $x$, $y \in G$.  Show that if $x$,
    $y$ and $xy$ all have order 2, then $x$ and $y$ commute.
  
  \item Show that $D_{6}$ and $S_{3}$ are isomorphic groups.
  
  \item Show that if $\Omega$ is a regular $n$-sided polygon in 
    $\reals^{2}$, then $\Sym(\Omega) \isom D_{6}$.
  
  \item Let $G = \{1, a, b, ab, a^{2}, a^{2}b\}$ with the relations $a^{3} = 1$,
    $b^{2} = 1$, and $ab = ba$.  Write out the Cayley table of this group, and
    show that $G$ is isomorphic to $C_{6}$.
  
  \item Show that $(\integers, + 0)$ is generated by the elements $2$ 
    and $3$.
    
    (*) Show that it is generated by any pair of coprime numbers.  
    (Hint: show that you can get $1$ as a sum of multiples of the 
    numbers.)
\end{exercises}

\section{Excursion: Introduction to Categories}

You may have noticed some similarities between the theory of groups as it has
been presented to this point, and the theory of abstract vector spaces.  In
both cases the objects were defined in terms of axioms which must hold for
various binary operations.  In both cases we have a concept called isomorphism
which tells us when two groups or two vector spaces are essentially the same.

Indeed, if you think about it, the concept of generating sets and
spanning sets are somewhat analogous: a set generates $G$ if you can get every
element of $G$ by applying the group operations to the elements of the generating
set; while a set spans a vector space $V$ if you can get every element of $V$
by applying linear operations (vector addition and scalar multiplication) to
the elements of the spanning set.

Clearly we should be careful not to take such analogies too far, since groups
and vector spaces \emph{are} different, but the analogies are useful for putting
the next few sections into context.

In the theory of vector spaces, there are three concepts we have not yet seen
in the context of groups:
\begin{itemize}
  \item direct sums of vector spaces: if $V$ and $W$ are vector spaces, we
    have the direct sum $V \oplus W$ which is the set $V \cross W$ together
    with appropriate vector space operations.
  
  \item subspaces: a subspace of a vector space is a subset which is also a
    vector space.
  
  \item linear transformations: a linear transformation is a function between
    vector spaces which preserves the vector space operations.
\end{itemize}

As you study more algebra you will notice that there are many similarities
like these between the theories of various types of algebraic objects.  Indeed,
we even get such similarities in other areas of pure mathematics, such as
analysis, topology and geometry.  In many cases the proofs of basic facts
in these theories are almost identical, but with appropriate change of
terminology.

Whenever you have similarities and patterns like this in mathematics, there
must be something going on. Category theory is the theory which deals with
and formalizes these similarities.  There are a handful of useful results
which have come out of category theory, but its primary significance is that
it provides a framework for much of modern mathematical theory.

We'll consider categories in more depth later.

\section{Direct Products}

You may recall the definition of a direct sum from linear algebra, if you
have two vector spaces $V$ and $W$ over the same scalar field, the direct
product is the set $V \cross W$ with vector addition
\[
  (v_{1}, w_{1}) + (v_{2}, w_{2}) = (v_{1} + v_{2}, w_{1} + w_{2}),
\]
scalar multiplication
\[
  \lambda (v, w) = (\lambda v, \lambda w)
\]
and zero vector $(0,0)$.  Note that the vector space operations are defined
simply by applying the appropriate vector space operation to each component.

\begin{theorem}
  Let $(G, \ast, e)$ and $(H, \circ, 1)$ be two groups.  If we define a
  binary operation $\bullet$ on $G \cross H$ by
  \[
    (g_{1}, h_{1}) \bullet (g_{2}, h_{2}) = (g_{1} \ast g_{2}, h_{1} \circ h_{2}),
  \]
  then $G \cross H = (G \cross H, \bullet, (e,1))$ is a group, and the inverse
  of $(g,h)$ is $(g^{-1}, h^{-1})$.
\end{theorem}
\begin{proof}
  The binary operation $\bullet$ is obviously well defined, so we only
  need to check that the three axioms hold.
  
  Associativity follows directly from the associativity of $G$ and $H$:
  \begin{align*}
    ((g_{1}, h_{1}) \bullet (g_{2}, h_{2})) \bullet (g_{3}, h_{3})
    &= (g_{1} \ast g_{2}, h_{1} \circ h_{2}) \bullet (g_{3}, h_{3}) \\
    &= ((g_{1} \ast g_{2}) \ast g_{3}, (h_{1} \circ h_{2}) \circ h_{3}) \\
    &= (g_{1} \ast (g_{2} \ast g_{3}), h_{1} \circ (h_{2} \circ h_{3})) \\
    &= (g_{1}, h_{1}) \bullet (g_{2} \ast g_{3}, h_{2} \circ h_{3}) \\
    &= (g_{1}, h_{1}) \bullet ((g_{2}, h_{2}) \bullet (g_{3}, h_{3}))
  \end{align*}
  
  It's straightforward to see that $(e,1)$ is an identity:
  \[
    (g, h) \bullet (e, 1) = (g \ast e, h \circ 1) = (g, h)
  \]
  and
  \[
    (e, 1) \bullet (g, h) = (e \ast g, e \circ h) = (g, h).
  \]
  
  Finally, we observe that
  \[
    (g^{-1}, h^{-1}) \bullet (g, h) = (g^{-1} \ast g, h^{-1} \circ h) = (e,1),
  \]
  so by Proposition~\ref{prop:uniqueinverse}, $(g,h)^{-1} = (g^{-1}, h^{-1})$,
  and so every element of $G \cross H$ has an inverse.
  
  So $G \cross H$ is a group.
\end{proof}

If $G$ and $H$ are finite groups, the multiplication principle tells us
that $G \cross H$ is a finite group, and the order of $G \cross H$ is
$|G||H|$.

Consider the following examples:

\begin{example}
  The group $C_{2} \cross C_{2}$ has $4$ elements: $(1,1)$, $(a,1)$, $(1,a)$
  and $(a,a)$.  We can draw up the Cayley table:
  \[
    \begin{array}{c|cccc}
      \bullet & (1,1) & (a,1) & (1,a) & (a,a) \\
      \hline
        (1,1) & (1,1) & (a,1) & (1,a) & (a,a) \\
        (a,1) & (a,1) & (1,1) & (a,a) & (1,a) \\
        (1,a) & (1,a) & (a,a) & (1,1) & (a,1) \\
        (a,a) & (a,a) & (1,a) & (a,1) & (1,1)
    \end{array}
  \]
  Hopefully you can immediately see that this group is isomorphic to the
  {\it vierergruppe} $V$ via the correspondence
  $(1,1) \leftrightarrow 1$, $(a,1) \leftrightarrow a$, $(1,a) \leftrightarrow b$, 
  and $(a,a) \leftrightarrow ab$.  Hence it is also isomorphic to the
  group of symmetries of the letter \textsl{H}.
\end{example}

\begin{example}
  The group $C_{2} \cross C_{3}$ has $6$ elements: $(1,1)$, $(a,1)$, $(1,b)$
  $(a,b)$, $(1, b^{2})$ and $(a,b^{2})$.  We note that
  \begin{align*}
    (a,b)^{2} &= (1,b^{2})\\
    (a,b)^{3} &= (a,1)\\
    (a,b)^{4} &= (1,b)\\
    (a,b)^{5} &= (a,b^{2})\\
    (a,b)^{6} &= (1,1)
  \end{align*}
  So $C_{2} \cross C_{3} = \langle (a,b) \rangle$, and so it is a cyclic group.
  Hence $C_{2} \cross C_{3}$ is isomorphic to the cyclic group of order $6$,
  $C_{6}$.
\end{example}

You may notice that in these example the groups are all Abelian.  This is a
consequence of the following proposition.

\begin{proposition}
  If $G$ and $H$ are Abelian groups, then so is $G \cross H$.
\end{proposition}
\begin{proof}
  Exercise.
\end{proof}

The converse to this proposition is also true, but it requires a lot more
theory to get it in its nicest form.

If we have several groups $G_{1}$, $G_{2}, \ldots, G_{n}$, we can define
the direct product $G_{1} \cross G_{2} \cross \cdots G_{n}$ in the obvious
way.  There is a fairly clear isomorphism between
$(G_{1} \cross G_{2}) \cross G_{3}$, $G_{1} \cross (G_{2} \cross G_{3})$, and
$G_{1} \cross G_{2} \cross G_{3}$ given by the correspondence
\[
  ((g_{1}, g_{2}), g_{3}) \leftrightarrow (g_{1}, (g_{2}, g_{3})) 
  \leftrightarrow (g_{1}, g_{2}, g_{3}).
\]
So just as we consider the vector spaces $(V_{1} \oplus V_{2}) \cross V_{3}$,
$V_{1} \oplus (V_{2} \cross V_{3})$ and  $V_{1} \oplus V_{2} \cross V_{3}$ as
being the same vector space, we blur the distinction between the above
direct products of groups and regard all three as the same group.
With this in mind, the direct product is then associative.  We will also
write
\[
  G^{n} = \underbrace{G \cross G \cross \cdots \cross G}_{\text{$n$ times}}.
\]

The following theorem will prove useful when we try to classify all the
groups of a given order.  It generalizes the isomorphism between $V$ and
$C_{2} \cross C_{2}$.

\begin{theorem}\label{thm:order2group}
  Let $G$ be a finite group such that $x^{2} = 1$ for every element $x \in G$,
  and $|G| \ge 2$.
  Then $G$ is isomorphic to $C_{2} \cross C_{2} \cross \cdots \cross C_{2}$.
\end{theorem}
\begin{proof}
  We first observe that $G$ must be Abelian.  Given any $x$ and $y \in G$,
  we have
  \[
    xyxy = (xy)^{2} = 1.
  \]
  But then
  \[
    x = x1 = x(xyxy) = x^{2}yxyx = yxy,
  \]
  and so
  \[
    yx = y(yxy) = y^{2}xy = xy.
  \]
  
  We now find elements $a_{k} \in G$ by the following inductive construction:
  \begin{theoremenum}
    \item Since $|G| \ge 2$, we can find some element $a_{1} \ne 1$.  So
      $\langle a_{1} \rangle = \{1, a_{1}\}$, since $a_{1}^{2} = 1$.
    
    \item Assume that we have found elements $a_{1}$, $a_{2}, \ldots, a_{r}$,
      such that $a_{k}$ is not in $\langle a_{1}, \ldots, a_{k-1} \rangle$ for
      all $k = 2, \ldots, r$.
      
      Then one of two things must be true: either
      $G = \langle a_{1}, \ldots, a_{r} \rangle$, or there is some $a_{r+1}$
      which is not in $\langle a_{1}, \ldots, a_{r} \rangle$. 
      But then the elements $a_{1}$, $a_{2}, \ldots, a_{r}, a_{r+1}$ satisfy
      the condition for $r+1$.
    
    \item Proceeding inductively, we must eventually exhaust all the elements
      of $G$.
  \end{theoremenum}
  
  So we have that $G = \langle a_{1}, a_{2}, \ldots, a_{n} \rangle$ for elements
  $a_{1}$, $a_{2}, \ldots, a_{n}$ such that $a_{k}$ is not in $\langle a_{1}, \ldots, a_{k-1} \rangle$ for
  all $k = 2, \ldots, n$.  So any element $x \in G$ can be written as a product
  of powers of the elements $a_{1}$, $a_{2}, \ldots, a_{n}$, and since $G$
  is Abelian, we can move all the powers of $a_{1}$ to the front of the
  product, $a_{2}$ to the next term, and so on.  So in general
  \[
    x = a_{1}^{p(1)}a_{2}^{p(2)}\cdots a_{n}^{p(n)},
  \]
  where $p(k)$ must be either $0$ or $1$.  Moreover, this is the only way that
  the element $x$ can be written, since if we also have
  \[
    x = a_{1}^{r(1)}a_{2}^{r(2)}\cdots a_{n}^{r(n)},
  \]
  then
  \begin{align*}
    1 &= a_{1}^{p(1)}a_{2}^{p(2)}\cdots a_{n}^{p(n)}(a_{1}^{r(1)}a_{2}^{r(2)}\cdots a_{n}^{r(n)})^{-1}\\
      &= a_{1}^{p(1) - r(1)}a_{2}^{p(2) - r(2)}\cdots a_{n}^{p(n) - r(n)}.
  \end{align*}
  But this implies that
  \[
    a_{n}^{p(n) - r(n)} = a_{1}^{r(1) - p(1)}a_{2}^{r(2) - p(2)}\cdots a_{n-1}^{r(n-1) - p(n-1)},
  \]
  and so $p(n) - r(n) = 0$, since if $p(n) - r(n) = 1$, then $a_{n}$ would be
  generated by $a_{1}$, $a_{2}, \ldots, a_{n-1}$, which contradicts our
  construction.  So
  \[
    1 = a_{1}^{p(1) - r(1)}a_{2}^{p(2) - r(2)}\cdots a_{n}^{p(n-1) - r(n-1)},
  \]
  and the same argument as for $n$ shows that $p(n-1) - r(n-1) = 0$.
  
  Proceeding inductively, we have that $p(k) - r(k) = 0$ for all $k$.  Hence
  $p(k) = r(k)$ for all $k$, and so there is only one such way to write $x$
  as a product of powers of $a_{1}$, $a_{2}, \ldots, a_{n}$ in that order.
  
  Now we can think of $C_{2} = \{1, a\}$, and so
  \[
    \underbrace{C_{2} \cross C_{2} \cross \cdots \cross C_{2}}_{\text{$n$ times}}
    = \{ (a^{p(1)}, a^{p(2)}, \ldots, a^{p(n)}) : p(k) \in \{0, 1\}\}.
  \]
  But we have a correspondence
  \[
    a_{1}^{p(1)}a_{2}^{p(2)}\cdots a_{n}^{p(n)} \leftrightarrow
    (a^{p(1)}, a^{p(2)}, \ldots, a^{p(n)}),
  \]
  and if you compare typical entries of the Cayley table you get
  \[
    \begin{array}{c|ccc}
      & \quad & a_{1}^{r(1)}a_{2}^{r(2)}\cdots a_{n}^{r(n)} & \quad \\
      \hline
      \\
    a_{1}^{p(1)}a_{2}^{p(2)}\cdots a_{n}^{p(n)} &&
       a_{1}^{p(1) + r(1)}a_{2}^{p(2) + r(2)}\cdots a_{n}^{p(n) + r(n)} \\
       &
    \end{array}
  \]
  and
  \[
    \begin{array}{c|ccc}
      & \quad & (a^{r(1)}, a^{r(2)}, \ldots, a^{r(n)}) & \quad \\
      \hline
      \\
    (a^{p(1)}, a^{p(2)}, \ldots, a^{p(n)}) &&
       (a^{p(1)+r(1)}, a^{p(2)+r(2)}, \ldots, a^{p(n)+r(n)}) \\
       &
    \end{array}
  \]
  and so the Cayley tables correspond.
  
  Hence $G \isom C_{2}^{n}$.
\end{proof}
\begin{corollary}
  Let $G$ be a finite group such that $x^{2} = 1$ for every element $x \in G$,
  and $|G| \ge 2$.
  Then $|G| = 2^{n}$ for some $n$.
\end{corollary}

\subsection*{Exercises}

\begin{exercises}
  \item Show that $D_{2n}$ and $C_{2} \cross C_{n}$ are not isomorphic for
    $n > 2$.
  
  \item Write down the Cayley tables of $C_{2} \cross C_{2} \cross C_{2}$
    and $C_{2} \cross C_{4}$.  Show that these two groups are not isomorphic.
  
    Show that $D_{8}$ and $C_{2} \cross C_{2} \cross C_{2}$ are not isomporhic.
  
  \item Let $p$ and $q$ be prime numbers.  Show that $C_{p} \cross C_{q} \isom
    C_{pq}$.
    
    More generally, show that if $p$ and $q$ are coprime, $C_{p} \cross C_{q} \isom
    C_{pq}$.

  \item Let $G$ be an Abelian group where $x^{3} = 1$ for every $x \in G$.
    Show that $G \isom C_{3} \cross C_{3} \cross \cdots \cross C_{3}$.
  
  \item (*) Find a group $G$ such that $x^{3} = 1$ for every $x \in
    G$, but $G \not\isom C_{3} \cross C_{3} \cross \cdots \cross C_{3}$
\end{exercises}

\section{Subgroups}

Groups often contain other groups.  You should be at least intuitively aware
of this fact, since the various additive groups of numbers are contained in
one another:
\[
  \integers \subset \rationals \subset \reals \subset \complex.
\]
Knowing the groups which are contained within a particular group can tell you
a lot about the group.

\begin{definition}
  Let $(G, \ast, e)$ be a group, and $H \subseteq G$.  We say $H$ is a
  \defn{subgroup}{subgroup} of $G$ if $(H, \ast, e)$ is a group (where
  $\ast$ is restricted to $H$).
  
  We denote this relationship by writing $H \le G$.  If $H \subset G$, 
  then we may write $H < G$.
\end{definition}

In general, we do not expect an arbitrary subset of a group to be a 
group.  In particular, we have to at least have $e \in H$.  
Furthermore, for $\ast$ to be a binary operation when restricted to 
$H$, it needs to be closed on $H$.  In other words, the product of 
elements of $H$ must again be an element of $H$.  Finally, we need 
that the inverse of every element of $H$ is an element of $H$.  But 
the good news is that we don't need to check associativity: that is 
guaranteed by the associativity of $\ast$ as an operation on $G$.  To 
summarize:

\begin{theorem}\label{thm:subgrouptest}
  Let $G = (G, \ast e)$ be a group, and $H \subseteq G$. If for every $x$ 
  and $y \in H$ we have
  \begin{theoremenum}
    \item $x \ast y \in H$, and
    
    \item $x^{-1} \in H$,
  \end{theoremenum}
  then $H$ is a subgroup.
\end{theorem}
\begin{proof}
  Condition (i) tells us that $\ast: H \cross H \to H$, so $\ast$ is a 
  binary operation on $H$.  Associativity is simple since $\ast$ is 
  associative on $G$, so the axiom still holds for a subset.  Since 
  $x^{-1} \in H$, we have both the inverse axiom, and the identity 
  being an element of $H$, since $e = x^{-1} \ast x \in H$ by (i).
  And $e$ is still an identity for $\ast$ on $H$, since it satisfies 
  the identity axiom for all elements of $G$, including those in $H$.
\end{proof}

In fact, we can make this theorem even slicker by combining the two conditions
into one:
\begin{corollary}\label{cor:subgrouptest}
  Let $G = (G, \ast e)$ be a group, and $H \subseteq G$. If for every $x$ 
  and $y \in H$ we have $xy^{-1} \in H$, then $H$ is a subgroup.
\end{corollary}
\begin{proof}
  We note that if $x \in H$, then $xx^{-1} = e \in H$, and hence $x^{-1} = ex^{-1}
  \in H$, giving condition (i) of the theorem.
  
  Additionally, since $y^{-1} \in H$, $xy = x(y^{-1})^{-1} \in H$, giving
  condition (ii) of the theorem.
  
  Hence $H$ is a subgroup.
\end{proof}

Note that if we use additive notation for a group, the conditions of
Theorem~\ref{thm:subgrouptest} become
\begin{theoremenum}
  \item $x + y \in H$, and
  \item $-x \in H$,
\end{theoremenum}
while the condition for Corollary~\ref{cor:subgrouptest} becomes $x - y \in H$.

We immediately note that a group $G$ is always a subgroup of itself, and the
set containing just the identity $\{e\}$ is always a group.  These two subgroups
are called the \defn{trivial subgroups}{subgroup!trivial} of $G$.  If $H$ is a
subgroup of $G$ which not trivial, we say that $H$ is a
\defn{proper subgroup}{subgroup!proper}.

For finite groups, the Cayley table of a subgroup is simply the Cayley table
of the whole group with every row and column corresponding to elements not in
the subgroup being removed.

\begin{example}
  The cyclic group of order $3$
  $C_{3} = \{1, a, a^{2}\}$ has the subgroups $\{1\}$, and $\{1, a,
  a^{2}\}$.  It has no proper subgroups.
  
  You can see that other subsets are not subgroups be inspection.
  For example, the set $\{1, a^{2}\}$ is not a subgroup because
  $a^{2}a^{2} = a^{4} = a$, and $a$ is not an element of the set.
\end{example}

\begin{example}\label{eg:C4subgroups}
  The cyclic group of order $4$ $C_{4} = \{1,
  a, a^{2}, a^{3}\}$ has the subgroups $\{1\}$, $\langle a^{2} \rangle
  = \{1, a^{2}\}$ and $\{1, a, a^{2}, a^{3}\}$.
  
  You can verify that $\{1, a^{2}\}$ is a subgroup by calculating every
  possible value of $xy^{-1}$ for $x$ and $y \in \{1, a^{2}\}$:
  \begin{alignat*}{2}
    1 \cdot 1^{-1} &= 1 \qquad & 1 \cdot (a^{2})^{-1} &= a^{-2} = a^{2}\\
    a^{2} \cdot 1^{-1} &= a^{2} & a^{2} \cdot (a^{2})^{-1} &= 1.
  \end{alignat*}
  In fact, we really only need to look at products where neither $x$ nor $y$
  is $1$, since those always leave the other term alone. We will see shortly
  that the fact that this is a set generated by an element guarantees that
  it is a subgroup.
  
  These subgroups are isomorphic to $C_{1}$, $C_{2}$ and $C_{4}$ 
  respectively.
\end{example}

\begin{example}\label{eg:4groupsubgroups}
  The vierergruppe $V = \{1, a, b, ab\}$ has the subgroups $\{1\}$,
  $\langle a \rangle = \{1, a\}$, $\langle b \rangle = \{1, b\}$,
  $\langle ab \rangle = \{1, ab\}$ and $\{1, a, b, ab\}$.
  
  These subgroups are isomorphic to $C_{1}$, $C_{2}$, $C_{2}$, $C_{2}$,
  and $V$ respectively.
\end{example}

\begin{example}\label{eg:C6subgroups}
  The cyclic group or order $6$, $C_{6} = \{1, a, a^{2}, a^{3}, a^{4}, a^{5}\}$
  has the subgroups $\{1\}$,
  $\langle a^{3} \rangle = \{1, a^{3}\}$,
  $\langle a^{2} \rangle = \{1, a^{2}, a^{4}\}$,
  and $C_{6}$.
  
  These subgroups are isomorphic to $C_{1}$, $C_{2}$, $C_{3}$, $C_{6}$,
  and $V$ respectively.
\end{example}

\begin{example}\label{eg:Cpsubgroups}
  Let $p$ be a prime number, and $C_{p}$ the cyclic group of order $p$.
  Since $a^{k}$ generates $C_{p}$ for all $k \ne 0$, any subgroup which
  contains any element other than $1$ must automatically contain all of
  $C_{p}$.  Hence $C_{p}$ only has the trivial subgroups $\{1\}$ and $C_{p}$.
\end{example}

\begin{example}
  If $s \in \reals$, then the set $\{sn : n \in \integers\}$ is a subgroup
  of the additive group of real numbers $(\reals, +, 0)$.  This follows
  because if we take two typical elements $ns$ and $ms$, then
  \[
    ns - ms = (n-m)s,
  \]
  and this is an element of the set $\{sn : n \in \integers\}$.
\end{example}

\begin{example}
  We have that $SL_{n}(\reals)$, $O_{n}(\reals)$ and $SO_{n}(\reals)$ are all
  proper subgroups of $GL_{n}(\reals)$.  In fact $SO_{n}(\reals)$ is also a
  proper subgroup of both $SL_{n}(\reals)$ and $O_{n}(\reals)$.
  
  We know that these are subgroups, since we showed that they were groups
  under matrix multiplication in an earlier example.
\end{example}

\begin{example}
  The alternating group $A_{n}$ is a subgroup of the corresponding symmetric
  group $S_{n}$.
\end{example}

\begin{example}
  If $G$ and $H$ are groups, then the subset $\{(x, e): x \in G\}$ of $G \cross H$
  is a subgroup of $G \cross H$.  Similarly, $\{(e, y): y \in H\}$ is a
  subgroup of $G \cross H$.
\end{example}

\begin{example}
  If $G$ is any finite group and $x \in G$, the set $\langle x \rangle$ is
  always a subgroup.  This follows since $x^{n}(x^{m})^{-1} = x^{n-m}$, which
  is a power of $x$ and so is an element of $\langle x \rangle$, and
  Corollary~\ref{cor:subgrouptest} tells us $\langle x \rangle$ is a subgroup.
\end{example}

In fact, we can extend the last example to the set generated by any set of
generators.

\begin{theorem}\label{thm:subgroupgenbyset}
  \sidebar{Finding Subgroups}{To find all the subgroups of a finite
  group, look at the subgroups generated by each element, then look at
  the subgroups generated by pairs of elements, then triples of
  elements, and so on.  This procedure works because of
  Theorem~\ref{thm:subgroupgenbyset}.\\ \medskip You can cut down the
  number of generating sets you need to check by noticing that if an
  element is in a subgroup, adding it to the set of generators of that
  subgroup gives nothing new.}
  Let $(G, \ast, e)$ be a group, and $X \subseteq G$.  Then $\langle X
  \rangle$ is the smallest subgroup of $G$ which contains $X$.
\end{theorem}
\begin{proof}
  First we must show that $\langle X \rangle$ is a subgroup.  This set
  consists of all products of powers of elements of $X$.  If we have
  \[
    x = x_{1}^{p_{1}}x_{2}^{p_{2}}\cdots x_{n}^{p_{n}} \qquad \text{and}
    \qquad y = y_{1}^{q_{1}}y_{2}^{q_{2}}\cdots y_{m}^{q_{m}},
  \]
  where $x_{k}$ and $y_{l} \in X$, $p_{k}$ and $q_{l} \in \integers$, then
  we have that
  \begin{align*}
    (x_{n}^{-p_{n}}x_{n-1}^{-p_{n-1}}\cdots x_{1}^{-p_{1}})x
     &= x_{n}^{-p_{n}}x_{n-1}^{-p_{n-1}}\cdots x_{1}^{-p_{1}}x_{1}^{p_{1}}x_{2}^{p_{2}}\cdots x_{n}^{p_{n}}\\
     &= x_{n}^{-p_{n}}x_{n-1}^{-p_{n-1}}\cdots x_{2}^{-p_{2}}x_{2}^{p_{2}}\cdots x_{n}^{p_{n}} \\
     &\vdots \\
     & = x_{n}^{-p_{n}}x_{n}^{p_{n}} = e,
  \end{align*}
  so $x^{-1} = x_{n}^{-p_{n}}x_{n-1}^{-p_{n-1}}\cdots x_{1}^{-p_{1}}$, which
  is a product of powers of elements of $X$, so $x^{-1} \in \langle X \rangle$.
  Similarly
  \[
    xy = x_{1}^{p_{1}}x_{2}^{p_{2}}\cdots x_{n}^{p_{n}}y_{1}^{q_{1}}y_{2}^{q_{2}}\cdots y_{m}^{q_{m}}
  \]
  is a product of powers of elements of $X$, so $xy \in \langle X \rangle$.
  
  So $\langle X \rangle$ satisfies conditions (i) and (ii) of Theorem~\ref{thm:subgrouptest},
  so it is a subgroup of $G$.
  
  Now assume that there is some subgroup $H$ of $G$ with $X \subseteq H \subset \langle X \rangle$.
  Then we can find some $x \in \langle X \rangle \setminus H$, so
  \[
    x = x_{1}^{p_{1}}x_{2}^{p_{2}}\cdots x_{n}^{p_{n}}
  \]
  where $x_{k} \in X$, $p_{k} \in \integers$.  A simple induction argument
  shows that $x_{k}^{p_{k}} \in H$ for all $k$, no matter what power we have
  of $p_{k}$.  But this means that $x_{1}^{p_{1}}x_{2}^{p_{2}} \in H$, since
  it is a product of elements of the subgroup $H$, and similarly
  $x_{1}^{p_{1}}x_{2}^{p_{2}}x_{3}^{p_{3}} \in H$.  Proceeding inductively, we
  have $x_{1}^{p_{1}}x_{2}^{p_{2}}\cdots x_{n}^{p_{n}} \in H$, and so $x \in H$.
  This is a contradiction, and hence there is no subgroup $H$.
  
  Therefore $\langle X \rangle$ is the smallest subgroup of $G$ containing $X$.
\end{proof}

It is worth noting that some texts actually define $\langle X \rangle$
to be the smallest subgroup of $G$ containing $X$.

\begin{example}\label{eg:D8subgroups}
  The dihedral group\index{group!dihedral} of order $8$, $D_{8} = \{1,
  a, a^{2}, a^{3}, b, ab, a^{2}, a^{3}\}$ has subgroups
  \begin{alignat*}{2}
    \langle 1 \rangle &= \{1\} \isom C_{1},
    &\langle a^{2} \rangle &= \{1, a^{2}\} \isom C_{2},\\
    \langle b \rangle &= \{1, b\} \isom C_{2},
    &\langle ab \rangle &= \{1, ab\} \isom C_{2},\\
    \langle a^{2}b \rangle &= \{1, a^{2}b\} \isom C_{2},
    &\langle a^{3}b \rangle &= \{1, a^{3}b\} \isom C_{2},\\
    \langle a \rangle &= \{1, a, a^{2}, a^{3}\} \isom C_{4},
    &\langle a^{2}, b \rangle &= \{1, a^{2}, b, a^{2}b\} \isom V,\\
    \langle a^{2}, ab \rangle &= \{1, a^{2}, ab, a^{3}b\} \isom V,
    \qquad &\langle a, b \rangle &= D_{8}.
  \end{alignat*}
\end{example}

\begin{example}
  Consider $S_{3} = \{e, (1,2,3), (1,3,2), (1,2), (2,3), (1,3)\}$.  We can
  find all the subgroups of this group by looking at the sets generated by
  each element:
  \begin{align*}
    \langle e \rangle &= e\\
    \langle (1,2,3) \rangle = \langle (1,3,2) \rangle &= \{e, (1,2,3), (1,3,2)\}\\
    \langle (1,2) \rangle &= \{e, (1,2)\}\\
    \langle (2,3) \rangle &= \{e, (2,3)\}\\
    \langle (1,3) \rangle &= \{e, (1,3)\}
  \end{align*}
  Now we need to consider the sets generated by pairs of elements.  For example,
  the set $\langle (1,2,3), (1,2) \rangle$ contains the elements $e$, $(1,2,3)$,
  $(1,3,2)$ and $(1,2)$ as well as
  \[
    (2,3) = (1,2,3)(1,2) \qquad (1,3) = (1,2)(1,2,3).
  \]
  So $\langle (1,2,3), (1,2) \rangle = S_{3}$. In fact, when we look at all
  possible pairings, we discover that
  \[
    \langle (1,2,3), (1,2) \rangle = \langle (1,2,3), (2,3) \rangle = \langle (1,2,3), (1,3) \rangle = S_{3}
  \]
  and
  \[
    \langle (1,2), (2,3) \rangle = \langle (1,2), (1,3) \rangle = \langle (2,3), (1,3) \rangle = S_{3}.
  \]
  So $S_{3}$ is the only other subgroup.
\end{example}

Notice in all the above examples of finite groups, the order of any
subgroup divides the order of the group.  This is always true, but we
need some new ideas before we can prove it.

\subsection*{Exercises}

\begin{exercises}
  \item Let $X$ be a subset of a group.  Show that any one of the 
    following is sufficient to show that $X$ is not a subgroup:
    \begin{theoremenum}
      \item $e \notin X$,
      
      \item there is an $x \in X$ with $x^{-1} \notin X$,
      
      \item there is an $x$ and $y \in X$ with $xy \notin X$,
      
      \item there is an $x$ and $y \in X$ with $xy^{-1} \notin X$.
    \end{theoremenum}
  
  \item For each of the following groups, find all its subgroups.  
    For each subgroup, determine if it is isomorphic to a known group.
    \begin{theoremenum}
      \item $D_{6}$
      
      \item $C_{8}$
    
      \item $C_{2} \cross C_{4}$
      
      \item $C_{2} \cross C_{2} \cross C_{2}$
      
      \item $D_{10}$

      \item $D_{12}$
      
      \item $A_{4}$
    \end{theoremenum}
    
  \item Let $s \in \reals$.  Show that $\{ n + ms : n, m \in \integers\}$
    is a subgroup of $\reals$.
  
  \item Show that $\naturals$ is not a subgroup of $(\integers, +, 0)$.
  
  \item Show that every subgroup of a cyclic group is a cyclic
    group\index{group!cyclic}.
  
    Show that every subgroup of an Abelian group is an Abelian
    group\index{group!Abelian}.
    
  \item Show that every group has a cyclic subgroup.
\end{exercises}

\section{Homomorphisms}\index{homomorphism|(}\index{-morphism!homo-|(}

Recall from abstract linear algebra that a linear transformation is a 
function from one vector space to another which preserves vector 
addition and scalar multiplication, ie.\ $T: V \to W$ is a linear 
transformation if and only if
\[
  T(v+w) = T(v) + T(w) \qquad \text{and} \qquad T(\lambda v) = 
  \lambda T(v),
\]
for all $v$, $w \in V$ and $\lambda \in \field$.

The analogue for groups should, then, be a function which preserves 
the group operation.

\begin{definition}
  Let $(G, \ast, e)$ and $(H, \star, 1)$ be groups.  A function
  $\alpha: G \to H$ is a \defn{(group)
  homomorphism}{homomorphism}\index{-morphism!homo-|emph} if
  \[
    \alpha(x \ast y) = \alpha(x) \star \alpha(y)
  \]
  for all $x$, $y \in X$.
\end{definition}
  
When using the multiplicative notation for groups, we will often 
simply write this condition as
\[
  \alpha(xy) = \alpha(x)\alpha(y).
\]

It is immediate from this definition that group homomorphisms preserve 
the identity and inverse.

\begin{proposition}
  Let $G$ and $H$ be groups and $\alpha: G \to H$ a homomorphism. 
  Then
  \begin{theoremenum}
    \item $\alpha(e_{G}) = e_{H}$,

    \item $\alpha(x^{-1}) = (\alpha(x))^{-1}$ for all $x \in X$.
  \end{theoremenum}
\end{proposition}
\begin{proof}
  (i) Let $y = \alpha(x)$ for some $x \in G$, so that
  \[
    y \alpha(e_{G}) = \alpha(x)\alpha(e_{G}) = 
    \alpha(xe_{G}) = \alpha(x) = y = y e_{H}.
  \]
  The cancellation law then tells us that $\alpha(e_{G}) = e_{H}$.
    
  (ii) Given any $x \in G$, we have that
  \[
    \alpha(x^{-1})\alpha(x) = \alpha(x^{-1}x) = \alpha(e_{G}) = e_{H}.
  \]
  So $\alpha(x^{-1}) = \alpha(x)^{-1}$.
\end{proof}

Using this proposition we can show, using induction if needed, that
\[
  \alpha(x^{n}) = \alpha(x)^{n}
\]
for any $n \in \integers$.

\begin{example}\label{eg:4grouphom}
  Let $V = \{1, a, b, ab\}$ be the four-group, and $C_{4} = \{1, a,
  a^{2}, a^{3}\}$ be the cyclic group of order $4$.  Consider the
  function $\alpha: V \to C_{4}$ definied by the following table:
  \[
    \begin{array}{c|c}
      x & \alpha(x) \\
      \hline
      1 & 1 \\
      a & a^{2} \\
      b & a^{2} \\
      ab & 1 \\
    \end{array}
  \]
  Checking by hand, we can verify that this is indeed a 
  homomorphism.  For example $a^{2} = 1$ in $V$ so $\alpha(a^{2}) = 
  \alpha(1) = 1$, and $\alpha(a)\alpha(a) = a^{2}a^{2} = 1$.
\end{example}

\begin{example}
  If we look at the groups $\integers_{3}$, where the group operation
  is addition modulo $3$, and the group $C_{6} = \{1, a, a^{2}, a^{3},
  a^{4}, a^{5}\}$, then we have the following homomorphisms from
  $\integers_{3} \to C_{6}$:
  \begin{align*}
    \alpha(x) &= 1\\
    \beta(x) &= a^{2x}\\
    \gamma(x) &= a^{-2x}
  \end{align*}
  To verify that $\beta$ is a homomoprhism, for example, we need to
  check that $\beta(x+y) = \beta(x)\beta(y)$\sidebar{Note}{Observe
  here that $\integers_{3}$ uses additive notation, while $C_{6}$ uses
  multiplicative notation, and we need to use the appropriate form of
  notation when verifying that we have a homomorphism.}:
  \begin{align*}
    \beta(x+y \pmod{3}) &= a^{2(x+y \pmod{3})} = a^{2x+2y \pmod{6}}\\
    \beta(x)\beta(y) &= a^{2x}a^{2y} = a^{2x+2y \pmod{6}},
  \end{align*}
  noting that in $C_{6}$, $a^{x}a^{y} = a^{x+y \pmod{6}}$.
  So $\beta$ is a homomorphism.  You could also verify this fact by 
  case-by-case checking of results.
  
  On the other hand, the function $\delta: \integers_{3} \to C_{6}$ 
  defined by $\delta(x) = a^{x}$ is not a homomorphism, because
  \[
    \delta(1 + 2 \pmod{3}) = \delta(0) = a^{0} = 1,
  \]
  but
  \[
    \delta(1)\delta(2) = a^{1}a^{2} = a^{3} \ne 1.
  \]
\end{example}

\begin{example}
  Let $GL_{n}(\reals)$ be the group of all invertible real-valued 
  matrices, and consider the determinant function $\det: 
  GL_{n}(\reals) \to R^{\cross}$ given by
  \[
    \det(A) = |A|.
  \]
  Since
  \[
    \det(AB) = |AB| = |A||B| = \det(A)\det(B),
  \]
  this is a homomorphism.
\end{example}

If we once again consider the analogy with abstract linear algebra, you may
recall that the image of a linear subspace under a linear transformation is
a subspace of the range.  If the analogy with linear algebra is to hold, the
same thing ought to be true for subgroups.

\begin{proposition}\label{prop:homsubgroup}
  If $G$ and $H$ are groups, $\alpha : G \to H$ is a homomorphism, and
  $K$ is a subgroup of $G$, then $\alpha(K)$ is a subgroup of $H$.
\end{proposition}
\begin{proof}
  Let $x$, $y \in \alpha(K)$, so that there are some $u$ and $v \in K$ such
  that $x = \alpha(u)$ and $y = \alpha(v)$.  Then, noting that $uv^{-1} \in K$,
  \[
    xy^{-1} = \alpha(u)(\alpha(v))^{-1} = \alpha(uv^{-1}) \in \alpha(K).
  \]
  So by Corollary~\ref{cor:subgrouptest}, $\alpha(K) \le H$.
\end{proof}

\begin{example}
  We know from
  Example~\ref{eg:4groupsubgroups} that the subgroups of $V$ are $\{1\}$,
  $\{1,a\}$, $\{1,b\}$, $\{1, ab\}$ and the whole group $V$.  The images of
  these sets under the homomorphism $\alpha$ of Example~\ref{eg:4grouphom}
  are $\{1\}$, $\{1,a^{2}\}$, $\{1,a^{2}\}$, $\{1\}$ and $\{1, a^{2}\}$,
  respectively.
\end{example}

Since $G \le G$, the following is an immediate corollary of the proposition.

\begin{corollary}
  If $G$ and $H$ are groups, and $\alpha : G \to H$ is a homomorphism,
  then $\alpha(G)$ is a subgroup of $H$.
\end{corollary}

In other words, the image of a homomorphism is a subgroup of the codomain.

We have a similar result for inverse images of subgroups.

\begin{proposition}\label{prop:inversehomsubgroup}
  If $G$ and $H$ are groups, $\alpha : G \to H$ is a homomorphism, and
  $K$ is a subgroup of $H$, then $\alpha^{-1}(K)$ is a subgroup of
  $G$.
\end{proposition}
\begin{proof}
  Let $x$, $y \in \alpha^{-1}(K)$, so that there are some $u$ and $v \in K$ such
  that $u = \alpha(x)$ and $v = \alpha(y)$.  Then,
  \[
    \alpha(xy^{-1}) = \alpha(x)(\alpha(y))^{-1} = uv^{-1} \in K,
  \]
  so $xy^{-1} \in \alpha^{-1}(K)$.
  Therefore by Corollary~\ref{cor:subgrouptest}, $\alpha^{-1}(K) \le G$.
\end{proof}

You may also recall from linear algebra that the kernel of a linear
transformation is a subspace of the domain.  This leads us to the
following definition and corollary.

\begin{definition}
  If $G$ and $H$ are groups, and $\alpha: G \to H$ is a homomorphism, 
  then the \defn{kernel}{kernel} of $\alpha$ is the set
  \[
    \ker \alpha = \{ x \in X : \alpha(x) = e_{H} \}
  \]
  of all elements of the group $G$ whose image is the identity.
\end{definition}

\begin{corollary}
  If $G$ and $H$ are groups, and $\alpha: G \to H$ is a homomorphism, 
  then $\ker \alpha$ is a subgroup of $G$.
\end{corollary}
\begin{proof}
  Notice that $\ker \alpha = \alpha^{-1}(\{e_{H}\})$, and $\{e_{H}\} \le H$, so
  by Proposition~\ref{prop:inversehomsubgroup}, $\ker \alpha \le G$.
\end{proof}

\begin{example}
  In Example~\ref{eg:4grouphom}, the kernel of the homomorphism is $\{1, ab\}$,
  which is a subgroup of $V$, and the image of the homomorphism is $\{1, a^{2}\}$,
  which is a subgroup of $C_{4}$.
\end{example}

\begin{example}
  The kernel of the determinant map from the previous example is the set
  of all matrices whose determinant is $1$, ie.
  \[
    \ker \det = SL_{n}(\reals).
  \]
  
  On the other hand, there are matrices whose determinant is any number you
  choose, so the image of $GL_{n}(\reals)$ under $\det$ is all of $\reals 
  \setminus \{0\}$.
\end{example}

\begin{example}
  Parity can be regarded as a function that takes permutations to elements
  of the multiplicative group of integers $\integers \setminus \{0\}$.
  Theorem~\ref{thm:parityproduct} tells us that this is a homomorphism.
  
  The kernel of $\parity$ is the subgroup $A_{n}$ of all permutations
  whose parity is $1$, ie.
  \[
    \ker \parity = A_{n}.
  \]
  
  On the other hand, the image of $S_{n}$ under the $\parity$ homomorphism
  is simply the set $\{1, -1\}$, which is a subgroup of the multiplicative
  group of integers.
\end{example}

Just as we are particularly interested in functions which are one-to-one,
onto, or bijective, we are interested in homomorphisms which are one-to-one,
onto, or bijective.

\begin{definition}
  Let $G$ and $H$ be groups, and $\alpha : G \to H$ a homomorphism.  If
  $\alpha$ is one-to-one (or injective), then we say that it is a
  \defn{monomorphism}{monomorphism}\index{-morphism!mono-|emph}.  If $\alpha$ is onto (or sujective),
  then we say that it is an \defn{epimorphism}{epimorphism}\index{-morphism!epi-|emph}.  If $\alpha$
  is a bijection, then we say that it is an \defn{isomorphism}{isomorphism}\index{-morphism!iso-|emph}.
  
  If $\alpha: G \to G$ is an isomorphism, then we call $\alpha$ an
  \defn{automorphism}{automorphism}\index{-morphism!auto-|emph}.  The set of all automorphisms of $G$
  is denoted $\Aut(G)$.
\end{definition}

This definition of isomorphism agrees with the definition that we have been
using, but can also be applied to infinite groups.  The concept of
``correspondence'' of elements that we have been informally using to translate
between the two Cayley tables is, formally, a bijection; while the fact that
the Cayley tables correspond means that the group operation is preserved by
the bijection, giving us a homomorphism.  If there is an isomorphism between
two groups $G$ and $H$, then we say that the groups are \defn{isomorphic}{isomorphic}
and write $G \isom H$ as usual.

\begin{example}
  Consider the additive group of reals numbers, $(\reals, +, 0)$, and the
  multiplicative group of positive real numbers, $(\reals^{+}, \times, 1)$.
  The exponential function
  \[
    \exp : x \mapsto e^{x}
  \]
  is a homomorphism between these two groups, since
  \[
    \exp(x+y) = e^{x+y} = e^{x}e^{y} = \exp(x)\exp(y).
  \]
  Furthermore, the exponential function is one-to-one (a fact you should be
  familiar with from elementary calculus), and the range of the exponential
  function is all positive real numbers.  Hence $\exp$ is an isomorphism, and
  the two groups are isomorphic.
  
  In fact, this isomorphism is not the only possible choice.  Any function
  of the form
  \[
    x \mapsto a^{x}
  \]
  for $a > 0$ is an isomorphism.  Going in the reverse direction, the
  corresponding logarithms
  \[
    x \mapsto \log_{a} x,
  \]
  regarded as functions from $\reals^{+}$ to $\reals$, are also isomorphisms.
\end{example}

The following proposition collects some useful facts about 
homomorphisms.

\begin{proposition}\label{prop:homomorphismfacts}
  Let $G$, $H$ and $K$ be groups, and let $\alpha: G \to H$ and 
  $\beta: H \to K$ be homomorphisms.
  \begin{theoremenum}
    \item $\beta \circ \alpha: G \to K$ is a homomorphism,
    
    \item if $\alpha$ and $\beta$ are monomorphisms, so is $\beta \circ 
      \alpha$,
    
    \item if $\alpha$ and $\beta$ are epimorphisms, so is $\beta \circ 
      \alpha$,
    
    \item if $\alpha$ and $\beta$ are isomorphisms, so is $\beta \circ 
      \alpha$,
    
    \item if $\alpha$ is an isomorphism, so is the inverse function 
      $\alpha^{-1}$,
    
    \item $\alpha$ is a monomorphism if and only if $\ker \alpha = 
      \{e\}$,
  \end{theoremenum}
\end{proposition}
\begin{proof}
  (i) We observe that
    \[
      \beta(\alpha(xy)) = \beta(\alpha(x)\alpha(y)) = 
      \beta(\alpha(x))\beta(\alpha(y)),
    \]
    so $\beta \circ \alpha$ is a homomorphism.
  
  (ii--iv) These follow immediately from (i) which tells us that the 
    composition is a homomorphism, and parts (ii--iv) of
    Proposition~\ref{prop:functionfacts} which tell us that the 
    composition is, respectively, one-to-one, onto, and bijective.
  
  (vi) If $\alpha$ is an isomorphism, then for any $x$, $y \in H$, we 
    have $u$ and $v \in G$ such that $\alpha(u) = x$ and $\alpha(v) = 
    y$, so
    \[
      xy = \alpha(u)\alpha(v) = \alpha(uv).
    \]
    But $u = \alpha^{-1}(x)$ and $v = \alpha^{-1}(y)$, so
    \[
      \alpha^{-1}(xy) = \alpha^{-1}(\alpha(uv)) = uv = 
      \alpha^{-1}(x)\alpha^{-1}(y).
    \]
    Hence $\alpha^{-1}$ is a homomorphism.
    
    Furthermore, parts (v) of Proposition~\ref{prop:functionfacts} 
    tells us that $\alpha^{-1}$ is a bijection, so $\alpha^{-1}$ is an 
    isomorphism.
  
  (v) If $\alpha$ is a monomorphism and $\alpha(x) = e$, then 
    $\alpha(x) = \alpha(e)$, and since $\alpha$ is injective, $x = e$.
    So $\ker \alpha = \{ e \}$.
  
    On the other hand, if $\ker \alpha = \{e\}$, then if we have $x$
    and $y \in G$ such that $\alpha(x) = \alpha(y)$, then
    \[
      \alpha(xy^{-1}) = \alpha(x)\alpha(y)^{-1} = 
      \alpha(x)\alpha(x)^{-1} = e,
    \]
    so $xy^{-1} \in \ker \alpha$, and hence $xy^{-1} = e$.  But then
    \[
      x = xy^{-1}y = ey = y,
    \]
    so $\alpha$ is injective and hence a monomorphism.
\end{proof}

\subsection*{Exercises}

\begin{exercises}
  \item Find all the homomorphisms from $C_{2}$ to $V$.  Describe the 
    kernel and image of each.
  
  \item Let $\integers$ be the additive group of integers.  Show that 
  for each $m \in \integers$, the function $\alpha_{m} : \integers 
  \to \integers$, given by
  \[
    \alpha_{m}(x) = mx
  \]
  is a homomorphism.  Show that if $m \ne 0$ then $\alpha_{m}$ is a 
  monomorphism.  Show that it is an epimorphism if and only if $m = 
  \pm 1$.
  
  \item Let $\rationals$ be the additive group of rational numbers.  Show that 
  for each $r \in \rationals$, the function $\alpha_{r}: \rationals 
  \to \rationals$ given by
  \[
    \alpha_{r}(x) = rx
  \]
  is a homomorphism.  Show that if $r \ne 0$ then $\alpha_{r}$ is an 
  automorphism.
  
  \item Consider the group $(G, \ast, e)$, where $G = \{(x,y) : x,y
    \in \reals, x \ne 0\}$, $(x,y) \ast (x',y') = (xx', xy' + y)$ and $e =
    (1,0)$, and the group $(H, \cdot, I_{2})$, where
    \[
      H = \left\{\begin{bmatrix}
        a & b \\
0 & 1
      \end{bmatrix} : a, b \in \reals, a \ne 0 \right\},
    \]
    $\cdot$ is matrix multiplication, and $I_{2}$ is the 2 by 2 
    identity matrix.  Show that $G$ and $H$ are isomorphic.
  
  \item Show that every isomorphism $\alpha : (\reals, +, 0) \to 
    (\reals^{+}, \times, 1)$ satisfies
    \[
      \alpha(x) = a^{x}
    \]
    for some number $a > 0$.
    
    Show that every isomorphism $\alpha : (\reals^{+}, \times, 1) \to 
    (\reals, +, 0)$ satisfies
    \[
      \alpha(x) = \log_{a} x
    \]
    for some number $a > 0$.
  
  \item\label{ex:directprodhoms} Let $G$ and $H$ be two groups. 
    Show that each of the following is a homomorphism:
    \begin{theoremenum}
      \item $\alpha: G \to G \cross H$, where $\alpha(g) = (g,1)$,

      \item $\alpha: H \to G \cross H$, where $\alpha(h) = (1,h)$,

      \item $\alpha: G \cross H \to G$, where $\alpha(g,h) = g$,

      \item $\alpha: G \cross H \to H$, where $\alpha(g,h) = h$,

      \item $\alpha: G \cross H \to H \cross G$, where 
        $\alpha(g,h) = (h,g)$,

      \item $\alpha: G \to G \cross G$, where $\alpha(g) = (g,g)$.
    \end{theoremenum}
    Which of these are monomorphisms, which are epimorphisms, and 
    which are isomorphisms?
  
  \item Find all the automorphisms of $V$.
  
  \item Let $G$ be a group.  Show that $(\Aut(G), \circ, \id)$ is a 
    group, where $\circ$ is composition and $\id: G \to G$ is the 
    identity function $\id(x) = x$ for all $x \in G$.
\end{exercises}
\index{homomorphism|)}\index{-morphism!homo-|)}


\newpage
\section*{Assignment 2}

The following exercises are due Friday, March 5th.

\begin{description}
  \item[2.1] Exercises 1, 2.
  \item[2.2] Exercises 2, 3, 5.
  \item[2.3] Exercises 2, 3, 4.
  \item[2.4] Exercises 1.
\end{description}

\section*{Assignment 3}

The following exercises are due Friday, March 12th.

\begin{description}
  \item[2.5] Exercises 1, 2, 7.
  \item[2.7] Exercises 1, 4.
  \item[2.8] Exercises 2, 3.
  \item[2.9] Exercises 1, 2, 4, 7.
\end{description}


\chapter{The Structure of Groups}

We are interested in understanding the structure of groups, particularly finite
groups, as a way of potentially distinguishing groups.  In this
section we will see a number of ways of looking at structure within a group.

\section{The Subgroup Lattice}

At a very coarse level, if two groups are isomorphic then their subgroups
must be in bijective correspondence with one another.

\begin{proposition}\label{prop:subgroupisom}
  Let $G$ and $H$ be two groups, and let $\Sub(G)$ and $\Sub(H)$ be the set
  of subgroups of $G$ and $H$ respectively.  If $G \isom H$ then there is a
  bijection between $\Sub(G)$ and $\Sub(H)$.
\end{proposition}
\begin{proof}
  Since $G \isom H$ there is an isomorphism $\alpha : G \to H$, and its
  inverse function $\alpha^{-1}: H \to G$ is also an isomorphism by
  Proposition~\ref{prop:homomorphismfacts}.  Since by
  Proposition~\ref{prop:homsubgroup}, the image $\alpha(K)$ of a subgroup $K$
  of $G$ is a subgroup of $H$, we can define a function
  \begin{align*}
    \overline{\alpha} : \Sub(G) & \to \Sub(H) \\
        K & \mapsto \alpha(K).
  \end{align*}
  
  The function $\overline{\alpha}$ is one-to-one, since if we have two
  subgroups $K_{1}$ and $K_{2}$ of $G$ such that $\overline{\alpha}(K_{1}) =
  \overline{\alpha}(K_{2})$, then
  \[
    K_{1} = \alpha^{-1}(\alpha(K_{1})) = \alpha^{-1}(\overline{\alpha}(K_{1})) =
    \alpha^{-1}(\overline{\alpha}(K_{12})) = \alpha^{-1}(\alpha(K_{2})) =
    K_{2},
  \]
  since $\alpha^{-1} \circ \alpha$ is the identity function.
  
  The function $\overline{\alpha}$ is onto, since if $K$ is a subgroup of
  $H$, then $\alpha^{-1}(K)$ is a subgroup of $G$, and
  \[
    \overline{\alpha}(\alpha^{-1}(K)) = \alpha(\alpha^{-1}(K)) = K
  \]
  since $\alpha \circ \alpha^{-1}$ is the identity function.
  
  So $\overline{\alpha}$ is a bijection.
\end{proof}

\begin{corollary}
  If $G$ and $H$ are finite groups with different numbers of subgroups, then
  $G$ and $H$ cannot be isomorphic.
\end{corollary}

\begin{example}
  From Example~\ref{eg:C4subgroups} $C_{4}$, we know that $C_{4}$ has 3
  subgroups.  On the other hand, Example~\ref{eg:4groupsubgroups} tells us
  that $V$ has $5$ subgroups, so $V$ is not isomorphic to $C_{4}$.
\end{example}

However, it is certainly conceivable that two groups may have the same
number of subgroups, but fail to be isomorphic.  In this case we need to
investigate the relationships between subgroups of a group.  For instance, if
we have a group $G$ and subgroups $H$ and $K$ of $G$ such that $K \subseteq H$,
then it is immediate from Corollary~\ref{cor:subgrouptest} that $K$ is a
subgroup of $H$.  So a good starting point is to consider which subgroups are
contained in other subgroups.

\begin{example}
  From Example~\ref{eg:C4subgroups} $C_{4} = \{1, a, a^{2}, a^{3}\}$ has the subgroups
  $\{1\}$, $\langle a^{2} \rangle = \{1, a^{2}\}$ and
  $\{1, a, a^{2}, a^{3}\}$.  We can easily see that $\{1\} \le \langle a^{2}
  \rangle \le C_{4}$.
 \end{example}
 
  Noting that each of these is subgroups is contained in the next, we can
  represent this situation diagramatically as follows:
  
  \begin{picture}(2,6)(-1,-1)
    \put(0,0){\makebox(0,0){$\{1\}$}}
    \put(0,0.5){\line(0,1){1}}
    \put(0,2){\makebox(0,0){$\langle a^{2} \rangle$}}
    \put(0,2.5){\line(0,1){1}}
    \put(0,4){\makebox(0,0){$C_{4}$}}
  \end{picture}

This sort of diagram is a graphical representation of the \defn{subgroup
lattice}{lattice!subgroup} of the group.  The idea is that if a
subgroup is contained in another, with no intermediate subgroups, we write
it higher on the page and join the two subgroups with a line.  We also try
to draw the diagram  so that subgroups with the same number of elements are
the same distance up the page.

Here are some more examples.

\begin{example}
  From Example~\ref{eg:4groupsubgroups}, the vierergruppe $V = \{1, a, b, ab\}$
has the subgroups $\{1\}$,
  $\langle a \rangle = \{1, a\}$, $\langle b \rangle = \{1, b\}$,
  $\langle ab \rangle = \{1, ab\}$ and $\{1, a, b, ab\}$.
  
  The subgroup lattice of $V$ is:
  
  \begin{picture}(6,6)(-3,-1)
    \put(0,0){\makebox(0,0){$\{1\}$}}
    \put(0,0.5){\line(0,1){1}}
    \put(0.5,0.5){\line(1,1){1}}
    \put(-0.5,0.5){\line(-1,1){1}}
    \put(-2,2){\makebox(0,0){$\langle a \rangle$}}
    \put(0,2){\makebox(0,0){$\langle b \rangle$}}
    \put(2,2){\makebox(0,0){$\langle ab \rangle$}}
    \put(0,2.5){\line(0,1){1}}
    \put(1.5,2.5){\line(-1,1){1}}
    \put(-1.5,2.5){\line(1,1){1}}
    \put(0,4){\makebox(0,0){$V$}}
  \end{picture}
\end{example}

\begin{example}
  From Example~\ref{eg:C6subgroups}, the cyclic group or order $6$, $C_{6} =
  \{1, a, a^{2}, a^{3}, a^{4}, a^{5}\}$
  has the subgroups $\{1\}$,
  $\langle a^{3} \rangle = \{1, a^{3}\}$,
  $\langle a^{2} \rangle = \{1, a^{2}, a^{4}\}$,
  and $C_{6}$.
  The subgroup lattice of $C_{6}$ is:
  
  \begin{picture}(6,8)(-3,-1)
    \put(0,0){\makebox(0,0){$\{1\}$}}
    \put(-0.5,0.5){\line(-1,1){1}}
    \put(0.25,0.5){\line(1,2){1.5}}
    \put(-2,2){\makebox(0,0){$\langle a^{3} \rangle$}}
    \put(2,4){\makebox(0,0){$\langle a^{2} \rangle$}}
    \put(-1.75,2.5){\line(1,2){1.5}}
    \put(1.5,4.5){\line(-1,1){1}}
    \put(0,6){\makebox(0,0){$C_{6}$}}
  \end{picture}
\end{example}

\begin{example}
  Let $p$ be a prime number, and $C_{p}$ the cyclic group of order $p$.
  From Example~\ref{eg:Cpsubgroups} $C_{p}$ only has the trivial subgroups $\{1\}$ and $C_{p}$.
  The subgroup lattice of $C_{p}$ is always:
  
  \begin{picture}(2,4)(-1,-1)
    \put(0,0){\makebox(0,0){$\{1\}$}}
    \put(0,0.5){\line(0,1){1}}
    \put(0,2){\makebox(0,0){$C_{p}$}}
  \end{picture}
\end{example}

\begin{example}
  From Example~\ref{eg:D8subgroups}, the dihedral group\index{group!dihedral}
of order $8$, $D_{8} = \{1, a, a^{2}, a^{3}, b, ab, a^{2}, a^{3}\}$
  has subgroups $\{1\}$,
  $\langle a^{2} \rangle = \{1, a^{2}\}$,
  $\langle b \rangle = \{1, b\}$,
  $\langle ab \rangle = \{1, ab\}$,
  $\langle a^{2}b \rangle = \{1, a^{2}b\}$,
  $\langle a^{3}b \rangle = \{1, a^{3}b\}$,
  $\langle a \rangle = \{1, a, a^{2}, a^{3}\}$,
  $\langle a^{2}, b \rangle = \{1, a^{2}, b, a^{2}b\}$,
  $\langle a^{2}, ab \rangle = \{1, a^{2}, ab, a^{3}b\}$,
  and $D_{8}$.
  The subgroup lattice of $D_{8}$ is:
  
  \begin{picture}(10,8)(-5,-1)
    \put(0,0){\makebox(0,0){$\{1\}$}}
    \put(0,0.5){\line(0,1){1}}
    \put(0.5,0.5){\line(1,1){1}}
    \put(0.5,0.25){\line(2,1){3}}
    \put(-0.5,0.5){\line(-1,1){1}}
    \put(-0.5,0.25){\line(-2,1){3}}
    \put(-4,2){\makebox(0,0){$\langle b \rangle$}}
    \put(-2,2){\makebox(0,0){$\langle a^{2}b \rangle$}}
    \put(0,2){\makebox(0,0){$\langle a^{2} \rangle$}}
    \put(2,2){\makebox(0,0){$\langle ab \rangle$}}
    \put(4,2){\makebox(0,0){$\langle a^{3}b \rangle$}}
    \put(0,2.5){\line(0,1){1}}
    \put(3.5,2.5){\line(-1,1){1}}
    \put(2,2.5){\line(0,1){1}}
    \put(0.5,2.5){\line(1,1){1}}
    \put(-3.5,2.5){\line(1,1){1}}
    \put(-0.5,2.5){\line(-1,1){1}}
    \put(-2,2.5){\line(0,1){1}}
    \put(-2,4){\makebox(0,0){$\langle a^{2}, b \rangle$}}
    \put(0,4){\makebox(0,0){$\langle a \rangle$}}
    \put(2,4){\makebox(0,0){$\langle a^{2}, ab \rangle$}}
    \put(0,4.5){\line(0,1){1}}
    \put(1.5,4.5){\line(-1,1){1}}
    \put(-1.5,4.5){\line(1,1){1}}
    \put(0,6){\makebox(0,0){$D_{8}$}}
  \end{picture}
\end{example}

Notice that in these diagrams, there is always a unique smallest subgroup which
is bigger than any pair of subgroups.  This is a corollary of
Theorem~\ref{thm:subgroupgenbyset}.

\begin{corollary}
  If $G$ is a group and $H$ and $K$ are subgroups of $G$, then $\langle H
  \union K \rangle$ is the smallest subgroup which contains both $H$ and $K$.
\end{corollary}

This subgroup is usually quite different from the union of the two sets. 
Indeed, we have the following:

\begin{proposition}
  Let $H$ and $K$ be subgroups of $G$.  Then $H \union K$ is a subgroup if and only
  if either $H \subseteq K$ or $K \subseteq H$.
\end{proposition}
\begin{proof}
  If $H \subseteq K$, then $H \union K = K$, so $H \union K$ is a subgroup.
  Similarly, if $K \subseteq H$, then $H \union K = H$, so $H \union K$ is
  a subgroup.
  
  Conversely, if neither $H$ nor $K$ is a subset of the other, then there is
  some $x \in H \setminus K$ and $y \in K \setminus H$.  Also $x^{-1} \in H$,
  since $H$ is a subgroup. But then if $xy \in H$, we have $x^{-1}(xy) = y
  \in H$, but since $y \in K \setminus H$, this means that $y \notin H$, which
  is a contradicition.  Therefore $xy \notin H$.  But a similar argument
  shows that $xy \notin K$.  So $xy \notin H \union K$.  So $H \union K$
  is not a subgroup of $G$.
\end{proof}

The following theorem shows us that there is also a subgroup which is contained
in both $H$ and $K$.

\begin{theorem}
  Let $(G, \ast, e)$ be a group, and $H$ and $K$ subgroups of $G$.  Then
  $H \intersect K$ is the largest subgroup of $G$ which is contained in both
  $H$ and $K$.
\end{theorem}
\begin{proof}
  We first need to show that $H \intersect K$ is a subgroup.  If $x$,
  $y \in H \intersect K$, then $xy^{-1} \in H$, since $H$ is a subgroup,
  and $xy^{-1} \in K$, since $K$ is a subgroup.  Therefore $xy^{-1} \in
  H \intersect K$, and so by Corollary~\ref{cor:subgrouptest},
  $H \intersect K$ is a subgroup of $G$.
  
  Since $H \intersect K$ is the largest set contained in both $H$ and $K$,
  it must also be the largest subgroup contained in both.
\end{proof}

We will denote $\langle H \union K \rangle$ by $H \vee K$ and call it the
\defn{join}{join!of subgroups} of $H$ and $K$.  We will denote $H \intersect K$
by $H \wedge K$, and call it the \defn{meet}{meet!of subgroups} of $H$ and $K$.
The reason for this terminology will be come clear when we look at abstract
lattices.

The significance of the lattice of subgroups is that if two groups do not have
similar lattices of subgroups, they cannot be isomorphic, so it provides a
nice pictorial way of demonstrating that two groups are distinct.  To show
this, we first need to show that homomorphisms preserve the relationship of
inclusion of subgroups.

\begin{proposition}\label{prop:homsubgrouporderpreserving}
  If $G$ and $H$ are groups, $\alpha : G \to H$ is a homomorphism, and $K_{1}$
  and $K_{2}$ are subgroups of $G$ with $K_{1} \subseteq K_{2}$, then
  $\alpha(K_{1})$ is a subgroup of $\alpha(K_{2})$.
  
  Furthermore, if $\alpha$ is a monomorphism, and $K_{1} \subset K_{2}$,
  then $\alpha(K_{1})$ not equal to $\alpha(K_{2})$.
\end{proposition}
\begin{proof}
  We know from Proposition~\ref{prop:homsubgroup} that $\alpha(K_{1})$ and
  $\alpha(K_{2})$ are subgroups of $H$, and it is immediate from the
  definition of the image of a set under a function that $\alpha(K_{1}) 
  \subseteq \alpha(K_{2})$ if $K_{1} \subseteq K_{2}$.  So $\alpha(K_{1}) 
  \le \alpha(K_{2})$.
  
  If $\alpha$ is a monomorphism in addition, then since there is some $g \in
  K_{2}$, but not in $K_{1}$, we cannot have $\alpha(h) = \alpha(g)$ for any
  $h \in K_{1}$ (otherwise $\alpha$ would not be one-to-one).  Hence
  $\alpha(K_{1})$ is properly contained in $\alpha(K_{2})$.
\end{proof}

\begin{corollary}\label{cor:hommeetandjoin}
  If $G$ and $H$ are groups, $\alpha : G \to H$ is a homomorphism, and $K_{1}$
  and $K_{2}$ are subgroups of $G$, then $\alpha(K_{1} \vee K_{2}) =
  \alpha(K_{1}) \vee \alpha(K_{2})$ and $\alpha(K_{1} \wedge K_{2}) =
  \alpha(K_{1}) \wedge \alpha(K_{2})$.
\end{corollary}
\begin{proof}
  We know that $\alpha(K_{1}) \vee \alpha(K_{2})$ is the smallest
  subgroup which contains both $\alpha(K_{1})$ and $\alpha(K_{2})$, but
  since $K_{1}$ and $K_{2} \subseteq K_{1} \vee K_{2}$, we have that
  $\alpha(K_{1})$ and $\alpha(K_{2}) \subseteq \alpha(K_{1} \vee K_{2})$,
  hence $\alpha(K_{1}) \vee \alpha(K_{2}) \subseteq \alpha(K_{1} \vee
  K_{2})$.

  Conversely, if $\alpha^{-1}(\alpha(K_{1}) \vee \alpha(K_{2}))$ is a
  subgroup of $G$ which contains both $K_{1}$ and $K_{2}$, so $K_{1} \vee
  K_{2} \subseteq \alpha^{-1}(\alpha(K_{1}) \vee \alpha(K_{2}))$, and hence
  \[
    \alpha(K_{1} \vee K_{2}) \subseteq \alpha(\alpha^{-1}(\alpha(K_{1}) \vee
    \alpha(K_{2}))) = \alpha(K_{1}) \vee \alpha(K_{2}).
  \]
  
  Hence $\alpha(K_{1}) \vee \alpha(K_{2}) = \alpha(K_{1} \vee K_{2})$.
  
  The proof of the case for $\wedge$ is left as an exercise.
\end{proof}

We will say that two groups $G$ and $H$ have corresponding, or
isomorphic, subgroup lattices if there is a bijection $f$ from
$\Sub(G)$ to $\Sub(H)$ so that $f(K_{1}) \vee f(K_{2}) = f(K_{1} \vee
K_{2})$, and $f(K_{1}) \wedge f(K_{2}) = f(K_{1} \wedge K_{2})$.

\begin{corollary}\label{cor:latticeisomtest}
  If $G$ and $H$ are two groups whose subgroup lattices do not
  correspond, then $G$ and $H$ are not isomorphic.
\end{corollary}
\begin{proof}
  If $G$ and $H$ are isomorphic, then
  Proposition~\ref{prop:subgroupisom} tells us that
  $\overline{\alpha}$ is a bijection from $\Sub(G)$ to $\Sub(H)$, and
  Corollary~\ref{cor:hommeetandjoin} says that
  $\overline{\alpha}(K_{1} \vee K_{2}) = \overline{\alpha}(K_{1}) \vee
  \overline{\alpha}(K_{2})$ and $\overline{\alpha}(K_{1} \wedge K_{2})
  = \overline{\alpha}(K_{1}) \wedge \overline{\alpha}(K_{2})$.  So 
  isomorphic groups have corresponding subgroup lattices.
  
  The contrapositive of this result is the corollary.
\end{proof}

\begin{example}
  The groups $C_{4}$ and $V$ have different subgroup lattices, so they 
  are not isomorphic.
\end{example}

Note that the converse of the corollary is not true.  We know, for example,
that if $p$ is prime, the groups $C_{p}$ all have corresponding subgroup
lattices, yet the groups are clearly not isomorphic.

\subsection*{Exercises}

\begin{exercises}
  \item Find the subgroup lattice of the group $D_{6}$.
  
  \item Find the subgroup lattice of the group $C_{8}$.
      
  \item Find the subgroup lattice of the group $C_{2} \cross C_{4}$.
      
  \item Find the subgroup lattice of the group  $C_{2} \cross C_{2} \cross
    C_{2}$.
  
  \item Find the subgroup lattice of the group $D_{10}$.
    
  \item Find the subgroup lattice of the group $D_{12}$.
      
  \item Find the subgroup lattice of the group $A_{4}$
  
  \item Complete the proof of Corollary~\ref{cor:hommeetandjoin}.
\end{exercises}

\section{Extension: Lattices}

The pattern that subgroups of a group make under inclusion is a particular
example of a general phenomenon.  The key idea is that we know when one
subgroup is ``larger'' than another, that we can find the largest thing
smaller than two subgroups (the meet) and that we can find the smallest thing 
larger than the two subgroups (the join).

This idea is essentially the same as what happens with general subsets of a set.
We know when one subset is ``larger'' than another, we can find the
largest thing smaller than two subsets (the intersection) and we can find the
smallest thing larger than the two subsets (the union).

To explore this similarity further, we need to introduce a general concept that
we can use to model the idea of one thing being larger then another.

\begin{definition}
  Let $X$ and $Y$ be sets.  A \defn{relation}{relation} between $X$ and $Y$
  is a subset $R$ of $X \cross Y$, where $x \in X$ and $y \in Y$ are
  considered to be related by $R$ if and only if $(x,y) \in R$.
  
  We write $xRy$ if $(x,y) \in R$.  If $X = Y$ we say that $R$ is a relation
  on $X$.
\end{definition}

Note that you should not confuse this definition of relation with the notion
of a relation on the elements of a group.

The concept of a relation is extremely general, and can be used to model a
great many fundamental mathematical concepts.

\begin{example}
  If $X$ is any set, equality can be regarded as the relation $R = \{(x,x) : x
  \in X\} \subseteq X \cross X$.  Here $xRy$ if and only if $(x,y) \in R$ if and
  only if $x = y$.
\end{example}

\begin{example}
  If $X$ and $Y$ are any sets and $f : X \to Y$ is a function, the graph
  $R = \{(x, f(x)): x \in X\} \subset X \cross Y$ is a relation where $xRy$ if and only if
  $y = f(x)$.  In fact functions are sometimes defined in this way as a
  special case of the concept of a relation.
\end{example}

\begin{example}
  In the real numbers, the set $L = \{ (x,y) : x \le y\} \in \reals \cross \reals$
  is a relation where $xLy$ if and only if $x \le y$.
\end{example}

\begin{example}
  If $\powerset(X)$ is the power set of some set $X$, then the set
  $\subseteq = \{(A,B): \text{$A$ is a subset of $B$}\}$ is a relation
  where $A \subseteq B$ if and only if $A$ is a subset of $B$.
\end{example}

\begin{example}
  If $G$ is a group, and $X$ is the set of subgroups of $G$, then the set
  $\le = \{(A,B): \text{$A$ is a subgroup of $B$}\}$ is a relation where
  $A \le B$ if and only $A$ is a subgroup of $B$.
\end{example}

Because of the generality of relations, we need to impose some additional
conditions to make them useful in modelling particular situations.

\begin{definition}
  Let $X$ and $Y$ be sets, and $R$ a relation on $X$.  Let $x, y$ and $z \in X$.
  
  We say that $R$ is \defn{reflexive}{relation!reflexive} if $xRx$ for all
  $x \in X$.
  
  We say that $R$ is \defn{symmetric}{relation! symmetric} if $xRy$ implies
  $yRx$.
  
  We say that $R$ is \defn{antisymmetric}{relation! antisymmetric} if $xRy$ and
  $yRx$ implies $x = y$.
  
  We say that $R$ is \defn{transitive}{relation!transitive} if $xRy$ and
  $yRz$ implies $xRz$.
  
  A \defn{partial order}{order!partial} on $X$ is a relation on $X$ which is
  reflexive, transitive and antisymmetric.
\end{definition}

\begin{example}
  The $\le$, $<$, $>$ and $\ge$ relations on any set of numbers. All are
  antisymmetric and transitive, but not symmetric.  The relations $\le$
  and $\ge$ are reflexive, but $<$ and $>$ are not.
\end{example}

\begin{example}
  The relation $\subseteq$ of $\powerset(X)$ is a partial order.
\end{example}

\begin{example}
  The relation $\le$ on the set of subgroups of a group is a partial order.
\end{example}

Notice that if $\preceq$ is a partial order on a
set $X$, and $x$, $y \in X$, then it may be the case that neither $x \preceq y$
nor $y \preceq x$.  If this is the case, we say that $x$ and $y$ are
\defn{incomparable elements}{incomparable elements} of $X$.

\begin{example}
  The sets $\{1\}$ and $\{2,3\}$ are incomparable elements of
  $\powerset(\{1,2,3\})$ under the subset partial order $\subseteq$.
\end{example}

So a partial order seem to encapsulate the general idea of something being
bigger than something else.  Now we need to model the idea of a meet and a
join.

\begin{definition}
  If $X$ is a set, $\preceq$ is a partial order on $X$, and $x$ and $y \in X$,
  then $a \in X$ is a \defn{lower bound}{bound!lower} for $x$ and $y$ if $a \preceq x$ and
  $a \preceq y$.  We say that $a$ is the \defn{greatest lower
  bound}{bound!greatest lower} of $x$ and $y$ if given any lower bound
  $z$ of $x$ and $y$, we have that $z \preceq a$.
  
  Similarly, $a$ is an \defn{upper bound}{bound!upper} for $x$ and $y$ if
  $x \preceq a$ and $y \preceq a$.  We say that $a$ is the \defn{least upper
  bound}{bound!least upper} of $x$ and $y$ if given any upper bound
  $z$ of $x$ and $y$, we have that $a \preceq z$.
  
  In general, we denote the greatest lower bound of $x$ and $y$ by $x \wedge y$,
  and the least upper bound of $x$ and $y$ by $x \vee y$.
  
  A \defn{lattice}{lattice} $(X, \preceq)$ is a set with a partial order such that
  every pair of elements has a greatest lower bound and a least upper bound.
\end{definition}

The examples of partial orders earlier in this section all have the lattice
property.

\begin{example}
  The pair $(\reals, \le)$ is a lattice.  Since for any $x$ and $y$, we have
  $x \le y$ or $y \le x$ (or both), then $x \vee y = \min x,y$ and 
  $x \wedge y = \max x,y$.
\end{example}

\begin{example}
  If $X$ is any set, then the pair $(\powerset(X), \subseteq)$ is a
  lattice.  Given any two subsets $A$ and $B$ of $X$, we have $A
  \wedge B = A \union B$ and $A \vee B = A \intersect B$.
\end{example}

\begin{example}
  The subgroup relation $\le$ on the set of subgroups $\Sub(G)$ of a
  group $G$ makes $(\Sub(G), \le)$ a lattice, since if $H$ and $K \in
  \Sub(G)$ we have $H \vee K = \langle H \union K \rangle$ and $H
  \wedge K = H \intersect K$.
\end{example}

\begin{example}\label{eg:abstractlattice}
  Let $X = \{0, x, y, 1\}$, and let $\preceq$ be the partial order
  defined by $0 \preceq x$, $0 \preceq y$, $0 \preceq x$, $0 \preceq
  1$, $x \preceq 1$, $y \preceq 1$ and $y \preceq 1$.  Then $(X,
  \preceq)$ is a lattice, and we can represent it diagramatically as

  \begin{picture}(6,6)(-3,-1)
    \put(0,0){\makebox(0,0){$0$}}
    \put(0,0.5){\line(0,1){1}}
    \put(0.5,0.5){\line(1,1){1}}
    \put(-0.5,0.5){\line(-1,1){1}}
    \put(-2,2){\makebox(0,0){$x$}}
    \put(0,2){\makebox(0,0){$y$}}
    \put(2,2){\makebox(0,0){$z$}}
    \put(0,2.5){\line(0,1){1}}
    \put(1.5,2.5){\line(-1,1){1}}
    \put(-1.5,2.5){\line(1,1){1}}
    \put(0,4){\makebox(0,0){$1$}}
  \end{picture}
\end{example}


In the previous section, we showed that if the subgroup lattice of two 
groups didn't agree, then the groups could not be isomorphic.  The 
heart of the result was showing that the isomorphism $\alpha$ between the 
groups produced a function between the subgroups $\overline{\alpha}$ 
which preserved meet and join.

\begin{definition}
  Let $(X, \preceq)$ and $(Y, \le)$ be two partially ordered sets.  A
  function $\alpha : X \to Y$ is an order-preserving function if whenever
  we have $x$ and $y \in X$ such that $x \preceq y$, we have $\alpha(x) 
  \le \alpha(y)$.
  
  If $(X, \preceq)$ and $(Y, \le)$ are lattices, then $\alpha: X \to
  Y$ is a lattice homomorphism if $\alpha(x \vee y) = \alpha(x) \vee
  \alpha(y)$, and $\alpha(x \wedge y) = \alpha(x) \wedge \alpha(y)$.
  If $\alpha$ is a bijective lattice homomorphism, we say that it is 
  a lattice isomorphism.
\end{definition}

\begin{example}
  Let $(X, \preceq)$ be as in Example~\ref{eg:abstractlattice}, and 
  $V$ be the four-group.
  The function $\alpha : X \to \Sub(V)$ defined by the table
  \[
    \begin{array}{cc}
      t & \alpha(t) \\
      \hline
      0 & \{e\} \\
      x & \{e, a\} \\
      y & \{e, b\} \\
      z & \{e, ab\} \\
      1 & V 
    \end{array}
  \]
  is a lattice isomorphism.  Perhaps the easiest way to grasp this 
  fact is to observe that the diagrams for each lattice correspond.
\end{example}


\begin{example}
  Consider the homomorphism $\alpha : V \to C_{4}$ of 
  Example~\ref{eg:4grouphom}.  The corresponding map between 
  subgroups of these groups is $\overline{\alpha}$, and is given by 
  the following table
  \[
    \begin{array}{cc}
      H & \overline{\alpha}(H) \\
      \hline
      \{e\} & \{e\} \\
      \{e, a\} & \{e, a^{2}\} \\
      \{e, b\} & \{e, a^{2}\} \\
      \{e, ab\} & \{e\} \\
      V & \{e, a^{2}\}
    \end{array}
  \]
  is a lattice homomorphism.
\end{example}

The essential content of
Proposition~\ref{prop:homsubgrouporderpreserving} and
Corollary~\ref{cor:hommeetandjoin} is then:

\begin{proposition}
  If $G$ and $H$ are groups, and $\alpha: G \to H$ is a homomorphism,
  then $\overline{\alpha} : \Sub(G) \to \Sub(H)$ is an
  order-preserving map.
\end{proposition}

\begin{corollary}\label{cor:grouphomtolathom}
  If $G$ and $H$ are groups, and $\alpha: G \to H$ is a homomorphism, 
  then $\overline{\alpha} : \Sub(G) \to \Sub(H)$ is a lattice 
  homomorphism.
\end{corollary}

So we can then re-phrase Corollary~\ref{cor:latticeisomtest} in the 
following way:

\begin{corollary}
  If $G$ and $H$ are groups, and the subgroup lattices of $G$ and $H$
  are not isomorphic, then $G$ and $H$ are not isomoprhic.
\end{corollary}

We can use Exercise~\ref{ex:orderfrommeetandjoin} to prove the 
following proposition.

\begin{proposition}\label{prop:latticehomisorderpreserve}
  Let $(X, \preceq)$ and $(Y, \le)$ be lattices, and $\alpha : X \to 
  Y$ a lattice homomorphism.  Then $\alpha$ is an order-preserving 
  function.
\end{proposition}
\begin{proof}
  See Exercise~\ref{ex:latticehomisorderpreserve}.
\end{proof}

In fact, we could prove Corollary~\ref{cor:hommeetandjoin} directly, 
and use this proposition to conclude that 
Proposition~\ref{prop:homsubgrouporderpreserving} must hold.

So the abstract concept of a lattice helps us understand the structure 
of subgroups within a group, just as the abstract structure of a group 
helps us understand concrete situations such as the symmetries of a 
set.

\subsection*{Exercises}

\begin{exercises}
  \item\label{ex:orderfrommeetandjoin} Let $\preceq$ be a partial
    order on $X$.  Show that if $x \preceq y$ that $x \wedge y = x$ and
    $x \vee y = y$.
  
  \item\label{ex:latticehomisorderpreserve} Prove
    Proposition~\ref{prop:latticehomisorderpreserve}.
  
  \item Consider the natural numbers $\naturals$ with the ``divides''
    relation $x \mid y$ if and only if $y = kx$ for some $k \in \naturals$
    (ie.\ if $x$ divides $y$).  Show that $\mid$ is a partial order, and that
    $x \wedge y$ is the greatest common divisor of $x$ and $y$ and $x \vee y$
    is the least common multiple of $x$ and $y$.
  
  \item Show that if $\preceq$ is a partial order on $X$, then the reverse
    relation $\succeq$ defined by $x \succeq y$ if and only if $y \preceq x$ is
    a partial order.  Show that $(X, \preceq)$ is a lattice if and only if
    $(X, \succeq)$ is a lattice.
    
  \item Let $F(D, \reals)$ be the set of real-valued functions on some fixed
    domain $D \subseteq \reals$ (ie.~the typical functions considered in calculus).
    Show that the relation defined by $f \le g$ if $f(x) \le g(x)$ for all
    $x \in D$ is a partial order.
    
    Give an example of two functions which are incomparable.
    
    Show that $(F(D, \reals), \le)$ is a lattice, where $f \wedge g$ and
    $f \vee g$ are the functions defined by
    \[
      (f \wedge g)(x) = \min f(x), g(x) \qquad \text{and} \qquad
        (f \vee g)(x) = \max f(x), g(x)
    \]
    respectively.
\end{exercises}

\section{The Centre and Centralizers}

Abelian groups are much nicer to work with algebraically than non-Abelian
groups.  However, even in the case of non-Abelian groups there may be large
parts of the group which commute with each other.

Recall that two elements $x$ and $y \in G$ commute with one another if
\[
  xy = yx.
\]
Multiplying both sides on the right by $x^{-1}$, we can state this
equivalently as saying that $x$ and $y$ commute if and only if
\[
  xyx^{-1} = y,
\]
or, multiplying on the other side, if and only if
\[
  x^{-1}yx = y.
\]

The set of all elements which commute with every other element of the 
group is called the \defn{centre}{centre} of the group, 
denoted by $Z(G)$.  At the very least we have the identity $e \in 
Z(G)$, but it is potentially much larger.

\begin{example}
  The centre of $D_{6}$ is $\{1\}$.  Clearly $1$ always commutes with any
  element of $D_{6}$.  For the other elements we can always find an element
  with which it does not commute.  For example, $a$ does not commute with
  $b$, since $ba = (a^{2})b \ne ab$.  The same equation shows that $a^{2}$
  does not commute with $b$.  So $a$, $a^{2}$ and $b$ are not in the centre.
  Similarly $(ab)a = a^{3}b = a^{2}(ab)$, so $ab$ does not commute with $a$,
  and $(a^{2}b)a = a^{4}b = a^{2}(a^{2}b)$, so $a^{2}b$ does not commute
  with $a$.
\end{example}

\begin{example}\label{eg:D8centre}
  The centre of $D_{8}$ is $\{1, a^{2}\}$.  This follows since $a^{n}a^{2} =
  a^{n+2} = a^{2}a^{n}$ and
  \[
    (a^{n}b)a^{2} = a^{n}a^{3}ba = a^{n+3}a^{3}b = a^{n+6}b = a^{n+2}b =
    a^{2}(a^{n}b).
  \]
   for $n = 0, 1, 2, 3$.  So $a^{2} \in Z(D_{8})$.
   
   On the other hand $(a^{n}b)a = a^{n}a^{3}b = a^{3}(a^{n}b)$ for $n = 0, 1,
   2, 3$, so $a$, $a^{3}$, and $a^{n}b$ are not in the centre.
\end{example}

\begin{example}
  The centre of $GL_{n}(\reals)$ is the set
  \[
   \reals^{\times} I_{n} = \left\{\begin{bmatrix}
     a & 0 & \cdots 0 \\
     0 & a & \cdots 0 \\
     \vdots & \vdots & \ddots & \vdots \\
     0 & 0 & \cdots a \\
   \end{bmatrix} : a \in \reals^{\times} \right\}.
  \]
  It is easy to verify with matrix multiplication that every element of this
  set commutes with every matrix.
  
  To simplify calculations, we will just look at the case $n = 2$.  To see that
  these are the only possible elements of the centre, we note that if
  \[
    \begin{bmatrix}
      a & b \\
      c & d
    \end{bmatrix} \in Z(GL_{2}(\reals))
  \]
  then we must have
  \[
    \begin{bmatrix}
      a & b \\
      c & d
    \end{bmatrix}
    \begin{bmatrix}
      0 & 1 \\
      1 & 0
    \end{bmatrix}
    =
    \begin{bmatrix}
      0 & 1 \\
      1 & 0
    \end{bmatrix}
    \begin{bmatrix}
      a & b \\
      c & d
    \end{bmatrix},
  \]
  or,
  \[
    \begin{bmatrix}
      b & a \\
      d & c
    \end{bmatrix}
    =
    \begin{bmatrix}
      c & d \\
      a & b
    \end{bmatrix}
  \]
  and so we conclude that $a = d$ and $b = c$.  We also must have
  \[
    \begin{bmatrix}
      a & b \\
      b & a
    \end{bmatrix}
    \begin{bmatrix}
      0 & -1 \\
      1 & 0
    \end{bmatrix}
    =
    \begin{bmatrix}
      0 & -1 \\
      1 & 0
    \end{bmatrix}
    \begin{bmatrix}
      a & b \\
      b & a
    \end{bmatrix},
  \]
  or,
  \[
    \begin{bmatrix}
      b & -a \\
      a & -b
    \end{bmatrix}
    =
    \begin{bmatrix}
      -b & -a \\
      a & b
    \end{bmatrix}
  \]
  so $b = -b$.  Hence $b = 0$, and so a matrix is in the centre only if it
  is in the set $\reals^{\times}I_{2}$.
\end{example}

The centre is an extremely nice subset of the group.

\begin{proposition}\label{prop:centresubgroup}
  Let $G$ be a group.  Then $Z(G)$ is an Abelian subgroup of $G$.
\end{proposition}
\begin{proof}
  Given $x$, $y \in Z(G)$, we observe that for any $z \in G$ we have
  \[
    (xy)z = xzy = z(xy),
  \]
  so $xy \in Z(G)$.  Similarly,
  \[
    x^{-1}z = x^{-1}zxx^{-1} = zx^{-1},
  \]
  so $x^{-1} \in Z(G)$.  Hence $Z(G)$ is a subgroup of $G$.
  
  It is immediate that $Z(G)$ is also Abelian, since every element of $Z(G)$ 
  commutes with every element of $G$, and $Z(G)$ is a subgroup.
\end{proof}

You can think of the centre as being a measure of how close the group $G$ is
to being Abelian.  If $Z(G) = G$, then the group is Abelian, while if $Z(G)$
is a large subgroup, then $G$ can be thought of having a large Abelian
component.  On the other hand, if $Z(G) = \{e\}$, the group is very far from
being Abelian.

\begin{proposition}
  Let $G$ and $H$ be isomorphic groups.  Then $Z(G)$ and $Z(H)$ are
  isomorphic groups.
\end{proposition}
\begin{proof}
  We know that there is an isomorphism $\alpha : G \to H$.  If $x \in Z(G)$,
  then for any $y \in H$, we have that there is some unique $u \in G$ such
  that $\alpha(u) = y$, and
  \[
    \alpha(x)y = \alpha(x)\alpha(u) = \alpha(xu) = \alpha(ux) =
    \alpha(u)\alpha(x) = y\alpha(x).
  \]
  So $\alpha(x) \in Z(H)$.  An identical argument shows that if $x \in
  Z(H)$, then $\alpha^{-1}(x) \in Z(G)$.
  
  Hence $\alpha(Z(G)) = Z(H)$, and the restriction of $\alpha$ to $Z(G)$ is
  an isomorphism between the centres.
\end{proof}

\begin{corollary}
  If $G$ and $H$ have centres which are not isomorphic, then $G$ and $H$ are
  not isomorphic.
\end{corollary}

More generally, rather than asking which elements commute with the entire
group we might ask which elements commute with some subset of the group.
If $X \subset G$ is any subset of the group $G$,
then the \defn{centralizer}{centralizer} of $X$ is the set of all elements 
which commute with every element of $X$, ie.
\[
  Z_{G}(X) = \{y \in G : yx = xy\} = \{y \in G : y^{-1}xy = x\} =
  \{y \in G : yxy^{-1} = x\}.
\]
This set always contains at least the identity element $e$.

\begin{proposition}
  Let $G$ be a group, and $X \subset G$.  Then $Z_{G}(X)$ is a
  subgroup of $G$.  If $Y \subseteq X$, then $Z_{G}(X)$ is a subgroup of
  $Z_{G}(Y)$.
\end{proposition}
\begin{proof}
  The proof of the first part of this proposition is essentially the same
  as the proof of the first part of Proposition~\ref{prop:centresubgroup}, but
  with $z \in X$ rather than $z \in G$.
  
  The second part follows from the fact that if $zx = xz$ for every $x \in
  X$, then it must also hold for every element of $Y$, since $Y \subseteq
  X$.  Therefore $Z_{G}(X) \subseteq Z_{G}(Y)$.
\end{proof}

In particular, this proposition implies that $Z(G) = Z_{G}(G) \subseteq
Z_{G}(X)$ for all $X$.

The case where $X$ contains a single element $g$ is important enough to have
its own, slightly different notation.  We define
\[
  Z_{G}(g) = \{y \in G : yg = gy\} = \{y \in G : y^{-1}gy = g\} =
  \{y \in G : ygy^{-1} = g\}.
\]

\begin{example}
  The consider the element $a$ of $D_{6}$.  Then $Z_{D_{6}}(a) = \{e, 
  a, a^{2}\}$.  The elements $b$, $ab$ and $a^{2}b$ are not elements 
  of $Z_{D_{6}}(a)$.  Similarly, we have:
  \begin{align*}
    Z_{D_{6}}(e) &= D_{6},\\
    Z_{D_{6}}(a^{2}) &= \{e, a, a^{2}\},\\
    Z_{D_{6}}(b) &= \{e, b\},\\
    Z_{D_{6}}(ab) &= \{e, ab\},\\
    Z_{D_{6}}(a^{2}b) &= \{e, a^{2}b\},\\
  \end{align*}
\end{example}

Once again, these subsets have very nice properties.

\begin{proposition}
  Let $G$ be a group and $g \in G$.  Then $\langle g \rangle$ is a subgroup of
  $G$. Furthermore, $Z_{G}(g) = G$ if and only if $g \in Z(G)$
\end{proposition}
\begin{proof}
  We note that $g \in Z_{G}(g)$, and since $\langle g \rangle$ is the
  smallest subgroup of $G$ containing $g$, we have that $\langle g \rangle$
  must be a subgroup of $Z_{G}(g)$.
  
  If $g \in Z(G)$, $gx = xg$ for all  $x \in G$, so $Z_{G}(g) = G$.
  
  On the other hand, if $Z_{G}(g) = G$, then that implies that for any $x
  \in G$, $xg = gx$, and so $g \in Z(G)$.
\end{proof}

You can think of the centralizer of $g$ as measuring how close $g$ is to
being an element of the centre, or how close it is to commuting with
everything.

The centralizer plays a key role in the discussion of conjugacy later
in this chapter.  We finish up with one last result which links the 
centre and centralizers of elements.

\begin{proposition}
  Let $G$ be a group.  Then $Z(G)$ is the intersection of all the 
  subgroups $Z_{G}(g)$.
\end{proposition}
\begin{proof}
  We know that $Z(G) \subseteq Z_{G}(g)$ for every $g$, so
  \[
    Z(G) \subseteq \bigcap_{g \in G} Z_{G}(g).
  \]
  On the other hand, if $x \in Z_{G}(g)$ for every $g \in G$, then 
  $xg = gx$ for every $g \in G$, and so $x \in Z(G)$.  Hence
  \[
    \bigcap_{g \in G} Z_{G}(g) \subseteq Z(G).
  \]
\end{proof}

\subsection*{Exercises}

\begin{exercises}
  \item  Find the centre of the group $C_{2} \cross D_{6}$.
  
  \item  Find the centre of the group $D_{10}$.
  
  \item  Show that $Z(D_{2n}) = \{1\}$ if $n$ is odd, and $Z(D_{2n}) = 
    \{1, a^{n/2}\}$ if $n$ is even.
  
  \item  Find the centralizers of each element of $D_{8}$.
  
  \item  Find the centralizers of each element of $C_{2} \cross D_{6}$.
  
  \item  Find the centralizers of each element of $S_{4}$.  What is 
    the center of this group?
  
  \item  Show that $Z_{G}(X) = Z_{G}(\langle X \rangle)$.
\end{exercises}


\section{Cosets}

There are other subsets of groups which it seems should have some
significance.  For example in $S_{3}$, the set of ``reflection
permutations'', ie.  those permutations with odd parity, is $\{ (1,2),
(1,3), (2,3) \}$ and is not a subgroup.  Nevertheless, the elements
have a commonality.  To understand such a situation, we need to
introduce some new notation.

If $\ast: A \cross B \to C$ is any binary relation, then given any 
$x \in X$ and $Y \subseteq B$, we define
\[
  x \ast Y = \{ x \ast y: y \in Y\} \subseteq C.
\]
Similarly, if $y \in B$ and $X \subseteq A$, we define
\[
  X \ast y = \{ x \ast y: x \in X\} \subseteq C.
\]
And analagously, we define
\[
  X \ast Y = \{ x \ast y: x \in X, y \in Y\} \subseteq C.
\]
We will often omit the operation $\ast$ and simply write $xY$, $Xy$ 
and $XY$ respectively.

If $\ast$ is a binary operation on $A$, and $X \subseteq A$, we will 
sometimes write
\[
  X^{n} = \underbrace{X \ast X \ast \cdots \ast X}_{\text{$n$ times}}.
\]
This is far from ideal notation, since it conflicts with the 
Cartesian product
\[
  X^{n} = \underbrace{X \cross X \cross \cdots \cross X}_{\text{$n$ times}}.
\]
However, it is usually clear from context which of the two 
possibilities we mean.

If $(G, \ast, e)$ is a group, and $X \subseteq G$ then we can also write
\[
  X^{-1} = \{x^{-1} : x \in X\}.
\]

Using this notation, we can write Theorem~\ref{thm:subgrouptest} and
Corollary~\ref{cor:subgrouptest} as follows:

\begin{corollary}
  Let $G = (G, \ast e)$ be a group, and $H \subseteq G$. If $H^{2} 
  \subseteq H$ and $H^{-1} \subseteq H$, then $H \le G$.
\end{corollary}

\begin{corollary}
  Let $G = (G, \ast e)$ be a group, and $H \subseteq G$. If $HH^{-1} 
  \subseteq H$, then $H \le G$.
\end{corollary}

In fact, if $H$ is a finite subset of $G$, we can use an even weaker 
condition:

\begin{proposition}\label{prop:subgrouptest3}
  Let $G = (G, \ast e)$ be a group, and $H$ be a finite subset $G$. If $H^{2} 
  = H$, then $H \le G$.
\end{proposition}

To prove this fact, we need a lemma which we will use often in what 
follows:

\begin{lemma}\label{lemma:translation}
  Let $G = (G, \ast e)$ be a group, $H \subseteq G$, and $g \in G$. 
  Then the \defn{right translation}{translation!right} by $g$ function
  $\rho_{g} : H \to Hg$ defined by $\rho_{g}(x) = xg$ is a bijection.
  
  Similarly, the \defn{left translation}{translation!left} by $g$
  function $\lambda_{g} : H \to gH$ defined by $\lambda_{g}(x) = gx$
  is a bijection.
  
  We also have $|H| = |gH| = |Hg|$.
\end{lemma}

\begin{proof}
  That $\rho_{g}$ is onto is trivial, for
  \[
    \rho_{g}(H) = \{ \rho_{g}(x) : x \in H \} = \{ xg : x \in H \} = 
    Hg.
  \]
  
  On the other hand, if $\rho_{g}(x_{1}) = \rho_{g}(x_{2})$, then 
  this means that $x_{1}g = x_{2}g$, and the cancellation law 
  (Proposition~\ref{prop:cancellation}) says that $x_{1} = x_{2}$.  
  Hence $\rho_{g}$ is one-to-one, and so $\rho_{g}$ is a bijection.
  
  The result for $\lambda_{g}$ is similar.
  
  Since we have bijections between $H$ and $Hg$, and $H$ and $gH$, all 
  three sets must have the same cardinality 
  (Proposition~\ref{prop:functionfacts}).
\end{proof}

With this in hand, the proof is simple.

\begin{proof}[Proposition~\ref{prop:subgrouptest3}]
  The hypothesis $H^{2} = H$ implies that $xy \in H$ for all $x$, $y 
  \in H$, so we need only prove that $x^{-1} \in H$.
  
  Let $|H| = n$.  Lemma~\ref{lemma:translation} tells us
  $|Hx| = |H| = n$.  Since $H$ is finite this, together with the fact that $Hx \subseteq
  H^{2} = H$, implies that $Hx = H$.
  
  Hence there must be some element $y \in H$ such that $yx =
  x$.  By the cancellation law, we have $y = e$, so $x \in H$.
  
  But now there must also be some $z \in H$ such that $zx = 
  e$, so $z = x^{-1}$.
  
  Hence $H$ is a subgroup of $G$.
\end{proof}

Getting back to the example at the start of this section, if $H = 
\langle (1,2,3) \rangle = \{e, (1,2,3), (1,3,2)\}$, then we can write
\[
  \{ (1,2), (1,3), (2,3) \} = (1,2)H.
\]
In fact, we also have
\[
  \{ (1,2), (1,3), (2,3) \} = (1,3)H = (2,3)H = H(1,2) = H(1,3) = H(2,3).
\]

This situation is important enough to give it a name.

\begin{definition}\label{defn:coset}
  Let $G$ be a group, and $H$ a subgroup of $G$.  Sets of the form
  $xH$ are called \defn{left cosets}{coset!left} of $H$ and sets of
  the form $Hx$ are called \defn{right cosets}{coset!right} of $H$,
  where $x$ is any element of $G$.
  
  If $G$ is Abelian, then $Hx = xH$, and we simply call the set a 
  \defn{coset}{coset} of $H$.
\end{definition}

So $\{ (1,2), (1,3), (2,3) \}$ is both a left and right coset of
$\{e, (1,2,3), (1,3,2)\}$.

\begin{example}
  Consider $C_{6} = \{ 1, a, a^{2}, a^{3}, a^{4}, a^{5}\}$.  We know
  that $H = \langle a^{3} \rangle$ is a subgroup, and its cosets are
  itself $H = a^{3}H$, $aH = a^{4}H = \{a, a^{4}\}$, and $a^{2}H =
  a^{5}H = \{a^{2}, a^{5}\}$.
  
  Because cyclic groups are Abelian, the left and right cosets are 
  the same.
\end{example}

Notice in the examples so far that several different choices for $x$ 
give the same coset.  This is typical.

\begin{theorem}\label{thm:cosetsequal}
  Let $G$ be a group, $H$ a subgroup of $G$, and $x$ and $y \in H$.  Then 
  $xH = yH$ if and only if $x^{-1}y \in H$.  On the other hand, $Hx = 
  Hy$ if and only if $xy^{-1} \in H$.
  
  Furthermore, either $xH = yH$ or $xH \intersect yH = \emptyset$.  
  Similarly, either $Hx = Hy$ or $Hx \intersect Hy = \emptyset$.
\end{theorem}
\begin{proof}
  A typical element of $yH$ is of the form $yz$, for some $z \in H$. 
  We note that $x^{-1}yz \in H$, as well, and so multiplying on the
  left by $x$ we have $x(x^{-1}yz) = yz \in xH$.  So $yH \subseteq
  xH$.  Similarly, since $(x^{-1}y)^{-1} = y^{-1}x \in H$, given a
  typical element $xz \in xH$ we have $y^{-1}xz \in H$, and so 
  $y(y^{-1}xz) = yz \in yH$.  Hence $xH \subseteq yH$, and we 
  conclude that $xH = yH$.
  
  Conversely, if $x^{-1}y \notin H$, we have that $y \notin xH$, since 
  if that were the case $y = xz$ for some $z \in H$, but $y = 
  xx^{-1}y$, and the cancellation law implies that then $z = 
  x^{-1}y$, so $z \notin H$, which is a contradiction.  However, $y = 
  ye \in yH$, so $xH$ and $yH$ are not equal.
  
  Assume that $xH \ne yH$, so that $xy^{-1} \notin H$.  If there were 
  some $z \in xH \intersect yH$, then we would have $z = xu = yv$ for 
  some $u$ and $v \in H$.  So $uv^{-1} \in H$.  But
  \[
    uv^{-1} = x^{-1}xuv^{-1} = x^{-1}yvv^{-1} = x^{-1}y \notin H.
  \]
  This is a contradiction, so there can be no such element $z$.
  
  Analagous arguments show that $Hx = Hy$ if and only if $xy^{-1}
  \in H$, and either $Hx = Hy$ or $Hx \intersect Hy = \emptyset$.
\end{proof}

Every element of the group must lie in some coset, since $x = xe \in 
xH$, so the cosets of a subgroup $H$ break the group $G$ up into a 
collection of disjoint subsets.  Furthermore, we know that each of 
these subsets has the same cardinality.  This is nice to know for 
infinite groups, but it is really useful when dealing with finite 
groups.

\begin{theorem}[Lagrange]\label{thm:lagrange}
  If $G$ is a finite group, and $H \le G$, then the number of left
  cosets of $H$ and the number of right cosets of $H$ both equal
  $|G|/|H|$.
\end{theorem}
\begin{proof}
  Assume that there are $n$ left cosets, and that $g_{1}H$, $g_{2}H,
  \ldots, g_{n}H$ is a complete list of the distinct left cosets of
  $H$, so $G$ is a disjoint union of these sets.  The
  inclusion-exclusion principle tells us that when we have a disjoint
  union of finite sets, the cardinality of the union is the sum of the
  cardinality of each set, ie.
  \[
    |G| = |g_{1}H| + |g_{2}H| + \cdots + |g_{n}H|.
  \]
  But Lemma~\ref{lemma:translation} tells is that $|g_{1}H| = 
  |g_{2}H| = \cdots = |g_{n}H| = |H|$, so
  \[
    |G| = \underbrace{|H| + |H| + \cdots + |H|}_{\text{$n$ times}} = 
    n|H|.
  \]
  Hence the number of left cosets is $n = |G|/|H|$.
  
  The argument for right cosets is analagous.
\end{proof}

The number of cosets of a subgroup is significant enough to be given 
its own name and notation.

\begin{definition}
  If $H$ is a subgroup of a group $G$, the \defn{index}{index!of a 
  subgroup} $[G : H]$ is the number of left (or right) cosets of $H$.
\end{definition}

The index is sometimes denoted $|G : H|$.  The following corollaries
of Lagrange's theorem are almost trivial, but they are important
enough that we state them explicitly.  We will use these some of these
facts as much, if not more often, than Lagrange's theorem itself.

\begin{corollary}
  If $H$ is a subgroup of a finite group $G$, $[G : H] = |G|/|H|$.
\end{corollary}

\begin{corollary}
  If $H$ is a subgroup of a finite group $G$, then both $[G : H]$ and 
  $|H|$ divide $|G|$.
\end{corollary}
\begin{proof}
  The numbers $[G : H] = |G|/|H|$, and $|H| = |G|/[G:H]$ are both
  natural numbers, so $|H| \divides |G|$ and $[G : H] \divides |G|$.
\end{proof}

\begin{corollary}
  If $G$ is a finite group, and $x \in G$, then the order of the 
  $x$, $o(x)$, divides $|G|$.
\end{corollary}
\begin{proof}
  We know that $o(x) = |\langle x \rangle|$, and we also know that 
  $\langle x \rangle$ is a subgroup.  Hence $|\langle x \rangle|$ 
  divides $|G|$, and so $o(x)$ divides $G$.
\end{proof}

This last corollary has some immediate consequences, both for 
properties of particular elements, and for classifying groups.

\begin{corollary}
  If $G$ is a finite group, and $n = |G|$, then given any $x \in G$,
  $x^{n} = e$.
\end{corollary}
\begin{proof}
  Assume that $o(x) = k$, so that in particular $x^{k} = e$.  By the
  previous corollary, $o(x)$ divides $n$, so we have $n = km$ for 
  some integer $m$.  But then
  \[
    x^{n} = x^{km} = (x^{k})^{m} = e^{m} = e.
  \]
\end{proof}

\subsection*{Exercises}

\begin{exercises}
  \item Identify the cosets of the subgroup $H = \{1, a^{2}\}$ of the group
    $D_{8}$.  What is the index of $H$ in $G$?
  
  \item Verify Lagrange's Theorem for $D_{8}$.
  
  \item Show that if $X \subseteq G$ is not a subgroup, then an element may
    lie in more than one set of the form $aX$.  Show that every element lies
    in at least one such set.
  
  \item Let $C_{n}$ be a cyclic group, and $m$ be a number that divides $n$.
    Show that there is an element of order $m$ in $C_{n}$.
  
  \item Show that if $G$ is a finite group, then the function
    $\alpha: \Sub(G) \to \naturals$ given by $\alpha(G) = |G|$ is an
    order-preserving function from the partial ordered set $(\Sub(G),
    \le)$ to the partially ordered set $(\naturals, \mid )$.
\end{exercises}

\section{Classifying Groups of Small Order}

We will use the corollaries of Lagrange's theorem to show that groups 
of low order fall into a few distinct isomorphism classes.  The aim of 
this section is to show that every group of order less than or equal 
to $8$ is isomorphic to one of a few standard groups.

\begin{corollary}
  Let $G$ be a group with $|G| = p$, a prime number.  Then $G$ has no 
  proper subgroups, and $G$ is cyclic.
\end{corollary}
\begin{proof}
  If $H$ is a subgroup of $G$, then $|H|$ divides $|G| = p$, so $|H|$ 
  must either be $1$ or $p$.  If $|H| = 1$, then $H = \{e\}$, since 
  every subgroup must contain the identity element.  On the other 
  hand, if $|H| = p$, then $H = G$, since $H$ must contain every 
  element of the group.  So $G$ has no proper subgroups.
  
  Now assume that $x \in G$, and $x \ne e$.  Then $H = \langle x
  \rangle$ is a subgroup of $G$, and $H$ contains at least two
  elements $x$ and $e$.  Hence $|H| \ne 1$, and so $|H| = p$, which 
  implies $H = G$.  So $G$ is generated by $x$, and hence is a cyclic 
  group.
\end{proof}

This last corollary means that every group of prime order is 
isomorphic to $C_{p}$.

We can also show that we have found all groups of order $4$.

\begin{proposition}
  Let $G$ be a group such that $|G| = 4$.  Then either $G \isom 
  C_{4}$ or $G \isom V \isom C_{2} \cross C_{2}$.
\end{proposition}
\begin{proof}
  We know that the order of every element of $G$ divides $4$, so the 
  order of an element which is not the idientity must be $2$ or $4$.
  
  If there is an element $x$ with $o(x) = 4$, then $x$ must generate 
  $G$.  Hence $G$ is cyclic, so $G \isom C_{4}$.
  
  If there is no element of order $4$, then we must have $x^{2} = e$ 
  for every $x \in G$.  But then Theorem~\ref{thm:order2group} tells 
  us that $G \isom C_{2} \cross C_{2}$.
\end{proof}

This table summarizes the typical groups of each order that we have 
discovered, up to order 12.

\medskip

  \begin{tabular}{|r|l|}
  \hline
  \textbf{Order} & \textbf{Known Groups} \\
  \hline
    1 & $C_{1}$ \\
    2 & $C_{2}$ \\
    3 & $C_{3}$ \\
    4 & $C_{4}$, $C_{2} \cross C_{2}$ \\
    5 & $C_{5}$ \\
    6 & $C_{6}$, $D_{6}$, ? \\
    7 & $C_{7}$ \\
    8 & $C_{8}$, $C_{2} \cross C_{2} \cross C_{2}$, $C_{4} \cross C_{2}$, 
    $D_{8}$, ? \\
    9 & $C_{9}$, $C_{3} \cross C_{3}$, ? \\
    10 & $C_{10}$, $D_{10}$, ? \\
    11 & $C_{11}$ \\
    12 & $C_{12}$, $C_{2} \cross C_{6}$, $D_{12}$, $A_{3}$, ? \\
  \hline
  \end{tabular}

\medskip

There may be other groups of order $6$, $8$, $10$ and $12$ that we do
not yet know of, indicated by the question marks in the table.

The following general result tells us that there are no more groups of 
order 6 and 10 than the ones listed on the table.

\begin{proposition}
  Let $G$ be a group with $|G| = 2p$, where $p$ is a prime number
  greater than $2$.  Then either $G$ is cyclic, or $G \isom D_{2p}$.
\end{proposition}
\begin{proof}
  The factors of $2p$ are $1$, $2$, $p$ and $2p$, so the order of each
  element must be one of those factors.  If $G$ has an element of
  order $2p$ it is cyclic.
  
  Assume that $G$ does not have an element of order $2p$.  If $G$ does
  not have an element of order $p$, then every non-identity element of
  $G$ must have order $2$, which means that $x^{2} = e$ for every
  element of $G$.  That imples that $G$ is a product of copies of 
  $C_{2}$ by Theorem~\ref{thm:order2group}, and in particular that 
  $|G| = 2^{n}$, which is impossible if $p > 2$.
  
  Hence there is some element $a \in G$ with $o(a) = p$.  Choose any
  element $b \notin \langle a \rangle$.  Then $G$ breaks into the two
  right cosets
  \begin{align*}
    \langle a \rangle &= \{1, a, a^{2}, \ldots, a^{p-1}\}, \\
    \langle a \rangle b &= \{b, ab, a^{2}b, \ldots, a^{p-1}b\}.
  \end{align*}
  Therefore $b^{2}$ and $(ba)^{2}$ must lie in one of these two cosets.
  
  If $b^{2} \in \langle a \rangle b$, then $b^{2} = a^{k}b$ for some 
  $k$, and the cancellation law tells us that $b = a^{k}$, which is a 
  contradiction.  Hence $b^{2} \in \langle a \rangle$, so $b^{2} = 
  a^{k}$ for some $k$.  If $k \ne 0$, then $b^{2} \ne e$, so $o(b) > 2$ 
  and hence $b$ must have order $p$.  But then since $p$ is odd, $p-1$ 
  is even, and
  \[
    b^{p} = b^{p-1}b = (b^{2})^{(p-1)/2}b = (a^{k})^{(p-1)/2}b = 
    a^{k(p-1)/2}b \ne e.
  \]
  So $b$ cannot have order $p$.  Hence $b^{2} = e$.
  
  The same argument with $ba$ in the place of $b$ shows that $(ba)^{2} = 
  e$.  Hence,
  \begin{align*}
    baba &= 1 \\
    bab &= a^{-1} = a^{p-1} \\
      ba &= a^{p-1}b^{-1} = a^{p-1}b.\\
  \end{align*}
  So $G = \{ 1, a, a^{2}, \ldots, a^{p-1}, b, ab, a^{2}b, \ldots, a^{p-1}b\}$
  and the relations
  \[
    a^{p} = 1, b = 1, \text{and} ba = a^{p-1}b,
  \]
  hold, which is precisely the definition of $D_{2p}$.  Hence $G \isom D_{2p}$.
\end{proof}

There may still be groups of order 8, 9 and 12 which we do not know
about.  We need to introduce a new concept to fully analyse these
situations, so we defer them to a later section.

\subsection*{Exercises}

\begin{exercises}
  \item Show that every Abelian group of order 8 is isomorphic to one of
  $C_{8}$, $C_{4} \cross C_{2}$, or $C_{2} \cross C_{2} \cross C_{2}$.
  
  \item\label{ex:quaterniongroup} For any natural number $n$, define the \defn{quaternion
  group}{group!quaternion} to be the group
  \[
    Q_{4n} = \{1, a, a^{2}, \ldots, a^{2n-1}, b, ab, a^{2}b, \ldots, a^{2n-1}b \}
  \]
  where the Cayley table is determined by the relations $a^{2n} = 1$, $b^{2} =
  a^{n}$, and $b^{-1}ab = a^{-1}$.
  
  Show that $Q_{4} \isom C_{2} \cross C_{2}$.
  
  Show that $Q_{8}$ is not isomorphic to any one of $C_{8}$, $C_{2} \cross C_{2} \cross
  C_{2}$, $C_{2} \cross C_{4}$, or $D_{8}$.  In other words, $Q_{8}$ is a new
  group of order $8$.
  
  Find the subgroup lattice for $Q_{8}$.
\end{exercises}

\section{Excursion: Equivalence Relations}

Underlying the concept of cosets is the notion of an equivalence
relation.  Equivalence relations are a powerful mathematical concept
which occur with regularity throughout abstract mathematics. 
Moreover, you are familiar with some of them already, even though you
may have not seen the concept formally explained.

\begin{definition}
  An \defn{equivalence relation}{relation!equivalence} on a set $X$ is a
  relation $R$ on $X$ which is reflexive, transitive and symmetric. 
  Equivalence relations are often denoted by the symbol $\sim$.
\end{definition}

\begin{example}
  The equality relation is an equivalence relation.  Indeed, it is the
  prototypical equivalence relation.
\end{example}

\begin{example}\label{eg:matrixconjugate}
  Let $M_{n}(\reals)$ be the set of $n$ by $n$ real-valued matrices.  Recall
  that $A$ and $B$ are equivalent matrices if there is an orthogonal matrix
  $U$ such that $A = U^{-1}BU$.  The relation $\sim$ defined by $A \sim B$ if
  and only if $A$ and $B$ are equivalence matrices is an equivalence relation.
\end{example}

\begin{example}\label{eg:modequivclass}
  Let $m$ be any natural number.  The relation $\equiv$ on $\integers$
  defined by $x \equiv y$ if and only if $m$ divides $x-y$ is an equivalence
  relation.  Another way of looking at it, is that it holds if and only if
  \[
    x \equiv y \pmod{m}.
  \]
  A third, useful, way of looking at this situation is that the set of
  all integers divisible by $m$ is the subset $H = m\integers =
  \{ mn : n \in \integers\}$ of $\integers$, and this is a subgroup, so
  $x \equiv y$ if and only if $x - y \in H$, and $x-y$ is the additive notation
  for $xy^{-1}$.  So this is the same condition we were using to test whether
  two elements were in the same coset.
\end{example}

We now get to the key example which links this section with what we have
just discussed by generalizing the above example.

\begin{example}
  Let $G$ be a group, $H$ a subgroup of $G$, and $x$, $y \in G$.  We define
  $x \equiv_{R} y \pmod{H}$ if $xy^{-1} \in H$.
  
  To prove this, you need to check that each of the three axioms for an 
  equivalence relation hold.
  
  Firstly, $xx^{-1} = e \in H$, so $x \equiv_{R} x \pmod{H}$, and $\equiv_{R} \pmod{H}$ is reflexive.
  
  Secondly, if $x \equiv_{R} y \pmod{H}$, then $xy^{-1} \in H$, then
  $yx^{-1} = (xy^{-1})^{-1} \in H$, so $y \equiv_{R} x \pmod{H}$, and $\equiv_{R} \pmod{H}$ is symmetric.
  
  Finally, if $x \equiv_{R} y \pmod{H}$ and $y \equiv_{R} z \pmod{H}$, then $xy^{-1}$ and $yz^{-1} \in H$,
  so $xz^{-1} = xy^{-1}yz^{-1} \in H$, so $x \equiv_{R} z \pmod{H}$, and $\equiv_{R} \pmod{H}$ is transitive.
  
  There is of course the equivalent ``left'' relation $x \equiv_{L} y \pmod{H}$ if $x^{-1}y \in H$,
  and this too is an equivalence relation.
\end{example}

Notice how the three group axioms correspond to the three equivalence relation
axioms in the previous example.  Notice also that with these equivalence
relations, the cosets are precisely the sets of elements which are equivalent
to one-another.

\begin{definition}
  Let $\sim$ be an equivalence relation on $X$.  The \defn{equivalence class}{equivalence class}
  of $x \in X$ is the set
  \[
    [x]_{\sim} = \{ y \in X : x \sim y \}.
  \]
  When the equivalence relation is clear, we typically write just $[x]$.
\end{definition}

\begin{example}
  For equality, the equivalence class $[x]_{=} = \{x\}$.
\end{example}

\begin{example}\label{eg:equivmodm}
  For $\equiv \pmod{m}$, the equivalence class of a number $n$,
  $[n]_{\equiv}$ is the set of all numbers with the same remainder when divided
  by $m$, or equivalently,
  \[
    [n] = \{\ldots, n-2m, n-m, n, n+m, n+2m, \ldots\}
  \]
\end{example}

\begin{example}
  The equivalence class of $x$ under the equivalence relation $\equiv_{R} \pmod{H}$
  is the right coset $Hx$.

  The equivalence class of $x$ under the equivalence relation $\equiv_{L} \pmod{H}$
  is the left coset $xH$.
\end{example}

A key fact about equivalence classes is that they partition the set $X$ into a
disjoint collection of subsets whose union is the whole set.

\begin{lemma}
  Let $\sim$ be an equivalence relation on $X$.  We have $x \sim y$ if and
  only if $[x] = [y]$.
  
  Also, either $[x] = [y]$ or $[x] \intersect [y] = \emptyset$.
\end{lemma}
\begin{proof}
  Assume that $x \sim y$, then given any $z \in [y]$, we have $y \sim z$,
  and since $\sim$ is transitive, $x \sim z$, so $z \in [x]$.  Hence $[x]
  \subseteq [y]$.  On the other hand, we know that by symmetry $x \sim y$
  implies $y \sim x$. If $z \in [x]$, we have $x \sim z$ and since $\sim$ is transitive, $x \sim z$, so $z \in [x]$.  Hence $[x]
  \subseteq [y]$.  Hence $[x] = [y]$.
  
  If $[x] = [y]$, then $y \in [y]$, since $y \sim y$ by reflexivity, and
  so $y \in [x]$.  Hence $x \sim y$ by definition.

  If $[x] \intersect [y] \ne \emptyset$, then there must be some $z$ in both
  sets, ie.~$x \sim z$ and $y \sim z$.  Now since $\sim$ is symmetric, $y \sim z$
  implies $z \sim y$, and since $\sim$ is transitive, $x \sim y$.  The first
  part of this lemma allows us to conclude that $[x] = [y]$.
\end{proof}

Theorem~\ref{thm:cosetsequal} is now a trivial corollary of this general fact
about equivalence relations.  Similarly, Lagrange's Theorem follows from the
following general fact.

\begin{theorem}\label{thm:countingequivalenceclasses}
  Let $X$ be a finite set, and let $\sim$ be an equivalence relation on $X$.
  Choose elements $x_{1}$, $x_{2}, \ldots, x_{n} \in X$ so that no two are
  equivalent, and so for every $x \in X$, there is some $x_{k}$ such that
  $[x] = [x_{k}]$ (ie.~this is a complete set of equivalence class
  representatives).  Then
  \[
    |X| = \sum_{k=1}^{n} |[x_{k}]|
  \]
\end{theorem}
\begin{proof}
  We have that $X$ is the disjoint union of the equivalence classes of the
  $x_{k}$, ie.
  \[
    X = [x_{1}] \union [x_{2}] \union \cdots \union [x_{n}],
  \]
  which follows from the fact that every $x$ lies in one of the $[x_{k}]$,
  and if $k \ne j$,
  \[
    [x_{j}] \intersect [x_{k}] = \emptyset,
  \]
  since $x_{k} \not\sim x_{j}$.
  
  Repeated application of the inclusion-exclusion principle then tells us
  that
  \[
    |X| = |[x_{1}]| + |[x_{2}]| + \cdots + |[x_{n}]|,
  \]
  since the intersections are empty.
\end{proof}

We will use this general theorem in the next section.

\subsection*{Exercises}

\begin{exercises}
  \item Verify that Example~\ref{eg:modequivclass} is an equivalence 
  relation.
  
  \item Verify that Example~\ref{eg:matrixconjugate} is an equivalence 
  relation.
  
  \item Let $A$ and $B \in M_{n}(\reals)$, and define a relation by $A 
    \sim B$ if $A = X^{-1}BX$ for some invertible matrix $X$.  SHow 
    that this is an equivalence relation.
  
  \item Can a relation be symmetric and transitive, but not reflexive? Give a proof or counterexample to justify your position.
  
  \item (*) Show that if $\sim$ is the equivalence relation of 
  Example~\ref{eg:matrixconjugate}, and
  \[
    A = \begin{bmatrix}
      1 & 0 \\
      0 & 2
    \end{bmatrix}
  \]
  then $[A]_{\sim}$ is the set of all symmetric $2 \times 2$ matrices with 
  eigenvalues $1$ and $2$.
\end{exercises}

\section{Conjugacy Classes}

There is another equivalence relation which is of interest which is a
generalization of the idea of conjugate matrices (see
Example~\ref{eg:matrixconjugate}).

\begin{definition}
  Let $G$ be a group and let $x$ and $y \in G$.  We say that $y$ is a
  conjugate of $x$ if $y = u^{-1}xu$ for some $u \in G$.  In this case, we
  write $x \sim y$.
\end{definition}

\begin{example}
  In the group $D_{6} = \{1, a, a^{2}, b, ab, a^{2}b\}$ we have that $a \sim
  a^{2}$, since $b^{-1}ab = bab = a^{2}bb = a^{2}$.  Similarly, we have that
  $b \sim ab$ since $a^{-1}ba = a^{2}ba = a^{2}a^{2}b = ab$.
\end{example}

\begin{proposition}
  If $G$ is a group, the conjugacy relation $\sim$ is an equivalence
  relation on $G$.
\end{proposition}
\begin{proof}
  We need to show that $\sim$ is reflexive, symmetric and transitive.
  
  Since $x = e^{-1}xe$, we have that $x \sim x$, so $\sim$ is reflexive.
  
  If $x \sim y$, then there is an element $u$ such that $y = u^{-1}xu$.  But
  then $x = uu^{-1}xuu^{-1} = uyu^{-1} = (u^{-1})^{-1}yu^{-1}$.  So $y \sim
  x$, and $\sim$ is symmetric.
  
  If $x \sim y$ and $y \sim z$, then $y = u^{-1}xu$ and $z = v^{-1}yv$ for
  some $u$ and $v$ in $G$, so $z = v^{-1}u^{-1}xuv = (uv)^{-1}x(uv)$. Hence
  $x \sim z$, and $\sim$ is transitive.
\end{proof}

We will call the equivalence class of $x \in G$ the \defn{conjugacy
class}{conjugacy class} of $x$.  We will denote the conjugacy class 
of an element $x \in G$ by $C(x)$.  In other words,
\[
  C(x) = \{ y \in G: y = u^{-1}xu~\text{for some $u \in G$} \}.
\]

\begin{example}
  The conjugacy classes of the four-group are $\{e\}$, $\{a\}$, $\{b\}$,
  $\{ab\}$.  In other words, each element is in its own conjugacy class.
\end{example}

\begin{example}
  The conjugacy classes of the group $D_{6} = \{1, a, a^{2}, b, ab, a^{2}b\}$
  are $\{1\}$, $\{a, a^{2}\}$, $\{b, ab, a^{2}b\}$.
\end{example}

The reason that the four-group has a conjugacy class for each element is
because it is an Abelian group.  In fact if $x$ commutes with everything in
$G$, then the only element of the conjugacy class of $x$ is itself.

\begin{proposition}\label{prop:centreconjugates}
  Let $G$ be a group and $x \in G$.  Then $C(x) = \{x\}$ if and only 
  if $x \in Z(G)$.
\end{proposition}
\begin{proof}
  Assume $x \in Z(G)$.  Let $y \in C(x)$, so that there is some $u \in
  G$ such that $y = u^{-1}xu$.  But then, since $xu = ux$,
  \[
    y = u^{-1}xu = u^{-1}ux = x.
  \]
  So $C(x) = \{x\}$.
  
  On the other hand, if $C(x) = \{x\}$, then for every $u \in G$, we 
  have $u^{-1}xu = x$, which means $x$ and $u$ commute.  Hence $x \in 
  Z(G)$.
\end{proof}

Notice in this case that we are saying that $u^{-1}xu = v^{-1}xv$ for
every $u$ and $v \in G$.  This is not going to be the case in general,
but it is of interest to know for which $u$ and $v$ it occurs.  For
example, if these quantities were never equal for different $u$ and
$v$, it would show that the conjugacy class of $x$ has many different 
elements.

\begin{proposition}
  Let $G$ be a group and $x \in G$.  Then $u^{-1}xu = v^{-1}xv$ if 
  and only if $uv^{-1} \in Z_{G}(x)$.
\end{proposition}
\begin{proof}
  If $u^{-1}xu = v^{-1}xv$, then we have
  \begin{align*}
    u^{-1}xu &= v^{-1}xv \\
    xu &= uv^{-1}xv \\
    xuv^{-1} &= uv^{-1}x.
  \end{align*}
  So we can see that $uv^{-1}$ commutes with $x$, and so $uv^{-1} \in 
  Z_{G}(x)$.
  
  On the other hand, if $uv^{-1} \in Z_{G}(x)$, we can run the 
  calculation backwards:
  \begin{align*}
    xuv^{-1} &= uv^{-1}x \\
    xu &= uv^{-1}xv \\
    u^{-1}xu &= v^{-1}xv,
  \end{align*}
  and so we have the result.
\end{proof}

Recalling Theorem~\ref{thm:cosetsequal}, this proposition cas the 
following corollary:

\begin{corollary}
  Let $G$ be a group and $x \in G$.  Then $u^{-1}xu = v^{-1}xv$ if 
  and only if $u$ and $v$ are in the same right coset of $Z_{G}(x)$, 
  ie.~if $Z_{G}(x)u = Z_{G}(x)v$.
\end{corollary}

Turning this around, it means that $u^{-1}xu \ne v^{-1}xv$ if and only
if $u$ and $v$ are in different right cosets of $Z_{G}(x)$, which
means that there is a distinct conjugate of $x$ for each distinct
right coset of $Z_{G}(x)$.  If $G$ is finite, then Lagrange's Theorem
tells us that there are exactly $|G : Z_{G}(x)| = |G|/|Z_{G}(x)|$ 
distinct right cosets of $Z_{G}(x)$.  This proves the following:

\begin{corollary}\label{cor:conjugacyclasssize}
  Let $G$ be a finite group and $x \in G$.  Then
  \[
    |C(x)| = |G : Z_{G}(x)|.
  \]
\end{corollary}

Note that this agrees with Proposition~\ref{prop:centreconjugates}, 
since $x \in Z(G)$ if and only if $Z_{G}(x) = G$, but this happens if 
and only if $|C(x)| = |G : Z_{G}(x)| = |G|/|G| = 1$.  In other words, 
if and only if the only conjugate of $x$ is itself.

\begin{example}
  For the group $D_{6}$ we know that $a$ has conjugacy class $\{a, 
  a^{2}\}$, and we know that $Z_{D_{6}}(a) = \{1, a, a^{2}\}$.  It is 
  simple to verify that
  \[
    |G|/|Z_{D_{6}}(a)| = 6/3 = 2 = |C(a)|.
  \]
  If we look at the other elements of $D_{g}$, we observe that
  \begin{alignat*}{2}
    |C(e)| &= 2 &\qquad |G|/|Z_{D_{6}}(e)| &= 6/6 = 1 \\
    |C(a^{2})| &= 2 &\qquad |G|/|Z_{D_{6}}(a^{2})| &= 6/3 = 2 \\
    |C(b)| &= 3 &\qquad |G|/|Z_{D_{6}}(b)| &= 6/2 = 3 \\
    |C(ab)| &= 3 &\qquad |G|/|Z_{D_{6}}(ab)| &= 6/2 = 3 \\
    |C(a^{2}b)| &= 3 &\qquad |G|/|Z_{D_{6}}(a^{2}b)| &= 6/2 = 3.
  \end{alignat*}
\end{example}

We can now apply Theorem~\ref{thm:countingequivalenceclasses} to the 
conjugacy equivalence relation.

\begin{theorem}[The Class Equation]
  Let $G$ be a finite group, and let $g_{1}$, $g_{2}$, $g_{n} \in G$ 
  be chosen so that no two are conjugate, and every conjugacy class 
  of $G$ occurs as one of the conjugacy classes $C(g_{k})$ (ie.~the 
  elements $g_{k}$ are a complete set of conjugacy class 
  representatives).  Then
  \[
    |G| = \sum_{k = 1}^{n} |C(g_{k})| = \sum_{k = 1}^{n} |G : 
    Z_{G}(g_{k})|.
  \]
  In fact, if we assume that $Z(G) = \{g_{1}, g_{2}, \ldots, 
  g_{m}\}$, then
  \[
    |G| = |Z(G)| + \sum_{k = m+1}^{n} |G : Z_{G}(g_{k})|.
  \]
\end{theorem}
\begin{proof}
  Since conjugacy is an equivalence relation, Theorem~\ref{thm:countingequivalenceclasses} 
  applies and tells us that
  \[
    |G| = |C(g_{1})| + |C(g_{2})| + \cdots + |C(g_{n})| = \sum_{k = 1}^{n} |C(g_{k})|.
  \]
  But then Corollary~\ref{cor:conjugacyclasssize} tells us that 
  $|C(g_{k})| = |G : Z_{G}(g_{k})|$, and so we can re-write the equation as
  \[
    |G| = \sum_{k = 1}^{n} |G : Z_{G}(g_{k})|.
  \]
  
  Now we know that if $g_{k} \in Z(G)$, then $|C(g_{k})| = 1$, so if
  $Z(G) = \{g_{1}, g_{2}, \ldots, g_{m}\}$, then
  \begin{align*}
    |G| &= |C(g_{1})| + |C(g_{2})| + \cdots + |C(g_{m})| + |C(g_{m + 1})| + 
    |C(g_{m + 2})| + \cdots + |C(g_{n})| \\
        &= \underbrace{1 + 1 + \cdots + 1}_{m~\text{times}} + |C(g_{m + 1})| + 
    |C(g_{m + 2})| + \cdots + |C(g_{n})| \\
        &= |Z(G)| + \sum_{k = m+1}^{n} |C(g_{k})| \\
        &= |Z(G)| + \sum_{k = m+1}^{n} |G : Z_{G}(g_{k})|.
  \end{align*}
\end{proof}

\begin{example}
  For the group $D_{6}$ we have that $1$, $a$, and $b$ are 
  representatives of each conjugacy class, and
  \[
    |C(1)| + |C(a)| + |C(b)| = 1 + 2 + 3 = 6 = |D_{6}|.
  \]
\end{example}

\begin{example}
  For the group $D_{8}$, we know from Example~\ref{eg:D8centre} that
  the centre $Z(D_{8}) = \{1, a^{2}\}$, so $C(1) = \{1\}$ and 
  $C(a^{2}) = \{a^{2}\}$.
  
  We know that $Z_{D_{8}}(a)$ contains $\langle a \rangle$ so 
  \[
    |Z_{D_{8}}(a)| \ge  |\langle a \rangle| = o(a) = 4,
  \]
  and from Lagrange's theorem $|Z_{D_{8}}(a)|$ divides $8$, so we 
  conclude that $|Z_{D_{8}}(a)|$ is $4$ or $8$.  But $a \not\in Z(D_{8})$, 
  so $Z_{D_{8}}(a) \ne D_{8}$, and hence $|Z_{D_{8}}(a)| = 4$.  Hence 
  $a$ has two conjugates, itself, and $b^{-1}ab = bab = a^{3}b^{2} = 
  a^{3}$.  So $C(a) = \{a, a^{3}\}$.
  
  Now $Z_{D_{8}}(b)$ contains $\langle b \rangle$, so $|Z_{D_{8}}(b)|
  \ge o(b) = 2$, and $|Z_{D_{8}}(b)|$ divides $8$.  Also $b$ is not in
  the centre of $D_{8}$, so $|Z_{D_{8}}(b)| \ne 8$, and hence
  $|Z_{D_{8}}(b)|$ is $2$ or $4$.  Now we know that the centre is a 
  subgroup of $Z_{D_{8}}(b)$, so $1$ and $a^{2}$ are in the subgroup, 
  as is $b$ itself.  Hence $|Z_{D_{8}}(b)| > 2$, and we conclude 
  that $|Z_{D_{8}}(b)| = 4$.  Hence $b$ has two conjugates, itself 
  and $a^{-1}ba = a^{-1}a^{3}b = a^{2}b$.  So $C(b) = \{b, a^{2}b\}$.
  
  Looking at $Z_{D_{8}}(ab)$, an analagous argument tells us that 
  $|Z_{D_{8}}(ab)| = 4$, and the conjucacy class is $C(ab) = \{ab, 
  a^{3}b\}$.
  
  We can verify that the class equation holds in this example: a 
  complete collection of equivalence class representatives is $1$, 
  $a^{2}$, $a$, $b$ and $ab$, and
  \[
    |Z(D_{8})| + |C(a)| + |C(b)| + |C(ab)| = 2 + 2 + 2 + 2 = 8 = 
    |D_{8}|.
  \]
\end{example}

We conclude with one last proposition which can help identify which 
elements are in different conjugacy classes.

\begin{proposition}
  Let $G$ be a finite group.  If $x$ and $y \in G$ are conjugate, 
  then $o(x) = o(y)$.
\end{proposition}
\begin{proof}
  Let $x = u^{-1}yu$.  Then if $o(y) = n$,
  \begin{align*}
    x^{n} &= (u^{-1}yu)^{n} \\
          &= \underbrace{(u^{-1}yu)(u^{-1}yu)\cdots 
          (u^{-1}yu)}{n~\text{times}} \\
          &= u^{-1}\underbrace{yy\ldots y}{n~\text{times}}u \\
&= u^{-1}y^{n}u \\
&= u^{-1}u \\
&= e.
  \end{align*}
  So $o(x)$ divides $n$.  However, if $o(x) = m$, then a similar 
  calculation shows that
  \begin{align*}
    e &= x^{m} \\
      &= u^{-1}y^{m}u,
  \end{align*}
  so $y^{m} = uu^{-1}y^{m}uu^{-1} = ueu^{-1} = e$.  Hence $n$ 
  divides $m$, and so $n = m$.
\end{proof}

Note that the converse is not true, since in the group $D_{8}$, the 
elements $a^{2}$, $b$ and $ab$ all have order $2$, but all are in 
distinct conjugacy classes.

Notice that every term in the class equation divides the order of 
the group.  We can use this to prove the following interesting result 
that will be useful when we look once again at groups of small order.

\begin{theorem}\label{thm:centreprimepower}
  Let $G$ be a finite group of prime power order, ie.  $|G| = p^{n}$
  where $p$ is prime and $n \ge 1$.  Then $|Z(G)| = p^{m}$ for some $m
  \ge 1$.
\end{theorem}
\begin{proof}
  Lagrange's theorem tells us that $|Z(G)| = p^{m}$ for some 
  $m \ge 0$.
  
  Let $x_{1}$, $x_{2}, \ldots, x_{k}$ be a complete set of conjugacy
  class representatives, and let $Z(G) = \{x_{1}, \ldots, x_{l}\}$. 
  Now let $n_{i} = |C(x_{i})| = |G : Z_{G}(x_{i})|$, so $n_{i} \mid
  p^{n}$.  Now for $i > l$, we must have $n_{i} > 1$, so $n_{i}$ must
  be a multiple of $p$.  Therefore,
  \[
    p^{n} = |G| = |Z(G)| + n_{l+1} + n_{l+2} + \cdots n_{k} = l + jp
  \]
  for some integer $j$.  But therefore $l$ is a multiple of $p$, and
  since $e \in Z(G)$, $|Z(G)| \ge 1$.  So we conclude that $|Z(G)| 
  \ge p$, and hence $m \ge 1$.
\end{proof}

In other words, groups of prime power must have more than just the 
identity in their centres.

\subsection*{Exercises}

\begin{exercises}
  \item In a group of order 15, what does the Class equation say are
  possible sizes of the conjugacy classes?  Potentially how many different
  ways can a group of order 15 be divided into conjugacy classes of these
  sizes (remembering that the number of conjugacy classes of size 1 has to
  equal $|Z(G)|$, which has to divide 15).
  
  Note: in actual fact, there turns out to be only one group of order 15,
  so there is just one way to do it once this is taken into account; however
  you should use the class equation to give you all the potentially possible
  ways that it could be done.
  
  \item Find the conjugacy classes of $D_{10}$, and verify that the 
  class equation holds.

  \item Find the conjugacy classes of $D_{12}$, and verify that the 
  class equation holds.

  \item Find the conjugacy classes of $A_{4}$, and verify that the 
  class equation holds.
  
  \item Let $G$ be a finite group, and $x \in G$.  Show that the
    conjugacy classes $C(x)$ and $C(x^{-1})$ have the same number of
    elements.
    
  \item Show that any group $G$ of even order must contain an
    element of even order, and use the previous exercise to conclude
    that there is at least one element $x \in G$ other than $e$ such
    that $C(x) = C(x^{-1})$.

\end{exercises}


\section{Normal Subgroups}

It turns out that many of the ideas relating to centralizers and conjugacy can
be applied to subgroups instead of individual elements.

\begin{proposition}
  Let $G$ be a group and $H$ a subgroup of $G$.  Given any $x \in G$, the
  set
  \[
    x^{-1}Hx = \{x^{-1}yx : y \in H \}
  \]
  of conjugates of elements of $H$ is a subgroup of $G$.
\end{proposition}
\begin{proof}
  Given any elements $u$ and $v \in x^{-1}Hx$, we have $y$ and $z \in H$
  such that $u = x^{-1}yx$ and $v = x^{-1}zx$.  Then
  \[
    uv^{-1} = x^{-1}yx(x^{-1}zx)^{-1} = x^{-1}yxx^{-1}z^{-1}x =
      x^{-1}yz^{-1}x,
  \]
  which is an element of $x^{-1}Hx$, since $yz^{-1} \in H$.
  
  Hence $x^{-1}Hx$ is a subgroup of $G$.
\end{proof}

If $x \in H$, then $x^{-1}Hx = H$, but if $x \notin H$ we may potentially
get something else.

\begin{example}
  In the group $D_{6}$, consider the subgroup $H = \{1, b\}$.  Since
  $a^{-1}1a = 1$, and $a^{-1}ba = a^{-1}a^{2}b = ab$ , we have
  \[
    a^{-1}Ha = \{1, ab\}.
  \]
  Similarly, we have
  \begin{align*}
    (a^{2})^{-1}Ha^{2} &= \{1, a^{2}b\} \\
    b^{-1}Hb &= \{1, b\} \\
    (ab)^{-1}H(ab) &= \{1, a^{2}b\} \\
    (a^{2}b)^{-1}H(a^{2}b) &= \{1, ab\} \\
  \end{align*}
  
  On the other hand, the subgroup $K = \{1, a, a^{2}\}$ has $x^{-1}Kx = K$
  for any $x$.  For example, since $b^{-1}1b = 1$, $b^{-1}ab = bab =
  a^{2}b^{2} = a^{2}$, and $b^{-1}a^{2}b = ba^{2}b = b^{2}a = a$, we have
  $b^{-1}Kb = K$.  Similar arguments give the remaining cases.
\end{example}

We will say that two subgroups $H$ and $K$ of $G$ are
\defn{conjugate}{conjugate!subgroups} if there is some $x \in G$ such that
\[
  K = x^{-1}Hx,
\]
and we will write $K \sim H$ if this is the case.  It is equivalent to
say that $K$ and $H$ are conjugate if and only if there is some $x \in G$
such that $xK = Hx$.

\begin{proposition}
  Let $G$ be a group.  The conjugacy relation $\sim$ is an equivalence relation
  on the set $\Sub(G)$ of all subgroups of $G$.  Furthermore if two
  subgroups are conjugate, they are isomorphic.
\end{proposition}
\begin{proof}
  We need to show that $\sim$ is reflexive, symmetric and transitive.
  
  Given a subgroup $H$, we have $H \sim H$ immediately from the fact that
  $e^{-1}He = H$.
  
  If $K$ and $H$ are subgroups with $K \sim H$, then there is some $x \in G$
  so that $K = x^{-1}Hx$.  But then $H = xKx^{-1} = (x^{-1})^{-1}Kx^{-1}$, and
  so $H \sim K$.
  
  Finally, if $H$, $K$ and $F$ are subgroups, with $H \sim K$ and $K \sim F$, 
  then there are elements $x$ and $y \in G$ such that $K = x^{-1}Hx$ and $F
  = y^{-1}Ky$.  But then $F = y^{-1}(x^{-1}Hx)y = (xy)^{-1}H(xy)$, and so
  $F \sim H$.
  
  So conjugacy of subgroups is an equivalence relation.
  
  If $K$ and $H$ are conjugate, with $K = x^{-1}Hx$, we define a function
  $\alpha : H \to K$ by $\alpha(y) = x^{-1}yx$.  This function is a
  homomorphism, since
  \[
    \alpha(y)\alpha(z) = x^{-1}yxx^{-1}zx = x^{-1}yzx = \alpha(yz).
  \]
  It is also onto, since $K = \alpha(H)$ by definition.  Finally, it is
  one-to-one since if $\alpha(y) = \alpha(z)$, then $x^{-1}yx = x^{-1}zx$,
  and using the cancellation law on the left and right gives $y = z$.
  
  So $\alpha$ is an isomorphism from $H$ to $K$, and $H$ and $K$ are
  isomorphic.
\end{proof}

\begin{example}
  Continuing the example of $D_{6}$ from above, we have that $\{1\}$ is only
  conjugate with itself, $\{1, a, a^{2}\}$ is only conjugate with itself,
  $\{1, b\} \sim \{1, ab\} \sim \{1, a^{2}b\}$, and $D_{6}$ is only
  conjugate with itself.
\end{example}

We recall that elements of a group whose conjugacy class was just themselves
were special: they formed the centre of the group.  Subgroups which are
conjugate only with themselves are also special.

\begin{definition}
  Let $G$ be a group and $K$ a subgroup of $G$.  If the only subgroup
  conjugate to $K$ is $K$ itself, that is, for any $x \in G$
  \[
    x^{-1}Kx = K,
  \]
  then we say that $K$ is a \defn{normal subgroup}{subgroup!normal}, and we
  write $K \lhd G$.
\end{definition}

Another way of representing the condition that $K$ is normal is that
\[
  Kx = xK
\]
for every $x \in G$, or in other words that the corresponding left- and right-
cosets of $K$ are identical.

\begin{example}
  In the group $D_{6}$ we have that the subgroups $\{1\}$, $\{1, a, a^{2}\}$
  and $D_{6}$ are normal.  The subgroups $\{1, b\}$, $\{1, ab\}$ and $\{1, 
  a^{2}b\}$ are not normal.
\end{example}

\begin{lemma}
  If $G$ is a group, then $\{e\}$ and $G$ are always normal subgroups of
  $G$.

  If $G$ is Abelian, then every subgroup of $G$ is normal.
\end{lemma}
\begin{proof}
  We know $x^{-1}ex = x^{-1}x = e$ for every $x$, so $x^{-1}\{e\}x = \{e\}$
  for all $x$.  We also know that $x^{-1}Gx = G$ for any $x$, since $x \in
  G$.

  If $G$ is Abelian, we recall that $Kx = xK$ for any subgroup and any $x
  \in G$ (see Definition~\ref{defn:coset}), hence $K$ is always normal.
\end{proof}

Not every element of a group is in the centre, and not every subgroup is
normal.  Similarly, just as the centralizer gives us information about the
conjugacy classes of elements, we have an analagous concept for conjugacy
classes of subgroups.

\begin{definition}
  Let $G$ be a group, and $H$ a subgroup of $G$.  We define the
  \defn{normalizer}{normalizer} of $H$ to be the set
  \[
    N_{G}(H) = \{ x \in G : H = x^{-1}Hx \}
  \]
\end{definition}

With this definition, we can duplicate most of the key results about
centralizers and conjugacy classes.

\begin{theorem}\label{thm:normalizers}
  Let $G$ be a group and $H$ a subgroup of $G$.  Then
  \begin{theoremenum}
    \item $N_{G}(H)$ is a subgroup of $G$,
    \item $H$ and $Z(G)$ are subgroups of $N_{G}(H)$,
    \item $N_{G}(H) = G$ if and only if $H$ is normal,
    \item $x^{-1}Hx = y^{-1}Hy$ if and only if $x$ and $y$ are in the same
      right coset of $N_{G}(H)$,
    \item The number of distinct conjugacy classes of $H$ is $|G :
      N_{G}(H)|$.
  \end{theoremenum}
\end{theorem}
\begin{proof}
  The proofs of these facts are analogous to the proofs of the corresponding
  results for centralizers, and are left as an exercise.
\end{proof}

Normal subgroups play a key role in the theory of groups, and we now turn to
study them in more detail.  We start with some ways of testing whether a
subgroup is normal or not, and discovering normal subgroups of a group.

\begin{theorem}
  Let $G$ be a group and $K$ a subgroup of $G$.  Then the following are
  equivalent:
  \begin{theoremenum}
    \item $K$ is normal,
    \item $x^{-1}Kx = K$ for all $x \in G$,
    \item $Kx = xK$ for all $x \in G$,
    \item $N_{G}(H) = G$,
    \item $x^{-1}yx \in K$ for all $y \in K$ and $x \in G$,
    \item $K$ is a union of some of the conjugacy classes of elements of $G$,
  \end{theoremenum}
\end{theorem}
\begin{proof}
  We have already seen that (i), (ii), (iii) and (iv) are equivalent.
  
  If $K$ is normal then $x^{-1}Kx = K$, so for any $y \in K$ we have that
  $x^{-1}yx \in K$.  Conversely, if $x^{-1}yx \in K$ for all $y \in
  K$ and $x \in G$, then if we fix $x$ we have that
  \[
    x^{-1}Kx = \{x^{-1}yx : y \in K\} \subseteq K.
  \]
  On the other hand, given any $y \in K$, we have that $(x^{-1})^{-1}yx^{-1}
  \in K$, and so
  \[
    x^{-1}((x^{-1})^{-1}yx^{-1})x = x^{-1}xyx^{-1}x = y,
  \]
  and so $K \subseteq x^{-1}Kx$.  Hence $x^{-1}Kx = K$ for every $x \in G$
  and so $K$ is normal.
  
  So we have just shown that (i) and (v) are equivalent.
  
  Another way of stating (v) is that if $y \in K$ then every conjugate of 
  $y$ is in $K$, so that $C(y) \subseteq K$.  Hence $K$ must be the union of
  all the conjugacy classes of its elements, ie.
  \[
    K = \bigcup_{y \in K} C(y)
  \]
  So (v) implies (vi).
  
  Conversely, if $K$ is a union of conjugacy classes, then given any element
  $y \in K$, the conjugacy class $C(y)$ of $y$ must be a subset of $K$, and
  so we have that $x^{-1}yx \in C(y) \subseteq K$.  Therefore $x^{-1}yx \in
  K$, and (vi) implies (v).
\end{proof}

\begin{example}
  In the group $D_{6}$, we have conjugacy classes $\{1\}$, $\{a, a^{2}\}$
  and $\{b, ab, a^{2}b\}$.  We can clearly see that each of the normal
  subgroups are unions of conjugacy classes:
  \begin{align*}
    \{1\} &= \{1\} \\
    \{1, a, a^{2}\} &= \{1\} \union \{a, a^{2}\} \\
    D_{6} &= \{1\} \union \{a, a^{2}\} \union \{b, ab, a^{2}b\}.
  \end{align*}
  
  Notice that not every union of conjugacy classes gives a normal subgroup,
  because some unions of conjugacy classes aren't subgroups.  For example,
  the set $\{1, b, ab, a^{2}b\} = \{1\} \union \{b, ab, a^{2}b\}$ is not a
  subgroup.
\end{example}

\begin{example}
  The group $D_{8}$ has conjugacy classes
  \[
    \{1\}, \{a^{2}\}, \{a, a^{3}\}, \{b, a^{2}b\}, \{ab, a^{3}b\}.
  \]
  The trivial subgroups $\{1\}$ and $D_{8}$ are automatically normal, but in
  addition, we have that the subgroups
  \begin{align*}
    \{1, a^{2}\} &= \{1\} \union \{a^{2}\} \\
    \{1, a, a^{2}, a^{3}\} &= \{1\} \union \{a^{2}\} \union \{a, a^{3}\} \\
    \{1, a^{2}, b, a^{2}b \} &= \{1\} \union \{a^{2}\} \union \{b, a^{2}b\} \\
    \{1, a^{2}, ab, a^{3}b \} &= \{1\} \union \{a^{2}\} \union \{ab, a^{3}b\}
  \end{align*}
  are all normal, as they can be written as unions of conjugacy classes as
  shown.  These are the only possible normal subgroups.
\end{example}

There are a number of conditions which guarantee that a subgroup is normal.

\begin{theorem}\label{thm:normalconditions}
  Let $G$ be a group, and $H$ a subgroup of $G$.  If any
  of the following holds, $H$ is normal:
  \begin{theoremenum}
    \item $H = \{e\}$ or $G$,
    \item $H \subseteq Z(G)$,
    \item $|G : H| = 2$,
    \item $H$ is the only subgroup of order $|H|$ in $G$.
  \end{theoremenum}
\end{theorem}
\begin{proof}
  We have already seen (i) is true.
  
  (ii) Recall that for any element $x \in Z(G)$, $C(x) = \{x\}$, so if $H
  \subseteq Z(G)$, then
  \[
    H = \bigcup_{x \in H} \{x\} = \bigcup_{x \in H} C(x),
  \]
  so $H$ is a union of conjugacy classes, and so $H$ is normal.
  
  (iii) If $|G : H| = 2$, then $H$ has two left cosets, $H$ itself and $xH$,
  where $x \notin H$.  Similarly, it has two right cosets $H$ and $Hx$.  Now
  since every element of $G$ is either in $H$ or $xH$, we have that $xH = G
  \setminus H$.  But we similarly have that $Hx = G \setminus H$.  Hence $xH
  = Hx$, and corresponding left and right cosets of $H$ are equal.  Hence
  $H$ is normal.
  
  (iv) We know that every conjugate of $H$ is a subgroup of $G$, and we must
  have $|x^{-1}Hx| = |H|$.  Hence if $H$ is the only subgroup of order $|H|$,
  we must have $H = x^{-1}Hx$ for all $x \in G$.  So $H$ is normal.
\end{proof}

\begin{example}
  In the dihedral group $D_{2n} = \{1, a, a^{2}, \ldots, a^{n-1}, b, ab,
  \ldots, a^{n-1}n\}$, the subgroup $\langle a \rangle = \{1, a, a^{2}, \ldots,
  a^{n-1}\}$ has order $|\langle a \rangle| = o(a) = n$.  So $|D_{2n} :
  \langle a \rangle| = |D_{2n}|/|\langle a \rangle| = 2n/n = 2$.
  
  So $\langle a \rangle$ is always a normal subgroup of $D_{2n}$.
\end{example}

Normal subgroups also have a nice relationship with the lattice structure of
subgroups.

\begin{proposition}\label{prop:normalproduct}
  Let $G$ be a group, $K$ a normal subgroup of $G$, and $H$ an
  arbitrary subgroup of $G$.  Then $HK$ is a subgroup of $G$, and
  furthermore $H \vee K = HK = KH$.
\end{proposition}
\begin{proof}
  We start by showing that $HK$ is a group.  Given $x$ and $y \in H$ and $u$
  and $v \in K$, we have that $xu$ and $yv$ are typical elements of $HK$.
  Now
  \[
    xu(yv)^{-1} = xuv^{-1}y^{-1} = xy^{-1}(y^{-1})^{-1}uv^{-1}y^{-1},
  \]
  and we know $xy^{-1} \in H$, and $(y^{-1})^{-1}uv^{-1}y^{-1} \in K$, since
  it is a conjugate of $uv^{-1} \in K$.  So $xu(yv)^{-1} \in HK$, and hence
  $HK$ is a subgroup of $G$.
  
  Now we need to show that $\langle H \union K \rangle = HK$.  We do this by
  showing that $HK$ is the smallest subgroup of $G$ containing both $H$ and
  $K$.  First observe that since $e \in K$, $H = He \subseteq HK$. 
  Similarly $K = eK \subseteq HK$.  So $HK$ contains both $H$ and $K$.  Now
  assume that $F$ is a subgroup which contains both $H$ and $K$.  Then given
  any $x \in H$ and $u \in K$, then $x$ and $u \in F$ and so $xu \in F$. 
  Hence $HK \subseteq F$.  So $HK$ is the smallest subgroup which contains
  both $H$ and $K$.
  
  A similar argument shows that $KH$ is a subgroup of $G$ and that 
  $H \vee K = KH$ as well.
  
  So we have that $HK = \langle H \union K \rangle = H \vee K = K \vee H =
  KH$.
\end{proof}

We can use this fact to show that normal subgroups form a sub-lattice 
within the lattice of subgroups of a group.

\begin{theorem}
  Let $G$ be a group, and let $H$ and $K$ be normal subgroups of $G$.  Then
  $H \vee K = \langle H \union K \rangle$ and $H \wedge K = H \intersect K$
  are both normal.
\end{theorem}
\begin{proof}
  We know that $H \vee K = HK$, so we will show that $HK$ is a normal
  subgroup of $G$.  If $x \in H$ and $u \in K$ and $z \in G$, then
  $xu$ is a typical element of $HK$ and we have that
  \[
    z^{-1}xuz = z^{-1}xzz^{-1}uz,
  \]
  and $z^{-1}xz \in H$, $z^{-1}uz \in K$, and so $z^{-1}xuz \in HK$.  Hence
  $HK$ is normal.

  Also, given any $y \in H \intersect K$, and any $x \in G$, we have
  that $x^{-1}yx \in H$ and $x^{-1}yx \in K$, so $x^{-1}yx \in H
  \intersect K$, and so $H \intersect K$ is normal.
\end{proof}

\begin{corollary}
  If $G$ is a group, then the set of normal subgroups of $G$ is a lattice.
\end{corollary}

We will use the symbol $\lhd$ to represent the order that gives this
lattice.  In other words, $H \lhd K$ if and only if both $H$ and $K$
are normal and $H \le K$.  This extends the use of $\lhd$ to indicate
a normal subgroup of a group.

\begin{example}
  The normal subgroup lattice of $D_{8}$ is:
  
  \begin{picture}(10,8)(-5,-1)
    \put(0,0){\makebox(0,0){$\{1\}$}}
    \put(0,0.5){\line(0,1){1}}
    \put(0,2){\makebox(0,0){$\langle a^{2} \rangle$}}
    \put(0,2.5){\line(0,1){1}}
    \put(0.5,2.5){\line(1,1){1}}
    \put(-0.5,2.5){\line(-1,1){1}}
    \put(-2,4){\makebox(0,0){$\langle a^{2}, b \rangle$}}
    \put(0,4){\makebox(0,0){$\langle a \rangle$}}
    \put(2,4){\makebox(0,0){$\langle a^{2}, ab \rangle$}}
    \put(0,4.5){\line(0,1){1}}
    \put(1.5,4.5){\line(-1,1){1}}
    \put(-1.5,4.5){\line(1,1){1}}
    \put(0,6){\makebox(0,0){$D_{8}$}}
  \end{picture}
  
  The lattice diagrams for the subgroup lattice and the normal subgroup
  lattice are sometimes combined by representing the normal subgroup lattice
  by thicker lines or double lines.
  
  \begin{picture}(10,8)(-5,-1)
    \put(0,0){\makebox(0,0){$\{1\}$}}
    \put(0.1,0.5){\line(0,1){1}}
    \put(-0.1,0.5){\line(0,1){1}}
    \put(0.5,0.5){\line(1,1){1}}
    \put(0.5,0.25){\line(2,1){3}}
    \put(-0.5,0.5){\line(-1,1){1}}
    \put(-0.5,0.25){\line(-2,1){3}}
    \put(-4,2){\makebox(0,0){$\langle b \rangle$}}
    \put(-2,2){\makebox(0,0){$\langle a^{2}b \rangle$}}
    \put(0,2){\makebox(0,0){$\langle a^{2} \rangle$}}
    \put(2,2){\makebox(0,0){$\langle ab \rangle$}}
    \put(4,2){\makebox(0,0){$\langle a^{3}b \rangle$}}
    \put(0.1,2.5){\line(0,1){1}}
    \put(-0.1,2.5){\line(0,1){1}}
    \put(3.5,2.5){\line(-1,1){1}}
    \put(2,2.5){\line(0,1){1}}
    \put(0.55,2.45){\line(1,1){1}}
    \put(0.45,2.55){\line(1,1){1}}
    \put(-3.5,2.5){\line(1,1){1}}
    \put(-0.45,2.55){\line(-1,1){1}}
    \put(-0.55,2.45){\line(-1,1){1}}
    \put(-2,2.5){\line(0,1){1}}
    \put(-2,4){\makebox(0,0){$\langle a^{2}, b \rangle$}}
    \put(0,4){\makebox(0,0){$\langle a \rangle$}}
    \put(2,4){\makebox(0,0){$\langle a^{2}, ab \rangle$}}
    \put(0.1,4.5){\line(0,1){1}}
    \put(-0.1,4.5){\line(0,1){1}}
    \put(1.55,4.55){\line(-1,1){1}}
    \put(1.45,4.45){\line(-1,1){1}}
    \put(-1.55,4.55){\line(1,1){1}}
    \put(-1.45,4.45){\line(1,1){1}}
    \put(0,6){\makebox(0,0){$D_{8}$}}
  \end{picture}
\end{example}

Part of the importance of normal subgroups is that they are closely related
to homomorphisms.  In fact every homomorphism gives you a normal subgroup.

\begin{theorem}\label{thm:inversenormal}
  Let $G$ and $H$ be groups and $\alpha: G \to H$ a homomorphism.  If $K$ is
  a normal subgroup of $H$, then $\alpha^{-1}(K)$ is a normal subgroup of
  $G$.
  
  In particular, $\ker \alpha$ is always normal.
\end{theorem}
\begin{proof}
  We know that $\alpha^{-1}(K)$ is a subgroup of $G$.  Given any $x \in G$,
  and any $y \in \alpha^{-1}(K)$, we have that
  \[
    \alpha(x^{-1}yx) = (\alpha(x))^{-1}\alpha(y)\alpha(x),
  \]
  and $\alpha(y) \in K$, so $(\alpha(x))^{-1}\alpha(y)\alpha(x) \in K$. 
  Therefore, $x^{-1}yx \in K$, and so $K$ is normal.
  
  We recall that $\ker \alpha = \alpha^{-1}(\{e\})$, and $\{e\}$ is always
  normal, so $\ker \alpha$ is normal.
\end{proof}

\begin{corollary}
  If $G$ and $H$ are groups, and $\alpha : G \to H$ is an isomorphism, then
  a subgroup $K$ of $G$ is normal if and only if $\alpha(K)$ is normal.
\end{corollary}

\begin{corollary}
  If $G$ and $H$ are isomorphic, then the lattice of normal subgroups of $G$
  and the lattice of normal subgroups of $H$ are isomorphic.
\end{corollary}

\begin{example}
  Consider the homomorphism $\alpha: D_{8} \to V$ defined by 
  $\alpha(a) = a$, and $\alpha(b) = b$.  Looking at the image of each 
  element, we see get the following table:
  \[
    \begin{array}{c|c}
      x & \alpha(x) \\
      \hline
      1 & 1 \\
      a & a \\
      a^{2} & 1 \\
      a^{3} & a \\
      b & b \\
      ab & ab \\
      a^{2}b & b \\
      a^{3}b & ab
    \end{array}
  \]
  
  Since $V$ is Abelian, every subgroup is normal, and the inverse 
  images of each subgroup are the normal subgroups
  \begin{align*}
     \alpha^{-1}(\{1\}) &= \{1, a^{2}\} \\
     \alpha^{-1}(\{1, a\}) &= \{1, a, a^{2}, a^{3}\} \\
     \alpha^{-1}(\{1, b\}) &= \{1, b, a^{2}, a^{2}b\} \\
     \alpha^{-1}(\{1, ab\}) &= \{1, ab, a^{2}, a^{3}b\} \\
     \alpha^{-1}(V) &= D_{8}.
  \end{align*}
\end{example}


\subsection*{Exercises}

\begin{exercises}
  \item Prove Theorem~\ref{thm:normalizers}.
  
  \item Let $H = \langle X \rangle$ be a subgroup of a group $G$. 
    Show that $H$ is normal if and only if $g^{-1}xg \in H$ for all $x
    \in X$.
  
  \item Find the normal subgroup lattice of $D_{10}$.

  \item Find the normal subgroup lattice of $D_{12}$.
  
  \item Find the normal subgroup lattice of $A_{4}$.
  
  \item\label{ex:commutatorsubgroup} Given elements $x$ and $y \in G$,
    their \defn{commutator}{communtator} is the element
    \[
      [x,y] = x^{-1}y^{-1}xy.
    \]
    The \defn{derived}{subgroup!derived} or \defn{commutator
    subgroup}{subgroup!commutator} is the subgroup generated by all 
    the commutators of elements of $G$
    \[
      G' = \langle \{ [x,y] : x, y \in G\} \rangle
    \]
    
    \begin{theoremenum}
      \item Show that if $x$ and $y$ commute, then $[x,y] = e$.  
        Conclude that the commutator subgroup of an Abelian group is 
        always $\{e\}$.
      
      \item Show that $G'$ is normal.
      
      \item Find the commutator subgroup of $D_{8}$.
      
      \item If $H$ and $K$ are normal subgroups of $G$, show that the 
        commutator $[x,y]$ of any pair $x \in H$ and $y \in K$ lies 
        in $H \wedge K$.  Show that if $H \wedge K = \{e\}$, then any 
        element of $H$ commutes with any element of $K$.

    \end{theoremenum}
\end{exercises}

\section{Groups of Small Order, Part II}

In previous sections we have discovered that all the groups of order 
less than $8$ are isomorphic to one of a small collection of groups.  
We do not know what groups there are of order $8$, however.

We have identified $C_{8}$, $C^{2} \cross C^{4}$, $C_{2} \cross C_{2}
\cross C_{2}$ and $D_{8}$, but there could potentially be more.  In fact,
in Exercise~\ref{ex:quaterniongroup}, there was the following definition.

\begin{definition}
  For any natural number $n$, the \defn{quaternion group}{group!quaternion} is
  the group with elements
  \[
    Q_{4n} = \{1, a, a^{2}, \ldots, a^{2n-1}, b, ab, a^{2}b, \ldots, a^{2n-1}b \}
  \]
  where the Cayley table is determined by the relations $a^{2n} = 1$, $b^{2} =
  a^{n}$, and $b^{-1}ab = a^{-1}$.
\end{definition}

In Exercise~\ref{ex:quaterniongroup} it was shown that $Q_{8}$
is not isomorphic to any other known group of order $8$.  In fact, $Q_{8}$
completes the set of isomorphism classes of groups of order $8$.

\begin{theorem}
  If $G$ is a group of order $8$, then $G$ is isomorphic to one of $C_{8}$,
  $C_{2} \cross C_{4}$, $C_{2} \cross C_{2} \cross C_{2}$, $D_{8}$ or $Q_{8}$.
\end{theorem}
\begin{proof}
  If $G$ has an element of order $8$, then Theorem~\ref{thm:cyclicgroups} tells
  us that $G$ is cyclic, and so $G \isom C_{8}$.

  If every element of $G$ other then $e$ has order $2$, then
  Theorem~\ref{thm:order2group} tells us that $G$ is a direct product of
  cyclic groups of order $2$. So $G \isom C_{2} \cross C_{2} \cross C_{2}$.

  If $G$ is not isomorphic to one of these two groups, it must have an element
  of order 4, and no elements of order 8.  Let $a$ be this element of 
  order $4$ in $G$, and let $H = \langle a \rangle$.  Since $|G : H| = 
  |G|/|H| = 8/4 = 2$, so if $b \in G \setminus H$, then $H$ has two cosets $H$
  and $Hb$, and $G$ is the disjoint union of $H$ and $Hb$.  So
  \[
    G = \{e, a, a^{2}, a^{3}, b, ab, a^{2}b, a^{3}b\}.
  \]

  If $G$ is Abelian, then $ab = ba$, and $b$ must either be of order 
  $2$ or $4$.  If $b^{2} = 1$, then $G \isom C_{4} \cross C_{2}$ where 
  the isomorphism is given by $\alpha(u^{k}, v^{l}) = a^{k}b^{l}$, 
  where $C_{4} = \langle u \rangle$ and $C_{2} = \langle v \rangle$.  
  If $b^{2} \ne e$, then we must have $b^{2} = a^{k}$, since if 
  $b^{2} = a^{k}b$, then the cancellation law tells us that $b = a^{k} 
  \in H$, which is a contradiction.  Furthermore if $k = 1$ or $k = 3$, 
  then $b^{4} = a^{2} \ne e$, which is a contradiction.  So then 
  $b^{2} = a^{2}$.  But in this case, $ab$ has order $2$, and $G \isom
  C_{4} \cross C_{2}$ where the isomorphism is given by $\alpha(u^{k}, v^{l}) 
  = a^{k}(ab)^{l}$.
  
  So the only Abelian groups of order $8$ are $C_{8}$, $C_{4} \cross C_{2}$
  and $C_{2} \cross C_{2} \cross C_{2}$.

  If $G$ is not Abelian, we note that Theorem~\ref{thm:normalconditions} says
  that since $|G : H| = 2$, $H$ is normal in $G$, which means that $b^{-1}ab
  \in H$, so
  \[
    b^{-1}ab = a^{k}
  \]
  where $k$ is one of $0$, $1$, $2$ or $3$.  But $k \ne 0$, since 
  otherwise $a \in C(e) = \{e\}$, which cannot happen.  If $k = 2$, then
  \[
   b^{-1}a^{2}b = b^{-1}abb^{-1}ab = a^{2}a^{2} = a^{4} = e,
  \]
  so $a^{2} \in C(e)$, which is also impossible.  If $k = 1$, then $b^{-1}ab
  = a$ implies $ab = ba$, so $G$ is Abelian.  So the only remaining
  possibility is that $k = 3$.

  As before, the order of $b$ must be either $2$ or $4$.  If $b$ has order $2$,
  then $b^{2} = 1$.  This means that $b = b^{-1}$, and so $b^{-1}ab = a^{3}$
  implies that $ba = a^{3}b$. So $G$ is determined by the relations
  $a^{4} = 1$, $b^{2} = 1$ and $ba = a^{3}b$, and $G$ is isomorphic to $D_{8}$
  under the trivial isomorphism $\alpha(a^{k}b^{l}) = a^{k}b^{l}$.
  
  Finally, if $b$ has order $4$, then the same argument as the Abelian case
  tells us that $b^{2} = a^{2}$.  We then observe that since $b^{3} = b_{-1}$,
  we have that $G$ is determined by the relations $a^{4} = 1$, $b^{2} = a^{2}$
  and $b^{-1}ab = a^{-1}$, and $G$ is isomorphic to $Q_{8}$
  under the trivial isomorphism $\alpha(a^{k}b^{l}) = a^{k}b^{l}$.
\end{proof}

The next lowest order that we don't have full information on is groups of
order $9$.  We know that we have $C_{9}$ and $C_{3} \cross C_{3}$, but there
could potentially be other groups of order $9$.

\begin{theorem}\label{thm:groupsoforder9}
  Let $G$ be a group of order $9$.  Then $G$ is isomorphic to one of $C_{9}$
  or $C_{3} \cross C_{3}$.
\end{theorem}
\begin{proof}
  From Lagrange's theorem, the elements of $G$ all have orders dividing $9$,
  so they have order $1$, $3$ or $9$.  If there is an element of order $9$,
  then Theorem~\ref{thm:cyclicgroups} tells us that $G \isom C_{9}$.  So if
  $G$ is not isomorphic to $C_{9}$, then every
  element other than the identity must have order $3$.
  
  Choosing an element $a \ne e$, we have that the subgroup $H = \langle
  a \rangle$ has order $3$.  Now choose $b \not\in H$.  Now $b^{2} \not\in H$,
  since that would imply $b^{2} = a^{k}$, where $k = 1$ or $2$, and in either
  case $b$ would have order $6$.  Similarly $b^{2} \notin Hb$ since then we
  would have $b^{2} = a^{k}b$, where $k = 1$ or $2$, and then since $G$ is
  Abelian $b^{3} = a^{k}b^{2} = a^{2k}b \ne e$, since $b \notin H$.  So the
  right cosets of $H$ are $H$, $Hb$ and $Hb^{2}$, and therefore
  \[
    G = \{e, a, a^{2}, b, ab, a^{2}b, b^{2}, ab^{2}, a^{2}b^{2}\}.
  \]

  Theorem~\ref{thm:centreprimepower} tells us that the centre of $G$ has
  $|Z(G)| = 3$ or $|Z(G)| = 9$.  If $|Z(G)| = 9$, then $G$ is Abelian, and
  $G \isom C_{3} \cross C_{3}$ via the isomorphism $\alpha(u^{k}, u^{l}) =
  a^{k}b^{l}$, where $C_{3} = \langle u \rangle$.
  
  So assume that $|Z(G)| = 3$.  Without loss of generality in the above
  discussion, we could have assumed that we chose $a \in Z(G)$, so that
  $H = Z(G)$.  Therefore $a$ commutes with every element of $G$, so in
  particular, $ba = ab$.  But since $G = \langle a, b \rangle$, $G$ is Abelian
  and $|Z(G)| = |G| = 9$.  So every group of order 9 is Abelian.
\end{proof}

At this point we have classified every group of order up to and including 11. 
It turns out that any group of order $12$ is isomorphic to one of $C_{12}$,
$C_{6} \cross C_{2}$, $D_{12}$, $Q_{12}$ or the alternating group $A_{4}$.
However, the proof of this fact requires considerably more powerful techniques
than we have available right now.

\subsection*{Exercises}

\begin{exercises}
  \item Show that if $p$ is a prime number, and $G$ is a group with $|G| =
  p^{2}$, then $G$ is isomorphic to one of $C_{p^{2}}$ or $C_{p} \cross C_{p}$.
  
  Hint: generalize the case of $|G| = 9$.
  
  \item Explain why there is one group of order $13$ and two groups 
    of order $14$.
  
  \item Show that $C_{15} \isom C_{5} \cross C_{3}$.
\end{exercises}

\section{Extension: Cayley Graphs}

Cayley graphs are closely related to generators, and give a nice way 
of picturing groups.  Unfortunately, they are not that useful in 
distingishing between groups which are not isomorphic, but they are 
of some independent interest, particularly when considering how one 
can computerize calculations involving groups.

\begin{definition}
  Let $(G, \ast, e)$ be a group, and let $S = \{ g_{1}, a_{2}, 
  \ldots, g_{n}\}$ be a finite set which generates $G$, ie. $G = 
  \langle S \rangle$.  The \defn{Cayley graph}{Cayley graph} of $G$ 
  with the generating set $S$ is the graph $\Gamma_{G} = (G, E)$ 
  with vertices being the elements of $G$, and two vertices $x$ and
  $y \in G$ are connected by an edge if $x = yg$ for some
  $g \in S \union S^{-1}$.
\end{definition}

Definitions vary somewhat from source to source.  Some may define the Cayley
graph as a directed graph with directed edges of the form $(x, xg)$ for
$g \in S \union S^{-1}$, while others allow a double edge $(x,y)$ if
both $x = yg$ and $y = xg$.

\begin{example}
  The cyclic group $C_{4} = \{1, a, a^{2}, a^{3}\}$ has a generating set
  $S = \{a\}$. The Cayley graph of $G$ with this generating set has edges
  $(1,a)$, $(a,a^{2})$, $(a^{2}, a^{3})$ and $(a^{3}, 1)$.  In other words,
  it looks like this:
  
  \begin{picture}(4,4)(-1,-1)
    \put(0,0){\line(1,0){2}}
    \put(0,0){\line(0,1){2}}
    \put(0,2){\line(1,0){2}}
    \put(2,0){\line(0,1){2}}
    \put(0,0){\circle*{0.2}}
    \put(0,2){\circle*{0.2}}
    \put(2,0){\circle*{0.2}}
    \put(2,2){\circle*{0.2}}
    \put(-0.5,-0.5){\makebox(0,0){$1$}}
    \put(-0.5,2.5){\makebox(0,0){$a$}}
    \put(2.5,-0.5){\makebox(0,0){$a^{3}$}}
    \put(2.5,2.5){\makebox(0,0){$a^{2}$}}
  \end{picture}
  
  This group is also generated (somewhat redundantly) by the set
  $S = \{a, a^{2}\}$.  This gives a Cayley graph which looks like this:
  
  \begin{picture}(4,4)(-1,-1)
    \put(0,0){\line(1,0){2}}
    \put(0,0){\line(0,1){2}}
    \put(0,0){\line(1,1){2}}
    \put(0,2){\line(1,0){2}}
    \put(0,2){\line(1,-1){2}}
    \put(2,0){\line(0,1){2}}
    \put(0,0){\circle*{0.2}}
    \put(0,2){\circle*{0.2}}
    \put(2,0){\circle*{0.2}}
    \put(2,2){\circle*{0.2}}
    \put(-0.5,-0.5){\makebox(0,0){$1$}}
    \put(-0.5,2.5){\makebox(0,0){$a$}}
    \put(2.5,-0.5){\makebox(0,0){$a^{3}$}}
    \put(2.5,2.5){\makebox(0,0){$a^{2}$}}
  \end{picture}
\end{example}

\begin{example}
  The four-group $V = \{1, a, b, ab\}$ has a generating set
  $S = \{a, b\}$. The Cayley graph of $G$ with this generating set has edges
  $(1,a)$, $(a,ab)$, $(1, b)$ and $(b, ab)$.  In other words,
  it looks like this:
  
  \begin{picture}(4,4)(-1,-1)
    \put(0,0){\line(1,0){2}}
    \put(0,0){\line(0,1){2}}
    \put(0,2){\line(1,0){2}}
    \put(2,0){\line(0,1){2}}
    \put(0,0){\circle*{0.2}}
    \put(0,2){\circle*{0.2}}
    \put(2,0){\circle*{0.2}}
    \put(2,2){\circle*{0.2}}
    \put(-0.5,-0.5){\makebox(0,0){$1$}}
    \put(-0.5,2.5){\makebox(0,0){$a$}}
    \put(2.5,-0.5){\makebox(0,0){$b$}}
    \put(2.5,2.5){\makebox(0,0){$ab$}}
  \end{picture}
  
  This group is also generated by the set
  $S = \{a, b, ab\}$.  This gives a Cayley graph which looks like this:
  
  \begin{picture}(4,4)(-1,-1)
    \put(0,0){\line(1,0){2}}
    \put(0,0){\line(0,1){2}}
    \put(0,0){\line(1,1){2}}
    \put(0,2){\line(1,0){2}}
    \put(0,2){\line(1,-1){2}}
    \put(2,0){\line(0,1){2}}
    \put(0,0){\circle*{0.2}}
    \put(0,2){\circle*{0.2}}
    \put(2,0){\circle*{0.2}}
    \put(2,2){\circle*{0.2}}
    \put(-0.5,-0.5){\makebox(0,0){$1$}}
    \put(-0.5,2.5){\makebox(0,0){$a$}}
    \put(2.5,-0.5){\makebox(0,0){$b$}}
    \put(2.5,2.5){\makebox(0,0){$ab$}}
  \end{picture}
\end{example}

Notice that in each case, the graphs look the same, even though the groups
are not isomorphic.

\begin{proposition}
  Let $G$ be a finite group.  The Cayley graph with generating set taken to
  be all of $G$ is the complete graph on $|G|$ vertices.
\end{proposition}
\begin{proof}
  Given any pair of vertices $x$, $y$, the element $g = y^{-1}x$ is in the
  generating set $G$, and $yg = yy^{-1}x = x$, so there is an edge joining
  $x$ and $y$.  Hence the Cayley graph is a complete graph.
\end{proof}

With careful selection of the generating set, however, the Cayley graph can
reveal a lot about the structure of the group.

\begin{example}
  The group of integers $(\integers, +, 0)$ is generated by the 
  element $1$.  The Cayley graph is infinite, but the region near $0$ 
  looks like:
  
  \begin{picture}(14,2)(-7,-1)
    \multiput(-6,0)(2,0){6}{\line(1,0){2}}
    \multiput(-6,0)(2,0){7}{\circle*{0.2}}    
    \put(-6,-0.5){\makebox(0,0){$-3$}}
    \put(-4,-0.5){\makebox(0,0){$-2$}}
    \put(-2,-0.5){\makebox(0,0){$-1$}}
    \put(0,-0.5){\makebox(0,0){$0$}}
    \put(2,-0.5){\makebox(0,0){$1$}}
    \put(4,-0.5){\makebox(0,0){$2$}}
    \put(6,-0.5){\makebox(0,0){$3$}}    
    \put(-6.5,0){\makebox(0,0){$\cdots$}}    
    \put(6.5,0){\makebox(0,0){$\cdots$}}    
  \end{picture}
  
\end{example}


Cayley graphs have some regularities that normal graphs do not.

\begin{proposition}
  Let $G$ be a group, and $S$ a finite set which generates $G$.
  The Cayley graph of $G$ with generating set $S$ has the following
  properties:
  \begin{theoremenum}
    \item the graph is connected.
    
    \item every vertex is an end of exactly $|S \union S^{-1}|$ edges
  \end{theoremenum}
\end{proposition}
\begin{proof}
  (i) Every element $x$ can be written as a product $x = x_{1}x_{2}\cdots x_{n}$
  of elements $x_{k} \in S \union S^{-1}$, so $e, x_{1}, x_{1}x_{2}, \ldots,
  x_{1}x_{2}\cdots x_{n-1}, x$ is a path from $e$ to $x$.  So the Cayley graph
  is connected.
  
  (ii) Given any $x \in G$, if we consider the set
  \[
    X = \{ xg : g \in S \union S^{-1} \},
  \]
  then we note that if $xg_{1} = xg_{2}$, the cancellation law says that
  $g_{1} = g_{2}$.  Therefore every edge $(x, xg)$ is distinct, and $|X| =
  |S \union S^{-1}|$.  Furthermore, if there is some $y$ such that $(y,x)$
  is an edge, then $x = yg$ for some $g \in S \union S^{-1}$, so $y =
  xg^{-1} \in X$.  So $X$ is exactly the set of all vertices connected to
  $x$ by one edge.  So $x$ is an end of exactly $|S \union S^{-1}|$ edges.
\end{proof}


\subsection*{Exercises}

\begin{exercises}
  \item Draw the Cayley graph of $C_{6}$ with generating set $\{a\}$.
  
    Draw the Cayley graph of $C_{6}$ with generating set $\{a^{2}, a^{3}\}$.

  \item Draw the Cayley graph of $D_{6}$ with generating set $\{a,b\}$.

  \item Draw the Cayley graph of $D_{8}$ with generating set $\{a,b\}$.

  \item Draw the Cayley graph of $C_{2} \cross C_{4}$ with generating set 
  $\{(a,1), (1,b)\}$.

  \item Draw the region near $0$ of the Cayley graph of $\integers$ with
    generating set $\{2, 3\}$.

  \item Draw the region near $0$ of the Cayley graph of $\integers^{2}$ with
    generating set $\{(1,0), (0,1)\}$.
    
    Draw the same region if the generating set is $\{(1,0), (0,1), 
    (1,1)\}$.
  
  \item Draw the region near $e$ of the Cayley graph of the free
    group\index{group!free} $F_{2}$ with generating set $\{a, b\}$.
\end{exercises}


\newpage
\section*{Assignment 4}

The following exercises are due Monday, April 19th.  Since this is a
long assignment, it will be worth twice as much as the other 3 assignments.

\begin{description}
  \item[3.1] Exercises 2, 4, 6.
  \item[3.2] Exercises 1, 3, 5.
  \item[3.3] Exercises 1, 3, 5.
  \item[3.4] Exercises 1, 4, 5.
  \item[3.5] Exercises 2.
  \item[3.6] Exercises 4.
  \item[3.7] Exercises 1, 2, 5, 6.
  \item[3.8] Exercises 2, 3, 6.
  \item[3.9] Exercises 1, 2.
  \item[3.10] Exercises 1, 2, 5, 6.
\end{description}



\chapter{Constructing Groups}

In Chapter 2 we saw a simple way that we could combine two groups to 
get a third group: the direct product.  In this chapter we look at 
other ways to construct new groups from already known groups.

\section{Quotient Groups}

Let $G$ be a group, and $N$ a normal subgroup of $G$.  We can (at least
potentially) define a binary operation $\ast$ on the set of cosets of $N$
by
\[
  Nx \ast Ny = Nxy.
\]
The difficulty with this definition is that there may be many
different choices for $x'$ and $y'$, so that $Nx' = Nx$
and $Ny = Ny'$, and it is not immediate why we should have $Nxy =
Nx'y'$.  However they are in fact equal since $x' = ux$ and $y' = vy$ for some
$u$ and $v \in N$, and since $N$ is normal, we have $z = xvx^{-1} \in N$, so
\[
  x'y' = uxvy = uxvx^{-1}xy = (uz)xy,
\]
so $Nx'y' = N(uz)xy = Nxy$.  So $\ast$ is a well-defined binary 
operation, and it is only well-defined if $N$ is normal.

\begin{proposition}
  Let $G$ be a group and $N$ a normal subgroup of $G$.  Let $G/N$ be the set
  of all cosets of $N$ in $G$. Then $(G/N, \ast, N)$ is a group.
\end{proposition}
\begin{proof}
  We first observe that $G/N$ is associative, since for any right cosets
  $Nx$, $Ny$ and $Nz \in G/N$, we have
  \[
    (Nx \ast Ny) \ast Nz = Nxy \ast Nz = Nxyz = Nx \ast Nyz = Nx 
    \ast (Ny \ast Nz).
  \]
  
  The set $N$ is an identity, since $N = Ne$, and so
  \[
    Ne \ast Nx = Nex = Nx \qquad \text{and} Nx \ast Ne = Nxe = Nx.
  \]
  
  Finally, $Nx^{-1}$ is the inverse of $Nx$, since
  \[
    Nx \ast Nx^{-1} = Nxx^{-1} = Ne = N \qquad \text{and} \qquad 
    Nx^{-1} \ast Nx = Nx^{-1}x = Ne = N.
  \]
\end{proof}

In fact, there is another way of looking at this product.  If we 
consider cosets $Nx$ and $Ny$, then we have that the product of the 
cosets as sets is
\[
  (Nx)(Ny) = (xN)(Ny) = (xN^{2})y = (xN)y = (Nx)y = Nxy,
\]
recalling that $Nx = xN$ since $N$ is normal, and $N^{2} = N$ since
$N$ is a subgroup.  In other words, $Nx \ast Ny$ is given by the
product of sets $(Nx)(Ny)$.  In some texts this is used as the 
definition of the product.

\begin{definition}
  If $G$ is a group, and $N$ a normal subgroup of $G$, then we call $G/N$ the 
  \defn{quotient group}{quotient group} of $N$ in $G$.  
\end{definition}

\begin{example}
  Let $D_{6} = \{1, a, a^{2}, b, ab, a^{2}b\}$ as usual.  If $N = 
  \{1, a, a^{2}\}$, then $N$ is normal, and the cosets are $N$ and 
  $Nb$.  Then the Cayley table of $D_{6}/N$ is simply
  \[
    \begin{array}{c|cc}
      \ast & N & Nb \\
      \hline
      N & N & Nb \\
      Nb & Nb & N
    \end{array}
  \]
  Clearly, $D_{6}/N \isom C_{2}$.
\end{example}

\begin{example}
  The additive group of integers is Abelian, so every subgroup is 
  normal.  If we have a subgroup of the form
  \[
    N = m\integers = \{mx : x \in \integers \},
  \]
  then we observed in Example~\ref{eg:equivmodm} that the cosets of 
  this subgroup are the sets of numbers which have the same remainer 
  modulo $m$, or more concretely,
  \[
    \integers/N = \{N, N + 1, N + 2, \ldots, N + (m-1)\}.
  \]
  The group operation on these cosets is just
  \[
    (N + x) + (N + y) = N + (x+y) = N + z,
  \]
  where $z = x + y \pmod{m}$.  In other words $\integers/m\integers 
  \isom Z_{m}$.  Indeed, this is a common way of defining addition modulo 
  $m$.
\end{example}

From the theory developed in the previous chapters, we can 
immediately conclude the following.

\begin{proposition}
  Let $G$ be a group, and $N$ a subgroup of $G$.  Then
  \begin{theoremenum}
    \item if $|G|$ is finite, then $|G/N| = |G|/|N|$,
    \item the function $\alpha(x) = Nx$ is a homomorphism from $G$ 
    to $G/N$, and $\ker \alpha = N$.
  \end{theoremenum}
\end{proposition}
\begin{proof}
  (i) This is immediate from the fact that $|G/N| = [G : N]$ and 
  Lagrange's Theorem.
  
  (ii) That $\alpha$ is a homomorphism follows immediately from the 
  fact that
  \[
    \alpha(x) \ast \alpha(y) = Nx \ast Ny = Nxy = \alpha(xy).
  \]
  It is immediate that $N \subseteq \ker \alpha$, since if $x \in 
  N$, $\alpha(x) = Nx = N$.  Similarly, if $\alpha(x) = N$, then 
  $Nx = N$, which only happens when $x \in N$. So $\ker \alpha = N$.
\end{proof}

\begin{corollary}
  A subgroup $N$ of a group $G$ is normal if and only if it is the 
  kernel of some homomorphism.
\end{corollary}

The relationship between quotient groups and homomorphisms is 
significantly deeper than this corollary, however.  These 
relationships are encapsulated in a trilogy of theorems called the 
Isomorphism Theorems.  Unfortunately there is little consensus about 
which of the three should be first, second and third.

\begin{theorem}[First Isomorphism Theorem]
  Let $G$ and $H$ be groups, and $\alpha: G \to H$ a homomorphism.  
  Then
  \[
    \alpha(G) \isom G/\ker \alpha.
  \]
\end{theorem}
\begin{proof}
  For simplicity of notation, let $N = \ker \alpha$.
  
  We would like to define a function $\beta : G/\ker \alpha \to
  \alpha(G)$ by $\beta(Nx) = \alpha(x)$, but it's not clear that this 
  is a well-defined function.  To verify that this definition is 
  good, we need to show that if $Nx = Ny$ then $\beta(Nx) = 
  \beta(Ny)$ so that the value of $\beta$ does not depend on the choice 
  of $x$.
  
  Now if $Nx = Ny$ we have that $xy^{-1} \in N$, so $\alpha(xy^{-1}) = 
  e$, and hence
  \[
    \alpha(y) = e\alpha(y) = \alpha(xy^{-1})\alpha(y) = 
    \alpha(xy^{-1}y) = \alpha(x).
  \]
  So we conclude that if $Nx = Ny$ then $beta(Nx) = \beta(Ny)$, and 
  so $\beta$ is well-defined.
  
  Furthermore, $\beta$ is a homomorphism, since
  \[
    \beta(Nx \ast Ny) = \beta(Nxy) = \alpha(xy) = 
    \alpha(x)\alpha(y) = \beta(Nx)\beta(Ny).
  \]
  
  We also have that $\beta$ is onto, since if $y \in \alpha(G)$, then 
  $y = \alpha(x)$ for some $x \in G$, but then $y = \beta(Nx)$.
  
  Finally, if $\beta(Nx) = \beta(Ny)$, then $\alpha(x) = \alpha(y)$, so
  \[
    \alpha(xy^{-1}) = \alpha(x)(\alpha(y))^{-1} = 
    \alpha(x)(\alpha(x))^{-1} = e.
  \]
  This means that $xy^{-1} \in \ker \alpha = N$, so $Nx = Ny$.  
  Hence $\beta$ is one-to-one.
  
  So $\beta$ is an isomorphism, and we conclude that $G/\ker \alpha 
  \isom \alpha(G)$.
\end{proof}

We will now turn to look at how the subgroup structure of $G$ and the
subgroup structure of $G/N$ are related.  Letting $\alpha: G \to G/N$
be given by $\alpha(x) = Nx$, we have that if $H \le G$, then
$\alpha(H) \le G/N$ and if $K \subseteq G/N$ then $\alpha^{-1}(K) \le
G$ from Propositions~\ref{prop:homsubgroup} and
\ref{prop:inversehomsubgroup}.  A deeper question is if there is any 
relationship between normal subgroups of $G$ and normal subgroups of 
$G/N$.

\begin{proposition}
  Let $G$ be a group and $N$ a normal subgroup of $G$.  Then every
  subgroup of $N$ is equal to $K/N$ where for some $K$ with $N \le K
  \le G$.  Furthermore $K/N$ is normal if and only if $K$ is normal.
\end{proposition}
\begin{proof}
  Let $\alpha: G \to G/N$ be given by $\alpha(x) = Nx$.

  Let $H$ be a subgroup of $G/N$, and let $K = \alpha^{-1}(H)$, so
  that $K$ is a subgroup of $G$, and since $N = \alpha^{-1}(\{e\}) 
  \subseteq \alpha^{-1}(H)$, so $N \le K$.  So $H = \alpha(K)$ and if we
  restrict $\alpha$ to $K$, the First Isomorphism Theorem tells us
  that
  \[
    \alpha(K) \isom K/N.
  \]
  Furthermore, recall from the proof of the First Isomorphism Theorem
  that this isomorphism is given by $\beta(Nx) = \alpha(x)$, and
  $\alpha(x) = Nx$, so $\beta$ is just the identity map, and so $H =
  K/N$.
  
  If $K$ is normal in $G$, then $x^{-1}Kx = K$ for each $x \in G$, and 
  so
  \[
    (Nx^{-1}) \ast \alpha(K) \ast (Nx) = \alpha(x^{-1}) \ast \alpha(K)
    \ast \alpha(x) = \alpha(x^{-1}Kx) = \alpha(K),
  \]
  so $H = \alpha(K)$ is normal in $G/N$.
  
  Conversely, if $H = K/N$ is normal in $G/N$, then
  Theorem~\ref{thm:inversenormal} tells us immediately that
  $\alpha^{-1}(H) = K$ is normal in $G$.
\end{proof}

Now if $N \lhd K \lhd G$, as in the last part of the Proposition, we
note that $N$ is normal when regarded as a subgroup of $K$ also, and
so we can take three quotients: $G/N$, $G/K$ and $K/N$.  The Second 
Isomorphism Theorem gives us a relationship between these three 
quotients.

\begin{theorem}[Second Isomorphism Theorem]
  Let $G$ be a group and $N$ and $K$ normal subgroups of $G$ with $N 
  \le K$.  Then
  \[
    (G/N)/(K/N) \isom G/K.
  \]
\end{theorem}
\begin{proof}
  We first note that since $K$ is normal, the previous proposition tells us 
  that $K/N$ is normal in $G/N$, and so $(G/N)/(K/N)$ is defined.
  
  We would like to define a function $\alpha: G/N \to G/K$ by 
  $\alpha(Nx) = Kx$, but once again we must be careful that this 
  well-defined, since there are multiple possible choices for $x$ 
  which give the same coset $Nx$.  If $Nx = Ny$, then we recall that 
  $xy^{-1} \in N$, and so $xy^{-1} \in K$ as well.  So $K = 
  K(xy^{-1})$,
  \[
    \alpha(Ny) = Ky = (Kxy^{-1})y = Kx = \alpha(Ny).
  \]
  So this is a well-defined function.
  
  Furthermore, $\alpha$ is a homomorphism, since
  \[
    \alpha(Nx \ast Ny) = \alpha(Nxy) = Kxy = Kx \ast Ky = \alpha(Nx) 
    \ast \alpha(Ny).
  \]
  We observe that $\alpha(Nx) = K$ if and only if $x \in K$, or
  equivalently, $Nx \in K/N$.  But this means that $\ker \alpha =
  K/N$.  We also have that since every coset of $K$ is of the form 
  $Kx$ for some $x \in G$, we have that $\alpha(G/N) = \{ \alpha(Nx) : 
  x \in G \} = \{ Kx : x \in G\} = G/K$, so $\alpha$ is onto.
  
  Now the First Isomorphism Theorem tells us that
  \[
    \alpha(G/N) \isom (G/N)/\ker \alpha,
  \]
  But we know that $\alpha(G/N) = G/K$ and $\ker \alpha = K/N$, so
  \[
    G/K \isom (G/N)/(K/N).
  \]
\end{proof}

Note that this theorem essentially says that the quotient operation
cancels in the way that you would expect a quotient to cancel: if $G$,
$K$ and $N$ were numbers you would expect the same equation to hold.

We can also ask what happens if $K$ is a general subgroup of $G$.  In
this case we can't talk about $K/N$, since we may not have $N
\subseteq K$.  However we do know that the meet of $K$ and $N$ is a
subgroup of $K$.  Indeed we have that for any $x \in K$, and $y \in K
\wedge N$ we have that $x^{-1}yx \in K$, since $K$ is a subgroup.  But
we also have that $x^{-1}yx \in N$, since $N$ is normal.  Hence $K
\wedge N$ is a normal subgroup of the subgroup $K$.  So we can
consider the quotient group $K/(K \wedge N)$.  Similarly, although $N$
is not a subgroup of $K$, we know that $N \vee K$ contains $N$, and
since $N$ is normal, we can consider the quotient group $(K \vee N)/N$.

\begin{theorem}[Third Isomorphism Theorem]
  Let $G$ be a group, $K$ a subgroup of $G$ and $N$ a normal subgroup 
  of $G$.  Then
  \[
    K/(K \wedge N) \isom (K \vee N)/N.
  \]
\end{theorem}
\begin{proof}
  Recall from Proposition~\ref{prop:normalproduct} that $K \vee N = 
  NK$ when $N$ is normal.
  
  We define a function $\alpha: K \to G/N$ by $\alpha(x) = Nx$.  This
  is a homomorphism since
  \[
    \alpha(xy) = Nxy = Nx \ast Ny = \alpha(x) \ast \alpha(y).
  \]
  The image of $\alpha$ is the set
  \[
    \alpha(K) = \{ Nx : x \in K \} = NK/N = (N \vee K)/N.
  \]
  Furthermore, the kernel of $\alpha$ is the set of $x$ such that 
  $\alpha(x) = N$, ie.~all $x \in K$ such that $Nx = N$.  But $Nx = N$ 
  if ans only if $x \in N$, so $x \in K \intersect N = K \wedge N$.
  
  So the First Isomorphism Theorem tells us that
  \[
    \alpha(K) = K/\ker \alpha,
  \]
  and so
  \[
    (N \vee K)/N \isom K/(K \wedge N).
  \]
\end{proof}


As you may expect, quotient groups can be used to shed some light on 
the structure of finite groups.

\begin{theorem}\label{thm:centrequotient}
  Let $G$ be a finite group which is not Abelian, and $Z(G)$ the
  centre of $G$.  Then $G/Z(G)$ cannot be cyclic.
\end{theorem}
\begin{proof}
  We first recall that $Z(G)$ is always normal, so $G/Z(G)$ is defined.
  If $G/Z(G)$ is cyclic, then we can find an element $t$ so that 
  $Z(G)t$ generates $G/Z(G)$, and so every coset of $Z(G)$ is of the 
  form $Z(G)t^{k}$ for some $k$. But then given arbitrary elements $x$ 
  and $y \in G$, we have that $x = ut^{k}$ and $y = vt^{l}$ for some 
  $u$ and $v \in Z(G)$.  So now, since $u$ and $v$ commute with all 
  elements of $G$,
  \[
    xy = ut^{k}vt^{l} = uvt^{k}t^{l} = uvt^{l}t^{k} = vt^{l}ut^{k} = 
    yx.
  \]
  So $G$ is Abelian, which is a contradiction.
\end{proof}

\begin{corollary}
  If $p$ is a prime number and $G$ is a group with order $p^{2}$, 
  then $G$ is Abelian.
\end{corollary}
\begin{proof}
  We know from Theorem~\ref{thm:centreprimepower} that the order of
  $Z(G)$ is either $p$ or $p^{2}$.  But if $|Z(G)| = p$, then $G$ is
  not Abelian, and we have that $|G/Z(G)| = |G|/|Z(G)| = p^{2}/p = p$,
  so $G/Z(G)$ must be a cyclic group of order $p$.  But the previous
  theorem showed that this cannot happen.
  
  Hence $|Z(G)| = p^{2}$, and so $G$ is Abelian.
\end{proof}

This corollary allows us to slightly simplify the proof of
Theorem~\ref{thm:groupsoforder9}, since the last paragraph can be replaced
by a reference to this corollary. Indeed, one can generalise that theorem to
all prime numbers.

\begin{theorem}
  If $p$ is a prime number and $G$ is a group with order $p^{2}$, 
  then $G$ is isomorphic to one of $C_{p^{2}}$ or $C_{p} \times C_{p}$.
\end{theorem}

\begin{proof}
  The previous corollary tells us that $G$ is Abelian.  If $G$ has an
  element of order $p^{2}$, then $G \isom C_{p^{2}}$.
  
  Otherwise every element of $G$ other then the identity $e$ has order $p$. 
  Let $a$ be such an element, and let $N = \langle a \rangle = \{1, a, a^{2},
  \ldots, a^{p-1}\}$. Lagrange's Theorem tells us that the quotient group $G/N$
  has order $p$, so $G/N \isom C_{p}$.  So there is some element $b$ with
  order $p$, such that $G/N$ is generated by the coset $Nb$, so $G/N = \{N, Nb,
  Nb^{2}, \ldots, Nb^{n-1}\}$.  Hence $G = \{a^{k}b^{l} : k, l = 0, 1, \ldots,
  p-1\}$ with $a^{p} = e$ and $b^{p} = e$.  Hence $G$ is isomorphic to
  $C_{p} \cross C_{p}$ via the isomorphism
  \[
    \alpha(u^{k},u^{l}) = a^{k}b^{l}.
  \]
\end{proof}



\subsection*{Exercises}

\begin{exercises}
  \item Consider the group $D_{8}$.  The subgroup $Z(D_{8}) = \{1,
    a^{2}\}$ is normal, since it is the centre of $D_{8}$.  Write down
    the Cayley table of $D_{8}/Z(D_{8})$.
    
    Explain why $D_{8}/Z(D_{8}) \isom V$.
  
  \item Consider the group $D_{12}$.  The subgroup $Z(D_{12}) = \{1,
    a^{3}\}$ is normal, since it is the centre of $D_{12}$.  Write down
    the Cayley table of $D_{12}/Z(D_{12})$.
    
    Explain why $D_{12}/Z(D_{12}) \isom D_{6}$?
  
  \item\label{ex:D2nquotient} Generalize the above two results, and show that if $n$ is 
    even, then $D_{2n}/Z(D_{2n}) \isom D_{n}$.
  
  \item Consider the normal subgroup $N = \{1, a^{2}, a^{4}\}$ of the 
    group $D_{12}$.  Write down the Cayley table of $D_{12}/N$.  Show 
    $D_{12}/N \isom V$.
  
  \item Let $G$ and $H$ be groups.  Show that the set
    \[
      K = \{ (g,e): g \in G \}
    \]
    is a normal subgroup of $G \cross H$.  Let $\pi: G \cross H \to H$
    be defined by $\pi(g,h) = h$ (this a homomorphism by
    Exercise~\ref{ex:directprodhoms}).  Use this homomorphism to show
    that
    \[
      (G \cross H) / K \isom H.
    \]
  
  \item Consider the additive group $\integers^{2}$.  Show that
    $\alpha(x,y) = 3x+2y$ is a homomorphism from $\integers^{2} \to
    \integers$, and $\ker \alpha = K = \{ (2k,-3k) : k \in integers \}$. 
    Show that
    \[
      \integers^{2} / K \isom \integers.
    \]
  
  \item Let $\alpha: G \to H$ be a homomorphism.  Use the first
    Isomorphism Theorem to show that $|\alpha(G)|$ divides both $|G|$ and
    $|H|$.  Show that if $|G|$ and $|H|$ have a greatest common
    divisor of $1$, then $\alpha(x) = e$ for all $x \in G$.
  
  \item Let $G$ be a group.  Recall the commutator subgroup $G'$
    defined in Exercise~\ref{ex:commutatorsubgroup} is normal.  Show
    that $G/G'$ is Abelian.
  
  \item Let $G$ be a group, and $N$ be a normal subgroup of $G$ such 
    that $G/N$ is Abelian.  Show that the commutator subgroup $G'$
    defined in Exercise~\ref{ex:commutatorsubgroup} is a normal subgroup
    of $N$.
\end{exercises}


\section{Automorphism Groups}

Recall that an automorphism of a group $G$ is an isomorphism from $G$ to
itself.  The set of all automorphisms of $G$ is denoted by $\Aut(G)$.  This
set is never empty since at the very least the identity map defined $\id(x)
= x$ is always an automorphism.

\begin{proposition}
  If $G$ is a group, then $(\Aut(G), \circ, \id)$ is a group, where $\circ$
  is function composition.
\end{proposition}
\begin{proof}
  Function composition of two automorphisms gives another automorphism, since
  if $\alpha$ and $\beta \in \Aut(G)$, then $\beta \circ \alpha: G \to G$ is
  an isomoprhism by Proposition~\ref{prop:homomorphismfacts}, so
  $\beta \circ \alpha \in \Aut(G)$.
  
  We already know that function composition is associative, so that group
  axiom holds.
  
  The identity map $\id$ is an identity under composition, since for any
  $x \in G$,
  \[
    (\id \circ \alpha)(x) = \id(\alpha(x)) = \alpha(x) \qquad \text{and}
    \qquad (\alpha \circ \id)(x) = \alpha(\id(x)) = \alpha(x),
  \]
  so we conclude that $\id \circ \alpha = \alpha \circ \id = \alpha$.
  
  Since $\alpha$ is an isomorphism, it has an inverse function which is also
  an isomorphism from $G$ to $G$ by Proposition~\ref{prop:homomorphismfacts}.
  We know that for any inverse function, $\alpha^{-1} \circ \alpha = \id =
  \alpha \circ \alpha^{-1}$, so $\alpha^{-1}$ is an inverse for $\alpha$ in
  the set of automorphisms.
\end{proof}

\begin{example}
  Consider the group $C_{4}$.  Any automorphism has to preserve the order of
  each of the elements, and since $a^{2}$ is the only element of order $2$,
  $\alpha(a^{2}) = a^{2}$ for every automorphism $\alpha$.  However, an
  automorphism could potentially swap $a$ and $a^{3}$.  Indeed, there are
  two automorphisms: $\id$ and the function $\alpha(x) = x^{-1}$, or more
  concretely,
  \[
    \begin{array}{c|c}
      x & \alpha(x) \\
      \hline
      1 & 1\\
      a & a^{3}\\
      a^{2} & a^{2}\\
      a^{3} & a
    \end{array}
  \]
  Verifying that this is an isomorphism is easy, since
  \[
    \alpha(a^{k}a^{l}) = \alpha(a^{k+l}) = a^{-k-l} = a^{-k}a^{-l}
    = \alpha(a^{k})\alpha(a^{l}),
  \]
  and it is clearly bijective.
  
  Since there are only two elements, $\Aut(C_{4}) \isom C_{2}$.
\end{example}

\begin{example}
  Consider the four-group $V$.  $V$ has $3$ elements of order $2$, $a$, $b$
  and $ab$, so an isomorphism could possibly interchange those elements.
  In fact any permutation of these three elements gives rise to a distinct
  automorphism.
  
  For example, the function $\alpha$ given by
  \[
    \begin{array}{c|c}
      x & \alpha(x) \\
      \hline
      1 & 1\\
      a & ab\\
      b & b\\
      ab & a
    \end{array}
  \]
  is an automorphism: it is clearly bijective, $\alpha(xx) = 1 =
  \alpha(x)\alpha(x)$ for any $x \in V$, since $x^{2} = 1$ for every element,
  $\alpha(1x) = \alpha(1)\alpha(x)$ for any $x \in V$, and the remaining cases
  are covered by
  \begin{align*}
      \alpha(a b) &= a = (ab)b = \alpha(a)\alpha(b)\\
      \alpha(a (ab)) &= b = (ab)a = \alpha(a)\alpha(ab)\\
      \alpha(b (ab)) &= ab = ba = \alpha(b)\alpha(ab).
  \end{align*}
  So $\alpha$ is indeed an automorphism.
  
  Given that every automorphism corresponds to a permutation of a set with
  3 elements, and function composition will correspond to composition of the
  permutations, we have that $\Aut(V) \isom S_{3}$.
\end{example}

As the previous example illustrates, finding all the automorphisms of a group
can be potentially difficult.  However, non-Abelian groups have a collection of
automorphisms which are easy to find.

\begin{proposition}
  If $G$ is a group, then the conjugation by x function
  \[
    \alpha_{x}(y) = x^{-1}yx
  \]
  is an automorphism.
\end{proposition}
\begin{proof}
  We first note that
  \[
    \alpha_{x}(yz) = x^{-1}yzx = x^{-1}yxx^{-1}zx = \alpha(y)\alpha(z),
  \]
  so $\alpha: G \to G$ is a homomorphism.  Now $\alpha_{x}(y) = e$ if and only if
  \[
    x^{-1}yx = e.
  \]
  But this implies that $y = xx^{-1} = e$.  So $\ker \alpha = \{e\}$, and
  so $\alpha_{x}$ is one-to-one.  Finally, $G$ is a normal subgroup of itself,
  so $x^{-1}Gx = G$, and so $\alpha_{x}(G) = G$.
  
  Hence $\alpha_{x}$ is an automorphism.
\end{proof}

We call such automorphisms \defn{inner automorphism}{automorphism!inner}, and
let $\Inn(G) = \{ \alpha_{x} : x \in G\}$ be the set of all inner automorphisms
of $G$.  Of course, not every one of these automorphisms need be distinct,
since if $x \in Z(G)$, then
\[
  \alpha_{x}(y) = x^{-1}yx = \id(y),
\]
for every $y$, and so $\alpha_{x} = y$.  So in particular, the collection of
inner automorphisms is just $\{\id\}$ if $G$ is Abelian.  In fact the set of
inner automorphisms is very closely related to the centre.

\begin{theorem}
  Let $G$ be a group.  Then $\Inn(G)$ is a subgroup of $\Aut(G)$, the function
  $\beta: x \mapsto \alpha_{x^{-1}}$ is a homomorphism from $G$ onto $\Inn(G)$, and
  \[
    \Inn(G) \isom G/Z(G).
  \]
\end{theorem}
\begin{proof}
  We start by observing that the function $\beta$ is a homomorphism:
  \[
    \beta(xy)(z) = \alpha_{(xy)^{-1}}(z) = xyzy^{-1}x = x(\alpha_{y^{-1}}(z))x
    = \alpha_{x^{-1}}(\alpha_{y^{-1}}(z)) = (\beta(x) \circ \beta(y))(z)
  \]
  for all $z \in G$, so $\beta(xy) = \beta(x) \circ \beta(y)$.
  
  Now the image of $G$ under $\beta$ is precisely $\Inn(G)$, so $\Inn(G)$ is
  a subgroup.  Furthermore, $\beta(x) = \id$ if and only if
  $\beta(x)(z) = z$ for all $z \in G$, or equivalently,
  \[
    xzx^{-1} = z
  \]
  for all $z \in G$.  So $x \in \ker \beta$ if and only if $x \in Z(G)$.
  Therefore $\ker \beta = Z(G)$.
  
  The First Isomorphism Theorem then tells us that
  \[
    \beta(G) \isom G/\ker \beta,
  \]
  so
  \[
    \Inn(G) \isom G/Z(G).
  \]
\end{proof}
\begin{corollary}
  If $G$ is a finite group which is not Abelian, then $\Inn(G)$ is never
  cyclic.
\end{corollary}
\begin{proof}
  This follows immediately from the above theorem and
  Theorem~\ref{thm:centrequotient}.
\end{proof}

\begin{example}
  The group $D_{6}$ has centre $Z(D_{6}) = \{1\}$, so $\Inn(D_{6}) \isom D_{6}$.
  In fact, since $D_{6}$ has $2$ elements of order $3$, and $3$ elements of
  order $2$, if $\alpha \in \Aut(D_{6})$ we must have
  \begin{align*}
    \alpha(a) &= a \text{ or } a^{2}\\
    \alpha(b) &= b, ab \text{ or } a^{2}b\\
  \end{align*}
  and since $\alpha(a^{k}b^{l}) = (\alpha(a))^{k}(\alpha(b))^{l}$, these
  choices completely determine the automorphism.  Therefore $|\Aut(D_{6})|
  \le 6$, and since $|\Inn(D_{6})| = |D_{6}| = 6$, every automorphism of
  $D_{6}$ is inner.
\end{example}

\begin{example}
  The group $D_{8}$ has centre $Z(D_{8}) = \{1, a^{2}\}$, so $|\Inn(G)| =
  |D_{8}|/|Z(D_{8})| = 8/2 = 4$.  But we know that $D_{8}/Z(D_{8})$ cannot be
  cyclic, so $\Inn(G) \isom V$.
\end{example}

It's worthwhile noting that if $H$ is a normal subgroup of $G$, then the
inner automorphisms of $G$ are automorphisms of $H$ when restricted to just $H$.
This follows because if $x \in G$, we have that
\[
  H = x^{-1}Hx = \alpha_{x}(H),
\]
so $\alpha_{x}$, regarded as a function defined on $H$, must be a bijective
homomorphism onto $H$, ie.~an element of $\Aut(H)$.

A \defn{characteristic subgroup}{subgroup!characteristic} $H$ of a group $G$
is a subgroup which is invariant under every automorphism of $G$, in other
words $\alpha(H) = H$ for all $\alpha \in \Aut(G)$.  Every characteristic
subgroup is automatically normal, since if $x \in G$, so $\alpha_{x} \in
\Inn(G)$, then
\[
  H = \alpha_{x}(H) = x^{-1}Hx.
\]
The centre of $G$ is always characteristic, since any isomorphism always
maps the centre to the centre.  The trivial subgroups $G$ and $\{e\}$ are also
always characteristic.

\begin{example}
  In the group $D_{6}$, the subgroup $H = \langle a \rangle$ is characteristic,
  since any automorphism must map $1$ to $1$ and elements of order $3$ to elements
  of order $3$, ie.~the set $\{a, a^{2}\}$ maps onto $\{a, a^{2}\}$.
\end{example}

\begin{proposition}
  If $G$ is a group, $N$ is a normal subgroup of $G$ and $H$ is a
  characteristic subgroup of $N$, then $H$ is a normal subgroup of $G$.
\end{proposition}
\begin{proof}
  Since $\alpha_{x} \in \Inn(G)$ is an automorphism of $N$, and $H$ is
  characteristic, then
  \[
    x^{-1}Hx = \alpha_{x}(H) = H.
  \]
\end{proof}

The inner automorphisms which come from characteristic subgroups are also
interesting.

\begin{theorem}
  Let $G$ be a group, and let $H$ be a characteristic subgroup of $G$.
  Then the set of inner automorphisms of the form $\{\alpha_{x} : x \in H \}$
  is a normal subgroup of $\Aut(G)$.
\end{theorem}
\begin{proof}
  We first note that since $\beta(x) = \alpha_{x^{-1}}$ is a homomorphism
  from $G$ to $\Inn(G)$, $\beta(H) = \{\alpha_{x} : x \in H \}$ is a subgroup
  of $\Aut(G)$.
  
  Let $x \in H$. Given any automorphism $\alpha$, we have that for any
  $z \in G$,
  \begin{align*}
    (\alpha^{-1} \circ \alpha_{x} \circ \alpha)(z)
      &= \alpha^{-1}(\alpha_{x}(\alpha(z)))\\
      &= \alpha^{-1}(x^{-1}\alpha(z)x)\\
      &= \alpha^{-1}(x^{-1})\alpha^{-1}(\alpha(z))\alpha^{-1}(x)\\
      &= (\alpha^{-1}(x))^{-1}z\alpha^{-1}(x)\\
      &= \alpha_{\alpha^{-1}(x)}(z).
  \end{align*}
  But since $H$ is characteristic, $\alpha^{-1}(x) \in H$, so $\alpha^{-1}
  \circ \alpha_{x} \circ \alpha \in \beta(H)$.  Hence $\beta(H)$ is a normal
  subgroup of $\Aut(G)$.
\end{proof}

\begin{corollary}
  If $G$ is a group, then $\Inn(G)$ is a normal subgroup of $\Aut(G)$.
\end{corollary}

\subsection*{Exercises}

\begin{exercises}
  \item Find the automorphism group of $C_{5}$.  Does $C_{5}$ have 
    any non-trivial inner automorphisms?
  
  \item Find the automorphism group of $C_{6}$.
  
  \item Find $\Aut(Q_{8})$.
  
  \item Show that the additive group of integers has only two
    automorphisms: $\id$ and $\iota(x) = -x$.  Conclude that
    $\Aut(\integers) \isom C_{2}$.
  
  \item Let $\integers_{m}$ be the additive group of integers modulo $m$,
    and let $\alpha : \integers_{m} \to \integers_{m}$ be a homomorphism.
    
    Show that if $\alpha(1) = k$, then $\alpha(x) = kx$.
    
    Show that $\alpha$ is an automorphism if and only if the greatest common
    divisor of $k$ and $m$ is $1$.
    
    Let $\alpha_{k}(x) = kx$ on $\integers_{m}$.  Show that $\alpha_{k} \circ
    \alpha_{j} = \alpha_{kj}$.
    
    Show that $\Aut(\integers_{m})$ is isomorphic to the multiplicative
    group $\integers_{m}^{*} = \{x : \gcd(x,m) = 1\}$.
    
    Show that $\Aut(\integers_{8}) \isom V$.
    
    Show that if $m = pq$ where $p$ and $q$ are distinct primes, then
    $|\Aut(\integers_{m})| = (p-1)(q-1)$.
  
  \item Show that if $(G, +, 0)$ is any finite Abelian group (written using
    additive notation), then the function
    \[
      \alpha_{k}(x) = \underbrace{x + x + x + \cdots + x}_{k\text{ times}}
    \]
    is a homomorphism from $G$ to $G$.  Show that if $k$ and $|G|$ have
    a greatest common divisor of $1$, then $\alpha_{k}(x) = 0$ if and only
    if $x = 0$.  Show that in this case, $\alpha_{k}$ is an automorphism.
    
    Show that $\alpha_{k} \circ \alpha_{j} = \alpha_{kj}$, so that the
    function $\beta : \integers_{|G|}^{*} \to \Aut(G)$ defined by
    $\beta(k) = \alpha_{k}$ is a homomorphism.
    
  \item Show that if $G$ is a group, the function 
    $\iota(x) = x^{-1}$ is an automorphism if and only if $G$ is Abelian.
  
  \item Show that if $G \isom H$, then $\Aut(G) \isom \Aut(H)$.
  
  \item Show that if $n$ is odd then $\Inn(D_{2n}) \isom D_{2n}$, 
    while if $n$ is even then $\Inn(D_{2n}) \isom D_{n}$ (Hint: you 
    may use Exercise~\ref{ex:D2nquotient} to prove this).
\end{exercises}


\section{Extension: Category Theory}

If you think about the basic outlines of the theory which we have developed
so far, you should notice some similarities between the theories of groups,
vector spaces, partial orders and lattices.  At a very abstract level we
have:

\begin{tabular}{l|l}
  Sets & Functions \\
  \hline
  Groups & Homomorphisms \\
  Vector Spaces & Linear Transformations \\
  Partially Ordered Sets & Order-preserving Functions \\
  Lattices & Lattice Homomorphisms
\end{tabular}

There are also similarities beyond this: in all cases there is the notion
of ``isomorphism'' between appropriate types of sets and the notion of a ``sub-''
object (like a subgroup or subspace), for example.

The model that we should keep in mind ofr what we are about to define
is simply a minimal set of axioms which sets and functions will satisfy:
\begin{enumerate}
  \item each function has a domain and codomain,
  
  \item if $\dom f = \cod g$ we can compose the functions,
  
  \item function composition is associative,
  
  \item for each set $X$, there is an identity function $\id_{A}: X
    \to X$, and this identity function has the property  $f \circ 
    \id_{A} = f$ and $\id_{A} \circ g = g$.
\end{enumerate}

Notice that group homomorphisms also satisfy all of these conditions.

\begin{definition}
  A \defn{category}{category} $\mathcal{C}$ consists of a set of
  \defn{objects}{objects}, $\mathcal{O}$; a set of
  \defn{arrows}{arrow} or \defn{morphisms}{morphism} $\mathcal{A}$,
  two functions
  \[
    \cod: \mathcal{A} \to \mathcal{O} \qquad \text{and} \qquad \dom:
    \mathcal{A} \to \mathcal{O}
  \]
  which assign to each arrow an object called, respectively, the
  \defn{domain}{domain} and \defn{codomain}{codomain} of the arrow; 
  a function
  \[
    \id: \mathcal{O} \to \mathcal{A},
  \]
  which assigns to each object $A$ an \defn{identity
  arrow}{arrow!identity} $\id_{A}$; and a
  \defn{composition}{composition} operation that assigns each to pair
  of arrows $(\alpha,\beta)$ with $\dom \alpha = \cod \beta$ an arrow
  $\gamma = \alpha \circ \beta$ with $\cod \gamma = \cod \alpha$ and
  $\dom \gamma = \dom \beta$.
  
  We will write $f : A \to B$ to denote that an arrow $f$ has domain 
  $A$ and codomain $B$, or diagramatically, write:
  \[
    A \to B
  \]
  
  These have to satisfy the following axioms:
  \begin{theoremenum}
    \item Associativity: if $f: B \to A$, $g: C \to B$, and $h: D \to
    C$, then $(f \circ g) \circ h = f \circ (g \circ h)$,
    
    \item Identity: for any $f : A \to B$, $f \circ \id_{A} = f$; and
    for any $g: B \to A$, $\id_{A} \circ g = g$,
  \end{theoremenum}
\end{definition}

Notice that these axioms are very similar to the definition of a 
group, but with added complexity because of the neccessity of dealing 
with the domains and codomains, and with no inverse axiom.

Categories are very closely related to directed graphs, and we can
often represent parts of a category graphically.  Many key facts in
category can be represented by succinctly by \defn{commuting 
diagrams}{commuting diagram}.  The key property of a commuting diagram 
is that any path following the arrows through a diagram that start 
from the same object and ends at the same object are equal.  For 
example, the associativity axiom can be represented by the following 
commuting diagram:

XXX Picture

Similarly, the following two diagrams represent the identity axioms:

XXX Picture

As is the case for associative binary operations, the associativity 
axiom for categories means that it doesn't matter where we put the 
parentheses in a composition of multiple arrows.  We can also show 
that for each $A$, $\id_{A}$ is unique.

\begin{example}
  The following are all categories:
  \begin{enumerate}
    \item $\mathbf{Set}$: the category with objects being all sets 
    contained in some universe $U$ and arrows being all functions 
    on those sets.
    
    \item $\mathbf{Grp}$: the category with objects being all groups
    contained in some universe $U$, and arrows being all group
    homomorphisms.
    
    \item $\mathbf{Abl}$: the category with objects being all Abelian
    groups contained in some universe $U$, and arrows being group
    homomorphisms between them.
    
    \item $\mathbf{Vec}(\field)$: the category with objects being all
    vector spaces over a field $\field$ (contained in some universe
    $U$), and arrows being linear transformations between them.
    
    \item $\mathbf{Lat}$: the category with objects being all lattices 
    contained in some universe $U$, and arrows being lattice 
    homomorphisms.
    
    \item $\mathbf{Set_{*}}$: the category whose objects are
    \defn{pointed sets}{set!pointed}: pairs $(X,x)$, where $X$ is a
    set contained in some universe $U$, and $x \in X$ is some
    distinguished point; and whose arrows are functions which map
    distingushed points to distinguished points: if $(X, x)$ and $(Y,
    y)$ are pointed sets, then $f:X \to Y$ is a morphism if and only
    if $f(x) = y$.
  \end{enumerate}
\end{example}

There are, of course, many, many other categories.  Indeed, whenever 
you encounter a new mathematical object, particularly in algebra, you 
should ask yourself ``what is the category that goes with this?''  If 
you can establish this, then you can immediately get a number of basic 
results for free.

For a good theory which encompasses the fundamentals of functions on
sets and group homomorphisms, we need to have more than just
composition and identity.  We also need to determine analogues of
injective functions (or group monomorphisms), surjective functions (or
group epimorphisms), and most importantly bijection (or group
isomorphisms).

\begin{definition}
  Let $\mathcal{C}$ be a category with objects $\mathcal{O}$ and
  arrows $\mathcal{A}$.  An arrow $\alpha: A \to B$ is
  \defn{invertible}{arrow!invertible} if there is an arrow
  $\alpha^{-1}: B \to A$ such that $\alpha^{-1} \circ \alpha =
  \id_{A}$ and $\alpha \circ \alpha^{-1} = \id_{B}$.  We say that two
  objects $A$ and $B$ are \defn{isomorphic}{isomorphic!objects} if there is an
  invertible arrow $\alpha: A \to B$.
  
  An arrow $\alpha: A \to B$ is \defn{monic}{arrow!monic} if whenever
  there are arrows $\beta_{1}$ and $\beta_{2}: C \to A$ such that
  $\alpha \circ \beta_{1} = \alpha \circ \beta_{2}$, then $\beta_{1} =
  \beta_{2}$ (ie.~we can cancel $\alpha$ on the left).  An arrow
  $\alpha: A \to B$ is \defn{epi}{arrow!epi} if whenever there are
  arrows $\beta_{1}$ and $\beta_{2}: B \to C$ such that $\beta_{1}
  \circ \alpha = \beta_{2} \circ \alpha$, then $\beta_{1} = \beta_{2}$
  (ie.~we can cancel $\alpha$ on the right).
  
  A \defn{right inverse}{arrow!inverse!right} of an arrow $\alpha: A
  \to B$ is an arrow $\rho: B \to A$ such that $\alpha \circ \rho =
  \id_{B}$.  A \defn{left inverse}{arrow!inverse!left} of an arrow
  $\alpha: A \to B$ is an arrow $\lambda: B \to A$ so that $\lambda
  \circ \alpha = \id_{A}$.  A right inverse of $\alpha$ is also called 
  a \defn{section}{section} of $\alpha$, while a left-inverse is 
  called a \defn{retraction}{retraction} of $\alpha$.
  
  If an object $A$ has the property that for any object $B$ we have a
  exactly one arrow $B \to A$, it is said to be
  \defn{terminal}{object!terminal}.  If instead it has the property
  that there is exactly one arrow from $A \to B$, then it is said to
  be \defn{initial}{object!initial}.
\end{definition}


\begin{example}
  In the categories of $\mathbf{Set}$ and $\mathbf{Grp}$, we have 
  that following correspondence:
  
  \begin{tabular}{l|ll}
     & $\mathbf{Set}$ & $\mathbf{Grp}$ \\
     \hline
     invertible arrow & bijective function & isomorphism \\
     monic arrow & injective function & monomorphism \\
     epi arrow & surjective function & epimorphism \\
     terminal object & any set with one element & any group with one 
     element \\
     initial object & the empty set & any group with one element
  \end{tabular}
  
\end{example}

We can prove a number of facts immediately:

\begin{proposition}
  If $\mathcal{C}$ is a category, then
  \begin{theoremenum}
    \item if an arrow $\alpha$ has a right inverse, then it is epi, 

    \item if an arrow $\alpha$ has a left inverse, then it is monic,
    
    \item an arrow $\alpha$ has both a left and right inverse if and
    only if it is invertible,
    
    \item if an arrow is invertible, it is both epi and monic.

  \end{theoremenum}
\end{proposition}

\begin{proof}
  (i) Let $\alpha: A \to B$ and let $\rho: B \to A$ be a right inverse
  of $\alpha$.  Then given any arrows $\beta_{1}$ and $\beta_{2}: B
  \to C$ such that $\beta_{1} \circ \alpha = \beta_{2} \circ \alpha$,
  we have that
  \begin{align*}
     \beta_{1} &= \beta_{1} \circ \id \\
     &= \beta_{1} \circ \alpha \circ \rho\\
     &= \beta_{2} \circ \alpha \circ \rho \\
     &= \beta_{2} \circ \id\\
     &= \beta_{2}.
  \end{align*}
  So $\alpha$ is epi.
  
  (ii) The proof of this is left as an exercise.
  
  (iii) If $\alpha$ is invertible, then the inverse is both a left 
  and right inverse, so $\alpha$ has both a left and right inverse.
  
  On the other hand, if $\alpha: A \to B$ and $\rho: B \to A$ and
  $\lambda: B \to A$ be right and left inverses of $\alpha$,
  respectively, then
  \begin{align*}
    \rho &= \id \circ \rho \\
         &= \lambda \circ \alpha \circ \rho \\
&= \lambda \circ \id \\
&= \lambda
  \end{align*}
  So $\rho = \lambda$ is an inverse for $\alpha$.
  
  (iv) This follows immediately from the first three parts.
  
\end{proof}

In the previous chapter, key results came from looking at an object in
another category which corresponded to the group, such as looking at
the subgroup lattice of a group in the category of lattices.  These
sorts of correspondences between categories were first recognised in
the study of algebraic topology, and are extremely powerful.  Indeed,
in some sense they are what justifies the study of categories as a
distinct topic.

\begin{definition}
  Let $\mathcal{C}_{1}$ and $\mathcal{C}_{2}$ be two categories with
  object sets $\mathcal{O}_{1}$ and $\mathcal{O}_{2}$, and arrow sets
  $\mathcal{A}_{1}$ and $\mathcal{A}_{2}$, respectively.  A 
  \defn{functor}{functor} $\mathcal{F} : \mathcal{C}_{1} \to 
  \mathcal{C}_{2}$ is a pair of functions $\mathcal{F_{O}} : 
  \mathcal{O}_{1} \to \mathcal{O}_{2}$ and $\mathcal{F_{A}} : 
  \mathcal{A}_{1} \to \mathcal{A}_{2}$ such that if $A \in 
  \mathcal{O_{1}}$, $\alpha$ and $\beta \in \mathcal{A_{1}}$, then
  \begin{alignat*}{2}
    \dom \mathcal{F_{A}}(\alpha) &= \mathcal{F_{O}}(\dom \alpha) & \cod
    \mathcal{F_{A}}(\alpha) &= \mathcal{F_{O}}(\cod \alpha)\\
    \mathcal{F_{A}}(\alpha \circ \beta) &= (\mathcal{F_{A}}(\alpha)
    \circ \mathcal{F_{A}}(\beta) \qquad & \mathcal{F_{A}}(\id_{A}) &=
    \id_{\mathcal{F_{O}}}(A)
  \end{alignat*}
  whenever $\alpha \circ \beta$ is defined.
\end{definition}

We usually don't distinguish between the functor, and the functions 
on the sets of objects and arrows, simply representing each of them 
by a single symbol $\mathcal{F}$.

\begin{example}\label{eg:funtorgrplat}
  Let $\mathbf{Grp}$ and $\mathbf{Lat}$ be the categories of groups
  and lattices defined earlier.  Then we know that the function
  $\mathcal{F_{O}}$ defined by $\mathcal{F_{O}}(G) = \Sub(G)$ is a
  function from the objects of $\mathbf{Grp}$ to the objects of
  $\mathbf{Lat}$.  Corollary~\ref{cor:grouphomtolathom} tells us that
  if we have a group homomorphism $\alpha: G \to H$, then we have a
  corresponding lattice homomorphism $\overline{\alpha} : \Sub(G) \to
  \Sub(H)$.  So we define $\mathcal{F_{A}}(\alpha) =
  \overline{\alpha}$.  This gives the first two of the four conditions
  that we need for $\mathcal{F}$ to be a functor.
  
  It is easily verified that $\mathcal{F_{A}}(\id_{G}) =
  \id_{\mathcal{F_{O}}(G)}$, while it is a little more work to check
  the remaining composition condition.  They do hold, however, so we 
  have a functor between the categories.
  
  This is not the only possible functor between these two categories,
  since we could also consider a functor which maps a group $G$ to the
  lattice consisting of the power set $\mathcal{P}(G)$ with meet and
  join being intersection and union.
\end{example}

\begin{example}
  The maps $\mathcal{F}(G) = G$ and $\mathcal{F}(\alpha) = \alpha$ 
  give a functor $\mathcal{F} : \mathbf{Grp} \to \mathbf{Set}$.
\end{example}

Functors such as the one in the last example are called
\defn{forgetful functors}{functor!forgetful} because we are forgetting
about all the extra structure that a group has and treating it just as
a set, and regarding homomoprhisms simply as functions.  Whenever we
have a category whose objects and arrows are specializations of
another categories objects and arrows we get a forgetful functor which
strips this additional structure away.

\begin{example}
  There is a forgetful functor $\mathcal{F} : \mathbf{Abl} \to
  \mathbf{Grp}$, since Abelian groups are simply groups with an
  additional requirement of commutativity, and homomorphisms between
  Abelian groups are still homomoprhisms.
\end{example}

When looking at the question of whether or not two objects within a
category are isomorphic or not, functors can help us say that that the two
objects are not isomorphic.

\begin{proposition}
  Let $\mathcal{C}$ be a category, and let $A$ and $B$ be two objects in
  $\mathcal{C}$.  If there is another category $\mathcal{D}$ and a functor
  $\mathcal{F}: \mathcal{C} \to \mathcal{D}$ such that $\mathcal{F}(A)$ and
  $\mathcal{F}(B)$ are not isomorphic in $\mathcal{D}$, then $A$ and $B$ are
  not isomorphic in $\mathcal{C}$.
\end{proposition}
\begin{proof}
  If $A$ and $B$ are isomorphic, then we have an invertible arrow $\alpha : A
  \to B$ and its inverse $\alpha^{-1} : B \to A$.  If $\mathcal{F}: \mathcal{C}
  \to \mathcal{D}$ is any functor, then
  \[
    \mathcal{F}(\alpha) \circ \mathcal{F}(\alpha^{-1}) = \mathcal{F}(\alpha
    \circ \alpha^{-1}) = \mathcal{F}(\id_{B}) = \id_{\mathcal{F}(B)}.
  \]
  Similarly $\mathcal{F}(\alpha^{-1}) \circ \mathcal{F}(\alpha) =
  \id_{\mathcal{F}(A)}$, and so $\mathcal{F}(\alpha) : \mathcal{F}(A) \to
  \mathcal{F}(B)$ has an inverse arrow $\mathcal{F}(\alpha^{-1})$, and hence
  $\mathcal{F}(A)$ is isomorphic to $\mathcal{F}(B)$.
  
  Hence if $\mathcal{F}(A)$ is not isomorphic to $\mathcal{F}(B)$, then $A$ and
  $B$ are not isomorphic.
\end{proof}

This very general result is at the core of many of the techniques we have
for distinguishing groups which are not isomorphic.  For example, the fact
that two groups with different subgroup lattices are not isomorphic is an
immediate corollary of this proposition, together with the functor of
Example~\ref{eg:funtorgrplat}.  The fact that two groups of different orders
are not isomorphic is an immediate corollary of this proposition, using the
forgetful functor from $\mathbf{Grp}$ to $\mathbf{Set}$.

Unfortunately, this doesn't help us in showing when two groups are
isomorphic, since there are many functors available, so checking every
possible functor is impossible.

The category of groups is not the only category of interest in mathematics,
of course, so it is useful to observe that in every category we find many
naturally occurring groups:

\begin{proposition}
  Let $\mathcal{C}$ be a category, and $A$ an object in $\mathcal{C}$.  Then
  the triple $(\Aut(A), \circ, \id_{A})$, where $\Aut(A)$ is the set of
  invertible arrows from $A$ to $A$, is a group.  We call this the
  \defn{automorphism group}{group!automorphism!of an object} of $A$.
\end{proposition}
\begin{proof}
  From the definition of a category, when $\circ$ is restricted to $\Aut(A)$,
  it is an associative binary operation, and $\id_{A}$ is an identity.  So the
  only thing that needs to be checked is that there is an inverse for every
  element, but this is guaranteed by the assumption that our arrows are all
  invertible.
\end{proof}

\begin{example}
  In the category $\mathrm{Grp}$, $\Aut(G)$ is precisely the automorphism
  group of $G$, as discussed earlier.
\end{example}

\begin{example}
  In the category of finite dimensional real vector spaces and linear
  transformations, we have that $\Aut(V)$ is the set of all invertible
  linear transformations from $V$ to $V$.  If $V$ has dimension $n$, then
  this group is isomorphic to $GL_{n}(\reals)$.
\end{example}

\begin{example}
  In the category $\mathrm{Set}$, the group $\Aut(X)$ consists of all
  bijections of $X$ onto itself, or in other words, the group of all
  permutations of $X$.  If $|X| = n$, then $\Aut(X) \isom S_{n}$.
\end{example}

\begin{example}\label{eg:symmetrycategory}
  One can consider a category whose objects are all subsets of $\reals^{n}$,
  and whose arrows are isometries which map one subset onto another.
  
  In this category, the automorphism group of an object is the set of all
  symmetries of the object.
\end{example}

As the above examples illustrate, the concept of an automorphism group
generalises the concept of symmetries that we first introduced in Chapter 1.
Whenever you see a new type of mathematical object being introduced, there
is usually a category associated with it, and hence there is some sort of
automorphism group associated with each object.  Given the wide variety of categories
that are of interest in mathematics, this underlines the importance of group
theory.

\begin{proposition}
  Let $\mathcal{C}$ be a category, and $A$ and $B$ two objects in
  $\mathcal{C}$.  If $A$ and $B$ are isomorphic, then $\Aut(A)$ and
  $\Aut(B)$ are isomorphic.
\end{proposition}
\begin{proof}
  Let $\alpha: A \to B$ be an invertible arrow.  Then we define a function
  $\overline{\alpha} : \Aut(A) \to \Aut(B)$ by
  \[
    \overline{\alpha}(\beta) = \alpha \circ \beta \circ \alpha^{-1}.
  \]
  One can easily verify that $\overline{\alpha}(\beta)$ is an
  invertible arrow, and hence the function is indeed into $\Aut(B)$, and
  furthermore
  \[
    \overline{\alpha}(\beta) \circ \overline{\alpha}(\gamma) = \alpha \circ \beta \circ
      \alpha^{-1} \circ \alpha \circ \gamma \circ \alpha^{-1} = \alpha \circ \beta
      \circ \gamma \circ \alpha^{-1} = \overline{\alpha}(\beta \circ \gamma),
  \]
  so $\overline{\alpha}$ is a homomorphism.  Similarly,
  $\overline{\alpha^{-1}} : \Aut(B) \to \Aut(A)$ is a homomorphism, and
  furthermore $\overline{\alpha} \circ \overline{\alpha^{-1}} =
  \id_{\Aut(B)}$ and $\overline{\alpha^{-1}} \circ \overline{\alpha} =
  \id_{\Aut(A)}$, so $\overline{\alpha}$ is an isomorphism.
\end{proof}

This means that in any time you have a category, you can use the automoprhism
groups to distinguish non-isomoprhic objects in the category via the
contrapositive of this result: if the automorphism groups are not
isomorphic, the objects are not isomorphic.

\begin{definition}
  If $\mathcal{C}$ is a category, and $A$ is an object in $\mathcal{C}$,
  then an \defn{action}{action} of a group $G$ on $A$ is a homomorphism
  $\alpha: G \to \Aut(A)$.
\end{definition}

For most categories of interest the objects of the category are
sets with additional structure and the arrows are functions with additional
conditions that they must satisfy. This means that $\alpha(g)$ is a
function of some sort from $A$ to $A$.  To avoid confusion, it is common to
write $\alpha(g) = \alpha_{g}$, so that we can use the less confusing
notation $\alpha_{g}(x)$ instead of $(\alpha(g))(x)$ to represent the image
of an element of $A$ under this automorphism.  In this case we have that
$\alpha_{gh} = \alpha_{g} \circ \alpha_{h}$.

\begin{example}\label{eg:inneraction}
  Given a group $G$, the map $\alpha: G \to \Aut(G)$ defined by $\alpha(g) =
  \alpha_{g}$, where $\alpha_{g}(x) = g^{-1}xg$ is an action of $G$ on
  itself.  The image of the homomorphism is $\Inn(G)$, and the kernel is
  $Z(G)$.
\end{example}

\begin{example}
  Given a group $G$ then for each $g \in G$ we have bijective functions
  $\lambda_{g}(x) = g^{-1}x$ and $\rho_{g}(x) = xg$.  Since $\lambda_{gh} =
  \lambda_{g} \circ \lambda_{h}$, and $\rho_{gh} = \rho_{g} \circ \rho_{h}$,
  the functions $\lambda: g \mapsto \lambda_{g}$ and $\rho: g \mapsto
  \rho_{g}$ are actions of $G$ on itself, when we consider it as an object
  in the category $\mathrm{Set}$.  We call these the
  \defn{left}{action!left} and \defn{right actions}{action!right} of $G$ on
  itself, respectively.
\end{example}

In a category in which the objects are sets with additional structure, and the
arrows are functions which satisfy some additional conditions, we define the
\defn{orbit}{orbit} of an element $x$ under an action of a group $G$ to be
the set
\[
  O(x) = \{ \alpha_{g}(x) : g \in G\}.
\]
We can define an equivalence relation using an action by $x \sim y$ if and
only if $y = \alpha_{g}(x)$ for some $g \in G$.  It is easy to verify that
this is indeed an equivalence relation, since the facts that $x =
\alpha_{e}(x)$; that if $x = \alpha_{g}(y)$ then $y = \alpha_{g^{-1}}(x)$; and
that if $x = \alpha_{g}(y)$ and $y = \alpha_{g}(z)$, then
\[
  x = \alpha_{g}(y) = \alpha_{g}(\alpha_{h}(z)) = \alpha_{gh}(z)
\]
give reflexivity, symmetry and transitivity, respectively.  We then have
that the orbit of $x$ is simply the equivalence class of $x$, ie.
\[
  O(x) = [x]_{\sim}.
\]

\begin{example}
  The orbit of an element $x \in G$ under the action of
  Example~\ref{eg:inneraction} is the conjugacy class of $x$, ie.~$O(x) =
  C(x)$.
\end{example}

\begin{example}
  The orbit of an element $x \in G$ under the left action of $G$ is $G$.
  
  However if we restrict the left action to some subgroup $H$ of $G$, so
  we only consider functions of the form $\lambda_{h}$ for $h \in H$, then
  $\lambda$ is an action of $H$ on $G$, and then $O(x)$ is the right coset
  $Hx$.
\end{example}

\begin{example}
  In the category of Example~\ref{eg:symmetrycategory}, consider closed disc
  of radius 1 with centre at the origin as the object $A$.  The
  automorphism group of this object consists of
  rotations $R_{\theta}$ and reflections $H_{\psi}$ of the disc, as discussed
  in Example~\ref{eg:circlesymmetry}.  There is an action of the additive
  group of real numbers $\alpha: \reals \to \Aut(A)$ given by $\alpha_{x} = R_{y}$,
  where $x = 2\pi k + y$, with $k \in \integers$ and $y \in [0,2\pi)$.
  
  In this case the orbit of the point $(1,0)$ is the unit circle, since we
  can find a rotation that maps that point to any other on the unit circle,
  and every rotation is in the image of the action.
  
  Indeed, the orbit of any point in the disc will be a circle.
\end{example}

The previous example should help you understand why an orbit is called an
orbit.

\section{Semidirect Products}

Group automorphisms and the actions of groups allow us to generalise the
notion of the direct product introduced in Chapter 2.  Let $G$ and $H$ be
groups, and let $\alpha$ be an action of $H$ on $G$.  We supply a binary
operation $\ast$ for the set
\[
  G \cross H = \{ (g,h) : g \in G, h \in H \}
\]
by
\[
  (g_{1}, h_{1}) \ast (g_{2}, h_{2}) = (\alpha_{h_{2}}(g_{1})g_{2}, h_{1}h_{2}).
\]

\begin{proposition}
  Let $G$ and $H$ be groups, and let $\alpha$ be an action of $H$ on $G$.
  The binary operation $\ast$ defined above makes $(G \cross H,
  \ast, (e,e))$ a group.  We call this group a \defn{semidirect product}{product!semidirect}
  of $G$ and $H$, and denote it symbolically by $G \rtimes_{\alpha} H$.
  If $H$ is a subgroup of $\Aut(G)$ acting in the obvious way, then we call
  this group the \defn{extension}{group extension} of $G$ by $H$.
  
  Moreover
  \begin{theoremenum}
    \item the subset $G' = \{(g,e): g \in G\}$ is a characteristic
    subgroup of the semidirect product which is isomorphic to $G$;
    
    \item $(G \rtimes_{\alpha} H) / G' \isom H$;
    
    \item the subset $H' = \{(e,h) : h \in H\}$ is a subgroup of the semidirect
    product which is isomorphic to $H$;
    
    \item the conjugate of an element $(g,e)$ of $G'$ by an element $(e,h)$
    of $H'$ is the element $(\alpha_{h}(g), e)$ of $G'$.  In other words,
    conjugation by elements of $H'$ is equivalent to the action $\alpha$ on $G$.
  \end{theoremenum}
\end{proposition}
\begin{proof}
  Showing that the semidirect product is a group is left as an easy exercise.
  It is also left as an exercise to show that $G' \isom G$ and $H' \isom H$.
    
  We note that the inverse of the element $(g,h)$ is the element
  $(\alpha_{h^{-1}}(g^{-1}), h^{-1})$.
  
  The function $\beta(g,h) = h$ from $G \rtimes_{\alpha} H$ to $H$ is a
  homomorphism, since
  \begin{align*}
    \beta((g_{1}, h_{1}) \ast (g_{2}, h_{2})) &= \beta(\alpha_{h_{2}}(g_{1})g_{2}, h_{1}h_{2}) \\
    &= h_{1}h_{2}\\
    &= \beta(g_{1}, h_{1})\beta(g_{2}, h_{2})).
  \end{align*}
  Now
  \[
    \ker \beta = \{ (g, h) : \beta(g,h) = h = e\} = \{ (g,e) : g \in G \} = G',
  \]
  so $G'$ is a normal subgroup, and furthermore the
  First Isomorphism Theorem says that since $\beta$ is onto,
  \[
    H \isom (G \rtimes_{\alpha} H)/\ker \beta = (G \rtimes_{\alpha} H)/G'
  \]
  
  Finally, by calculation,
  \begin{align*}
    (e,h)^{-1} \ast (g, e) \ast (e,h) &= (e,h^{-1}) \ast (g, e) \ast (e,h) \\
    &= (\alpha_{e}(e)g, h^{-1}e) \ast (e,h) \\
    &= (g, h^{-1}) \ast (e,h) \\
    &= (\alpha_{h}(g)e, h^{-1}h) \\
    &= (\alpha_{h}(g), e).
  \end{align*}
\end{proof}

\begin{example}
  For any $G$ and $H$, there is always the trivial action $\alpha_{h}(g) = g$.
  Under this action, the semidirect product $G \rtimes_{\alpha} H$ has product
  \[
    (g_{1}, h_{1}) \ast (g_{2}, h_{2}) =
      (\alpha_{h_{2}}(g_{1})g_{2}, h_{1}h_{2})
      = (g_{1}g_{2}, h_{1}h_{2}).
  \]
  In other words, this is simply the direct product of the two groups.
\end{example}

\begin{example}
  If $G$ is any Abelian group, there is an action of the multiplicative
  group $\{1, -1\} \isom C_{2}$ given by $\alpha_{1}(g) = g$ and
  $\alpha_{-1}(g) = g^{-1}$.  It is trivial that $\alpha_{1}$ is a
  homomorphism, and $\alpha_{-1}(gh) = (gh)^{-1} = h^{-1}g^{-1} =
  g^{-1}h^{-1} = \alpha_{-1}(g)\alpha_{-1}(h)$, since $G$ is Abelian.
  
  The semidirect product of $G$ and this group with this action is a group
  of order $2|G|$, and if $G$ has any element of order greater than $2$, this
  is distinct from the direct product.  Indeed, if it is distinct from
  the direct product, the semidirect product is not Abelian, since if $g_{1}$
  is an element of order greater than $2$, we have
  \[
    (g_{1}, 1) \ast (e, -1) = (g_{1}^{-1}, -1),
  \]
  but
  \[
    (e, -1) \ast (g_{1}, 1) = (g_{1}, -1),
  \]
  and these are only equal if $g_{1}^{-1} = g_{1}$, which implies that
  $g_{1}$ has order $2$.  So these elements do not commute.
  
  One can show that as a particular example of this action giving a semidirect
  product, we have $D_{2n} \isom C_{n} \rtimes_{\alpha} C_{2}$, where the
  generators $a$ and $b$ of $D_{2n}$ correspond to the elements $(u,1)$ and
  $(1,-1)$, respectively.
\end{example}


\newpage
\section*{Assignment 5}

The following exercises are due on day of final.

\begin{description}
  \item[4.1] Exercises 1, 2, 4, 5, 7.
  \item[4.2] Exercises 1, 4, 5, 8, 9.
\end{description}


\chapter{Introduction}

In this section we will look at a number of concrete examples where we
extract an algebraic system from various mathematical concepts.  This will
hopefully give you some solid examples of the sorts of objects that we will
be discussing for the remained of the course, and why they are of significance
in pretty much all areas of science, mathematics, and even art.

\section{Symmetry}

You are probably familiar, in an informal way, with the idea of symmetry from
Euclidean geometry and calculus.  For example, the letter ``\textsf{A}'' has
reflective symmetry in its vertical axis, ``\textsf{E}'' has reflective
symmetry in its horizontal axis, ``\textsf{N}'' has rotational symmetry of
$\pi$ radians about its centre, and ``\textsf{H}'' has all three types of
symmetry.  And the letter ``\textsf{F}'' has none of these symmetries.

Symmetry is also important in understanding real world phenomena.  As some
examples:
\begin{itemize}
  \item The symmetries of molecules can affect possible chemical reactions.
    For example, many proteins and amino acids (the basic building blocks of
    life) have ``left-handed'' and ``right-handed'' versions which are
    reflections of one-another.  Life on earth uses the ``left-handed''
    versions almost exclusively.
    
  \item Crystals have very strong symmetries, largely determined by the
    symmetries of the atoms or molecules of which the crystal is built.
    
  \item Most animals and plants have some sort of symmetry in their
    body-shapes, although they are never perfectly symmetrical.  Most
    animals have bilateral symmetry, while plants often have five-fold
    or six-fold rotational symmetry.
  
  \item In art and design, symmetrical patterns are often found to be more
    pleasing to the eye than asymmetrical patterns, or simply more practical.
    
  \item waves in fluids, or the vibrations of a drumhead or string are often
    symmetrical, or built out of symmetric components.  These symmetries
    are usually inherent to the underlying equations that we use to model
    such systems, and understanding the symmetry can be crucial in finding
    solutions to these equations.
\end{itemize}

But what, precisely, do we mean by symmetry?

\begin{definition}
  Let $\Omega$ be a subset of $\reals^{n}$.  A \defn{symmetry}{symmetry} of
  $\Omega$ is a function $T: \reals^{n} \to \reals^{n}$ such that
  \begin{theoremenum}
    \item $T(\Omega) = \Omega$, and
    \item $T$ preserves distances between points
  \end{theoremenum}
\end{definition}

Functions which preserve distance in $\reals^{n}$ are sometimes called
\defn{Euclidean transforms}{Euclidean transform}.  We will see these again
later.

We also note that every set $\Omega$ has the trivial \defn{identity
symmetry}{symmetry!identity} $I(x) = x$.

\begin{proposition}\label{prop:symmetryfacts}
  Let $\Omega$ be a subset of $\reals^{n}$, and let $S$ and $T$ be
  symmetries of $\Omega$.  Then
  \begin{theoremenum}
    \item $T$ is one-to-one and onto.
    
    \item the inverse function $T^{-1}$ is a symmetry of $\Omega$
    
    \item the composition $T \circ S$ is a symmetry of $\Omega$

    \item the compositions $T \circ T^{-1}$ and $T^{-1} \circ T$ always
       equal the identity symmetry $I$.
  \end{theoremenum}
\end{proposition}
\begin{proof}
  If $T(x_{1}) = T(x_{2})$ then $d(T(x_{1}), T(x_{2})) = 0$ so the fact that
$T$ preserves distances means that $d(x_{1}, x_{2}) = 0$.  But this implies
that $x_{1} = x_{2}$, so $T$ is one-to-one.

  We also know that if $B_{r} = \{x \in \reals^{n} : d(x,0) \le r \}$ is the
$n$-dimensional ``ball'' of radius $r$ centred at the origin of $\reals^{n}$,
then $T(B_{r})$ must also be a ball of radius $r$ (since $T$ preserves
distances).  Let $c$ be the centre of the ball $T(B_{r})$, and given any point
$y \in \reals^{n}$ we have $y \in T(B_{r})$ for all $r \ge d(y,c)$.  Hence
there is some $x \in \reals^{n}$ such that $T(x) = y$, and so $T$ is onto.

  Since $T$ is one-to-one and onto, it has an inverse function $T^{-1}$.
 We observe that $T^{-1}(\Omega) = T^{-1}(T(\Omega)) = \Omega$, and also that
$d(T^{-1}(x), T^{-1}(y)) = d(T(T^{-1}(x)), T(T^{-1}(y))) = d(x,y)$.  Hence
$T^{-1}$ is a symmetry of $\Omega$.

  Parts (iii) and (iv) are left as a simple exercise.
\end{proof}

We\sidebar{Notation}{Many algebra texts write $ST$ for $T \circ S$, because
$S$ is applied first, then $T$.  In these notes, however, we will remain
consistent with the traditional function composition order, but you must
keep this clear in your head to avoid confusion.} will sometimes write the composed symmetry $T \circ S$ as simply $TS$. 
Remember that because function composition works from right to left, $TS$
means that the symmetry $S$ is applied first, followed by the symmetry $T$.

\begin{figure}\label{fig:symmetryofH}
  \vspace{1in}
  \caption{The set $\Omega$ of Example~\ref{eg:symmetryofH}}
\end{figure}

\begin{example}\label{eg:symmetryofH}
  Let $\Omega \subseteq \reals^{2}$ be the H-shaped set illustrated in
Figure~\ref{fig:symmetryofH}.  Then $\Omega$ has the following symmetries:
  \begin{alignat*}{4}
    I(x,y) &= (x,y) &\qquad& \text{(Identity)} \\
    H(x,y) &= (x,-y) && \text{(Reflection in the $x$-axis)} \\
    V(x,y) &= (-x,y) && \text{(Reflection in the $y$-axis)} \\
    R(x,y) &= (-x,-y) && \text{(Rotation by $\pi$ radians about the origin)}
  \end{alignat*}
  
  These are the only symmetries, as it turns out.  We can confirm by direct
calculation that $I^{-1} = I$, $H^{-1} = H$, $V^{-1} = V$ and $R^{-1} = R$. 
In other words, each of these transformations is its own inverse.  We also
have the following compositions of symmetries:
  \begin{alignat*}{6}
    H \circ H &= I & \qquad & H \circ V &= R & \qquad & H \circ R &= V\\
    V \circ H &= R && V \circ V &= I && V \circ R &= H\\
    R \circ H &= V && R \circ V &= H && R \circ R &= I
  \end{alignat*}
  In fact we can write this out as a ``multiplication table'' of sorts:
  
  \medskip
  \hspace{1.5in}\begin{tabular}{c|cccc}
    $\circ$ & $I$ & $H$ & $V$ & $R$ \\
    \hline
    $I$ & $I$ & $H$ & $V$ & $R$ \\
    $H$ & $H$ & $I$ & $R$ & $V$ \\
    $V$ & $V$ & $R$ & $I$ & $H$ \\
    $R$ & $R$ & $V$ & $H$ & $I$ \\
  \end{tabular}
  
  \medskip
  
  This sort of multiplication table is called a \defn{Cayley table}{Cayley
  table}.
  
  One can check that this ``multiplication'' of symmetries satisfies
associative and commutative laws.
\end{example}

There is nothing really special about the set $\Omega$ in the above example:
every set will have a collection of symmetries which give a ``multiplication
table'' which satisfies an associative law, although it turns out that it may
not satisfy a commutative law.

\begin{example}\label{eg:symmtriangle}
  Let $\Omega \subseteq \reals^{2}$ be an equilateral triangle centred at
  the origin and with one vertex at $(1,0)$.  Then $\Omega$ has the following
  symmetries:

  \begin{tabular}{lp{3.5in}}
    $I$ & Identity \\
    $R_{1}$ & Rotation by $2\pi/3$ radians clockwise \\
    $R_{2}$ & Rotation by $2\pi/3$ radians anticlockwise \\
    $H_{0}$ & Reflection in the $x$-axis \\
    $H_{1}$ & Reflection in the line through $(0,0)$ and $(-\sqrt{3}/2,1/2)$ \\
    $H_{2}$ & Reflection in the line through $(0,0)$ and $(-\sqrt{3}/2,-1/2)$
  \end{tabular}
  
  The precise formulas for these symmetries are an exercise.  A little thought
  tells us that $I^{-1} = I$, $R_{1}^{-1} = R_{2}$, $R_{2}^{-1} = R_{1}$,
  $H_{1}^{-1} = H_{1}$, $H_{2}^{-1} = H_{2}$, and $H_{3}^{-1} = H_{3}$.
  The Cayley table for these symmetries is:
    
  \medskip
  \hspace{1in}\begin{tabular}{c|cccccc}
    $\circ$ & $I$ & $R_{1}$ & $R_{2}$ & $H_{0}$ & $H_{1}$ & $H_{2}$ \\
    \hline
    $I$ & $I$ & $R_{1}$ & $R_{2}$ & $H_{0}$ & $H_{1}$ & $H_{2}$ \\
    $R_{1}$ & $R_{1}$ & $R_{2}$ & $I$ & $H_{1}$ & $H_{2}$ & $H_{0}$ \\
    $R_{2}$ & $R_{2}$ & $I$ & $R_{1}$ & $H_{2}$ & $H_{0}$ & $H_{1}$ \\
    $H_{0}$ & $H_{0}$ & $H_{2}$ & $H_{1}$ & $I$ & $R_{1}$ & $R_{2}$ \\
    $H_{1}$ & $H_{1}$ & $H_{0}$ & $H_{2}$ & $R_{2}$ & $I$ & $R_{1}$ \\
    $H_{2}$ & $H_{2}$ & $H_{1}$ & $H_{0}$ & $R_{1}$ & $R_{2}$ & $I$ \\
  \end{tabular}
  
  \medskip
  
  This operation is associative, but it is clearly \emph{not} commutative:
  $H_{0} \circ H_{1} = R_{1}$, but $H_{1} \circ H_{0} = R_{2}$, for example.
\end{example}

The reason that the composition operation of symmetries is always associative
is because it is really just composition of functions, and function composition
is an associative operation.

\begin{example}\label{eg:circlesymmetry}
  Let $\Omega$ be the unit circle.  Then $\Omega$ has infinitely many
  symmetries, which fall into two classes:
  
  \begin{tabular}{lp{3.5in}}
    $R_{\theta}$ & Rotation by $\theta$ radians clockwise,
      $0 \le \theta < 2\pi$ \\
    $H_{\varphi}$ & Reflection in the line which makes an angle $\varphi$
      to the $x$-axis at the origin, $0 \le \varphi < \pi$.
  \end{tabular}
  
  The identity is $R_{0}$, rotation by $0$ radians.  We can also check that
  the inverse of $R_{\theta}$ is $R_{2\pi - \theta}$ for $0 < \theta < 2\pi$,
  and the inverse of $H_{\varphi}$ is $H_{\varphi}$.
  
  Because the set of symmetries is infinite, we can't write down a Cayley
  table, but we can list how the generic symmetries compose:
  
  \begin{alignat*}{4}
    R_{\theta} \circ R_{\omega} &= R_{\theta + \omega} &\qquad&
    R_{\theta} \circ H_{\varphi} &= \\
    H_{\varphi} \circ R_{\theta} &= H_{\varphi - \theta/2}&&
    H_{\varphi} \circ H_{\psi} &= R_{2\varphi}
  \end{alignat*}
  
\end{example}

All the examples so far have used rotational and reflective symmetries, but
some regions have translational symmetry.

\begin{example}
  Let $\Omega$ be the $x$-axis in $\reals^{2}$.  Then $\Omega$ has symmetries
  of the form
  \[
    T_{c}(x,y) = (x + c,y),
  \]
  ie. right translation by $c$, for any $c \in \reals$.
  
  The identity symmetry is $T_{0}$, the inverse symmetry of $T_{c}$ is
  $T_{-c}$, and symmetries of this set compose by the rule
  \[
    T_{a} \circ T_{b} = T_{a+b}.
  \]
\end{example}

\subsection*{Exercises}

\begin{enumerate}
  \item Find the set of symmetries for each capital letter of the alphabet.
  
  \item Prove Proposition~\ref{prop:symmetryfacts} (iii-iv).
  
  \item Write down formulas for each of the symmetries in
    Example~\ref{eg:symmtriangle}.
    
    Hint 1: the point $(x,y) \in \reals^{2}$ rotated clockwise by an angle
    $\theta$ about the origin is $(x\cos \theta + y\sin \theta, -x\sin \theta +
    y\cos \theta)$.
    
    Hint 2: from the Cayley table, we have $H_{1} = R_{1} \circ H_{0}$ and
    $H_{2} = R_{2} \circ H_{0}$, and it is easy to find the formula of a
    composition of functions.
  
  \item Let $\Omega \subseteq \reals^{2}$ be a square, centred at the origin,
    with side length 1.
    Find all 8 symmetries of $\Omega$, and write down the formula for each. 
    Find the inverses of each symmetry.
    Write out the Cayley table for the symmetries of a square.
  
  \item\label{ex:symtetra} (*) Let $\Omega \subseteq \reals^{3}$ be a regular tetrahedron
    centred at the origin.  Show that $\Omega$ has 24 symmetries.
  
  \item (*) Let $\Omega = \integers^{2} \subseteq \reals^{2}$ be the integer
    lattice in the plane, ie.
    \[
      \integers^{2} = \{ (m,n) \in \reals^{2} : m, n \in \integers \}.
    \]
    Classify the symmetries of $\integers^{2}$.
    Find the inverses of each symmetry.
    As in Example~\ref{eg:circlesymmetry}, find the product of typical symmetries.
\end{enumerate}

\section{Permutations}

A \defn{permutation}{permutation} of a set $X$ is simply a re-arrangement of
the elements, or more precisely a function $p$ that maps each element of $X$ to
an element of $X$ with no two distinct elements being mapped to the same
element (and for infinite sets, we also need $p(X) = X$).  Another way of
saying this is that a permutation of $X$ is simply a bijection $p: X \to X$.

Normally we are interested only in permutations of finite sets, and we
really only care how many elements there are to permute.  Hence it is
customary to consider permutations of the set $\{1, 2, 3, ..., n\}$.

Since permutations are just functions,we can define them as we would any other
function, by specifying the value that the function takes at each point in the
domain.  Unfortunately, unlike the usual functions you see in a calculus course,
you usually can't specify permutations using a formula.

\begin{example}
  The following function $p$ is a permutation of the set $\{1,2,3,4,5,6,7,8\}$:
  \begin{alignat*}{8}
    p(1) &= 2 &\qquad&
    p(2) &= 4 &\qquad&
    p(3) &= 6 &\qquad&
    p(4) &= 8 \\
    p(5) &= 7 &&
    p(6) &= 5 &&
    p(7) &= 3 &&
    p(8) &= 1  
  \end{alignat*}
\end{example}

A more compact way of writing down a permutation is to write it as an array of
numbers, with $1$, through $n$ on the top row, and the respective image of each
in the second row, like so:
\[
  p = \begin{pmatrix}
    1 & 2 & 3 & \ldots & n \\
    p(1) & p(2) & p(3) & \ldots & p(n)
  \end{pmatrix}
\]

\begin{example}
  The permutation $p$ of the previous example can be written as follows:
  \[
    p = \begin{pmatrix}
      1 & 2 & 3 & 4 & 5 & 6 & 7 & 8 \\
      2 & 4 & 6 & 8 & 7 & 5 & 3 & 1
    \end{pmatrix}
  \]
\end{example}

We denote the set of all permutations of $\{1,2,3,\ldots,n\}$ by $S_{n}$.

\begin{example}\label{eg:perm3part1}
  The set $S_{3}$ is
  \[
    \left\{\begin{pmatrix}
      1 & 2 & 3 \\
      1 & 2 & 3
    \end{pmatrix},
    \begin{pmatrix}
      1 & 2 & 3 \\
      3 & 1 & 2
    \end{pmatrix},
    \begin{pmatrix}
      1 & 2 & 3 \\
      2 & 3 & 1
    \end{pmatrix},
    \begin{pmatrix}
      1 & 2 & 3 \\
      1 & 3 & 2
    \end{pmatrix},
    \begin{pmatrix}
      1 & 2 & 3 \\
      2 & 1 & 3
    \end{pmatrix},
    \begin{pmatrix}
      1 & 2 & 3 \\
      3 & 2 & 1
    \end{pmatrix} \right\}
  \]
\end{example}

Note that, as in the above example, the \defn{identity
permutation}{permutation!identity} $p(k) = k$ is always a permutation.

Since every permutation is a one-to-one and onto function, there is an inverse
function $p^{-1}$ associated with every permutation $p$.

We can ``multiply'' two permutations by applying the first permutation, and
then using the second permutation to permute the result.  If $p$ and $q$ are
permutations of the same set, $pq(k)$ is the what you get from applying $q$
to $p(k)$, ie.\ $pq(k) = q(p(k))$, so $pq = q \circ p$ (note the reversal of
terms in the product versus the composition).

\begin{proposition}\label{prop:permgroup}
  Let $X$ be any set, and $p$ and $q$ be permutations of $X$, then
  \begin{theoremenum}
    \item $p^{-1}$ is a permutation of $X$,
    \item $pq$ is a permutation of $X$,
    \item the product satisfies an associative law: $(pq)r = p(qr)$.
  \end{theoremenum}
\end{proposition}
\begin{proof}
  These follow immediately from Proposition~\ref{prop:functionfacts}:
  the inverse
  function of a bijection is a bijection, proving (i); the composition
  of bijective functions is a bijective function, proving (ii); and
  composition of functions is associative, so 
  \[
    (pq)r = r \circ (q \circ p) = (r \circ q) \circ p = p(qr),
  \]
  proving (iii).
\end{proof}

\begin{example}
  Let
  \[
    p = \begin{pmatrix}
      1 & 2 & 3 \\
      3 & 2 & 1
    \end{pmatrix}
    \qquad \text{and} \qquad
    q = \begin{pmatrix}
      1 & 2 & 3 \\
      3 & 1 & 2
    \end{pmatrix}.
  \]
  We can find $pq$ fairly easily: for example if
  $k=1$, we know that $p(1) = 3$, and $q(3) = 2$, so $pq(1) = 2$. Repeating
  for $k = 2$ and $3$, we get So
  we have
  \[
    pq = \begin{pmatrix}
      1 & 2 & 3 \\
      2 & 1 & 3
    \end{pmatrix}.
  \]
\end{example}

\begin{example}\label{eg:perm3part2}
  We listed all the elements of $S_{3}$ in Example~\ref{eg:perm3part1}.
  To simplify notation we will give each of these a symbol to identify it:
  \begin{alignat*}{6}
    p_{0} &= \begin{pmatrix}
      1 & 2 & 3 \\
      1 & 2 & 3
    \end{pmatrix} &\qquad&
    p_{1} &= \begin{pmatrix}
      1 & 2 & 3 \\
      3 & 1 & 2
    \end{pmatrix} &\qquad&
    p_{2} &= \begin{pmatrix}
      1 & 2 & 3 \\
      2 & 3 & 1
    \end{pmatrix}\\
    p_{3} &= \begin{pmatrix}
      1 & 2 & 3 \\
      1 & 3 & 2
    \end{pmatrix} &&
    p_{4} &= \begin{pmatrix}
      1 & 2 & 3 \\
      2 & 1 & 3
    \end{pmatrix} &&
    p_{5} &= \begin{pmatrix}
      1 & 2 & 3 \\
      3 & 2 & 1
    \end{pmatrix}
  \end{alignat*}
  It is easy to verify that $p_{0}^{-1} = p_{0}$, $p_{1}^{-1} = p_{2}$,
  $p_{2}^{-1} = p_{1}$, $p_{3}^{-1} = p_{3}$, $p_{4}^{-1} = p_{4}$, 
  and $p_{5}^{-1} = p_{5}$.
  
  Just as with symmetries, we can write out a Cayley table for the products
  of these permutations:
  \[
  \begin{array}{c|cccccc}
               & p_{0} & p_{1} & p_{2} & p_{3} & p_{4} & p_{5} \\
    \hline
    p_{0} & p_{0} & p_{1} & p_{2} & p_{3} & p_{4} & p_{5} \\
    p_{1} & p_{1} & p_{2} & p_{0} & p_{4} & p_{5} & p_{3} \\
    p_{2} & p_{2} & p_{0} & p_{1} & p_{5} & p_{3} & p_{4} \\
    p_{3} & p_{3} & p_{5} & p_{4} & p_{0} & p_{2} & p_{1} \\
    p_{4} & p_{4} & p_{3} & p_{5} & p_{1} & p_{0} & p_{2} \\
    p_{5} & p_{5} & p_{4} & p_{3} & p_{2} & p_{1} & p_{0} 
  \end{array}
  \]
  This product is not commutative.
  
  It's probably not immediately obvious, but if you look closely you will see
  that the pattern of this Cayley table is exactly the same as the pattern of
  the Cayley table of Example~\ref{eg:symmtriangle}, with the correspondences
    $p_{0} \leftrightarrow I$,
    $p_{1} \leftrightarrow R_{1}$,
    $p_{2} \leftrightarrow R_{2}$,
    $p_{3} \leftrightarrow H_{0}$,
    $p_{4} \leftrightarrow H_{1}$,
    $p_{5} \leftrightarrow H_{2}$.
  Indeed, the inverses of each element have the same pattern under these same
  correspondences.
  
  In other words, if we look at these two examples abstractly, we seem to be
  getting the same underlying mathematical object.
  
  This correspondence can be made even more concrete in the following way: if
  we label the vertices of the equilateral triangle of Example~\ref{eg:symmtriangle}
  with the numbers 1, 2 and 3, starting at $(0,0)$ and working clockwise, we
  find that the symmetries of the triangle permute the vertices in exactly
  the same way that the corresponding permutations permute the corresponding
  numbers.
\end{example}

\subsection{Cycles}

Even with the current notation, expressing and working with permutations can
be cumbersome.  There is another, alternative, notation which can speed up
the process of working with permutations.  This notation works by looking at
the \defn{cycles}{cycle} withing a permutation.  If $p$ is a permutation
of the set $X$, the cycle of an element $k$ of $X$ in $p$ is the sequence
of elements $(k, p(k), p^{2}(k), \ldots, p^{m}(k))$ (where $p^{l}$ is the product
of $p$ with itself $l$ times) such that $m$ is the smallest number such that,
$p^{m+1}(k) = k$.

Note that the order of the elements in a cycle is important, but not where we
start in the cycle.  For example, we regard $(k, p(k), p^{2}(k), \ldots, p^{m}(k))$,
$(p(k),$ $p^{2}(k),\ldots, p^{m}(k), k)$, $(p^{2}(k), \ldots, p^{m}(k), k, p(k))$, etc.\ as
representing the same cycle.  If $X$ is the set $\{1, 2, \ldots n\}$, it is
usual to write a cycle starting with the smallest number in the cycle.

A cycle with $m$ elements is called an \defn{$m$-cycle}{$m$-cycle}.  A 2-cycle
is sometimes called a \defn{transposition}{transposition}, since it transposes
two elements.

\begin{example}
  In the following permutation
  \[
    \begin{pmatrix}
      1 & 2 & 3 & 4 & 5 & 6 & 7 & 8 \\
      2 & 4 & 6 & 8 & 7 & 5 & 3 & 1
    \end{pmatrix}
  \]
  we have $1 \to 2$, $2 \to 4$, $4 \to 8$ and $8 \to 1$, so $(1,2,4,8)$ is
  a cycle.  We could also write this cycle as $(2, 4, 8, 1)$, $(4,8, 1, 2)$,
  or $(8, 1, 2, 4)$.
  
  The smallest element not in this cycle is $3$, and we have
  $3 \to 6$, $6 \to 5$, $5 \to 7$ and $7 \to 3$, so $(3, 6, 5, 7)$ is another
  cycle.
  
  Since every element is in one of these two cycles, these are the only cycles
  in this permutation.
\end{example}

If we find all of the cycles of a permutation, we can represent the permutation
as a whole as a product of its cycles.

\begin{example}
  The permutation
  \[
    \begin{pmatrix}
      1 & 2 & 3 & 4 & 5 & 6 & 7 & 8 \\
      2 & 4 & 6 & 8 & 7 & 5 & 3 & 1
    \end{pmatrix}
  \]
  can be written as $(1, 2, 4, 8)(3, 6, 5, 7)$ or $(3, 6, 5, 7)(1, 2, 4, 8)$.
\end{example}

\begin{example}
  The elements of $S_{3}$ can be represented in cycle form as follows:
  \begin{alignat*}{6}
    \begin{pmatrix}
      1 & 2 & 3 \\
      1 & 2 & 3
    \end{pmatrix} &= (1)(2)(3) &\qquad&
    \begin{pmatrix}
      1 & 2 & 3 \\
      3 & 1 & 2
    \end{pmatrix} &= (1, 3, 2) &\qquad&
    \begin{pmatrix}
      1 & 2 & 3 \\
      2 & 3 & 1
    \end{pmatrix} &= (1, 2, 3)\\
    \begin{pmatrix}
      1 & 2 & 3 \\
      1 & 3 & 2
    \end{pmatrix} &= (1)(2, 3) &&
    \begin{pmatrix}
      1 & 2 & 3 \\
      2 & 1 & 3
    \end{pmatrix} &= (1, 2)(3)&&
    \begin{pmatrix}
      1 & 2 & 3 \\
      3 & 2 & 1
    \end{pmatrix} &= (1, 3)(2)
  \end{alignat*}
\end{example}

To work out how a product of cycles permutes a particular element $k$, all you
need do is work from left to right until you find the element in a cycle, and
then find the element which follows it in that cycle.  When you are finding
products of permutations (as we will see below), you continue on with the new
element in the remaining cycles until you reach the end of the product.

\begin{example}
  Consider the permutation $p = (1, 3, 5)(2)(4, 6)$ of the set
  $\{1, 2, 3, 4, 5, 6\}$.  We can calculate $p(1)$ be looking at the first
  cycle, where we see that the element after $1$ in that cycle is $3$, and we also
  note that $3$ does not occur in any cycle after the first, so $p(1) = 3$.
  Similarly, we have $p(2) = 2$, $p(3) = 5$, $p(4) = 6$, $p(5) = 1$ and
  $p(6) = 4$.  This permutation could also be written as
  \[
    \begin{pmatrix}
      1 & 2 & 3 & 4 & 5 & 6\\
      3 & 2 & 5 & 6 & 1 & 4
    \end{pmatrix}.
  \]
\end{example}

Notice that there would be no difference in the above example if the cycle
$(2)$ was omitted.  It is common practise to leave such single-element
cycles out, particularly when the set which is being permuted is clear.

\begin{example}
  Consider the product of cycles $p = (1, 3, 5)(2, 3)(4, 6, 5)$ in the set
  $\{1, 2, 3, 4, 5,$ $6\}$.  We can calculate $p(1)$ be looking at the first
  cycle, where we see that the element after $1$ in that cycle is $3$; however
  $3$ occurs in the second cycle, and the element after it in the cycle is $2$;
  and $2$ does not occur in the remaining cycle, so $p(1) = 2$.  Similarly,
  we have $3 \to 5$ in the first cycle, and $5 \to 4$ in the last cycle, so
  $p(3) = 4$.
  Calculating everything out, we have $p(2) = 3$, $p(4) = 6$, $p(5) = 1$ and
  $p(6) = 5$.  This permutation could also be written as
  \[
    \begin{pmatrix}
      1 & 2 & 3 & 4 & 5 & 6\\
      2 & 3 & 4 & 6 & 1 & 5
    \end{pmatrix},
  \]
  or more simply in cycle notation as $(1, 2, 3, 4, 6, 5)$.
\end{example}

We can then calculate the product of two permutations in cycle notation by
writing all the cycles as one long product of cycles, and then reducing to
the standard form of the cycles by the above process.

\begin{theorem}
  Every permutation of $S_{n}$ can be written as a product of disjoint cycles.
  (Two cycles are disjoint if they have not elements in common.)
\end{theorem}
\begin{proof}
  Let $p$ be a permutation of $S_{n}$.  We let $c_{1}$ be the cycle which
  includes $1$,
  \[
    c_{1} = \{1, p(1), p^{2}(1), \ldots, p^{m_{1}}(1)\},
  \]
  and we let $p_{1}$ be the permutation defined by
  \[
    p_{1}(k) = \begin{cases}
      k & \text{if $k \in c_{1}$,}\\
      p(k) & \text{otherwise}.
    \end{cases}
  \]
  Then it is clear that $p = c_{1}p_{1}$.
  
  Now if we have written $p = c_{1}\ldots c_{l}p_{l}$, where $c_{1}, \ldots,\
  c_{l}$ are disjoint cycles, and $p_{l}$ is a permutation which satisfies
  has $p_{l}(k) = k$ whenever $k$ is in one of the cycles, then one of two
  things must be true: either every element of $\{1, 2, \ldots, n\}$ is an
  element of one of the cycles, or there is some smallest element $k_{l}$
  which is not in any of the cycles.
  
  In the first case, we have that $p_{l}$ must be the identity permutation,
  so $p = c_{1}\ldots c_{l}$, and we are done.
  
  In the second case, we let $c_{l+1}$ be the cycle including $k_{l}$,
  \[
    c_{l+1} = \{k_{l}, p(k_{l}), p^{2}(k_{l}), \ldots, p^{m_{l}}(k_{l})\},
  \]
  and let $p_{l+1}$ be the permutation defined by
  \[
    p_{l+1}(k) = \begin{cases}
      k & \text{if $k$ is an element of any cycle $c_{1}$, $c_{2}, \ldots, c_{l+1}$}\\
      p(k) & \text{otherwise}.
    \end{cases}
  \]
  Then $p = c_{1}\ldots c_{l+1}p_{l+1}$.
  
  Since $\{1,2,3,\ldots, n\}$ is a finite set, an induction argument using
  this construction proves the result.
\end{proof}

\subsection{Parity}

Informally, if we compare the permuations
\[
  \begin{pmatrix}
    1 & 2 & 3\\
    3 & 1 & 2
  \end{pmatrix}
  \qquad \text{and} \qquad
  \begin{pmatrix}
    1 & 2 & 3\\
    3 & 2 & 1
  \end{pmatrix},
\]
we observe that the first ``rotates'' the elements to the right, while
the second ``reflects'' the elements.  Indeed, if we consider the
correspondence between these permutations and the symmetries of a triangle
discussed in Example~\ref{eg:perm3part2}, we see that the first corresponds
to a rotation, and the second to a reflection.  In this section, we will
generalize this idea to arbitrary permutations.

The starting point of this discussion is a comparison between the following
two products: if $p \in S_{n}$ we define
\[
  D_{n} = \prod_{1 \le i < j \le n} (j - i)
\]
and
\[
  D(p) = \prod_{1 \le i < j \le n} (p(j) - p(i)).
\]
Given any pair of distinct elements $k,l \in \{1, 2, \ldots, n\}$, both of
these products contain exactly one factor which is a difference of $k$ and $l$.
This is easy to see in the product $D_{n}$, but a little thought will convince
you that it is also the case for $D(p)$.  The difference between the two
products is that in $D(p)$ it may not necessarily be the larger term minus
the smaller term.  Hence $D_{n}$ and $D(p)$ have the same magnitude, but may differ
in sign.

\begin{definition}
  Let $p \in S_{n}$.  The \defn{parity}{parity} of $p$ is
  \[
    \parity(p) = \frac{D(p)}{D_{n}}.
  \]
\end{definition}

Clearly the parity of $p$ is $1$ if $D(p) > 0$ and $-1$ if $D(p) < 0$.

\begin{example}
  Consider the permutations
  \[
    p = \begin{pmatrix}
      1 & 2 & 3\\
      3 & 1 & 2
    \end{pmatrix}
    \qquad \text{and} \qquad
    q = \begin{pmatrix}
      1 & 2 & 3\\
      3 & 2 & 1
    \end{pmatrix}.
  \]
  In both cases
  \[
    D_{3} = (3 - 1)(3 - 2)(2 - 1) = 2 \times 1 \times 1 = 2.
  \]
  Now
  \[
    D(p) = (p(3) - p(1))(p(3) - p(2))(p(2) - p(1)) = (2 - 3)(2 - 1)(1 - 3)
      = -1 \times 1 \times -2 = 2,
  \]
  so
  \[
    \parity(p) = 2/2 = 1.
  \]
  On the other hand,
  \[
    D(q) = (q(3) - q(1))(q(3) - q(2))(q(2) - q(1)) = (1 - 3)(1 - 2)(2 - 3)
      = -2 \times -1 \times -1 = -2,
  \]
  so
  \[
    \parity(q) = -2/2 = -1.
  \]
\end{example}

Calculating the parity in the last example was fairly straightforward,
but calculating the parity of a general permutation can be quite time
consuming: a simple counting argument tells us that if $p \in S_{n}$ we
have $n(n-1)/2$ terms in the product $D(p)$.  We need a better way to
calculate the parity.

It turns out that there is nothing particularly special about $D_{n}$ in
the definition of parity.  Let $(x_{1}, x_{2}, \ldots, x_{n})$ be a sequence of distinct numbers, and $p$
a permutation of $\{1, 2, \ldots n\}$.  We define
\begin{equation}\label{eqn:permutationaction}
  p(x_{1}, x_{2}, \ldots, x_{n}) = (x_{p(1)}, x_{p(2)}, \ldots, x_{p(n)}),
\end{equation}
and
\[
  D(x_{1}, x_{2}, \ldots, x_{n}) = \prod_{1 \le i < j \le n} (x_{j} - x_{i}).
\]
The following technical lemma shows that we can use these instead to find the
parity of $p$.

\begin{lemma}
  If $p$ is a permutation in $S_{n}$, then
  \[
    \parity(p) = \frac{D(x_{p(1)}, x_{p(2)}, \ldots, x_{p(n)})}{D(x_{1}, x_{2},
    \ldots, x_{n})}.
  \]
  We say that $p$ is an \defn{even permutation}{permutation!even} if
  $\parity(p) = 1$, and $p$ is an \defn{odd permutation}{permutation!odd} if
  $\parity(p) = -1$.
\end{lemma}
\begin{proof}
  Given any number $k \in \{1, 2, \ldots, n\}$, let $a_{k} = p^{-1}(k)$.
  Then given any $k > l$, we have that $k = p(a_{k})$ and $l = p(a_{l})$.
  
  If $a_{k} > a_{l}$, in which case the corresponding terms in each of the
  sums $D_{n}$, $D(p)$, $D(x_{1}, \ldots, x_{n})$ and $D(x_{p(1)}, \ldots,
  x_{p(n)})$ are, respectively, $k - l$, $p(a_{k}) - p(a_{l})$, $x_{k} - x_{l}$,
  and $x_{p(a_{k})} - x_{p(a_{l})}$, with the first two being equal and the
  second two being equal, and so these terms in the quotients $D(p)/D_{n}$
  and $D(x_{p(1)}, \ldots, x_{p(n)})/D(x_{1}, \ldots, x_{n})$, respectively,
  cancel each other out.

  On the other hand, if $a_{k} < a_{l}$, the corresponding terms in each of the
  sums $D_{n}$, $D(p)$, $D(x_{1}, \ldots, x_{n})$ and $D(x_{p(1)}, \ldots,
  x_{p(n)})$ are, respectively, $k - l$, $p(a_{l}) - p(a_{k})$, $x_{k} - x_{l}$,
  and $x_{p(a_{l})} - x_{p(a_{k})}$, with the first two being negatives and the
  second two being negatives, and so these terms in the quotients $D(p)/D_{n}$
  and $D(x_{p(1)}, \ldots, x_{p(n)})/D(x_{1}, \ldots, x_{n})$, respectively,
  give a factor of $-1$.
  
  Hence the number of terms giving each of the factors $1$ and $-1$ in each
  quotient are equal, so
  \[
    \parity(p) = \frac{D(p)}{D_{n}}
      = \frac{D(x_{p(1)}, x_{p(2)}, \ldots, x_{p(n)})}{D(x_{1}, x_{2}, \ldots, x_{n})}.
  \]
\end{proof}

Note that $D_{n} = D(1, 2, \ldots, n)$, and $D(p) = D(p(1), p(2), \ldots, p(n))$.

With this lemma in hand, we can easily prove the following important result:

\begin{theorem}
  Let $p$ and $q \in S_{n}$.  Then
  \[
    \parity(pq) = \parity(p)\parity(q).
  \]
\end{theorem}
\begin{proof}
  The key observation here is that if we have permutations $p$ and $q$,
  then
  \[
    \parity(p) = \frac{D(pq)}{D(q)}.
  \]
  Letting $a_{k} = q(k)$, so that
  \[
    D(pq) = D(q(p(1)), q(p(2)), \ldots, q(p(n))) = D(a_{p(1)}, \ldots, a_{p(n)}),
  \]
  and
  \[
    D(q) = D(a_{1}, \ldots, a_{n}),
  \]
  the lemma tells us that
  \[
    \parity(p) = \frac{D(a_{p(1)}, \ldots, a_{p(n)})}{D(a_{1}, \ldots, a_{n})}
     = \frac{D(pq)}{D(q)}.
  \]
  
  It is immediate form this that
  \[
    \parity(p) \times \parity(q) = \frac{D(pq)}{D(q)} \times \frac{D(q)}{D_{n}}
    = \frac{D(pq)}{D_{n}} = \parity(pq).
  \]
\end{proof}

Thinking in terms of cycles also helps us to calculate the parity of a
permutation, as this result shows:

\begin{theorem}
  Let $c = (k_{1}, k_{2}, \ldots, k_{m})$ be a cycle.  Then
  \[
    \parity(c) = \begin{cases}
      1 & \text{if $m$ is odd} \\
      -1 & \text{if $m$ is even}.
    \end{cases}
  \]
\end{theorem}
\begin{proof}
  First we observe that if $p = (1, 2)$, then all the factors in $D(p)$
  are positive, except for $p(2) - p(1) = 1 - 2 = -1$.  Hence $D(p)$ is
  negative, so $\parity(p) = -1$.
  
  Now if $i, j > 2$, and $i \ne j$, then simple checking shows that
  $(i,j) = (1,i)(2,j)(1,2)(1,i)(2,j)$,
  and, given this fact, the previous theorem tells us
  \begin{align*}
    \parity((i,j)) &= \parity((1,i)(2,j)) \times \parity(1,2) \times \parity((1,i)(2,j)) \\
    &= -\parity((1,i)(2,j))^{2}\\
    &= -1
  \end{align*}
  
  Finally, we observe that
  \begin{equation}\label{eqn:cycleproduct}
    c = (k_{1}, k_{2}, \ldots, k_{m}) = (k_{1}, k_{m})(k_{2}, k_{m})\cdots (k_{m-1}, k_{m}),
  \end{equation}
  and so
  \begin{align*}
    \parity(c) &= \parity((k_{1}, k_{m})) \times \parity((k_{2}, k_{m})) \times \cdots \times \parity((k_{m-1}, k_{m}))\\
    &= (-1)^{m-1} \\
    &= \begin{cases}
    1 & \text{if $m$ is odd} \\
    -1 & \text{if $m$ is even.}
    \end{cases}
  \end{align*}
\end{proof}

\begin{corollary}
  A permutation $p$ is even iff when it is expressed as a product of
  cycles there are an even number of commas in the expression.
\end{corollary}

Another interesting fact that can be squeezed out of the previous theorem
is the following:

\begin{proposition}
  Any permutation $p$ can be written as a product of $2$-cycles, and the
  number of $2$-cycles is even iff $p$ is even.
\end{proposition}
\begin{proof}
  The first fact follows from the fact that every permutation can be written
  as a product of cycles, and equation (\ref{eqn:cycleproduct}) shows that every cycle
  is a product of $2$-cycles.
  
  The second follows from the previous corollary, coupled with the fact that
  every $2$-cycle has a single comma.
\end{proof}

\begin{example}
  Let $p_{0}$, $p_{1}, \ldots, p_{5}$ be as in Example~\ref{eg:perm3part2}.
  Then $\parity(p_{0}) = \parity(p_{1}) = \parity(p_{2}) = 1$, and
  $\parity(p_{3}) = \parity(p_{4}) = \parity(p_{5}) = -1$.
  
  Observe that the ``reflections'' have parity -1, while the ``rotations''
  have parity 1.
\end{example}

\begin{example}
  The permutation
  \[
    p = \begin{pmatrix}
     1 & 2 & 3 & 4 & 5 & 6 & 7 & 8 & 9 & 10 \\
     3 & 5 & 2 & 7 & 8 & 6 & 1 & 4 & 10 & 9
    \end{pmatrix},
  \]
  can be written as $p = (1, 3, 2, 5, 8, 4, 7)(9, 10)$, and since it has $7$
  commas in the expression, it has parity $-1$.
\end{example}

\subsection{Permutation Matrices}

Another way of looking at permutations is very closely related to
Equation~\ref{eqn:permutationaction}.  Let $e_{k}$ be the $k$th standard
orthonormal basis vector in $\reals^{n}$, ie.\ $e_{k}$ is the vector with $0$
in every entry except the $k$th entry, which is $1$.


Since $x = (x_{1},x_{2},\ldots,x_{n})$ is a vector in $\reals^{n}$, the function
it implicitly defines, $T_{p}: \reals^{n} \to \reals^{n}$, where
\[
  T_{p}x = (x_{p(1)}, x_{p(2)}, \ldots, x_{p(n)})
\]
is a linear transformation:
\begin{align*}
  T_{p}(x+y) &= (x_{p(1)} + y_{p(1)}, x_{p(2)} + y_{p(2)}, \ldots, x_{p(n)} + y_{p(n)})\\
    &= (x_{p(1)}, x_{p(2)}, \ldots, x_{p(n)}) + (y_{p(1)}, y_{p(2)}, \ldots, y_{p(n)})
    = T_{p}x + T_{p}y
\end{align*}
and
\begin{align*}
  T_{p}(\lambda x) &= (\lambda x_{p(1)}, \lambda x_{p(2)}, \ldots, \lambda x_{p(n)})\\
    &= \lambda (x_{p(1)}, x_{p(2)}, \ldots, x_{p(n)})
    = \lambda T_{p}x.
\end{align*}

By looking at the image of each standard basis vector $e_{k}$ under the
transformation $T_{p}$, we can find a corresponding $n \times n$ matrix
which we will also call $T_{p}$.
We note that $T_{p}e_{k} = e_{p(k)}$, so $T_{p}$ always takes basis vectors
to basis vectors. Hence every column in the matrix $T_{p}$ is a basis vector,
and every entry of each column is $0$, except for a $1$ in the $p(k)$th row.
Furthermore, since $p$ is a permutation $p(k) = p(j)$ if and only if $k = j$,
so each row is $0$, except for one entry which is $1$.

\begin{proposition}
  Let $p \in S_{n}$ be a permutation.  Then
  \begin{theoremenum}
    \item $T_{p}$ is an orthogonal matrix
    \item $T_{p}$ is the matrix with $1$s in the $p(k)$th row of the $k$th
      column, for $k = 1,2,\ldots,n$, and $0$ everywhere else.
    \item $T_{e} = I_{n}$.
    \item $T_{p}T_{q} = T_{p \circ q} = T_{qp}$.
    
  \end{theoremenum}
\end{proposition}

\subsection*{Exercises}

\begin{enumerate}
  \item Let
    \[
      p = \begin{pmatrix}
        1 & 2 & 3 & 4 & 5 & 6 \\
        4 & 2 & 5 & 1 & 6 & 3
      \end{pmatrix}
      \qquad \text{and} \qquad
      q = \begin{pmatrix}
        1 & 2 & 3 & 4 & 5 & 6 \\
        3 & 4 & 5 & 1 & 2 & 6
      \end{pmatrix}.
    \]
    Find $pq$ and $qp$.  Write both permutations using cycle notation.
    Determine the parity of $p$ and $q$.
  
  \item Let $p = (1, 5, 3, 2)(4, 6, 8)$ and $q = (1, 7, 4, 3)(8, 2)(5, 6)$.
    Find $pq$ and $qp$.  Write both permutations using array notation.
    Determine the parity of $p$ and $q$.

  \item How many distinct permutations are there of the set $\{1, 2,
    \ldots, n\}$? (Hint: they're called \emph{permutations}.)
  
  \item Let $p \in S_{3}$.  Use the Cayley table for $S_{3}$ to show that
    $p^{6}$ is always the identity permutation.
  
  \item Write down all the elements of $S_{4}$ in both array and cycle form.
    Calculate the parity of each element.
    Find the inverse of each element.
    Choose 5 pairs of non-identity elements, and calculate their product.
  
  \item Let $c = (k_{1}, k_{2}, \ldots, k_{m})$ be a cycle.  What is $c^{-1}$?
    Use your answer to calculate the inverse of the permutation
    $p = (1,3,4)(2,5)$.
  
  \item Show that $D_{n} = (n-1)! (n-2)! \ldots 2! 1!$.
  
  \item Show that exactly half the permutations of $S_{n}$ are even, and half
    are odd.
  
  \item (*) Show that $S_{4}$ and the set of symmetries of a regular tetrahedron
    (see Section~\ref{section:symmetry} Exercise~\ref{ex:symtetra}) correspond
    in the same way as $S_{3}$ and the set of symmetries of an equilateral
    triangle.
    
    Hint: you could do this by calculating all $576$ entries in the Cayley
    table of each, and comparing the two; however it is more practical
  
  \item (**) Write a computer program that calculates and prints out the
    Cayley table for $S_{4}$.  Generalize it to print out the Cayley table
    for $S_{n}$ for any $n$.
\end{enumerate}

\section{Modulo Arithmetic}

We say that two numbers $x$ and $y$ are \defn{equal (modulo $m$)}{modulo|equality} if $x$ and
$y$ differ by a multiple of $m$, and we write
\[
  x \equiv y \pmod m
\]
to denote this situation.  Another equivalent (and useful) way to think of
this situation is that $x$ and $y$ have the same remainder when you divide
by $m$.  Since any number greater than $m$ is equal (modulo $m$) to a number
less than $m$, it is customary when working modulo $m$ to reduce your answer
to a number in the range $[0,m)$.

For example
\[
  -1 \equiv 7 \equiv 1023 \pmod 8,
\]
and we would usually write any of these three numbers as $7 \mod 8$ if it
were the solution to a problem.

When we are working modulo $m$, we can perform the operations of addition,
multiplication and subtraction as normal, but we reduce our answers to the
range $[0,m)$.  Indeed, in complicated expressions, one can reduce at
intermediate steps to simplify calculations:
\[
  7 \times 6 + 4 \times 3 \equiv 42 + 12 \equiv 54 \equiv 6 \pmod 8
\]
could be instead calculated as
\[
  7 \times 6 + 4 \times 3 \equiv 42 + 12 \equiv 2 + 4 \equiv 6 \pmod 8.
\]

Division is a trickier topic, but since we are usually performing modulo
arithmetic with integers the na\"{i}ve way of defining modulo division does
not make sense in most cases.  Nevertheless, we will see later on that in
some cases division does make sense.

We can write out addition and multiplication tables for operations modulo
some base, and we call these Cayley tables, just as before.

\begin{example}\label{eg:mod6}
  The addition and multiplication tables, modulo 6 are as follows:
  \[
    \begin{array}{c|cccccc}
      + & 0 & 1 & 2 & 3 & 4 & 5 \\
      \hline
      0 & 0 & 1 & 2 & 3 & 4 & 5 \\
      1 & 1 & 2 & 3 & 4 & 5 & 0 \\
      2 & 2 & 3 & 4 & 5 & 0 & 1 \\
      3 & 3 & 4 & 5 & 0 & 1 & 2 \\
      4 & 4 & 5 & 0 & 1 & 2 & 3 \\
      5 & 5 & 0 & 1 & 2 & 3 & 4
    \end{array}
  \qquad
    \begin{array}{c|cccccc}
      \times & 0 & 1 & 2 & 3 & 4 & 5 \\
      \hline
      0 & 0 & 0 & 0 & 0 & 0 & 0 \\
      1 & 0 & 1 & 2 & 3 & 4 & 5 \\
      2 & 0 & 2 & 4 & 0 & 2 & 4 \\
      3 & 0 & 3 & 0 & 3 & 0 & 3 \\
      4 & 0 & 4 & 2 & 0 & 4 & 2 \\
      5 & 0 & 5 & 4 & 2 & 3 & 1
    \end{array}
  \]
\end{example}


\chapter{Groups}

Algebra concerns the abstraction of simple arithmetic operations to
situations where the quantities involved are unknown.  In developing rules
for algebra, we discover that there are certain rules which always apply,
such as the commutative and associative laws of addition and multiplication,
and that these laws allow us to manipulate and simplify algebraic
expressions. As we learnt more mathematics, we saw similar rules appear over
and over again.  For example, we know that addition of vectors, addition of
matrices, and multiplication of matrices also all satisfy associative laws, and
the first two are also commutative.

In the previous chapter, we also saw that products of symmetries and
permutations also follow an associative law, and sometimes are commutative as
well.

In abstract algebra, instead of concentrating on specific algebraic
settings (such as algebra with numbers, vectors or, now, permutations or
symmetries) we instead look
at the \emph{rules} of algebra and ask what we can infer from reasonable
collections of such rules.  We can then apply the knowledge so gained to a
surprisingly wide collection of concrete situations which happen to satisfy
such rules.

You may have already seen such an approach in linear algebra, where one
eventually considers abstract vector spaces (as opposed to concrete ones, such
as $\reals^{n}$).  One then finds that, for example, the theory applies to
differentiable functions, with differentiation being a linear operator,
giving new insight into calculus that you may not have had before.

The approach in this course is to start with the simplest reasonable
sets of rules, and work up to more complex situations.  Our starting point,
then will be the \defn{group}{group}, an object which encapsulates a reasonable set
of rules for a single algebraic operation.

\section{Binary Operations}

A \defn{binary operation}{binary operation} is a special type of function that we shall be
using with some regularity in this section.  A binary operation $\ast$
is a function
\begin{align*}
  \ast : A \cross B &\to C\\
         (x,y) &\mapsto x \ast y.
\end{align*}
Instead of using ``function-style'' notation $\ast(x,y)$, it is traditional to
write the operation ``in-line'' as $x \ast y$.  In this sense, binary
operations are no more than special notation for certain functions.  Often $A$,
$B$ and $C$ are the same set, in which case we say that a binary operation
$\ast : A \cross A \to A$ is a binary operation on $A$.

A binary operation is \defn{commutative}{commutative} if
\[
  x \ast y = y \ast x.
\]
It is \defn{associative}{associative} if
\[
  (x \ast y) \ast z = x \ast (y \ast z) = x \ast y \ast z.
\]
An element $e$ of $A$ is an \defn{identity}{identity} for the binary operation if
\[
  e \ast x = x \qquad \text{and} \qquad x \ast e = x
\]
for every $x \in A$.  More generally, one can have a \defn{left identity}{identity!left}
$e$ which merely satisfies
\[
  e \ast x = x
\]
for every $x$.  A \defn{right identity}{identity!right} is defined analagously.

\begin{lemma}
  If $\ast: A \cross A \to A$ is a binary operation, and $e$ is an identity for
  $\ast$, then it is the only identity element.
\end{lemma}
\begin{proof}
  Assume that there is another element $e'$ so that $e' \ast x = x \ast e' =
x$.  Then in particular, if we let $x = e$, we have $e' \ast e = e$.  But by
assumption, $e$ is an identity, and so $e' \ast e = e'$.  Hence $e = e'$.
\end{proof}

Notationally, if $\ast$ behaves in a ``multiplication-like'' fashion, or it
is clear from context which binary operation we are using, we will often simply
write $xy$ for $x \ast y$.

For example, the addition operation is a binary operation in the integers
\begin{align*}
  + : \integers \cross \integers &\to \integers\\
         (x,y) &\mapsto x + y.
\end{align*}
In this case we could write $+(2,3) = 2 + 3 = 5$.  Addition is, of course,
both associative and commutative.  $0$ is an identity for addition.  In fact
addition is also an associative and commutative binary operation on just about
any reasonable set of numbers you care to consider, and if $0$ is in the
set, then $0$ is an identity.

Multiplication is a binary operation on the set $\reals$, and it is associative
and commutative, and $1$ is an identity.  Again, like addition,
multiplication is communtative and associative on many reasonable sets of
numbers.

If we consider the set $M_{n}(\reals)$ of $n \times n$ real-valued matrices,
then matrix addition and matrix multiplication are binary operations.  Both
operations are associative, but only matrix addition is commutative.  The
zero matrix is an identity for addition, the identity matrix $I_{n}$ (the
matrix with $1$ down the diagonal and $0$ elsewhere) is an identity for matrix
multiplication.

We also have scalar multiplication as a binary operation $\reals \cross
M_{n}(\reals) \to M_{n}(\reals)$.  This cannot be commutative or
associative, but it does have $1$ as a left identity.

More generally, the inner or dot product on $\reals^{n}$ is a binary
operation $\cdot: \reals^{n} \cross \reals^{n} \to \reals$ which is
commutative, but cannot be associative (since the codomain is $\reals$, and
one cannot take a dot product of an element of $\reals$ and an element of
$\reals^{n}$).  Similarly, there is no identity element of any sort.

One can define arbitrary binary products which are of little or no interest.
For example, $x \ast y = e^{x}(x + \sin(y))$ is a binary product.  But it is
neither associative, commutative, nor has an identity.  So clearly simple
binary operations are not enough to encapsulate the sorts of rules that we
expect algebraic operations to have.

\subsection*{Exercises}

\begin{enumerate}
  \item Let $\ast: A \cross A \to A$ be a commutative binary operation.  If
$e$ is a left identity, show that it also a right identity (and hence simply
an identity).

  \item (*) Let $X$ be any set, and let $\powerset(X)$ be the power set of
$X$ (ie. the set of all subsets of $X$).  Show that $\union : \powerset(X)
\cross \powerset(X) \to \powerset(X)$ is an associative, commutative binary
operation, and that $\emptyset$ is an identity for this operation.

  Similarly, show that $\intersect : \powerset(X) \cross \powerset(X) \to
\powerset(X)$ is an associative, commutative binary operation, and that
$X$ is an identity for this operation.

\end{enumerate}

\section{Groups}

If you look carefully at the discussion of symmetries and permutations, you
will note that not only was there a binary operation, but there was an inverse.
We should have some model for this additional operation.

A \defn{group}{group}, $\mathbf{G} = (G, \ast, e)$, consists of a set $G$, a binary
operation $\ast: G \cross G \to G$, and an element $e \in G$
satisfying the following three conditions:
\begin{enumerate}
  \item $\ast$ is associative
  \item $e$ is an identity for $\ast$
  \item every element $x \in G$ has an \defn{inverse element}{inverse element} $x^{-1} \in G$
such that $x \ast x^{-1} = x^{-1} \ast x = e$.
\end{enumerate}
We call $\ast$ the \defn{group operation}{group operation}.

Note that there is no requirement that the group operation is commutative. 
If it does happen to be commutative, then we say that the group is an
\defn{commutative}{group!commutative} or \defn{Abelian group}{group!Abelian}.

This means that in general $x \ast y$ and $y \ast x$ are distinct elements,
but sometimes they are not.  If
\[
  x \ast y = y \ast x
\]
for a particular $x$ and $y \in G$, we say that $x$ and $y$ \defn{commute}{commuting elements}.

A number of different notations are used when working with groups elements. 
Most commonly we will omit the group operation entirely and simply write $xy$
for $x \ast y$, just as is done for multiplication.  In this case we use the
following clear notation for repeated applications of the group operations:
\[
  x^{k} = \underbrace{x x x ... x}_{k\text{ times}}
\]
for any natural number $k$.  To make this notation mesh nicely with the
expected behaviour of power laws, we define
\[
  x^{-k} = (x^{-1})^{k} \qquad \text{and} \qquad x^{0} = e.
\]
When then have the standard power laws
\[
  x^{m}x^{k} = x^{m+k} \qquad \text{and} \qquad (x^{m})^{k} = x^{mk},
\]
for any integers $m$ and $k$.  Its also not hard to see that $(x^{k})^{-1} =
x^{-k}$.  However, we have that
\[
  (xy)^{k} \ne x^{k}y^{k}
\]
in general.  In the case that $x$ and $y$ commute, then we do have equality.

In the case of Abelian groups, we will sometimes instead use an additive
notation.  We use $+$ for the group operation, and we customarily write the
identity element as $0$, and the inverse element of $x$ as $-x$. We then use
the notation
\[
  kx = \underbrace{x + x + \cdots + x}_{k\text{ times}}
\]
for any natural number $k$, and
\[
  -kx = k(-x) \qquad \text{and}  \qquad 0x = 0.
\]
We then have the natural rules that
\[
  kx + mx = (k + m)x, \qquad k(mx) = (km)x \qquad \text{and} \qquad kx + ky
= k(x+y)
\]
for any integers $k$ and $m$.

If the set $G$ has a finite number of elements, we say that the
\defn{order}{order!of a group} of the group is the number of elements of $G$. 
If $G$ is an infinite set, we say that the group has infinite order.  We denote
the order of the group by $|G|$.

\begin{example}[Addition and Multiplication]
  Since a principle motivation for the definition of groups are standard
  algebraic operations, it should be no surprise that the following are all
  Abelian groups:
  \begin{itemize}
    \item the additive group of real numbers $(\reals, +, 0)$
    \item the additive group of complex numbers $(\complex, +, 0)$
    \item the additive group of rational numbers $(\rationals, +, 0)$
    \item the additive group of integers $(\integers, +, 0)$
    \item the multiplicative group of real numbers $(\reals \setminus \{0\}, \times, 1)$
    \item the multiplicative group of complex numbers $(\complex \setminus \{0\}, \times, 1)$
    \item the multiplicative group of rational numbers $(\rationals \setminus \{0\}, \times, 1)$
    \item the multiplicative group of integers $(\integers \setminus \{0\}, \times, 1)$
    \item the multiplicative group of natural numbers $(\naturals, \times, 1)$
  \end{itemize}
  Note that for the multiplicative groups, we need to exclude $0$, since $0$
  has no multiplicative inverse.
  
  All of these groups have infinite order.
\end{example}

\begin{example}[Modulo Addition]
  If $m$ is any natural number, the additive group of integers modulo $m$ is
  the group $\integers_{m} = (\{0, 1, 2, \ldots, m-1\}, +, 0)$, where addition
  is performed modulo $m$.  To confirm that it is a group, we need to check
  that the axioms hold.
  
  Associativity follows from the fact that regular addition is associative and commutative.
  Given $x$, $y$ and $z$, we have $x + y = a + km$ for some $a$ and $k$, so
  $(x + y) + z \equiv a + z \pmod{m}$.  But $y = a - x + km$, so $y + z
  \equiv a - x + z \pmod{m}$, and hence $x + (y + z) \equiv x + a - x + z \equiv a
  + z \pmod{m}$.
  
  The fact that $0$ is an identity is trivial: $0 + x = x$, so $0 + x \equiv x \pmod{m}$
  follows immediately.
  
  If $x \in \{1, 2, \ldots, m-1\}$, we know that $-x \equiv m-x \pmod{m}$, and so
  $(m - x) + x \equiv 0 \pmod{m}$ and $x + (m - x) \equiv 0 \pmod{m}$.  Also
  $0$ is its own inverse. So every element has an inverse.
  
  These groups are also clearly Abelian, since regular addition is commutative.
  
  The order of $\integers_{m}$ is $m$.
\end{example}

The previous example shows that there are groups of all orders.

\begin{example}
  Multiplication modulo $m$ does not, in general, give a group structure.
  Multiplication modulo $m$ is associative, and $1$ is an identity.
  We have to exclude $0$ from the group, because it clearly does not
  have a multiplicative inverse, but even with this restriction, some other
  elements may not have multiplicative inverses.
  
  If you consider multiplication modulo $6$, as in Example~\ref{eg:mod6},
  you can see that there are no inverses for $2$, $3$, and $4$, since none
  of them have a number which you can multiply them by to give $1$.  Indeed,
  there is a somewhat deeper problem in that some products give $0$, which
  cannot be an element of the group.
  
  Multiplication modulo $m$ {\em does} sometimes give you a group, however.
  The multiplication table (omitting $0$) for multiplication modulo $5$ is
  as follows:
  \[
    \begin{array}{c|cccc}
      \times & 1 & 2 & 3 & 4 \\
      \hline
      1 & 1 & 2 & 3 & 4 \\
      2 & 2 & 4 & 1 & 3 \\
      3 & 3 & 1 & 4 & 2 \\
      4 & 4 & 3 & 2 & 1
    \end{array}
  \]
  A quick check shows that every element has an inverse.  Hence
  $(\{1, 2, 3, 4\}, \times, 1)$ is a group, where $\times$ is multiplication
  modulo $5$.
\end{example}

\begin{example}
  The \defn{symmetric group}{group!symmetric} is the group $S_{n} = (S_{n},
  \cdot, e)$ of all permutations, with the multiplication of permutations
  being the group operation, and $e(k) = k$ being the identity
  permutation.  That this is a group is largely the content of
  Proposition~\ref{prop:permgroup}.  The only thing that needs to be checked
  is that the identity permutation is in fact a group identity, and that is
  fairly straightforward: if $p$ is any permutation in $S_{n}$,
  \[
    (pe)(k) = e(p(k)) = p(k) \qquad \text{and} \qquad (ep)(k) = p(e(k)) = p(k),
  \]
  for all $k$, so $ep = pe = p$, and $e$ is therefore the identity for this
  group operation.
\end{example}

\begin{example}[Matrix Groups]
  Recall that a matrix $A$ is invertible if and only if $\det(A) \ne 0$.
  If we are going to find groups of matrices with matrix multiplication as
  the group operation, then they must be invertible at least.
  
  The following are all groups:
  \begin{itemize}
    \item the \defn{general linear group}{group!general linear} of $n \times n$ 
    matrices $(GL_{n}(\reals), \times, I_{n})$, where
    \[
      GL_{n}(\reals) = \{ A \in M_{n}(\reals) : \det(A) \ne 0\}.
    \]
    
    \item the \defn{orthogonal group}{group!orthogonal} of $n \times n$ 
    matrices $(O_{n}(\reals), \times, I_{n})$, where $O_{n}(\reals)$
    is the set of orthogonal matrices (ie.\ matrices whose columns form an
    orthonormal basis or, equivalently, which satisfy $A^{-1} = A^{t}$).
    
    \item the \defn{special linear group}{group!special linear} of $n \times
    n$ matrices $(SL_{n}(\reals), \times, I_{n})$, where
    \[
      SL_{n}(\reals) = \{ A \in M_{n}(\reals) : \det(A) = 1\}.
    \]
    
    \item the \defn{special orthogonal group}{group!special orthogonal} of
    $n \times n$ matrices $(SO_{n}(\reals), \times, I_{n})$, where $SO_{n}(\reals)$
    is the set of orthogonal matrices with determinant 1.
  \end{itemize}
  
  There isn't anything particularly special about $\reals$-valued matrices
  in the above.  Once can define $GL_{n}(\field)$,  $SL_{n}(\field)$, 
  $O_{n}(\field)$, and $SO_{n}(\field)$ for any field $\field$ (such as
  the complex numbers $\complex$, or the rational numbers $\rationals$).
  
  A \defn{unitary matrix}{matrix!unitary} is a complex-valued matrix which
  satisfies $A^{-1} = A^{*}$, where $A^{*}$ is the conjugate transpose matrix
  of $A$.  More precisely, if $A = [a_{i,j}]_{i,j=1}^{n}$, then
  \[
    A^{*} = [\overline{a_{i,j}}]^{t}.
  \]
  We then have two additional complex matrix groups
  \begin{itemize}
   \item the \defn{unitary group}{group!unitary} of $n \times n$ 
    matrices $(U_{n}(\complex), \times, I_{n})$, where $U_{n}$
    is the set of unitary matrices.
    
    \item the \defn{special unitary group}{group!special unitary} of
    $n \times n$ matrices $(SU_{n}(\complex), \times, I_{n})$, where $SU_{n}(\complex)$
    is the set of unitary matrices with determinant 1.
  \end{itemize}
\end{example}

\subsection*{Exercises}

\begin{enumerate}
  \item Let $(G, \ast, e)$ be a group, and let $x$ and $y$ be two elements
of $G$ which commute.  Prove that for any $k \in \integers$, $(xy)^{k} =
x^{k}y^{k}$.

  \item Give an example of a group and two elements of that group such that
  \[
    (xy)^{2} \ne x^{2}y^{2}.
  \]
  Provide concrete calculations to demonstrate this fact for your example.
\end{enumerate}

\subsection*{Extension: Less Than A Group}

There are algebraic objects which do not quite satisfy all the axioms of a
group, but which are nevertheless of interest.

\printindex

\end{document}
\end

