\chapter{The Structure of Groups}

We are interested in understanding the structure of groups, particularly finite
groups, as a way of potentially distinguishing groups.  In this
section we will see a number of ways of looking at structure within a group.

\section{The Subgroup Lattice}

At a very coarse level, if two groups are isomorphic then their subgroups
must be in bijective correspondence with one another.

\begin{proposition}\label{prop:subgroupisom}
  Let $G$ and $H$ be two groups, and let $\Sub(G)$ and $\Sub(H)$ be the set
  of subgroups of $G$ and $H$ respectively.  If $G \isom H$ then there is a
  bijection between $\Sub(G)$ and $\Sub(H)$.
\end{proposition}
\begin{proof}
  Since $G \isom H$ there is an isomorphism $\alpha : G \to H$, and its
  inverse function $\alpha^{-1}: H \to G$ is also an isomorphism by
  Proposition~\ref{prop:homomorphismfacts}.  Since by
  Proposition~\ref{prop:homsubgroup}, the image $\alpha(K)$ of a subgroup $K$
  of $G$ is a subgroup of $H$, we can define a function
  \begin{align*}
    \overline{\alpha} : \Sub(G) & \to \Sub(H) \\
        K & \mapsto \alpha(K).
  \end{align*}
  
  The function $\overline{\alpha}$ is one-to-one, since if we have two
  subgroups $K_{1}$ and $K_{2}$ of $G$ such that $\overline{\alpha}(K_{1}) =
  \overline{\alpha}(K_{2})$, then
  \[
    K_{1} = \alpha^{-1}(\alpha(K_{1})) = \alpha^{-1}(\overline{\alpha}(K_{1})) =
    \alpha^{-1}(\overline{\alpha}(K_{12})) = \alpha^{-1}(\alpha(K_{2})) =
    K_{2},
  \]
  since $\alpha^{-1} \circ \alpha$ is the identity function.
  
  The function $\overline{\alpha}$ is onto, since if $K$ is a subgroup of
  $H$, then $\alpha^{-1}(K)$ is a subgroup of $G$, and
  \[
    \overline{\alpha}(\alpha^{-1}(K)) = \alpha(\alpha^{-1}(K)) = K
  \]
  since $\alpha \circ \alpha^{-1}$ is the identity function.
  
  So $\overline{\alpha}$ is a bijection.
\end{proof}

\begin{corollary}
  If $G$ and $H$ are finite groups with different numbers of subgroups, then
  $G$ and $H$ cannot be isomorphic.
\end{corollary}

\begin{example}
  From Example~\ref{eg:C4subgroups} $C_{4}$, we know that $C_{4}$ has 3
  subgroups.  On the other hand, Example~\ref{eg:4groupsubgroups} tells us
  that $V$ has $5$ subgroups, so $V$ is not isomorphic to $C_{4}$.
\end{example}

However, it is certainly conceivable that two groups may have the same
number of subgroups, but fail to be isomorphic.  In this case we need to
investigate the relationships between subgroups of a group.  For instance, if
we have a group $G$ and subgroups $H$ and $K$ of $G$ such that $K \subseteq H$,
then it is immediate from Corollary~\ref{cor:subgrouptest} that $K$ is a
subgroup of $H$.  So a good starting point is to consider which subgroups are
contained in other subgroups.

\begin{example}
  From Example~\ref{eg:C4subgroups} $C_{4} = \{1, a, a^{2}, a^{3}\}$ has the subgroups
  $\{1\}$, $\langle a^{2} \rangle = \{1, a^{2}\}$ and
  $\{1, a, a^{2}, a^{3}\}$.  We can easily see that $\{1\} \le \langle a^{2}
  \rangle \le C_{4}$.
 \end{example}
 
  Noting that each of these is subgroups is contained in the next, we can
  represent this situation diagramatically as follows:
  
  \begin{picture}(2,6)(-1,-1)
    \put(0,0){\makebox(0,0){$\{1\}$}}
    \put(0,0.5){\line(0,1){1}}
    \put(0,2){\makebox(0,0){$\langle a^{2} \rangle$}}
    \put(0,2.5){\line(0,1){1}}
    \put(0,4){\makebox(0,0){$C_{4}$}}
  \end{picture}

This sort of diagram is a graphical representation of the \defn{subgroup
lattice}{lattice!subgroup} of the group.  The idea is that if a
subgroup is contained in another, with no intermediate subgroups, we write
it higher on the page and join the two subgroups with a line.  We also try
to draw the diagram  so that subgroups with the same number of elements are
the same distance up the page.

Here are some more examples.

\begin{example}
  From Example~\ref{eg:4groupsubgroups}, the vierergruppe $V = \{1, a, b, ab\}$
has the subgroups $\{1\}$,
  $\langle a \rangle = \{1, a\}$, $\langle b \rangle = \{1, b\}$,
  $\langle ab \rangle = \{1, ab\}$ and $\{1, a, b, ab\}$.
  
  The subgroup lattice of $V$ is:
  
  \begin{picture}(6,6)(-3,-1)
    \put(0,0){\makebox(0,0){$\{1\}$}}
    \put(0,0.5){\line(0,1){1}}
    \put(0.5,0.5){\line(1,1){1}}
    \put(-0.5,0.5){\line(-1,1){1}}
    \put(-2,2){\makebox(0,0){$\langle a \rangle$}}
    \put(0,2){\makebox(0,0){$\langle b \rangle$}}
    \put(2,2){\makebox(0,0){$\langle ab \rangle$}}
    \put(0,2.5){\line(0,1){1}}
    \put(1.5,2.5){\line(-1,1){1}}
    \put(-1.5,2.5){\line(1,1){1}}
    \put(0,4){\makebox(0,0){$V$}}
  \end{picture}
\end{example}

\begin{example}
  From Example~\ref{eg:C6subgroups}, the cyclic group or order $6$, $C_{6} =
  \{1, a, a^{2}, a^{3}, a^{4}, a^{5}\}$
  has the subgroups $\{1\}$,
  $\langle a^{3} \rangle = \{1, a^{3}\}$,
  $\langle a^{2} \rangle = \{1, a^{2}, a^{4}\}$,
  and $C_{6}$.
  The subgroup lattice of $C_{6}$ is:
  
  \begin{picture}(6,8)(-3,-1)
    \put(0,0){\makebox(0,0){$\{1\}$}}
    \put(-0.5,0.5){\line(-1,1){1}}
    \put(0.25,0.5){\line(1,2){1.5}}
    \put(-2,2){\makebox(0,0){$\langle a^{3} \rangle$}}
    \put(2,4){\makebox(0,0){$\langle a^{2} \rangle$}}
    \put(-1.75,2.5){\line(1,2){1.5}}
    \put(1.5,4.5){\line(-1,1){1}}
    \put(0,6){\makebox(0,0){$C_{6}$}}
  \end{picture}
\end{example}

\begin{example}
  Let $p$ be a prime number, and $C_{p}$ the cyclic group of order $p$.
  From Example~\ref{eg:Cpsubgroups} $C_{p}$ only has the trivial subgroups $\{1\}$ and $C_{p}$.
  The subgroup lattice of $C_{p}$ is always:
  
  \begin{picture}(2,4)(-1,-1)
    \put(0,0){\makebox(0,0){$\{1\}$}}
    \put(0,0.5){\line(0,1){1}}
    \put(0,2){\makebox(0,0){$C_{p}$}}
  \end{picture}
\end{example}

\begin{example}
  From Example~\ref{eg:D8subgroups}, the dihedral group\index{group!dihedral}
of order $8$, $D_{8} = \{1, a, a^{2}, a^{3}, b, ab, a^{2}, a^{3}\}$
  has subgroups $\{1\}$,
  $\langle a^{2} \rangle = \{1, a^{2}\}$,
  $\langle b \rangle = \{1, b\}$,
  $\langle ab \rangle = \{1, ab\}$,
  $\langle a^{2}b \rangle = \{1, a^{2}b\}$,
  $\langle a^{3}b \rangle = \{1, a^{3}b\}$,
  $\langle a \rangle = \{1, a, a^{2}, a^{3}\}$,
  $\langle a^{2}, b \rangle = \{1, a^{2}, b, a^{2}b\}$,
  $\langle a^{2}, ab \rangle = \{1, a^{2}, ab, a^{3}b\}$,
  and $D_{8}$.
  The subgroup lattice of $D_{8}$ is:
  
  \begin{picture}(10,8)(-5,-1)
    \put(0,0){\makebox(0,0){$\{1\}$}}
    \put(0,0.5){\line(0,1){1}}
    \put(0.5,0.5){\line(1,1){1}}
    \put(0.5,0.25){\line(2,1){3}}
    \put(-0.5,0.5){\line(-1,1){1}}
    \put(-0.5,0.25){\line(-2,1){3}}
    \put(-4,2){\makebox(0,0){$\langle b \rangle$}}
    \put(-2,2){\makebox(0,0){$\langle a^{2}b \rangle$}}
    \put(0,2){\makebox(0,0){$\langle a^{2} \rangle$}}
    \put(2,2){\makebox(0,0){$\langle ab \rangle$}}
    \put(4,2){\makebox(0,0){$\langle a^{3}b \rangle$}}
    \put(0,2.5){\line(0,1){1}}
    \put(3.5,2.5){\line(-1,1){1}}
    \put(2,2.5){\line(0,1){1}}
    \put(0.5,2.5){\line(1,1){1}}
    \put(-3.5,2.5){\line(1,1){1}}
    \put(-0.5,2.5){\line(-1,1){1}}
    \put(-2,2.5){\line(0,1){1}}
    \put(-2,4){\makebox(0,0){$\langle a^{2}, b \rangle$}}
    \put(0,4){\makebox(0,0){$\langle a \rangle$}}
    \put(2,4){\makebox(0,0){$\langle a^{2}, ab \rangle$}}
    \put(0,4.5){\line(0,1){1}}
    \put(1.5,4.5){\line(-1,1){1}}
    \put(-1.5,4.5){\line(1,1){1}}
    \put(0,6){\makebox(0,0){$D_{8}$}}
  \end{picture}
\end{example}

Notice that in these diagrams, there is always a unique smallest subgroup which
is bigger than any pair of subgroups.  This is a corollary of
Theorem~\ref{thm:subgroupgenbyset}.

\begin{corollary}
  If $G$ is a group and $H$ and $K$ are subgroups of $G$, then $\langle H
  \union K \rangle$ is the smallest subgroup which contains both $H$ and $K$.
\end{corollary}

This subgroup is usually quite different from the union of the two sets. 
Indeed, we have the following:

\begin{proposition}
  Let $H$ and $K$ be subgroups of $G$.  Then $H \union K$ is a subgroup if and only
  if either $H \subseteq K$ or $K \subseteq H$.
\end{proposition}
\begin{proof}
  If $H \subseteq K$, then $H \union K = K$, so $H \union K$ is a subgroup.
  Similarly, if $K \subseteq H$, then $H \union K = H$, so $H \union K$ is
  a subgroup.
  
  Conversely, if neither $H$ nor $K$ is a subset of the other, then there is
  some $x \in H \setminus K$ and $y \in K \setminus H$.  Also $x^{-1} \in H$,
  since $H$ is a subgroup. But then if $xy \in H$, we have $x^{-1}(xy) = y
  \in H$, but since $y \in K \setminus H$, this means that $y \notin H$, which
  is a contradicition.  Therefore $xy \notin H$.  But a similar argument
  shows that $xy \notin K$.  So $xy \notin H \union K$.  So $H \union K$
  is not a subgroup of $G$.
\end{proof}

The following theorem shows us that there is also a subgroup which is contained
in both $H$ and $K$.

\begin{theorem}
  Let $(G, \ast, e)$ be a group, and $H$ and $K$ subgroups of $G$.  Then
  $H \intersect K$ is the largest subgroup of $G$ which is contained in both
  $H$ and $K$.
\end{theorem}
\begin{proof}
  We first need to show that $H \intersect K$ is a subgroup.  If $x$,
  $y \in H \intersect K$, then $xy^{-1} \in H$, since $H$ is a subgroup,
  and $xy^{-1} \in K$, since $K$ is a subgroup.  Therefore $xy^{-1} \in
  H \intersect K$, and so by Corollary~\ref{cor:subgrouptest},
  $H \intersect K$ is a subgroup of $G$.
  
  Since $H \intersect K$ is the largest set contained in both $H$ and $K$,
  it must also be the largest subgroup contained in both.
\end{proof}

We will denote $\langle H \union K \rangle$ by $H \vee K$ and call it the
\defn{join}{join!of subgroups} of $H$ and $K$.  We will denote $H \intersect K$
by $H \wedge K$, and call it the \defn{meet}{meet!of subgroups} of $H$ and $K$.
The reason for this terminology will be come clear when we look at abstract
lattices.

The significance of the lattice of subgroups is that if two groups do not have
similar lattices of subgroups, they cannot be isomorphic, so it provides a
nice pictorial way of demonstrating that two groups are distinct.  To show
this, we first need to show that homomorphisms preserve the relationship of
inclusion of subgroups.

\begin{proposition}\label{prop:homsubgrouporderpreserving}
  If $G$ and $H$ are groups, $\alpha : G \to H$ is a homomorphism, and $K_{1}$
  and $K_{2}$ are subgroups of $G$ with $K_{1} \subseteq K_{2}$, then
  $\alpha(K_{1})$ is a subgroup of $\alpha(K_{2})$.
  
  Furthermore, if $\alpha$ is a monomorphism, and $K_{1} \subset K_{2}$,
  then $\alpha(K_{1})$ not equal to $\alpha(K_{2})$.
\end{proposition}
\begin{proof}
  We know from Proposition~\ref{prop:homsubgroup} that $\alpha(K_{1})$ and
  $\alpha(K_{2})$ are subgroups of $H$, and it is immediate from the
  definition of the image of a set under a function that $\alpha(K_{1}) 
  \subseteq \alpha(K_{2})$ if $K_{1} \subseteq K_{2}$.  So $\alpha(K_{1}) 
  \le \alpha(K_{2})$.
  
  If $\alpha$ is a monomorphism in addition, then since there is some $g \in
  K_{2}$, but not in $K_{1}$, we cannot have $\alpha(h) = \alpha(g)$ for any
  $h \in K_{1}$ (otherwise $\alpha$ would not be one-to-one).  Hence
  $\alpha(K_{1})$ is properly contained in $\alpha(K_{2})$.
\end{proof}

\begin{corollary}\label{cor:hommeetandjoin}
  If $G$ and $H$ are groups, $\alpha : G \to H$ is a homomorphism, and $K_{1}$
  and $K_{2}$ are subgroups of $G$, then $\alpha(K_{1} \vee K_{2}) =
  \alpha(K_{1}) \vee \alpha(K_{2})$ and $\alpha(K_{1} \wedge K_{2}) =
  \alpha(K_{1}) \wedge \alpha(K_{2})$.
\end{corollary}
\begin{proof}
  We know that $\alpha(K_{1}) \vee \alpha(K_{2})$ is the smallest
  subgroup which contains both $\alpha(K_{1})$ and $\alpha(K_{2})$, but
  since $K_{1}$ and $K_{2} \subseteq K_{1} \vee K_{2}$, we have that
  $\alpha(K_{1})$ and $\alpha(K_{2}) \subseteq \alpha(K_{1} \vee K_{2})$,
  hence $\alpha(K_{1}) \vee \alpha(K_{2}) \subseteq \alpha(K_{1} \vee
  K_{2})$.

  Conversely, if $\alpha^{-1}(\alpha(K_{1}) \vee \alpha(K_{2}))$ is a
  subgroup of $G$ which contains both $K_{1}$ and $K_{2}$, so $K_{1} \vee
  K_{2} \subseteq \alpha^{-1}(\alpha(K_{1}) \vee \alpha(K_{2}))$, and hence
  \[
    \alpha(K_{1} \vee K_{2}) \subseteq \alpha(\alpha^{-1}(\alpha(K_{1}) \vee
    \alpha(K_{2}))) = \alpha(K_{1}) \vee \alpha(K_{2}).
  \]
  
  Hence $\alpha(K_{1}) \vee \alpha(K_{2}) = \alpha(K_{1} \vee K_{2})$.
  
  The proof of the case for $\wedge$ is left as an exercise.
\end{proof}

We will say that two groups $G$ and $H$ have corresponding, or
isomorphic, subgroup lattices if there is a bijection $f$ from
$\Sub(G)$ to $\Sub(H)$ so that $f(K_{1}) \vee f(K_{2}) = f(K_{1} \vee
K_{2})$, and $f(K_{1}) \wedge f(K_{2}) = f(K_{1} \wedge K_{2})$.

\begin{corollary}\label{cor:latticeisomtest}
  If $G$ and $H$ are two groups whose subgroup lattices do not
  correspond, then $G$ and $H$ are not isomorphic.
\end{corollary}
\begin{proof}
  If $G$ and $H$ are isomorphic, then
  Proposition~\ref{prop:subgroupisom} tells us that
  $\overline{\alpha}$ is a bijection from $\Sub(G)$ to $\Sub(H)$, and
  Corollary~\ref{cor:hommeetandjoin} says that
  $\overline{\alpha}(K_{1} \vee K_{2}) = \overline{\alpha}(K_{1}) \vee
  \overline{\alpha}(K_{2})$ and $\overline{\alpha}(K_{1} \wedge K_{2})
  = \overline{\alpha}(K_{1}) \wedge \overline{\alpha}(K_{2})$.  So 
  isomorphic groups have corresponding subgroup lattices.
  
  The contrapositive of this result is the corollary.
\end{proof}

\begin{example}
  The groups $C_{4}$ and $V$ have different subgroup lattices, so they 
  are not isomorphic.
\end{example}

Note that the converse of the corollary is not true.  We know, for example,
that if $p$ is prime, the groups $C_{p}$ all have corresponding subgroup
lattices, yet the groups are clearly not isomorphic.

\subsection*{Exercises}

\begin{exercises}
  \item Find the subgroup lattice of the group $D_{6}$.
  
  \item Find the subgroup lattice of the group $C_{8}$.
      
  \item Find the subgroup lattice of the group $C_{2} \cross C_{4}$.
      
  \item Find the subgroup lattice of the group  $C_{2} \cross C_{2} \cross
    C_{2}$.
  
  \item Find the subgroup lattice of the group $D_{10}$.
    
  \item Find the subgroup lattice of the group $D_{12}$.
      
  \item Find the subgroup lattice of the group $A_{4}$
  
  \item Complete the proof of Corollary~\ref{cor:hommeetandjoin}.
\end{exercises}

\section{Extension: Lattices}

The pattern that subgroups of a group make under inclusion is a particular
example of a general phenomenon.  The key idea is that we know when one
subgroup is ``larger'' than another, that we can find the largest thing
smaller than two subgroups (the meet) and that we can find the smallest thing 
larger than the two subgroups (the join).

This idea is essentially the same as what happens with general subsets of a set.
We know when one subset is ``larger'' than another, we can find the
largest thing smaller than two subsets (the intersection) and we can find the
smallest thing larger than the two subsets (the union).

To explore this similarity further, we need to introduce a general concept that
we can use to model the idea of one thing being larger then another.

\begin{definition}
  Let $X$ and $Y$ be sets.  A \defn{relation}{relation} between $X$ and $Y$
  is a subset $R$ of $X \cross Y$, where $x \in X$ and $y \in Y$ are
  considered to be related by $R$ if and only if $(x,y) \in R$.
  
  We write $xRy$ if $(x,y) \in R$.  If $X = Y$ we say that $R$ is a relation
  on $X$.
\end{definition}

Note that you should not confuse this definition of relation with the notion
of a relation on the elements of a group.

The concept of a relation is extremely general, and can be used to model a
great many fundamental mathematical concepts.

\begin{example}
  If $X$ is any set, equality can be regarded as the relation $R = \{(x,x) : x
  \in X\} \subseteq X \cross X$.  Here $xRy$ if and only if $(x,y) \in R$ if and
  only if $x = y$.
\end{example}

\begin{example}
  If $X$ and $Y$ are any sets and $f : X \to Y$ is a function, the graph
  $R = \{(x, f(x)): x \in X\} \subset X \cross Y$ is a relation where $xRy$ if and only if
  $y = f(x)$.  In fact functions are sometimes defined in this way as a
  special case of the concept of a relation.
\end{example}

\begin{example}
  In the real numbers, the set $L = \{ (x,y) : x \le y\} \in \reals \cross \reals$
  is a relation where $xLy$ if and only if $x \le y$.
\end{example}

\begin{example}
  If $\powerset(X)$ is the power set of some set $X$, then the set
  $\subseteq = \{(A,B): \text{$A$ is a subset of $B$}\}$ is a relation
  where $A \subseteq B$ if and only if $A$ is a subset of $B$.
\end{example}

\begin{example}
  If $G$ is a group, and $X$ is the set of subgroups of $G$, then the set
  $\le = \{(A,B): \text{$A$ is a subgroup of $B$}\}$ is a relation where
  $A \le B$ if and only $A$ is a subgroup of $B$.
\end{example}

Because of the generality of relations, we need to impose some additional
conditions to make them useful in modelling particular situations.

\begin{definition}
  Let $X$ and $Y$ be sets, and $R$ a relation on $X$.  Let $x, y$ and $z \in X$.
  
  We say that $R$ is \defn{reflexive}{relation!reflexive} if $xRx$ for all
  $x \in X$.
  
  We say that $R$ is \defn{symmetric}{relation! symmetric} if $xRy$ implies
  $yRx$.
  
  We say that $R$ is \defn{antisymmetric}{relation! antisymmetric} if $xRy$ and
  $yRx$ implies $x = y$.
  
  We say that $R$ is \defn{transitive}{relation!transitive} if $xRy$ and
  $yRz$ implies $xRz$.
  
  A \defn{partial order}{order!partial} on $X$ is a relation on $X$ which is
  reflexive, transitive and antisymmetric.
\end{definition}

\begin{example}
  The $\le$, $<$, $>$ and $\ge$ relations on any set of numbers. All are
  antisymmetric and transitive, but not symmetric.  The relations $\le$
  and $\ge$ are reflexive, but $<$ and $>$ are not.
\end{example}

\begin{example}
  The relation $\subseteq$ of $\powerset(X)$ is a partial order.
\end{example}

\begin{example}
  The relation $\le$ on the set of subgroups of a group is a partial order.
\end{example}

Notice that if $\preceq$ is a partial order on a
set $X$, and $x$, $y \in X$, then it may be the case that neither $x \preceq y$
nor $y \preceq x$.  If this is the case, we say that $x$ and $y$ are
\defn{incomparable elements}{incomparable elements} of $X$.

\begin{example}
  The sets $\{1\}$ and $\{2,3\}$ are incomparable elements of
  $\powerset(\{1,2,3\})$ under the subset partial order $\subseteq$.
\end{example}

So a partial order seem to encapsulate the general idea of something being
bigger than something else.  Now we need to model the idea of a meet and a
join.

\begin{definition}
  If $X$ is a set, $\preceq$ is a partial order on $X$, and $x$ and $y \in X$,
  then $a \in X$ is a \defn{lower bound}{bound!lower} for $x$ and $y$ if $a \preceq x$ and
  $a \preceq y$.  We say that $a$ is the \defn{greatest lower
  bound}{bound!greatest lower} of $x$ and $y$ if given any lower bound
  $z$ of $x$ and $y$, we have that $z \preceq a$.
  
  Similarly, $a$ is an \defn{upper bound}{bound!upper} for $x$ and $y$ if
  $x \preceq a$ and $y \preceq a$.  We say that $a$ is the \defn{least upper
  bound}{bound!least upper} of $x$ and $y$ if given any upper bound
  $z$ of $x$ and $y$, we have that $a \preceq z$.
  
  In general, we denote the greatest lower bound of $x$ and $y$ by $x \wedge y$,
  and the least upper bound of $x$ and $y$ by $x \vee y$.
  
  A \defn{lattice}{lattice} $(X, \preceq)$ is a set with a partial order such that
  every pair of elements has a greatest lower bound and a least upper bound.
\end{definition}

The examples of partial orders earlier in this section all have the lattice
property.

\begin{example}
  The pair $(\reals, \le)$ is a lattice.  Since for any $x$ and $y$, we have
  $x \le y$ or $y \le x$ (or both), then $x \vee y = \min x,y$ and 
  $x \wedge y = \max x,y$.
\end{example}

\begin{example}
  If $X$ is any set, then the pair $(\powerset(X), \subseteq)$ is a
  lattice.  Given any two subsets $A$ and $B$ of $X$, we have $A
  \wedge B = A \union B$ and $A \vee B = A \intersect B$.
\end{example}

\begin{example}
  The subgroup relation $\le$ on the set of subgroups $\Sub(G)$ of a
  group $G$ makes $(\Sub(G), \le)$ a lattice, since if $H$ and $K \in
  \Sub(G)$ we have $H \vee K = \langle H \union K \rangle$ and $H
  \wedge K = H \intersect K$.
\end{example}

\begin{example}\label{eg:abstractlattice}
  Let $X = \{0, x, y, 1\}$, and let $\preceq$ be the partial order
  defined by $0 \preceq x$, $0 \preceq y$, $0 \preceq x$, $0 \preceq
  1$, $x \preceq 1$, $y \preceq 1$ and $y \preceq 1$.  Then $(X,
  \preceq)$ is a lattice, and we can represent it diagramatically as

  \begin{picture}(6,6)(-3,-1)
    \put(0,0){\makebox(0,0){$0$}}
    \put(0,0.5){\line(0,1){1}}
    \put(0.5,0.5){\line(1,1){1}}
    \put(-0.5,0.5){\line(-1,1){1}}
    \put(-2,2){\makebox(0,0){$x$}}
    \put(0,2){\makebox(0,0){$y$}}
    \put(2,2){\makebox(0,0){$z$}}
    \put(0,2.5){\line(0,1){1}}
    \put(1.5,2.5){\line(-1,1){1}}
    \put(-1.5,2.5){\line(1,1){1}}
    \put(0,4){\makebox(0,0){$1$}}
  \end{picture}
\end{example}


In the previous section, we showed that if the subgroup lattice of two 
groups didn't agree, then the groups could not be isomorphic.  The 
heart of the result was showing that the isomorphism $\alpha$ between the 
groups produced a function between the subgroups $\overline{\alpha}$ 
which preserved meet and join.

\begin{definition}
  Let $(X, \preceq)$ and $(Y, \le)$ be two partially ordered sets.  A
  function $\alpha : X \to Y$ is an order-preserving function if whenever
  we have $x$ and $y \in X$ such that $x \preceq y$, we have $\alpha(x) 
  \le \alpha(y)$.
  
  If $(X, \preceq)$ and $(Y, \le)$ are lattices, then $\alpha: X \to
  Y$ is a lattice homomorphism if $\alpha(x \vee y) = \alpha(x) \vee
  \alpha(y)$, and $\alpha(x \wedge y) = \alpha(x) \wedge \alpha(y)$.
  If $\alpha$ is a bijective lattice homomorphism, we say that it is 
  a lattice isomorphism.
\end{definition}

\begin{example}
  Let $(X, \preceq)$ be as in Example~\ref{eg:abstractlattice}, and 
  $V$ be the four-group.
  The function $\alpha : X \to \Sub(V)$ defined by the table
  \[
    \begin{array}{cc}
      t & \alpha(t) \\
      \hline
      0 & \{e\} \\
      x & \{e, a\} \\
      y & \{e, b\} \\
      z & \{e, ab\} \\
      1 & V 
    \end{array}
  \]
  is a lattice isomorphism.  Perhaps the easiest way to grasp this 
  fact is to observe that the diagrams for each lattice correspond.
\end{example}


\begin{example}
  Consider the homomorphism $\alpha : V \to C_{4}$ of 
  Example~\ref{eg:4grouphom}.  The corresponding map between 
  subgroups of these groups is $\overline{\alpha}$, and is given by 
  the following table
  \[
    \begin{array}{cc}
      H & \overline{\alpha}(H) \\
      \hline
      \{e\} & \{e\} \\
      \{e, a\} & \{e, a^{2}\} \\
      \{e, b\} & \{e, a^{2}\} \\
      \{e, ab\} & \{e\} \\
      V & \{e, a^{2}\}
    \end{array}
  \]
  is a lattice homomorphism.
\end{example}

The essential content of
Proposition~\ref{prop:homsubgrouporderpreserving} and
Corollary~\ref{cor:hommeetandjoin} is then:

\begin{proposition}
  If $G$ and $H$ are groups, and $\alpha: G \to H$ is a homomorphism,
  then $\overline{\alpha} : \Sub(G) \to \Sub(H)$ is an
  order-preserving map.
\end{proposition}

\begin{corollary}\label{cor:grouphomtolathom}
  If $G$ and $H$ are groups, and $\alpha: G \to H$ is a homomorphism, 
  then $\overline{\alpha} : \Sub(G) \to \Sub(H)$ is a lattice 
  homomorphism.
\end{corollary}

So we can then re-phrase Corollary~\ref{cor:latticeisomtest} in the 
following way:

\begin{corollary}
  If $G$ and $H$ are groups, and the subgroup lattices of $G$ and $H$
  are not isomorphic, then $G$ and $H$ are not isomoprhic.
\end{corollary}

We can use Exercise~\ref{ex:orderfrommeetandjoin} to prove the 
following proposition.

\begin{proposition}\label{prop:latticehomisorderpreserve}
  Let $(X, \preceq)$ and $(Y, \le)$ be lattices, and $\alpha : X \to 
  Y$ a lattice homomorphism.  Then $\alpha$ is an order-preserving 
  function.
\end{proposition}
\begin{proof}
  See Exercise~\ref{ex:latticehomisorderpreserve}.
\end{proof}

In fact, we could prove Corollary~\ref{cor:hommeetandjoin} directly, 
and use this proposition to conclude that 
Proposition~\ref{prop:homsubgrouporderpreserving} must hold.

So the abstract concept of a lattice helps us understand the structure 
of subgroups within a group, just as the abstract structure of a group 
helps us understand concrete situations such as the symmetries of a 
set.

\subsection*{Exercises}

\begin{exercises}
  \item\label{ex:orderfrommeetandjoin} Let $\preceq$ be a partial
    order on $X$.  Show that if $x \preceq y$ that $x \wedge y = x$ and
    $x \vee y = y$.
  
  \item\label{ex:latticehomisorderpreserve} Prove
    Proposition~\ref{prop:latticehomisorderpreserve}.
  
  \item Consider the natural numbers $\naturals$ with the ``divides''
    relation $x \mid y$ if and only if $y = kx$ for some $k \in \naturals$
    (ie.\ if $x$ divides $y$).  Show that $\mid$ is a partial order, and that
    $x \wedge y$ is the greatest common divisor of $x$ and $y$ and $x \vee y$
    is the least common multiple of $x$ and $y$.
  
  \item Show that if $\preceq$ is a partial order on $X$, then the reverse
    relation $\succeq$ defined by $x \succeq y$ if and only if $y \preceq x$ is
    a partial order.  Show that $(X, \preceq)$ is a lattice if and only if
    $(X, \succeq)$ is a lattice.
    
  \item Let $F(D, \reals)$ be the set of real-valued functions on some fixed
    domain $D \subseteq \reals$ (ie.~the typical functions considered in calculus).
    Show that the relation defined by $f \le g$ if $f(x) \le g(x)$ for all
    $x \in D$ is a partial order.
    
    Give an example of two functions which are incomparable.
    
    Show that $(F(D, \reals), \le)$ is a lattice, where $f \wedge g$ and
    $f \vee g$ are the functions defined by
    \[
      (f \wedge g)(x) = \min f(x), g(x) \qquad \text{and} \qquad
        (f \vee g)(x) = \max f(x), g(x)
    \]
    respectively.
\end{exercises}

\section{The Centre and Centralizers}

Abelian groups are much nicer to work with algebraically than non-Abelian
groups.  However, even in the case of non-Abelian groups there may be large
parts of the group which commute with each other.

Recall that two elements $x$ and $y \in G$ commute with one another if
\[
  xy = yx.
\]
Multiplying both sides on the right by $x^{-1}$, we can state this
equivalently as saying that $x$ and $y$ commute if and only if
\[
  xyx^{-1} = y,
\]
or, multiplying on the other side, if and only if
\[
  x^{-1}yx = y.
\]

The set of all elements which commute with every other element of the 
group is called the \defn{centre}{centre} of the group, 
denoted by $Z(G)$.  At the very least we have the identity $e \in 
Z(G)$, but it is potentially much larger.

\begin{example}
  The centre of $D_{6}$ is $\{1\}$.  Clearly $1$ always commutes with any
  element of $D_{6}$.  For the other elements we can always find an element
  with which it does not commute.  For example, $a$ does not commute with
  $b$, since $ba = (a^{2})b \ne ab$.  The same equation shows that $a^{2}$
  does not commute with $b$.  So $a$, $a^{2}$ and $b$ are not in the centre.
  Similarly $(ab)a = a^{3}b = a^{2}(ab)$, so $ab$ does not commute with $a$,
  and $(a^{2}b)a = a^{4}b = a^{2}(a^{2}b)$, so $a^{2}b$ does not commute
  with $a$.
\end{example}

\begin{example}\label{eg:D8centre}
  The centre of $D_{8}$ is $\{1, a^{2}\}$.  This follows since $a^{n}a^{2} =
  a^{n+2} = a^{2}a^{n}$ and
  \[
    (a^{n}b)a^{2} = a^{n}a^{3}ba = a^{n+3}a^{3}b = a^{n+6}b = a^{n+2}b =
    a^{2}(a^{n}b).
  \]
   for $n = 0, 1, 2, 3$.  So $a^{2} \in Z(D_{8})$.
   
   On the other hand $(a^{n}b)a = a^{n}a^{3}b = a^{3}(a^{n}b)$ for $n = 0, 1,
   2, 3$, so $a$, $a^{3}$, and $a^{n}b$ are not in the centre.
\end{example}

\begin{example}
  The centre of $GL_{n}(\reals)$ is the set
  \[
   \reals^{\times} I_{n} = \left\{\begin{bmatrix}
     a & 0 & \cdots 0 \\
     0 & a & \cdots 0 \\
     \vdots & \vdots & \ddots & \vdots \\
     0 & 0 & \cdots a \\
   \end{bmatrix} : a \in \reals^{\times} \right\}.
  \]
  It is easy to verify with matrix multiplication that every element of this
  set commutes with every matrix.
  
  To simplify calculations, we will just look at the case $n = 2$.  To see that
  these are the only possible elements of the centre, we note that if
  \[
    \begin{bmatrix}
      a & b \\
      c & d
    \end{bmatrix} \in Z(GL_{2}(\reals))
  \]
  then we must have
  \[
    \begin{bmatrix}
      a & b \\
      c & d
    \end{bmatrix}
    \begin{bmatrix}
      0 & 1 \\
      1 & 0
    \end{bmatrix}
    =
    \begin{bmatrix}
      0 & 1 \\
      1 & 0
    \end{bmatrix}
    \begin{bmatrix}
      a & b \\
      c & d
    \end{bmatrix},
  \]
  or,
  \[
    \begin{bmatrix}
      b & a \\
      d & c
    \end{bmatrix}
    =
    \begin{bmatrix}
      c & d \\
      a & b
    \end{bmatrix}
  \]
  and so we conclude that $a = d$ and $b = c$.  We also must have
  \[
    \begin{bmatrix}
      a & b \\
      b & a
    \end{bmatrix}
    \begin{bmatrix}
      0 & -1 \\
      1 & 0
    \end{bmatrix}
    =
    \begin{bmatrix}
      0 & -1 \\
      1 & 0
    \end{bmatrix}
    \begin{bmatrix}
      a & b \\
      b & a
    \end{bmatrix},
  \]
  or,
  \[
    \begin{bmatrix}
      b & -a \\
      a & -b
    \end{bmatrix}
    =
    \begin{bmatrix}
      -b & -a \\
      a & b
    \end{bmatrix}
  \]
  so $b = -b$.  Hence $b = 0$, and so a matrix is in the centre only if it
  is in the set $\reals^{\times}I_{2}$.
\end{example}

The centre is an extremely nice subset of the group.

\begin{proposition}\label{prop:centresubgroup}
  Let $G$ be a group.  Then $Z(G)$ is an Abelian subgroup of $G$.
\end{proposition}
\begin{proof}
  Given $x$, $y \in Z(G)$, we observe that for any $z \in G$ we have
  \[
    (xy)z = xzy = z(xy),
  \]
  so $xy \in Z(G)$.  Similarly,
  \[
    x^{-1}z = x^{-1}zxx^{-1} = zx^{-1},
  \]
  so $x^{-1} \in Z(G)$.  Hence $Z(G)$ is a subgroup of $G$.
  
  It is immediate that $Z(G)$ is also Abelian, since every element of $Z(G)$ 
  commutes with every element of $G$, and $Z(G)$ is a subgroup.
\end{proof}

You can think of the centre as being a measure of how close the group $G$ is
to being Abelian.  If $Z(G) = G$, then the group is Abelian, while if $Z(G)$
is a large subgroup, then $G$ can be thought of having a large Abelian
component.  On the other hand, if $Z(G) = \{e\}$, the group is very far from
being Abelian.

\begin{proposition}
  Let $G$ and $H$ be isomorphic groups.  Then $Z(G)$ and $Z(H)$ are
  isomorphic groups.
\end{proposition}
\begin{proof}
  We know that there is an isomorphism $\alpha : G \to H$.  If $x \in Z(G)$,
  then for any $y \in H$, we have that there is some unique $u \in G$ such
  that $\alpha(u) = y$, and
  \[
    \alpha(x)y = \alpha(x)\alpha(u) = \alpha(xu) = \alpha(ux) =
    \alpha(u)\alpha(x) = y\alpha(x).
  \]
  So $\alpha(x) \in Z(H)$.  An identical argument shows that if $x \in
  Z(H)$, then $\alpha^{-1}(x) \in Z(G)$.
  
  Hence $\alpha(Z(G)) = Z(H)$, and the restriction of $\alpha$ to $Z(G)$ is
  an isomorphism between the centres.
\end{proof}

\begin{corollary}
  If $G$ and $H$ have centres which are not isomorphic, then $G$ and $H$ are
  not isomorphic.
\end{corollary}

More generally, rather than asking which elements commute with the entire
group we might ask which elements commute with some subset of the group.
If $X \subset G$ is any subset of the group $G$,
then the \defn{centralizer}{centralizer} of $X$ is the set of all elements 
which commute with every element of $X$, ie.
\[
  Z_{G}(X) = \{y \in G : yx = xy\} = \{y \in G : y^{-1}xy = x\} =
  \{y \in G : yxy^{-1} = x\}.
\]
This set always contains at least the identity element $e$.

\begin{proposition}
  Let $G$ be a group, and $X \subset G$.  Then $Z_{G}(X)$ is a
  subgroup of $G$.  If $Y \subseteq X$, then $Z_{G}(X)$ is a subgroup of
  $Z_{G}(Y)$.
\end{proposition}
\begin{proof}
  The proof of the first part of this proposition is essentially the same
  as the proof of the first part of Proposition~\ref{prop:centresubgroup}, but
  with $z \in X$ rather than $z \in G$.
  
  The second part follows from the fact that if $zx = xz$ for every $x \in
  X$, then it must also hold for every element of $Y$, since $Y \subseteq
  X$.  Therefore $Z_{G}(X) \subseteq Z_{G}(Y)$.
\end{proof}

In particular, this proposition implies that $Z(G) = Z_{G}(G) \subseteq
Z_{G}(X)$ for all $X$.

The case where $X$ contains a single element $g$ is important enough to have
its own, slightly different notation.  We define
\[
  Z_{G}(g) = \{y \in G : yg = gy\} = \{y \in G : y^{-1}gy = g\} =
  \{y \in G : ygy^{-1} = g\}.
\]

\begin{example}
  The consider the element $a$ of $D_{6}$.  Then $Z_{D_{6}}(a) = \{e, 
  a, a^{2}\}$.  The elements $b$, $ab$ and $a^{2}b$ are not elements 
  of $Z_{D_{6}}(a)$.  Similarly, we have:
  \begin{align*}
    Z_{D_{6}}(e) &= D_{6},\\
    Z_{D_{6}}(a^{2}) &= \{e, a, a^{2}\},\\
    Z_{D_{6}}(b) &= \{e, b\},\\
    Z_{D_{6}}(ab) &= \{e, ab\},\\
    Z_{D_{6}}(a^{2}b) &= \{e, a^{2}b\},\\
  \end{align*}
\end{example}

Once again, these subsets have very nice properties.

\begin{proposition}
  Let $G$ be a group and $g \in G$.  Then $\langle g \rangle$ is a subgroup of
  $G$. Furthermore, $Z_{G}(g) = G$ if and only if $g \in Z(G)$
\end{proposition}
\begin{proof}
  We note that $g \in Z_{G}(g)$, and since $\langle g \rangle$ is the
  smallest subgroup of $G$ containing $g$, we have that $\langle g \rangle$
  must be a subgroup of $Z_{G}(g)$.
  
  If $g \in Z(G)$, $gx = xg$ for all  $x \in G$, so $Z_{G}(g) = G$.
  
  On the other hand, if $Z_{G}(g) = G$, then that implies that for any $x
  \in G$, $xg = gx$, and so $g \in Z(G)$.
\end{proof}

You can think of the centralizer of $g$ as measuring how close $g$ is to
being an element of the centre, or how close it is to commuting with
everything.

The centralizer plays a key role in the discussion of conjugacy later
in this chapter.  We finish up with one last result which links the 
centre and centralizers of elements.

\begin{proposition}
  Let $G$ be a group.  Then $Z(G)$ is the intersection of all the 
  subgroups $Z_{G}(g)$.
\end{proposition}
\begin{proof}
  We know that $Z(G) \subseteq Z_{G}(g)$ for every $g$, so
  \[
    Z(G) \subseteq \bigcap_{g \in G} Z_{G}(g).
  \]
  On the other hand, if $x \in Z_{G}(g)$ for every $g \in G$, then 
  $xg = gx$ for every $g \in G$, and so $x \in Z(G)$.  Hence
  \[
    \bigcap_{g \in G} Z_{G}(g) \subseteq Z(G).
  \]
\end{proof}

\subsection*{Exercises}

\begin{exercises}
  \item  Find the centre of the group $C_{2} \cross D_{6}$.
  
  \item  Find the centre of the group $D_{10}$.
  
  \item  Show that $Z(D_{2n}) = \{1\}$ if $n$ is odd, and $Z(D_{2n}) = 
    \{1, a^{n/2}\}$ if $n$ is even.
  
  \item  Find the centralizers of each element of $D_{8}$.
  
  \item  Find the centralizers of each element of $C_{2} \cross D_{6}$.
  
  \item  Find the centralizers of each element of $S_{4}$.  What is 
    the center of this group?
  
  \item  Show that $Z_{G}(X) = Z_{G}(\langle X \rangle)$.
\end{exercises}


\section{Cosets}

There are other subsets of groups which it seems should have some
significance.  For example in $S_{3}$, the set of ``reflection
permutations'', ie.  those permutations with odd parity, is $\{ (1,2),
(1,3), (2,3) \}$ and is not a subgroup.  Nevertheless, the elements
have a commonality.  To understand such a situation, we need to
introduce some new notation.

If $\ast: A \cross B \to C$ is any binary relation, then given any 
$x \in X$ and $Y \subseteq B$, we define
\[
  x \ast Y = \{ x \ast y: y \in Y\} \subseteq C.
\]
Similarly, if $y \in B$ and $X \subseteq A$, we define
\[
  X \ast y = \{ x \ast y: x \in X\} \subseteq C.
\]
And analagously, we define
\[
  X \ast Y = \{ x \ast y: x \in X, y \in Y\} \subseteq C.
\]
We will often omit the operation $\ast$ and simply write $xY$, $Xy$ 
and $XY$ respectively.

If $\ast$ is a binary operation on $A$, and $X \subseteq A$, we will 
sometimes write
\[
  X^{n} = \underbrace{X \ast X \ast \cdots \ast X}_{\text{$n$ times}}.
\]
This is far from ideal notation, since it conflicts with the 
Cartesian product
\[
  X^{n} = \underbrace{X \cross X \cross \cdots \cross X}_{\text{$n$ times}}.
\]
However, it is usually clear from context which of the two 
possibilities we mean.

If $(G, \ast, e)$ is a group, and $X \subseteq G$ then we can also write
\[
  X^{-1} = \{x^{-1} : x \in X\}.
\]

Using this notation, we can write Theorem~\ref{thm:subgrouptest} and
Corollary~\ref{cor:subgrouptest} as follows:

\begin{corollary}
  Let $G = (G, \ast e)$ be a group, and $H \subseteq G$. If $H^{2} 
  \subseteq H$ and $H^{-1} \subseteq H$, then $H \le G$.
\end{corollary}

\begin{corollary}
  Let $G = (G, \ast e)$ be a group, and $H \subseteq G$. If $HH^{-1} 
  \subseteq H$, then $H \le G$.
\end{corollary}

In fact, if $H$ is a finite subset of $G$, we can use an even weaker 
condition:

\begin{proposition}\label{prop:subgrouptest3}
  Let $G = (G, \ast e)$ be a group, and $H$ be a finite subset $G$. If $H^{2} 
  = H$, then $H \le G$.
\end{proposition}

To prove this fact, we need a lemma which we will use often in what 
follows:

\begin{lemma}\label{lemma:translation}
  Let $G = (G, \ast e)$ be a group, $H \subseteq G$, and $g \in G$. 
  Then the \defn{right translation}{translation!right} by $g$ function
  $\rho_{g} : H \to Hg$ defined by $\rho_{g}(x) = xg$ is a bijection.
  
  Similarly, the \defn{left translation}{translation!left} by $g$
  function $\lambda_{g} : H \to gH$ defined by $\lambda_{g}(x) = gx$
  is a bijection.
  
  We also have $|H| = |gH| = |Hg|$.
\end{lemma}

\begin{proof}
  That $\rho_{g}$ is onto is trivial, for
  \[
    \rho_{g}(H) = \{ \rho_{g}(x) : x \in H \} = \{ xg : x \in H \} = 
    Hg.
  \]
  
  On the other hand, if $\rho_{g}(x_{1}) = \rho_{g}(x_{2})$, then 
  this means that $x_{1}g = x_{2}g$, and the cancellation law 
  (Proposition~\ref{prop:cancellation}) says that $x_{1} = x_{2}$.  
  Hence $\rho_{g}$ is one-to-one, and so $\rho_{g}$ is a bijection.
  
  The result for $\lambda_{g}$ is similar.
  
  Since we have bijections between $H$ and $Hg$, and $H$ and $gH$, all 
  three sets must have the same cardinality 
  (Proposition~\ref{prop:functionfacts}).
\end{proof}

With this in hand, the proof is simple.

\begin{proof}[Proposition~\ref{prop:subgrouptest3}]
  The hypothesis $H^{2} = H$ implies that $xy \in H$ for all $x$, $y 
  \in H$, so we need only prove that $x^{-1} \in H$.
  
  Let $|H| = n$.  Lemma~\ref{lemma:translation} tells us
  $|Hx| = |H| = n$.  Since $H$ is finite this, together with the fact that $Hx \subseteq
  H^{2} = H$, implies that $Hx = H$.
  
  Hence there must be some element $y \in H$ such that $yx =
  x$.  By the cancellation law, we have $y = e$, so $x \in H$.
  
  But now there must also be some $z \in H$ such that $zx = 
  e$, so $z = x^{-1}$.
  
  Hence $H$ is a subgroup of $G$.
\end{proof}

Getting back to the example at the start of this section, if $H = 
\langle (1,2,3) \rangle = \{e, (1,2,3), (1,3,2)\}$, then we can write
\[
  \{ (1,2), (1,3), (2,3) \} = (1,2)H.
\]
In fact, we also have
\[
  \{ (1,2), (1,3), (2,3) \} = (1,3)H = (2,3)H = H(1,2) = H(1,3) = H(2,3).
\]

This situation is important enough to give it a name.

\begin{definition}\label{defn:coset}
  Let $G$ be a group, and $H$ a subgroup of $G$.  Sets of the form
  $xH$ are called \defn{left cosets}{coset!left} of $H$ and sets of
  the form $Hx$ are called \defn{right cosets}{coset!right} of $H$,
  where $x$ is any element of $G$.
  
  If $G$ is Abelian, then $Hx = xH$, and we simply call the set a 
  \defn{coset}{coset} of $H$.
\end{definition}

So $\{ (1,2), (1,3), (2,3) \}$ is both a left and right coset of
$\{e, (1,2,3), (1,3,2)\}$.

\begin{example}
  Consider $C_{6} = \{ 1, a, a^{2}, a^{3}, a^{4}, a^{5}\}$.  We know
  that $H = \langle a^{3} \rangle$ is a subgroup, and its cosets are
  itself $H = a^{3}H$, $aH = a^{4}H = \{a, a^{4}\}$, and $a^{2}H =
  a^{5}H = \{a^{2}, a^{5}\}$.
  
  Because cyclic groups are Abelian, the left and right cosets are 
  the same.
\end{example}

Notice in the examples so far that several different choices for $x$ 
give the same coset.  This is typical.

\begin{theorem}\label{thm:cosetsequal}
  Let $G$ be a group, $H$ a subgroup of $G$, and $x$ and $y \in H$.  Then 
  $xH = yH$ if and only if $x^{-1}y \in H$.  On the other hand, $Hx = 
  Hy$ if and only if $xy^{-1} \in H$.
  
  Furthermore, either $xH = yH$ or $xH \intersect yH = \emptyset$.  
  Similarly, either $Hx = Hy$ or $Hx \intersect Hy = \emptyset$.
\end{theorem}
\begin{proof}
  A typical element of $yH$ is of the form $yz$, for some $z \in H$. 
  We note that $x^{-1}yz \in H$, as well, and so multiplying on the
  left by $x$ we have $x(x^{-1}yz) = yz \in xH$.  So $yH \subseteq
  xH$.  Similarly, since $(x^{-1}y)^{-1} = y^{-1}x \in H$, given a
  typical element $xz \in xH$ we have $y^{-1}xz \in H$, and so 
  $y(y^{-1}xz) = yz \in yH$.  Hence $xH \subseteq yH$, and we 
  conclude that $xH = yH$.
  
  Conversely, if $x^{-1}y \notin H$, we have that $y \notin xH$, since 
  if that were the case $y = xz$ for some $z \in H$, but $y = 
  xx^{-1}y$, and the cancellation law implies that then $z = 
  x^{-1}y$, so $z \notin H$, which is a contradiction.  However, $y = 
  ye \in yH$, so $xH$ and $yH$ are not equal.
  
  Assume that $xH \ne yH$, so that $xy^{-1} \notin H$.  If there were 
  some $z \in xH \intersect yH$, then we would have $z = xu = yv$ for 
  some $u$ and $v \in H$.  So $uv^{-1} \in H$.  But
  \[
    uv^{-1} = x^{-1}xuv^{-1} = x^{-1}yvv^{-1} = x^{-1}y \notin H.
  \]
  This is a contradiction, so there can be no such element $z$.
  
  Analagous arguments show that $Hx = Hy$ if and only if $xy^{-1}
  \in H$, and either $Hx = Hy$ or $Hx \intersect Hy = \emptyset$.
\end{proof}

Every element of the group must lie in some coset, since $x = xe \in 
xH$, so the cosets of a subgroup $H$ break the group $G$ up into a 
collection of disjoint subsets.  Furthermore, we know that each of 
these subsets has the same cardinality.  This is nice to know for 
infinite groups, but it is really useful when dealing with finite 
groups.

\begin{theorem}[Lagrange]\label{thm:lagrange}
  If $G$ is a finite group, and $H \le G$, then the number of left
  cosets of $H$ and the number of right cosets of $H$ both equal
  $|G|/|H|$.
\end{theorem}
\begin{proof}
  Assume that there are $n$ left cosets, and that $g_{1}H$, $g_{2}H,
  \ldots, g_{n}H$ is a complete list of the distinct left cosets of
  $H$, so $G$ is a disjoint union of these sets.  The
  inclusion-exclusion principle tells us that when we have a disjoint
  union of finite sets, the cardinality of the union is the sum of the
  cardinality of each set, ie.
  \[
    |G| = |g_{1}H| + |g_{2}H| + \cdots + |g_{n}H|.
  \]
  But Lemma~\ref{lemma:translation} tells is that $|g_{1}H| = 
  |g_{2}H| = \cdots = |g_{n}H| = |H|$, so
  \[
    |G| = \underbrace{|H| + |H| + \cdots + |H|}_{\text{$n$ times}} = 
    n|H|.
  \]
  Hence the number of left cosets is $n = |G|/|H|$.
  
  The argument for right cosets is analagous.
\end{proof}

The number of cosets of a subgroup is significant enough to be given 
its own name and notation.

\begin{definition}
  If $H$ is a subgroup of a group $G$, the \defn{index}{index!of a 
  subgroup} $[G : H]$ is the number of left (or right) cosets of $H$.
\end{definition}

The index is sometimes denoted $|G : H|$.  The following corollaries
of Lagrange's theorem are almost trivial, but they are important
enough that we state them explicitly.  We will use these some of these
facts as much, if not more often, than Lagrange's theorem itself.

\begin{corollary}
  If $H$ is a subgroup of a finite group $G$, $[G : H] = |G|/|H|$.
\end{corollary}

\begin{corollary}
  If $H$ is a subgroup of a finite group $G$, then both $[G : H]$ and 
  $|H|$ divide $|G|$.
\end{corollary}
\begin{proof}
  The numbers $[G : H] = |G|/|H|$, and $|H| = |G|/[G:H]$ are both
  natural numbers, so $|H| \divides |G|$ and $[G : H] \divides |G|$.
\end{proof}

\begin{corollary}
  If $G$ is a finite group, and $x \in G$, then the order of the 
  $x$, $o(x)$, divides $|G|$.
\end{corollary}
\begin{proof}
  We know that $o(x) = |\langle x \rangle|$, and we also know that 
  $\langle x \rangle$ is a subgroup.  Hence $|\langle x \rangle|$ 
  divides $|G|$, and so $o(x)$ divides $G$.
\end{proof}

This last corollary has some immediate consequences, both for 
properties of particular elements, and for classifying groups.

\begin{corollary}
  If $G$ is a finite group, and $n = |G|$, then given any $x \in G$,
  $x^{n} = e$.
\end{corollary}
\begin{proof}
  Assume that $o(x) = k$, so that in particular $x^{k} = e$.  By the
  previous corollary, $o(x)$ divides $n$, so we have $n = km$ for 
  some integer $m$.  But then
  \[
    x^{n} = x^{km} = (x^{k})^{m} = e^{m} = e.
  \]
\end{proof}

\subsection*{Exercises}

\begin{exercises}
  \item Identify the cosets of the subgroup $H = \{1, a^{2}\}$ of the group
    $D_{8}$.  What is the index of $H$ in $G$?
  
  \item Verify Lagrange's Theorem for $D_{8}$.
  
  \item Show that if $X \subseteq G$ is not a subgroup, then an element may
    lie in more than one set of the form $aX$.  Show that every element lies
    in at least one such set.
  
  \item Let $C_{n}$ be a cyclic group, and $m$ be a number that divides $n$.
    Show that there is an element of order $m$ in $C_{n}$.
  
  \item Show that if $G$ is a finite group, then the function
    $\alpha: \Sub(G) \to \naturals$ given by $\alpha(G) = |G|$ is an
    order-preserving function from the partial ordered set $(\Sub(G),
    \le)$ to the partially ordered set $(\naturals, \mid )$.
\end{exercises}

\section{Classifying Groups of Small Order}

We will use the corollaries of Lagrange's theorem to show that groups 
of low order fall into a few distinct isomorphism classes.  The aim of 
this section is to show that every group of order less than or equal 
to $8$ is isomorphic to one of a few standard groups.

\begin{corollary}
  Let $G$ be a group with $|G| = p$, a prime number.  Then $G$ has no 
  proper subgroups, and $G$ is cyclic.
\end{corollary}
\begin{proof}
  If $H$ is a subgroup of $G$, then $|H|$ divides $|G| = p$, so $|H|$ 
  must either be $1$ or $p$.  If $|H| = 1$, then $H = \{e\}$, since 
  every subgroup must contain the identity element.  On the other 
  hand, if $|H| = p$, then $H = G$, since $H$ must contain every 
  element of the group.  So $G$ has no proper subgroups.
  
  Now assume that $x \in G$, and $x \ne e$.  Then $H = \langle x
  \rangle$ is a subgroup of $G$, and $H$ contains at least two
  elements $x$ and $e$.  Hence $|H| \ne 1$, and so $|H| = p$, which 
  implies $H = G$.  So $G$ is generated by $x$, and hence is a cyclic 
  group.
\end{proof}

This last corollary means that every group of prime order is 
isomorphic to $C_{p}$.

We can also show that we have found all groups of order $4$.

\begin{proposition}
  Let $G$ be a group such that $|G| = 4$.  Then either $G \isom 
  C_{4}$ or $G \isom V \isom C_{2} \cross C_{2}$.
\end{proposition}
\begin{proof}
  We know that the order of every element of $G$ divides $4$, so the 
  order of an element which is not the idientity must be $2$ or $4$.
  
  If there is an element $x$ with $o(x) = 4$, then $x$ must generate 
  $G$.  Hence $G$ is cyclic, so $G \isom C_{4}$.
  
  If there is no element of order $4$, then we must have $x^{2} = e$ 
  for every $x \in G$.  But then Theorem~\ref{thm:order2group} tells 
  us that $G \isom C_{2} \cross C_{2}$.
\end{proof}

This table summarizes the typical groups of each order that we have 
discovered, up to order 12.

\medskip

  \begin{tabular}{|r|l|}
  \hline
  \textbf{Order} & \textbf{Known Groups} \\
  \hline
    1 & $C_{1}$ \\
    2 & $C_{2}$ \\
    3 & $C_{3}$ \\
    4 & $C_{4}$, $C_{2} \cross C_{2}$ \\
    5 & $C_{5}$ \\
    6 & $C_{6}$, $D_{6}$, ? \\
    7 & $C_{7}$ \\
    8 & $C_{8}$, $C_{2} \cross C_{2} \cross C_{2}$, $C_{4} \cross C_{2}$, 
    $D_{8}$, ? \\
    9 & $C_{9}$, $C_{3} \cross C_{3}$, ? \\
    10 & $C_{10}$, $D_{10}$, ? \\
    11 & $C_{11}$ \\
    12 & $C_{12}$, $C_{2} \cross C_{6}$, $D_{12}$, $A_{3}$, ? \\
  \hline
  \end{tabular}

\medskip

There may be other groups of order $6$, $8$, $10$ and $12$ that we do
not yet know of, indicated by the question marks in the table.

The following general result tells us that there are no more groups of 
order 6 and 10 than the ones listed on the table.

\begin{proposition}
  Let $G$ be a group with $|G| = 2p$, where $p$ is a prime number
  greater than $2$.  Then either $G$ is cyclic, or $G \isom D_{2p}$.
\end{proposition}
\begin{proof}
  The factors of $2p$ are $1$, $2$, $p$ and $2p$, so the order of each
  element must be one of those factors.  If $G$ has an element of
  order $2p$ it is cyclic.
  
  Assume that $G$ does not have an element of order $2p$.  If $G$ does
  not have an element of order $p$, then every non-identity element of
  $G$ must have order $2$, which means that $x^{2} = e$ for every
  element of $G$.  That imples that $G$ is a product of copies of 
  $C_{2}$ by Theorem~\ref{thm:order2group}, and in particular that 
  $|G| = 2^{n}$, which is impossible if $p > 2$.
  
  Hence there is some element $a \in G$ with $o(a) = p$.  Choose any
  element $b \notin \langle a \rangle$.  Then $G$ breaks into the two
  right cosets
  \begin{align*}
    \langle a \rangle &= \{1, a, a^{2}, \ldots, a^{p-1}\}, \\
    \langle a \rangle b &= \{b, ab, a^{2}b, \ldots, a^{p-1}b\}.
  \end{align*}
  Therefore $b^{2}$ and $(ba)^{2}$ must lie in one of these two cosets.
  
  If $b^{2} \in \langle a \rangle b$, then $b^{2} = a^{k}b$ for some 
  $k$, and the cancellation law tells us that $b = a^{k}$, which is a 
  contradiction.  Hence $b^{2} \in \langle a \rangle$, so $b^{2} = 
  a^{k}$ for some $k$.  If $k \ne 0$, then $b^{2} \ne e$, so $o(b) > 2$ 
  and hence $b$ must have order $p$.  But then since $p$ is odd, $p-1$ 
  is even, and
  \[
    b^{p} = b^{p-1}b = (b^{2})^{(p-1)/2}b = (a^{k})^{(p-1)/2}b = 
    a^{k(p-1)/2}b \ne e.
  \]
  So $b$ cannot have order $p$.  Hence $b^{2} = e$.
  
  The same argument with $ba$ in the place of $b$ shows that $(ba)^{2} = 
  e$.  Hence,
  \begin{align*}
    baba &= 1 \\
    bab &= a^{-1} = a^{p-1} \\
      ba &= a^{p-1}b^{-1} = a^{p-1}b.\\
  \end{align*}
  So $G = \{ 1, a, a^{2}, \ldots, a^{p-1}, b, ab, a^{2}b, \ldots, a^{p-1}b\}$
  and the relations
  \[
    a^{p} = 1, b = 1, \text{and} ba = a^{p-1}b,
  \]
  hold, which is precisely the definition of $D_{2p}$.  Hence $G \isom D_{2p}$.
\end{proof}

There may still be groups of order 8, 9 and 12 which we do not know
about.  We need to introduce a new concept to fully analyse these
situations, so we defer them to a later section.

\subsection*{Exercises}

\begin{exercises}
  \item Show that every Abelian group of order 8 is isomorphic to one of
  $C_{8}$, $C_{4} \cross C_{2}$, or $C_{2} \cross C_{2} \cross C_{2}$.
  
  \item\label{ex:quaterniongroup} For any natural number $n$, define the \defn{quaternion
  group}{group!quaternion} to be the group
  \[
    Q_{4n} = \{1, a, a^{2}, \ldots, a^{2n-1}, b, ab, a^{2}b, \ldots, a^{2n-1}b \}
  \]
  where the Cayley table is determined by the relations $a^{2n} = 1$, $b^{2} =
  a^{n}$, and $b^{-1}ab = a^{-1}$.
  
  Show that $Q_{4} \isom C_{2} \cross C_{2}$.
  
  Show that $Q_{8}$ is not isomorphic to any one of $C_{8}$, $C_{2} \cross C_{2} \cross
  C_{2}$, $C_{2} \cross C_{4}$, or $D_{8}$.  In other words, $Q_{8}$ is a new
  group of order $8$.
  
  Find the subgroup lattice for $Q_{8}$.
\end{exercises}

\section{Excursion: Equivalence Relations}

Underlying the concept of cosets is the notion of an equivalence
relation.  Equivalence relations are a powerful mathematical concept
which occur with regularity throughout abstract mathematics. 
Moreover, you are familiar with some of them already, even though you
may have not seen the concept formally explained.

\begin{definition}
  An \defn{equivalence relation}{relation!equivalence} on a set $X$ is a
  relation $R$ on $X$ which is reflexive, transitive and symmetric. 
  Equivalence relations are often denoted by the symbol $\sim$.
\end{definition}

\begin{example}
  The equality relation is an equivalence relation.  Indeed, it is the
  prototypical equivalence relation.
\end{example}

\begin{example}\label{eg:matrixconjugate}
  Let $M_{n}(\reals)$ be the set of $n$ by $n$ real-valued matrices.  Recall
  that $A$ and $B$ are equivalent matrices if there is an orthogonal matrix
  $U$ such that $A = U^{-1}BU$.  The relation $\sim$ defined by $A \sim B$ if
  and only if $A$ and $B$ are equivalence matrices is an equivalence relation.
\end{example}

\begin{example}\label{eg:modequivclass}
  Let $m$ be any natural number.  The relation $\equiv$ on $\integers$
  defined by $x \equiv y$ if and only if $m$ divides $x-y$ is an equivalence
  relation.  Another way of looking at it, is that it holds if and only if
  \[
    x \equiv y \pmod{m}.
  \]
  A third, useful, way of looking at this situation is that the set of
  all integers divisible by $m$ is the subset $H = m\integers =
  \{ mn : n \in \integers\}$ of $\integers$, and this is a subgroup, so
  $x \equiv y$ if and only if $x - y \in H$, and $x-y$ is the additive notation
  for $xy^{-1}$.  So this is the same condition we were using to test whether
  two elements were in the same coset.
\end{example}

We now get to the key example which links this section with what we have
just discussed by generalizing the above example.

\begin{example}
  Let $G$ be a group, $H$ a subgroup of $G$, and $x$, $y \in G$.  We define
  $x \equiv_{R} y \pmod{H}$ if $xy^{-1} \in H$.
  
  To prove this, you need to check that each of the three axioms for an 
  equivalence relation hold.
  
  Firstly, $xx^{-1} = e \in H$, so $x \equiv_{R} x \pmod{H}$, and $\equiv_{R} \pmod{H}$ is reflexive.
  
  Secondly, if $x \equiv_{R} y \pmod{H}$, then $xy^{-1} \in H$, then
  $yx^{-1} = (xy^{-1})^{-1} \in H$, so $y \equiv_{R} x \pmod{H}$, and $\equiv_{R} \pmod{H}$ is symmetric.
  
  Finally, if $x \equiv_{R} y \pmod{H}$ and $y \equiv_{R} z \pmod{H}$, then $xy^{-1}$ and $yz^{-1} \in H$,
  so $xz^{-1} = xy^{-1}yz^{-1} \in H$, so $x \equiv_{R} z \pmod{H}$, and $\equiv_{R} \pmod{H}$ is transitive.
  
  There is of course the equivalent ``left'' relation $x \equiv_{L} y \pmod{H}$ if $x^{-1}y \in H$,
  and this too is an equivalence relation.
\end{example}

Notice how the three group axioms correspond to the three equivalence relation
axioms in the previous example.  Notice also that with these equivalence
relations, the cosets are precisely the sets of elements which are equivalent
to one-another.

\begin{definition}
  Let $\sim$ be an equivalence relation on $X$.  The \defn{equivalence class}{equivalence class}
  of $x \in X$ is the set
  \[
    [x]_{\sim} = \{ y \in X : x \sim y \}.
  \]
  When the equivalence relation is clear, we typically write just $[x]$.
\end{definition}

\begin{example}
  For equality, the equivalence class $[x]_{=} = \{x\}$.
\end{example}

\begin{example}\label{eg:equivmodm}
  For $\equiv \pmod{m}$, the equivalence class of a number $n$,
  $[n]_{\equiv}$ is the set of all numbers with the same remainder when divided
  by $m$, or equivalently,
  \[
    [n] = \{\ldots, n-2m, n-m, n, n+m, n+2m, \ldots\}
  \]
\end{example}

\begin{example}
  The equivalence class of $x$ under the equivalence relation $\equiv_{R} \pmod{H}$
  is the right coset $Hx$.

  The equivalence class of $x$ under the equivalence relation $\equiv_{L} \pmod{H}$
  is the left coset $xH$.
\end{example}

A key fact about equivalence classes is that they partition the set $X$ into a
disjoint collection of subsets whose union is the whole set.

\begin{lemma}
  Let $\sim$ be an equivalence relation on $X$.  We have $x \sim y$ if and
  only if $[x] = [y]$.
  
  Also, either $[x] = [y]$ or $[x] \intersect [y] = \emptyset$.
\end{lemma}
\begin{proof}
  Assume that $x \sim y$, then given any $z \in [y]$, we have $y \sim z$,
  and since $\sim$ is transitive, $x \sim z$, so $z \in [x]$.  Hence $[x]
  \subseteq [y]$.  On the other hand, we know that by symmetry $x \sim y$
  implies $y \sim x$. If $z \in [x]$, we have $x \sim z$ and since $\sim$ is transitive, $x \sim z$, so $z \in [x]$.  Hence $[x]
  \subseteq [y]$.  Hence $[x] = [y]$.
  
  If $[x] = [y]$, then $y \in [y]$, since $y \sim y$ by reflexivity, and
  so $y \in [x]$.  Hence $x \sim y$ by definition.

  If $[x] \intersect [y] \ne \emptyset$, then there must be some $z$ in both
  sets, ie.~$x \sim z$ and $y \sim z$.  Now since $\sim$ is symmetric, $y \sim z$
  implies $z \sim y$, and since $\sim$ is transitive, $x \sim y$.  The first
  part of this lemma allows us to conclude that $[x] = [y]$.
\end{proof}

Theorem~\ref{thm:cosetsequal} is now a trivial corollary of this general fact
about equivalence relations.  Similarly, Lagrange's Theorem follows from the
following general fact.

\begin{theorem}\label{thm:countingequivalenceclasses}
  Let $X$ be a finite set, and let $\sim$ be an equivalence relation on $X$.
  Choose elements $x_{1}$, $x_{2}, \ldots, x_{n} \in X$ so that no two are
  equivalent, and so for every $x \in X$, there is some $x_{k}$ such that
  $[x] = [x_{k}]$ (ie.~this is a complete set of equivalence class
  representatives).  Then
  \[
    |X| = \sum_{k=1}^{n} |[x_{k}]|
  \]
\end{theorem}
\begin{proof}
  We have that $X$ is the disjoint union of the equivalence classes of the
  $x_{k}$, ie.
  \[
    X = [x_{1}] \union [x_{2}] \union \cdots \union [x_{n}],
  \]
  which follows from the fact that every $x$ lies in one of the $[x_{k}]$,
  and if $k \ne j$,
  \[
    [x_{j}] \intersect [x_{k}] = \emptyset,
  \]
  since $x_{k} \not\sim x_{j}$.
  
  Repeated application of the inclusion-exclusion principle then tells us
  that
  \[
    |X| = |[x_{1}]| + |[x_{2}]| + \cdots + |[x_{n}]|,
  \]
  since the intersections are empty.
\end{proof}

We will use this general theorem in the next section.

\subsection*{Exercises}

\begin{exercises}
  \item Verify that Example~\ref{eg:modequivclass} is an equivalence 
  relation.
  
  \item Verify that Example~\ref{eg:matrixconjugate} is an equivalence 
  relation.
  
  \item Let $A$ and $B \in M_{n}(\reals)$, and define a relation by $A 
    \sim B$ if $A = X^{-1}BX$ for some invertible matrix $X$.  SHow 
    that this is an equivalence relation.
  
  \item Can a relation be symmetric and transitive, but not reflexive? Give a proof or counterexample to justify your position.
  
  \item (*) Show that if $\sim$ is the equivalence relation of 
  Example~\ref{eg:matrixconjugate}, and
  \[
    A = \begin{bmatrix}
      1 & 0 \\
      0 & 2
    \end{bmatrix}
  \]
  then $[A]_{\sim}$ is the set of all symmetric $2 \times 2$ matrices with 
  eigenvalues $1$ and $2$.
\end{exercises}

\section{Conjugacy Classes}

There is another equivalence relation which is of interest which is a
generalization of the idea of conjugate matrices (see
Example~\ref{eg:matrixconjugate}).

\begin{definition}
  Let $G$ be a group and let $x$ and $y \in G$.  We say that $y$ is a
  conjugate of $x$ if $y = u^{-1}xu$ for some $u \in G$.  In this case, we
  write $x \sim y$.
\end{definition}

\begin{example}
  In the group $D_{6} = \{1, a, a^{2}, b, ab, a^{2}b\}$ we have that $a \sim
  a^{2}$, since $b^{-1}ab = bab = a^{2}bb = a^{2}$.  Similarly, we have that
  $b \sim ab$ since $a^{-1}ba = a^{2}ba = a^{2}a^{2}b = ab$.
\end{example}

\begin{proposition}
  If $G$ is a group, the conjugacy relation $\sim$ is an equivalence
  relation on $G$.
\end{proposition}
\begin{proof}
  We need to show that $\sim$ is reflexive, symmetric and transitive.
  
  Since $x = e^{-1}xe$, we have that $x \sim x$, so $\sim$ is reflexive.
  
  If $x \sim y$, then there is an element $u$ such that $y = u^{-1}xu$.  But
  then $x = uu^{-1}xuu^{-1} = uyu^{-1} = (u^{-1})^{-1}yu^{-1}$.  So $y \sim
  x$, and $\sim$ is symmetric.
  
  If $x \sim y$ and $y \sim z$, then $y = u^{-1}xu$ and $z = v^{-1}yv$ for
  some $u$ and $v$ in $G$, so $z = v^{-1}u^{-1}xuv = (uv)^{-1}x(uv)$. Hence
  $x \sim z$, and $\sim$ is transitive.
\end{proof}

We will call the equivalence class of $x \in G$ the \defn{conjugacy
class}{conjugacy class} of $x$.  We will denote the conjugacy class 
of an element $x \in G$ by $C(x)$.  In other words,
\[
  C(x) = \{ y \in G: y = u^{-1}xu~\text{for some $u \in G$} \}.
\]

\begin{example}
  The conjugacy classes of the four-group are $\{e\}$, $\{a\}$, $\{b\}$,
  $\{ab\}$.  In other words, each element is in its own conjugacy class.
\end{example}

\begin{example}
  The conjugacy classes of the group $D_{6} = \{1, a, a^{2}, b, ab, a^{2}b\}$
  are $\{1\}$, $\{a, a^{2}\}$, $\{b, ab, a^{2}b\}$.
\end{example}

The reason that the four-group has a conjugacy class for each element is
because it is an Abelian group.  In fact if $x$ commutes with everything in
$G$, then the only element of the conjugacy class of $x$ is itself.

\begin{proposition}\label{prop:centreconjugates}
  Let $G$ be a group and $x \in G$.  Then $C(x) = \{x\}$ if and only 
  if $x \in Z(G)$.
\end{proposition}
\begin{proof}
  Assume $x \in Z(G)$.  Let $y \in C(x)$, so that there is some $u \in
  G$ such that $y = u^{-1}xu$.  But then, since $xu = ux$,
  \[
    y = u^{-1}xu = u^{-1}ux = x.
  \]
  So $C(x) = \{x\}$.
  
  On the other hand, if $C(x) = \{x\}$, then for every $u \in G$, we 
  have $u^{-1}xu = x$, which means $x$ and $u$ commute.  Hence $x \in 
  Z(G)$.
\end{proof}

Notice in this case that we are saying that $u^{-1}xu = v^{-1}xv$ for
every $u$ and $v \in G$.  This is not going to be the case in general,
but it is of interest to know for which $u$ and $v$ it occurs.  For
example, if these quantities were never equal for different $u$ and
$v$, it would show that the conjugacy class of $x$ has many different 
elements.

\begin{proposition}
  Let $G$ be a group and $x \in G$.  Then $u^{-1}xu = v^{-1}xv$ if 
  and only if $uv^{-1} \in Z_{G}(x)$.
\end{proposition}
\begin{proof}
  If $u^{-1}xu = v^{-1}xv$, then we have
  \begin{align*}
    u^{-1}xu &= v^{-1}xv \\
    xu &= uv^{-1}xv \\
    xuv^{-1} &= uv^{-1}x.
  \end{align*}
  So we can see that $uv^{-1}$ commutes with $x$, and so $uv^{-1} \in 
  Z_{G}(x)$.
  
  On the other hand, if $uv^{-1} \in Z_{G}(x)$, we can run the 
  calculation backwards:
  \begin{align*}
    xuv^{-1} &= uv^{-1}x \\
    xu &= uv^{-1}xv \\
    u^{-1}xu &= v^{-1}xv,
  \end{align*}
  and so we have the result.
\end{proof}

Recalling Theorem~\ref{thm:cosetsequal}, this proposition cas the 
following corollary:

\begin{corollary}
  Let $G$ be a group and $x \in G$.  Then $u^{-1}xu = v^{-1}xv$ if 
  and only if $u$ and $v$ are in the same right coset of $Z_{G}(x)$, 
  ie.~if $Z_{G}(x)u = Z_{G}(x)v$.
\end{corollary}

Turning this around, it means that $u^{-1}xu \ne v^{-1}xv$ if and only
if $u$ and $v$ are in different right cosets of $Z_{G}(x)$, which
means that there is a distinct conjugate of $x$ for each distinct
right coset of $Z_{G}(x)$.  If $G$ is finite, then Lagrange's Theorem
tells us that there are exactly $|G : Z_{G}(x)| = |G|/|Z_{G}(x)|$ 
distinct right cosets of $Z_{G}(x)$.  This proves the following:

\begin{corollary}\label{cor:conjugacyclasssize}
  Let $G$ be a finite group and $x \in G$.  Then
  \[
    |C(x)| = |G : Z_{G}(x)|.
  \]
\end{corollary}

Note that this agrees with Proposition~\ref{prop:centreconjugates}, 
since $x \in Z(G)$ if and only if $Z_{G}(x) = G$, but this happens if 
and only if $|C(x)| = |G : Z_{G}(x)| = |G|/|G| = 1$.  In other words, 
if and only if the only conjugate of $x$ is itself.

\begin{example}
  For the group $D_{6}$ we know that $a$ has conjugacy class $\{a, 
  a^{2}\}$, and we know that $Z_{D_{6}}(a) = \{1, a, a^{2}\}$.  It is 
  simple to verify that
  \[
    |G|/|Z_{D_{6}}(a)| = 6/3 = 2 = |C(a)|.
  \]
  If we look at the other elements of $D_{g}$, we observe that
  \begin{alignat*}{2}
    |C(e)| &= 2 &\qquad |G|/|Z_{D_{6}}(e)| &= 6/6 = 1 \\
    |C(a^{2})| &= 2 &\qquad |G|/|Z_{D_{6}}(a^{2})| &= 6/3 = 2 \\
    |C(b)| &= 3 &\qquad |G|/|Z_{D_{6}}(b)| &= 6/2 = 3 \\
    |C(ab)| &= 3 &\qquad |G|/|Z_{D_{6}}(ab)| &= 6/2 = 3 \\
    |C(a^{2}b)| &= 3 &\qquad |G|/|Z_{D_{6}}(a^{2}b)| &= 6/2 = 3.
  \end{alignat*}
\end{example}

We can now apply Theorem~\ref{thm:countingequivalenceclasses} to the 
conjugacy equivalence relation.

\begin{theorem}[The Class Equation]
  Let $G$ be a finite group, and let $g_{1}$, $g_{2}$, $g_{n} \in G$ 
  be chosen so that no two are conjugate, and every conjugacy class 
  of $G$ occurs as one of the conjugacy classes $C(g_{k})$ (ie.~the 
  elements $g_{k}$ are a complete set of conjugacy class 
  representatives).  Then
  \[
    |G| = \sum_{k = 1}^{n} |C(g_{k})| = \sum_{k = 1}^{n} |G : 
    Z_{G}(g_{k})|.
  \]
  In fact, if we assume that $Z(G) = \{g_{1}, g_{2}, \ldots, 
  g_{m}\}$, then
  \[
    |G| = |Z(G)| + \sum_{k = m+1}^{n} |G : Z_{G}(g_{k})|.
  \]
\end{theorem}
\begin{proof}
  Since conjugacy is an equivalence relation, Theorem~\ref{thm:countingequivalenceclasses} 
  applies and tells us that
  \[
    |G| = |C(g_{1})| + |C(g_{2})| + \cdots + |C(g_{n})| = \sum_{k = 1}^{n} |C(g_{k})|.
  \]
  But then Corollary~\ref{cor:conjugacyclasssize} tells us that 
  $|C(g_{k})| = |G : Z_{G}(g_{k})|$, and so we can re-write the equation as
  \[
    |G| = \sum_{k = 1}^{n} |G : Z_{G}(g_{k})|.
  \]
  
  Now we know that if $g_{k} \in Z(G)$, then $|C(g_{k})| = 1$, so if
  $Z(G) = \{g_{1}, g_{2}, \ldots, g_{m}\}$, then
  \begin{align*}
    |G| &= |C(g_{1})| + |C(g_{2})| + \cdots + |C(g_{m})| + |C(g_{m + 1})| + 
    |C(g_{m + 2})| + \cdots + |C(g_{n})| \\
        &= \underbrace{1 + 1 + \cdots + 1}_{m~\text{times}} + |C(g_{m + 1})| + 
    |C(g_{m + 2})| + \cdots + |C(g_{n})| \\
        &= |Z(G)| + \sum_{k = m+1}^{n} |C(g_{k})| \\
        &= |Z(G)| + \sum_{k = m+1}^{n} |G : Z_{G}(g_{k})|.
  \end{align*}
\end{proof}

\begin{example}
  For the group $D_{6}$ we have that $1$, $a$, and $b$ are 
  representatives of each conjugacy class, and
  \[
    |C(1)| + |C(a)| + |C(b)| = 1 + 2 + 3 = 6 = |D_{6}|.
  \]
\end{example}

\begin{example}
  For the group $D_{8}$, we know from Example~\ref{eg:D8centre} that
  the centre $Z(D_{8}) = \{1, a^{2}\}$, so $C(1) = \{1\}$ and 
  $C(a^{2}) = \{a^{2}\}$.
  
  We know that $Z_{D_{8}}(a)$ contains $\langle a \rangle$ so 
  \[
    |Z_{D_{8}}(a)| \ge  |\langle a \rangle| = o(a) = 4,
  \]
  and from Lagrange's theorem $|Z_{D_{8}}(a)|$ divides $8$, so we 
  conclude that $|Z_{D_{8}}(a)|$ is $4$ or $8$.  But $a \not\in Z(D_{8})$, 
  so $Z_{D_{8}}(a) \ne D_{8}$, and hence $|Z_{D_{8}}(a)| = 4$.  Hence 
  $a$ has two conjugates, itself, and $b^{-1}ab = bab = a^{3}b^{2} = 
  a^{3}$.  So $C(a) = \{a, a^{3}\}$.
  
  Now $Z_{D_{8}}(b)$ contains $\langle b \rangle$, so $|Z_{D_{8}}(b)|
  \ge o(b) = 2$, and $|Z_{D_{8}}(b)|$ divides $8$.  Also $b$ is not in
  the centre of $D_{8}$, so $|Z_{D_{8}}(b)| \ne 8$, and hence
  $|Z_{D_{8}}(b)|$ is $2$ or $4$.  Now we know that the centre is a 
  subgroup of $Z_{D_{8}}(b)$, so $1$ and $a^{2}$ are in the subgroup, 
  as is $b$ itself.  Hence $|Z_{D_{8}}(b)| > 2$, and we conclude 
  that $|Z_{D_{8}}(b)| = 4$.  Hence $b$ has two conjugates, itself 
  and $a^{-1}ba = a^{-1}a^{3}b = a^{2}b$.  So $C(b) = \{b, a^{2}b\}$.
  
  Looking at $Z_{D_{8}}(ab)$, an analagous argument tells us that 
  $|Z_{D_{8}}(ab)| = 4$, and the conjucacy class is $C(ab) = \{ab, 
  a^{3}b\}$.
  
  We can verify that the class equation holds in this example: a 
  complete collection of equivalence class representatives is $1$, 
  $a^{2}$, $a$, $b$ and $ab$, and
  \[
    |Z(D_{8})| + |C(a)| + |C(b)| + |C(ab)| = 2 + 2 + 2 + 2 = 8 = 
    |D_{8}|.
  \]
\end{example}

We conclude with one last proposition which can help identify which 
elements are in different conjugacy classes.

\begin{proposition}
  Let $G$ be a finite group.  If $x$ and $y \in G$ are conjugate, 
  then $o(x) = o(y)$.
\end{proposition}
\begin{proof}
  Let $x = u^{-1}yu$.  Then if $o(y) = n$,
  \begin{align*}
    x^{n} &= (u^{-1}yu)^{n} \\
          &= \underbrace{(u^{-1}yu)(u^{-1}yu)\cdots 
          (u^{-1}yu)}{n~\text{times}} \\
          &= u^{-1}\underbrace{yy\ldots y}{n~\text{times}}u \\
&= u^{-1}y^{n}u \\
&= u^{-1}u \\
&= e.
  \end{align*}
  So $o(x)$ divides $n$.  However, if $o(x) = m$, then a similar 
  calculation shows that
  \begin{align*}
    e &= x^{m} \\
      &= u^{-1}y^{m}u,
  \end{align*}
  so $y^{m} = uu^{-1}y^{m}uu^{-1} = ueu^{-1} = e$.  Hence $n$ 
  divides $m$, and so $n = m$.
\end{proof}

Note that the converse is not true, since in the group $D_{8}$, the 
elements $a^{2}$, $b$ and $ab$ all have order $2$, but all are in 
distinct conjugacy classes.

Notice that every term in the class equation divides the order of 
the group.  We can use this to prove the following interesting result 
that will be useful when we look once again at groups of small order.

\begin{theorem}\label{thm:centreprimepower}
  Let $G$ be a finite group of prime power order, ie.  $|G| = p^{n}$
  where $p$ is prime and $n \ge 1$.  Then $|Z(G)| = p^{m}$ for some $m
  \ge 1$.
\end{theorem}
\begin{proof}
  Lagrange's theorem tells us that $|Z(G)| = p^{m}$ for some 
  $m \ge 0$.
  
  Let $x_{1}$, $x_{2}, \ldots, x_{k}$ be a complete set of conjugacy
  class representatives, and let $Z(G) = \{x_{1}, \ldots, x_{l}\}$. 
  Now let $n_{i} = |C(x_{i})| = |G : Z_{G}(x_{i})|$, so $n_{i} \mid
  p^{n}$.  Now for $i > l$, we must have $n_{i} > 1$, so $n_{i}$ must
  be a multiple of $p$.  Therefore,
  \[
    p^{n} = |G| = |Z(G)| + n_{l+1} + n_{l+2} + \cdots n_{k} = l + jp
  \]
  for some integer $j$.  But therefore $l$ is a multiple of $p$, and
  since $e \in Z(G)$, $|Z(G)| \ge 1$.  So we conclude that $|Z(G)| 
  \ge p$, and hence $m \ge 1$.
\end{proof}

In other words, groups of prime power must have more than just the 
identity in their centres.

\subsection*{Exercises}

\begin{exercises}
  \item In a group of order 15, what does the Class equation say are
  possible sizes of the conjugacy classes?  Potentially how many different
  ways can a group of order 15 be divided into conjugacy classes of these
  sizes (remembering that the number of conjugacy classes of size 1 has to
  equal $|Z(G)|$, which has to divide 15).
  
  Note: in actual fact, there turns out to be only one group of order 15,
  so there is just one way to do it once this is taken into account; however
  you should use the class equation to give you all the potentially possible
  ways that it could be done.
  
  \item Find the conjugacy classes of $D_{10}$, and verify that the 
  class equation holds.

  \item Find the conjugacy classes of $D_{12}$, and verify that the 
  class equation holds.

  \item Find the conjugacy classes of $A_{4}$, and verify that the 
  class equation holds.
  
  \item Let $G$ be a finite group, and $x \in G$.  Show that the
    conjugacy classes $C(x)$ and $C(x^{-1})$ have the same number of
    elements.
    
  \item Show that any group $G$ of even order must contain an
    element of even order, and use the previous exercise to conclude
    that there is at least one element $x \in G$ other than $e$ such
    that $C(x) = C(x^{-1})$.

\end{exercises}


\section{Normal Subgroups}

It turns out that many of the ideas relating to centralizers and conjugacy can
be applied to subgroups instead of individual elements.

\begin{proposition}
  Let $G$ be a group and $H$ a subgroup of $G$.  Given any $x \in G$, the
  set
  \[
    x^{-1}Hx = \{x^{-1}yx : y \in H \}
  \]
  of conjugates of elements of $H$ is a subgroup of $G$.
\end{proposition}
\begin{proof}
  Given any elements $u$ and $v \in x^{-1}Hx$, we have $y$ and $z \in H$
  such that $u = x^{-1}yx$ and $v = x^{-1}zx$.  Then
  \[
    uv^{-1} = x^{-1}yx(x^{-1}zx)^{-1} = x^{-1}yxx^{-1}z^{-1}x =
      x^{-1}yz^{-1}x,
  \]
  which is an element of $x^{-1}Hx$, since $yz^{-1} \in H$.
  
  Hence $x^{-1}Hx$ is a subgroup of $G$.
\end{proof}

If $x \in H$, then $x^{-1}Hx = H$, but if $x \notin H$ we may potentially
get something else.

\begin{example}
  In the group $D_{6}$, consider the subgroup $H = \{1, b\}$.  Since
  $a^{-1}1a = 1$, and $a^{-1}ba = a^{-1}a^{2}b = ab$ , we have
  \[
    a^{-1}Ha = \{1, ab\}.
  \]
  Similarly, we have
  \begin{align*}
    (a^{2})^{-1}Ha^{2} &= \{1, a^{2}b\} \\
    b^{-1}Hb &= \{1, b\} \\
    (ab)^{-1}H(ab) &= \{1, a^{2}b\} \\
    (a^{2}b)^{-1}H(a^{2}b) &= \{1, ab\} \\
  \end{align*}
  
  On the other hand, the subgroup $K = \{1, a, a^{2}\}$ has $x^{-1}Kx = K$
  for any $x$.  For example, since $b^{-1}1b = 1$, $b^{-1}ab = bab =
  a^{2}b^{2} = a^{2}$, and $b^{-1}a^{2}b = ba^{2}b = b^{2}a = a$, we have
  $b^{-1}Kb = K$.  Similar arguments give the remaining cases.
\end{example}

We will say that two subgroups $H$ and $K$ of $G$ are
\defn{conjugate}{conjugate!subgroups} if there is some $x \in G$ such that
\[
  K = x^{-1}Hx,
\]
and we will write $K \sim H$ if this is the case.  It is equivalent to
say that $K$ and $H$ are conjugate if and only if there is some $x \in G$
such that $xK = Hx$.

\begin{proposition}
  Let $G$ be a group.  The conjugacy relation $\sim$ is an equivalence relation
  on the set $\Sub(G)$ of all subgroups of $G$.  Furthermore if two
  subgroups are conjugate, they are isomorphic.
\end{proposition}
\begin{proof}
  We need to show that $\sim$ is reflexive, symmetric and transitive.
  
  Given a subgroup $H$, we have $H \sim H$ immediately from the fact that
  $e^{-1}He = H$.
  
  If $K$ and $H$ are subgroups with $K \sim H$, then there is some $x \in G$
  so that $K = x^{-1}Hx$.  But then $H = xKx^{-1} = (x^{-1})^{-1}Kx^{-1}$, and
  so $H \sim K$.
  
  Finally, if $H$, $K$ and $F$ are subgroups, with $H \sim K$ and $K \sim F$, 
  then there are elements $x$ and $y \in G$ such that $K = x^{-1}Hx$ and $F
  = y^{-1}Ky$.  But then $F = y^{-1}(x^{-1}Hx)y = (xy)^{-1}H(xy)$, and so
  $F \sim H$.
  
  So conjugacy of subgroups is an equivalence relation.
  
  If $K$ and $H$ are conjugate, with $K = x^{-1}Hx$, we define a function
  $\alpha : H \to K$ by $\alpha(y) = x^{-1}yx$.  This function is a
  homomorphism, since
  \[
    \alpha(y)\alpha(z) = x^{-1}yxx^{-1}zx = x^{-1}yzx = \alpha(yz).
  \]
  It is also onto, since $K = \alpha(H)$ by definition.  Finally, it is
  one-to-one since if $\alpha(y) = \alpha(z)$, then $x^{-1}yx = x^{-1}zx$,
  and using the cancellation law on the left and right gives $y = z$.
  
  So $\alpha$ is an isomorphism from $H$ to $K$, and $H$ and $K$ are
  isomorphic.
\end{proof}

\begin{example}
  Continuing the example of $D_{6}$ from above, we have that $\{1\}$ is only
  conjugate with itself, $\{1, a, a^{2}\}$ is only conjugate with itself,
  $\{1, b\} \sim \{1, ab\} \sim \{1, a^{2}b\}$, and $D_{6}$ is only
  conjugate with itself.
\end{example}

We recall that elements of a group whose conjugacy class was just themselves
were special: they formed the centre of the group.  Subgroups which are
conjugate only with themselves are also special.

\begin{definition}
  Let $G$ be a group and $K$ a subgroup of $G$.  If the only subgroup
  conjugate to $K$ is $K$ itself, that is, for any $x \in G$
  \[
    x^{-1}Kx = K,
  \]
  then we say that $K$ is a \defn{normal subgroup}{subgroup!normal}, and we
  write $K \lhd G$.
\end{definition}

Another way of representing the condition that $K$ is normal is that
\[
  Kx = xK
\]
for every $x \in G$, or in other words that the corresponding left- and right-
cosets of $K$ are identical.

\begin{example}
  In the group $D_{6}$ we have that the subgroups $\{1\}$, $\{1, a, a^{2}\}$
  and $D_{6}$ are normal.  The subgroups $\{1, b\}$, $\{1, ab\}$ and $\{1, 
  a^{2}b\}$ are not normal.
\end{example}

\begin{lemma}
  If $G$ is a group, then $\{e\}$ and $G$ are always normal subgroups of
  $G$.

  If $G$ is Abelian, then every subgroup of $G$ is normal.
\end{lemma}
\begin{proof}
  We know $x^{-1}ex = x^{-1}x = e$ for every $x$, so $x^{-1}\{e\}x = \{e\}$
  for all $x$.  We also know that $x^{-1}Gx = G$ for any $x$, since $x \in
  G$.

  If $G$ is Abelian, we recall that $Kx = xK$ for any subgroup and any $x
  \in G$ (see Definition~\ref{defn:coset}), hence $K$ is always normal.
\end{proof}

Not every element of a group is in the centre, and not every subgroup is
normal.  Similarly, just as the centralizer gives us information about the
conjugacy classes of elements, we have an analagous concept for conjugacy
classes of subgroups.

\begin{definition}
  Let $G$ be a group, and $H$ a subgroup of $G$.  We define the
  \defn{normalizer}{normalizer} of $H$ to be the set
  \[
    N_{G}(H) = \{ x \in G : H = x^{-1}Hx \}
  \]
\end{definition}

With this definition, we can duplicate most of the key results about
centralizers and conjugacy classes.

\begin{theorem}\label{thm:normalizers}
  Let $G$ be a group and $H$ a subgroup of $G$.  Then
  \begin{theoremenum}
    \item $N_{G}(H)$ is a subgroup of $G$,
    \item $H$ and $Z(G)$ are subgroups of $N_{G}(H)$,
    \item $N_{G}(H) = G$ if and only if $H$ is normal,
    \item $x^{-1}Hx = y^{-1}Hy$ if and only if $x$ and $y$ are in the same
      right coset of $N_{G}(H)$,
    \item The number of distinct conjugacy classes of $H$ is $|G :
      N_{G}(H)|$.
  \end{theoremenum}
\end{theorem}
\begin{proof}
  The proofs of these facts are analogous to the proofs of the corresponding
  results for centralizers, and are left as an exercise.
\end{proof}

Normal subgroups play a key role in the theory of groups, and we now turn to
study them in more detail.  We start with some ways of testing whether a
subgroup is normal or not, and discovering normal subgroups of a group.

\begin{theorem}
  Let $G$ be a group and $K$ a subgroup of $G$.  Then the following are
  equivalent:
  \begin{theoremenum}
    \item $K$ is normal,
    \item $x^{-1}Kx = K$ for all $x \in G$,
    \item $Kx = xK$ for all $x \in G$,
    \item $N_{G}(H) = G$,
    \item $x^{-1}yx \in K$ for all $y \in K$ and $x \in G$,
    \item $K$ is a union of some of the conjugacy classes of elements of $G$,
  \end{theoremenum}
\end{theorem}
\begin{proof}
  We have already seen that (i), (ii), (iii) and (iv) are equivalent.
  
  If $K$ is normal then $x^{-1}Kx = K$, so for any $y \in K$ we have that
  $x^{-1}yx \in K$.  Conversely, if $x^{-1}yx \in K$ for all $y \in
  K$ and $x \in G$, then if we fix $x$ we have that
  \[
    x^{-1}Kx = \{x^{-1}yx : y \in K\} \subseteq K.
  \]
  On the other hand, given any $y \in K$, we have that $(x^{-1})^{-1}yx^{-1}
  \in K$, and so
  \[
    x^{-1}((x^{-1})^{-1}yx^{-1})x = x^{-1}xyx^{-1}x = y,
  \]
  and so $K \subseteq x^{-1}Kx$.  Hence $x^{-1}Kx = K$ for every $x \in G$
  and so $K$ is normal.
  
  So we have just shown that (i) and (v) are equivalent.
  
  Another way of stating (v) is that if $y \in K$ then every conjugate of 
  $y$ is in $K$, so that $C(y) \subseteq K$.  Hence $K$ must be the union of
  all the conjugacy classes of its elements, ie.
  \[
    K = \bigcup_{y \in K} C(y)
  \]
  So (v) implies (vi).
  
  Conversely, if $K$ is a union of conjugacy classes, then given any element
  $y \in K$, the conjugacy class $C(y)$ of $y$ must be a subset of $K$, and
  so we have that $x^{-1}yx \in C(y) \subseteq K$.  Therefore $x^{-1}yx \in
  K$, and (vi) implies (v).
\end{proof}

\begin{example}
  In the group $D_{6}$, we have conjugacy classes $\{1\}$, $\{a, a^{2}\}$
  and $\{b, ab, a^{2}b\}$.  We can clearly see that each of the normal
  subgroups are unions of conjugacy classes:
  \begin{align*}
    \{1\} &= \{1\} \\
    \{1, a, a^{2}\} &= \{1\} \union \{a, a^{2}\} \\
    D_{6} &= \{1\} \union \{a, a^{2}\} \union \{b, ab, a^{2}b\}.
  \end{align*}
  
  Notice that not every union of conjugacy classes gives a normal subgroup,
  because some unions of conjugacy classes aren't subgroups.  For example,
  the set $\{1, b, ab, a^{2}b\} = \{1\} \union \{b, ab, a^{2}b\}$ is not a
  subgroup.
\end{example}

\begin{example}
  The group $D_{8}$ has conjugacy classes
  \[
    \{1\}, \{a^{2}\}, \{a, a^{3}\}, \{b, a^{2}b\}, \{ab, a^{3}b\}.
  \]
  The trivial subgroups $\{1\}$ and $D_{8}$ are automatically normal, but in
  addition, we have that the subgroups
  \begin{align*}
    \{1, a^{2}\} &= \{1\} \union \{a^{2}\} \\
    \{1, a, a^{2}, a^{3}\} &= \{1\} \union \{a^{2}\} \union \{a, a^{3}\} \\
    \{1, a^{2}, b, a^{2}b \} &= \{1\} \union \{a^{2}\} \union \{b, a^{2}b\} \\
    \{1, a^{2}, ab, a^{3}b \} &= \{1\} \union \{a^{2}\} \union \{ab, a^{3}b\}
  \end{align*}
  are all normal, as they can be written as unions of conjugacy classes as
  shown.  These are the only possible normal subgroups.
\end{example}

There are a number of conditions which guarantee that a subgroup is normal.

\begin{theorem}\label{thm:normalconditions}
  Let $G$ be a group, and $H$ a subgroup of $G$.  If any
  of the following holds, $H$ is normal:
  \begin{theoremenum}
    \item $H = \{e\}$ or $G$,
    \item $H \subseteq Z(G)$,
    \item $|G : H| = 2$,
    \item $H$ is the only subgroup of order $|H|$ in $G$.
  \end{theoremenum}
\end{theorem}
\begin{proof}
  We have already seen (i) is true.
  
  (ii) Recall that for any element $x \in Z(G)$, $C(x) = \{x\}$, so if $H
  \subseteq Z(G)$, then
  \[
    H = \bigcup_{x \in H} \{x\} = \bigcup_{x \in H} C(x),
  \]
  so $H$ is a union of conjugacy classes, and so $H$ is normal.
  
  (iii) If $|G : H| = 2$, then $H$ has two left cosets, $H$ itself and $xH$,
  where $x \notin H$.  Similarly, it has two right cosets $H$ and $Hx$.  Now
  since every element of $G$ is either in $H$ or $xH$, we have that $xH = G
  \setminus H$.  But we similarly have that $Hx = G \setminus H$.  Hence $xH
  = Hx$, and corresponding left and right cosets of $H$ are equal.  Hence
  $H$ is normal.
  
  (iv) We know that every conjugate of $H$ is a subgroup of $G$, and we must
  have $|x^{-1}Hx| = |H|$.  Hence if $H$ is the only subgroup of order $|H|$,
  we must have $H = x^{-1}Hx$ for all $x \in G$.  So $H$ is normal.
\end{proof}

\begin{example}
  In the dihedral group $D_{2n} = \{1, a, a^{2}, \ldots, a^{n-1}, b, ab,
  \ldots, a^{n-1}n\}$, the subgroup $\langle a \rangle = \{1, a, a^{2}, \ldots,
  a^{n-1}\}$ has order $|\langle a \rangle| = o(a) = n$.  So $|D_{2n} :
  \langle a \rangle| = |D_{2n}|/|\langle a \rangle| = 2n/n = 2$.
  
  So $\langle a \rangle$ is always a normal subgroup of $D_{2n}$.
\end{example}

Normal subgroups also have a nice relationship with the lattice structure of
subgroups.

\begin{proposition}\label{prop:normalproduct}
  Let $G$ be a group, $K$ a normal subgroup of $G$, and $H$ an
  arbitrary subgroup of $G$.  Then $HK$ is a subgroup of $G$, and
  furthermore $H \vee K = HK = KH$.
\end{proposition}
\begin{proof}
  We start by showing that $HK$ is a group.  Given $x$ and $y \in H$ and $u$
  and $v \in K$, we have that $xu$ and $yv$ are typical elements of $HK$.
  Now
  \[
    xu(yv)^{-1} = xuv^{-1}y^{-1} = xy^{-1}(y^{-1})^{-1}uv^{-1}y^{-1},
  \]
  and we know $xy^{-1} \in H$, and $(y^{-1})^{-1}uv^{-1}y^{-1} \in K$, since
  it is a conjugate of $uv^{-1} \in K$.  So $xu(yv)^{-1} \in HK$, and hence
  $HK$ is a subgroup of $G$.
  
  Now we need to show that $\langle H \union K \rangle = HK$.  We do this by
  showing that $HK$ is the smallest subgroup of $G$ containing both $H$ and
  $K$.  First observe that since $e \in K$, $H = He \subseteq HK$. 
  Similarly $K = eK \subseteq HK$.  So $HK$ contains both $H$ and $K$.  Now
  assume that $F$ is a subgroup which contains both $H$ and $K$.  Then given
  any $x \in H$ and $u \in K$, then $x$ and $u \in F$ and so $xu \in F$. 
  Hence $HK \subseteq F$.  So $HK$ is the smallest subgroup which contains
  both $H$ and $K$.
  
  A similar argument shows that $KH$ is a subgroup of $G$ and that 
  $H \vee K = KH$ as well.
  
  So we have that $HK = \langle H \union K \rangle = H \vee K = K \vee H =
  KH$.
\end{proof}

We can use this fact to show that normal subgroups form a sub-lattice 
within the lattice of subgroups of a group.

\begin{theorem}
  Let $G$ be a group, and let $H$ and $K$ be normal subgroups of $G$.  Then
  $H \vee K = \langle H \union K \rangle$ and $H \wedge K = H \intersect K$
  are both normal.
\end{theorem}
\begin{proof}
  We know that $H \vee K = HK$, so we will show that $HK$ is a normal
  subgroup of $G$.  If $x \in H$ and $u \in K$ and $z \in G$, then
  $xu$ is a typical element of $HK$ and we have that
  \[
    z^{-1}xuz = z^{-1}xzz^{-1}uz,
  \]
  and $z^{-1}xz \in H$, $z^{-1}uz \in K$, and so $z^{-1}xuz \in HK$.  Hence
  $HK$ is normal.

  Also, given any $y \in H \intersect K$, and any $x \in G$, we have
  that $x^{-1}yx \in H$ and $x^{-1}yx \in K$, so $x^{-1}yx \in H
  \intersect K$, and so $H \intersect K$ is normal.
\end{proof}

\begin{corollary}
  If $G$ is a group, then the set of normal subgroups of $G$ is a lattice.
\end{corollary}

We will use the symbol $\lhd$ to represent the order that gives this
lattice.  In other words, $H \lhd K$ if and only if both $H$ and $K$
are normal and $H \le K$.  This extends the use of $\lhd$ to indicate
a normal subgroup of a group.

\begin{example}
  The normal subgroup lattice of $D_{8}$ is:
  
  \begin{picture}(10,8)(-5,-1)
    \put(0,0){\makebox(0,0){$\{1\}$}}
    \put(0,0.5){\line(0,1){1}}
    \put(0,2){\makebox(0,0){$\langle a^{2} \rangle$}}
    \put(0,2.5){\line(0,1){1}}
    \put(0.5,2.5){\line(1,1){1}}
    \put(-0.5,2.5){\line(-1,1){1}}
    \put(-2,4){\makebox(0,0){$\langle a^{2}, b \rangle$}}
    \put(0,4){\makebox(0,0){$\langle a \rangle$}}
    \put(2,4){\makebox(0,0){$\langle a^{2}, ab \rangle$}}
    \put(0,4.5){\line(0,1){1}}
    \put(1.5,4.5){\line(-1,1){1}}
    \put(-1.5,4.5){\line(1,1){1}}
    \put(0,6){\makebox(0,0){$D_{8}$}}
  \end{picture}
  
  The lattice diagrams for the subgroup lattice and the normal subgroup
  lattice are sometimes combined by representing the normal subgroup lattice
  by thicker lines or double lines.
  
  \begin{picture}(10,8)(-5,-1)
    \put(0,0){\makebox(0,0){$\{1\}$}}
    \put(0.1,0.5){\line(0,1){1}}
    \put(-0.1,0.5){\line(0,1){1}}
    \put(0.5,0.5){\line(1,1){1}}
    \put(0.5,0.25){\line(2,1){3}}
    \put(-0.5,0.5){\line(-1,1){1}}
    \put(-0.5,0.25){\line(-2,1){3}}
    \put(-4,2){\makebox(0,0){$\langle b \rangle$}}
    \put(-2,2){\makebox(0,0){$\langle a^{2}b \rangle$}}
    \put(0,2){\makebox(0,0){$\langle a^{2} \rangle$}}
    \put(2,2){\makebox(0,0){$\langle ab \rangle$}}
    \put(4,2){\makebox(0,0){$\langle a^{3}b \rangle$}}
    \put(0.1,2.5){\line(0,1){1}}
    \put(-0.1,2.5){\line(0,1){1}}
    \put(3.5,2.5){\line(-1,1){1}}
    \put(2,2.5){\line(0,1){1}}
    \put(0.55,2.45){\line(1,1){1}}
    \put(0.45,2.55){\line(1,1){1}}
    \put(-3.5,2.5){\line(1,1){1}}
    \put(-0.45,2.55){\line(-1,1){1}}
    \put(-0.55,2.45){\line(-1,1){1}}
    \put(-2,2.5){\line(0,1){1}}
    \put(-2,4){\makebox(0,0){$\langle a^{2}, b \rangle$}}
    \put(0,4){\makebox(0,0){$\langle a \rangle$}}
    \put(2,4){\makebox(0,0){$\langle a^{2}, ab \rangle$}}
    \put(0.1,4.5){\line(0,1){1}}
    \put(-0.1,4.5){\line(0,1){1}}
    \put(1.55,4.55){\line(-1,1){1}}
    \put(1.45,4.45){\line(-1,1){1}}
    \put(-1.55,4.55){\line(1,1){1}}
    \put(-1.45,4.45){\line(1,1){1}}
    \put(0,6){\makebox(0,0){$D_{8}$}}
  \end{picture}
\end{example}

Part of the importance of normal subgroups is that they are closely related
to homomorphisms.  In fact every homomorphism gives you a normal subgroup.

\begin{theorem}\label{thm:inversenormal}
  Let $G$ and $H$ be groups and $\alpha: G \to H$ a homomorphism.  If $K$ is
  a normal subgroup of $H$, then $\alpha^{-1}(K)$ is a normal subgroup of
  $G$.
  
  In particular, $\ker \alpha$ is always normal.
\end{theorem}
\begin{proof}
  We know that $\alpha^{-1}(K)$ is a subgroup of $G$.  Given any $x \in G$,
  and any $y \in \alpha^{-1}(K)$, we have that
  \[
    \alpha(x^{-1}yx) = (\alpha(x))^{-1}\alpha(y)\alpha(x),
  \]
  and $\alpha(y) \in K$, so $(\alpha(x))^{-1}\alpha(y)\alpha(x) \in K$. 
  Therefore, $x^{-1}yx \in K$, and so $K$ is normal.
  
  We recall that $\ker \alpha = \alpha^{-1}(\{e\})$, and $\{e\}$ is always
  normal, so $\ker \alpha$ is normal.
\end{proof}

\begin{corollary}
  If $G$ and $H$ are groups, and $\alpha : G \to H$ is an isomorphism, then
  a subgroup $K$ of $G$ is normal if and only if $\alpha(K)$ is normal.
\end{corollary}

\begin{corollary}
  If $G$ and $H$ are isomorphic, then the lattice of normal subgroups of $G$
  and the lattice of normal subgroups of $H$ are isomorphic.
\end{corollary}

\begin{example}
  Consider the homomorphism $\alpha: D_{8} \to V$ defined by 
  $\alpha(a) = a$, and $\alpha(b) = b$.  Looking at the image of each 
  element, we see get the following table:
  \[
    \begin{array}{c|c}
      x & \alpha(x) \\
      \hline
      1 & 1 \\
      a & a \\
      a^{2} & 1 \\
      a^{3} & a \\
      b & b \\
      ab & ab \\
      a^{2}b & b \\
      a^{3}b & ab
    \end{array}
  \]
  
  Since $V$ is Abelian, every subgroup is normal, and the inverse 
  images of each subgroup are the normal subgroups
  \begin{align*}
     \alpha^{-1}(\{1\}) &= \{1, a^{2}\} \\
     \alpha^{-1}(\{1, a\}) &= \{1, a, a^{2}, a^{3}\} \\
     \alpha^{-1}(\{1, b\}) &= \{1, b, a^{2}, a^{2}b\} \\
     \alpha^{-1}(\{1, ab\}) &= \{1, ab, a^{2}, a^{3}b\} \\
     \alpha^{-1}(V) &= D_{8}.
  \end{align*}
\end{example}


\subsection*{Exercises}

\begin{exercises}
  \item Prove Theorem~\ref{thm:normalizers}.
  
  \item Let $H = \langle X \rangle$ be a subgroup of a group $G$. 
    Show that $H$ is normal if and only if $g^{-1}xg \in H$ for all $x
    \in X$.
  
  \item Find the normal subgroup lattice of $D_{10}$.

  \item Find the normal subgroup lattice of $D_{12}$.
  
  \item Find the normal subgroup lattice of $A_{4}$.
  
  \item\label{ex:commutatorsubgroup} Given elements $x$ and $y \in G$,
    their \defn{commutator}{communtator} is the element
    \[
      [x,y] = x^{-1}y^{-1}xy.
    \]
    The \defn{derived}{subgroup!derived} or \defn{commutator
    subgroup}{subgroup!commutator} is the subgroup generated by all 
    the commutators of elements of $G$
    \[
      G' = \langle \{ [x,y] : x, y \in G\} \rangle
    \]
    
    \begin{theoremenum}
      \item Show that if $x$ and $y$ commute, then $[x,y] = e$.  
        Conclude that the commutator subgroup of an Abelian group is 
        always $\{e\}$.
      
      \item Show that $G'$ is normal.
      
      \item Find the commutator subgroup of $D_{8}$.
      
      \item If $H$ and $K$ are normal subgroups of $G$, show that the 
        commutator $[x,y]$ of any pair $x \in H$ and $y \in K$ lies 
        in $H \wedge K$.  Show that if $H \wedge K = \{e\}$, then any 
        element of $H$ commutes with any element of $K$.

    \end{theoremenum}
\end{exercises}

\section{Groups of Small Order, Part II}

In previous sections we have discovered that all the groups of order 
less than $8$ are isomorphic to one of a small collection of groups.  
We do not know what groups there are of order $8$, however.

We have identified $C_{8}$, $C^{2} \cross C^{4}$, $C_{2} \cross C_{2}
\cross C_{2}$ and $D_{8}$, but there could potentially be more.  In fact,
in Exercise~\ref{ex:quaterniongroup}, there was the following definition.

\begin{definition}
  For any natural number $n$, the \defn{quaternion group}{group!quaternion} is
  the group with elements
  \[
    Q_{4n} = \{1, a, a^{2}, \ldots, a^{2n-1}, b, ab, a^{2}b, \ldots, a^{2n-1}b \}
  \]
  where the Cayley table is determined by the relations $a^{2n} = 1$, $b^{2} =
  a^{n}$, and $b^{-1}ab = a^{-1}$.
\end{definition}

In Exercise~\ref{ex:quaterniongroup} it was shown that $Q_{8}$
is not isomorphic to any other known group of order $8$.  In fact, $Q_{8}$
completes the set of isomorphism classes of groups of order $8$.

\begin{theorem}
  If $G$ is a group of order $8$, then $G$ is isomorphic to one of $C_{8}$,
  $C_{2} \cross C_{4}$, $C_{2} \cross C_{2} \cross C_{2}$, $D_{8}$ or $Q_{8}$.
\end{theorem}
\begin{proof}
  If $G$ has an element of order $8$, then Theorem~\ref{thm:cyclicgroups} tells
  us that $G$ is cyclic, and so $G \isom C_{8}$.

  If every element of $G$ other then $e$ has order $2$, then
  Theorem~\ref{thm:order2group} tells us that $G$ is a direct product of
  cyclic groups of order $2$. So $G \isom C_{2} \cross C_{2} \cross C_{2}$.

  If $G$ is not isomorphic to one of these two groups, it must have an element
  of order 4, and no elements of order 8.  Let $a$ be this element of 
  order $4$ in $G$, and let $H = \langle a \rangle$.  Since $|G : H| = 
  |G|/|H| = 8/4 = 2$, so if $b \in G \setminus H$, then $H$ has two cosets $H$
  and $Hb$, and $G$ is the disjoint union of $H$ and $Hb$.  So
  \[
    G = \{e, a, a^{2}, a^{3}, b, ab, a^{2}b, a^{3}b\}.
  \]

  If $G$ is Abelian, then $ab = ba$, and $b$ must either be of order 
  $2$ or $4$.  If $b^{2} = 1$, then $G \isom C_{4} \cross C_{2}$ where 
  the isomorphism is given by $\alpha(u^{k}, v^{l}) = a^{k}b^{l}$, 
  where $C_{4} = \langle u \rangle$ and $C_{2} = \langle v \rangle$.  
  If $b^{2} \ne e$, then we must have $b^{2} = a^{k}$, since if 
  $b^{2} = a^{k}b$, then the cancellation law tells us that $b = a^{k} 
  \in H$, which is a contradiction.  Furthermore if $k = 1$ or $k = 3$, 
  then $b^{4} = a^{2} \ne e$, which is a contradiction.  So then 
  $b^{2} = a^{2}$.  But in this case, $ab$ has order $2$, and $G \isom
  C_{4} \cross C_{2}$ where the isomorphism is given by $\alpha(u^{k}, v^{l}) 
  = a^{k}(ab)^{l}$.
  
  So the only Abelian groups of order $8$ are $C_{8}$, $C_{4} \cross C_{2}$
  and $C_{2} \cross C_{2} \cross C_{2}$.

  If $G$ is not Abelian, we note that Theorem~\ref{thm:normalconditions} says
  that since $|G : H| = 2$, $H$ is normal in $G$, which means that $b^{-1}ab
  \in H$, so
  \[
    b^{-1}ab = a^{k}
  \]
  where $k$ is one of $0$, $1$, $2$ or $3$.  But $k \ne 0$, since 
  otherwise $a \in C(e) = \{e\}$, which cannot happen.  If $k = 2$, then
  \[
   b^{-1}a^{2}b = b^{-1}abb^{-1}ab = a^{2}a^{2} = a^{4} = e,
  \]
  so $a^{2} \in C(e)$, which is also impossible.  If $k = 1$, then $b^{-1}ab
  = a$ implies $ab = ba$, so $G$ is Abelian.  So the only remaining
  possibility is that $k = 3$.

  As before, the order of $b$ must be either $2$ or $4$.  If $b$ has order $2$,
  then $b^{2} = 1$.  This means that $b = b^{-1}$, and so $b^{-1}ab = a^{3}$
  implies that $ba = a^{3}b$. So $G$ is determined by the relations
  $a^{4} = 1$, $b^{2} = 1$ and $ba = a^{3}b$, and $G$ is isomorphic to $D_{8}$
  under the trivial isomorphism $\alpha(a^{k}b^{l}) = a^{k}b^{l}$.
  
  Finally, if $b$ has order $4$, then the same argument as the Abelian case
  tells us that $b^{2} = a^{2}$.  We then observe that since $b^{3} = b_{-1}$,
  we have that $G$ is determined by the relations $a^{4} = 1$, $b^{2} = a^{2}$
  and $b^{-1}ab = a^{-1}$, and $G$ is isomorphic to $Q_{8}$
  under the trivial isomorphism $\alpha(a^{k}b^{l}) = a^{k}b^{l}$.
\end{proof}

The next lowest order that we don't have full information on is groups of
order $9$.  We know that we have $C_{9}$ and $C_{3} \cross C_{3}$, but there
could potentially be other groups of order $9$.

\begin{theorem}\label{thm:groupsoforder9}
  Let $G$ be a group of order $9$.  Then $G$ is isomorphic to one of $C_{9}$
  or $C_{3} \cross C_{3}$.
\end{theorem}
\begin{proof}
  From Lagrange's theorem, the elements of $G$ all have orders dividing $9$,
  so they have order $1$, $3$ or $9$.  If there is an element of order $9$,
  then Theorem~\ref{thm:cyclicgroups} tells us that $G \isom C_{9}$.  So if
  $G$ is not isomorphic to $C_{9}$, then every
  element other than the identity must have order $3$.
  
  Choosing an element $a \ne e$, we have that the subgroup $H = \langle
  a \rangle$ has order $3$.  Now choose $b \not\in H$.  Now $b^{2} \not\in H$,
  since that would imply $b^{2} = a^{k}$, where $k = 1$ or $2$, and in either
  case $b$ would have order $6$.  Similarly $b^{2} \notin Hb$ since then we
  would have $b^{2} = a^{k}b$, where $k = 1$ or $2$, and then since $G$ is
  Abelian $b^{3} = a^{k}b^{2} = a^{2k}b \ne e$, since $b \notin H$.  So the
  right cosets of $H$ are $H$, $Hb$ and $Hb^{2}$, and therefore
  \[
    G = \{e, a, a^{2}, b, ab, a^{2}b, b^{2}, ab^{2}, a^{2}b^{2}\}.
  \]

  Theorem~\ref{thm:centreprimepower} tells us that the centre of $G$ has
  $|Z(G)| = 3$ or $|Z(G)| = 9$.  If $|Z(G)| = 9$, then $G$ is Abelian, and
  $G \isom C_{3} \cross C_{3}$ via the isomorphism $\alpha(u^{k}, u^{l}) =
  a^{k}b^{l}$, where $C_{3} = \langle u \rangle$.
  
  So assume that $|Z(G)| = 3$.  Without loss of generality in the above
  discussion, we could have assumed that we chose $a \in Z(G)$, so that
  $H = Z(G)$.  Therefore $a$ commutes with every element of $G$, so in
  particular, $ba = ab$.  But since $G = \langle a, b \rangle$, $G$ is Abelian
  and $|Z(G)| = |G| = 9$.  So every group of order 9 is Abelian.
\end{proof}

At this point we have classified every group of order up to and including 11. 
It turns out that any group of order $12$ is isomorphic to one of $C_{12}$,
$C_{6} \cross C_{2}$, $D_{12}$, $Q_{12}$ or the alternating group $A_{4}$.
However, the proof of this fact requires considerably more powerful techniques
than we have available right now.

\subsection*{Exercises}

\begin{exercises}
  \item Show that if $p$ is a prime number, and $G$ is a group with $|G| =
  p^{2}$, then $G$ is isomorphic to one of $C_{p^{2}}$ or $C_{p} \cross C_{p}$.
  
  Hint: generalize the case of $|G| = 9$.
  
  \item Explain why there is one group of order $13$ and two groups 
    of order $14$.
  
  \item Show that $C_{15} \isom C_{5} \cross C_{3}$.
\end{exercises}

\section{Extension: Cayley Graphs}

Cayley graphs are closely related to generators, and give a nice way 
of picturing groups.  Unfortunately, they are not that useful in 
distingishing between groups which are not isomorphic, but they are 
of some independent interest, particularly when considering how one 
can computerize calculations involving groups.

\begin{definition}
  Let $(G, \ast, e)$ be a group, and let $S = \{ g_{1}, a_{2}, 
  \ldots, g_{n}\}$ be a finite set which generates $G$, ie. $G = 
  \langle S \rangle$.  The \defn{Cayley graph}{Cayley graph} of $G$ 
  with the generating set $S$ is the graph $\Gamma_{G} = (G, E)$ 
  with vertices being the elements of $G$, and two vertices $x$ and
  $y \in G$ are connected by an edge if $x = yg$ for some
  $g \in S \union S^{-1}$.
\end{definition}

Definitions vary somewhat from source to source.  Some may define the Cayley
graph as a directed graph with directed edges of the form $(x, xg)$ for
$g \in S \union S^{-1}$, while others allow a double edge $(x,y)$ if
both $x = yg$ and $y = xg$.

\begin{example}
  The cyclic group $C_{4} = \{1, a, a^{2}, a^{3}\}$ has a generating set
  $S = \{a\}$. The Cayley graph of $G$ with this generating set has edges
  $(1,a)$, $(a,a^{2})$, $(a^{2}, a^{3})$ and $(a^{3}, 1)$.  In other words,
  it looks like this:
  
  \begin{picture}(4,4)(-1,-1)
    \put(0,0){\line(1,0){2}}
    \put(0,0){\line(0,1){2}}
    \put(0,2){\line(1,0){2}}
    \put(2,0){\line(0,1){2}}
    \put(0,0){\circle*{0.2}}
    \put(0,2){\circle*{0.2}}
    \put(2,0){\circle*{0.2}}
    \put(2,2){\circle*{0.2}}
    \put(-0.5,-0.5){\makebox(0,0){$1$}}
    \put(-0.5,2.5){\makebox(0,0){$a$}}
    \put(2.5,-0.5){\makebox(0,0){$a^{3}$}}
    \put(2.5,2.5){\makebox(0,0){$a^{2}$}}
  \end{picture}
  
  This group is also generated (somewhat redundantly) by the set
  $S = \{a, a^{2}\}$.  This gives a Cayley graph which looks like this:
  
  \begin{picture}(4,4)(-1,-1)
    \put(0,0){\line(1,0){2}}
    \put(0,0){\line(0,1){2}}
    \put(0,0){\line(1,1){2}}
    \put(0,2){\line(1,0){2}}
    \put(0,2){\line(1,-1){2}}
    \put(2,0){\line(0,1){2}}
    \put(0,0){\circle*{0.2}}
    \put(0,2){\circle*{0.2}}
    \put(2,0){\circle*{0.2}}
    \put(2,2){\circle*{0.2}}
    \put(-0.5,-0.5){\makebox(0,0){$1$}}
    \put(-0.5,2.5){\makebox(0,0){$a$}}
    \put(2.5,-0.5){\makebox(0,0){$a^{3}$}}
    \put(2.5,2.5){\makebox(0,0){$a^{2}$}}
  \end{picture}
\end{example}

\begin{example}
  The four-group $V = \{1, a, b, ab\}$ has a generating set
  $S = \{a, b\}$. The Cayley graph of $G$ with this generating set has edges
  $(1,a)$, $(a,ab)$, $(1, b)$ and $(b, ab)$.  In other words,
  it looks like this:
  
  \begin{picture}(4,4)(-1,-1)
    \put(0,0){\line(1,0){2}}
    \put(0,0){\line(0,1){2}}
    \put(0,2){\line(1,0){2}}
    \put(2,0){\line(0,1){2}}
    \put(0,0){\circle*{0.2}}
    \put(0,2){\circle*{0.2}}
    \put(2,0){\circle*{0.2}}
    \put(2,2){\circle*{0.2}}
    \put(-0.5,-0.5){\makebox(0,0){$1$}}
    \put(-0.5,2.5){\makebox(0,0){$a$}}
    \put(2.5,-0.5){\makebox(0,0){$b$}}
    \put(2.5,2.5){\makebox(0,0){$ab$}}
  \end{picture}
  
  This group is also generated by the set
  $S = \{a, b, ab\}$.  This gives a Cayley graph which looks like this:
  
  \begin{picture}(4,4)(-1,-1)
    \put(0,0){\line(1,0){2}}
    \put(0,0){\line(0,1){2}}
    \put(0,0){\line(1,1){2}}
    \put(0,2){\line(1,0){2}}
    \put(0,2){\line(1,-1){2}}
    \put(2,0){\line(0,1){2}}
    \put(0,0){\circle*{0.2}}
    \put(0,2){\circle*{0.2}}
    \put(2,0){\circle*{0.2}}
    \put(2,2){\circle*{0.2}}
    \put(-0.5,-0.5){\makebox(0,0){$1$}}
    \put(-0.5,2.5){\makebox(0,0){$a$}}
    \put(2.5,-0.5){\makebox(0,0){$b$}}
    \put(2.5,2.5){\makebox(0,0){$ab$}}
  \end{picture}
\end{example}

Notice that in each case, the graphs look the same, even though the groups
are not isomorphic.

\begin{proposition}
  Let $G$ be a finite group.  The Cayley graph with generating set taken to
  be all of $G$ is the complete graph on $|G|$ vertices.
\end{proposition}
\begin{proof}
  Given any pair of vertices $x$, $y$, the element $g = y^{-1}x$ is in the
  generating set $G$, and $yg = yy^{-1}x = x$, so there is an edge joining
  $x$ and $y$.  Hence the Cayley graph is a complete graph.
\end{proof}

With careful selection of the generating set, however, the Cayley graph can
reveal a lot about the structure of the group.

\begin{example}
  The group of integers $(\integers, +, 0)$ is generated by the 
  element $1$.  The Cayley graph is infinite, but the region near $0$ 
  looks like:
  
  \begin{picture}(14,2)(-7,-1)
    \multiput(-6,0)(2,0){6}{\line(1,0){2}}
    \multiput(-6,0)(2,0){7}{\circle*{0.2}}    
    \put(-6,-0.5){\makebox(0,0){$-3$}}
    \put(-4,-0.5){\makebox(0,0){$-2$}}
    \put(-2,-0.5){\makebox(0,0){$-1$}}
    \put(0,-0.5){\makebox(0,0){$0$}}
    \put(2,-0.5){\makebox(0,0){$1$}}
    \put(4,-0.5){\makebox(0,0){$2$}}
    \put(6,-0.5){\makebox(0,0){$3$}}    
    \put(-6.5,0){\makebox(0,0){$\cdots$}}    
    \put(6.5,0){\makebox(0,0){$\cdots$}}    
  \end{picture}
  
\end{example}


Cayley graphs have some regularities that normal graphs do not.

\begin{proposition}
  Let $G$ be a group, and $S$ a finite set which generates $G$.
  The Cayley graph of $G$ with generating set $S$ has the following
  properties:
  \begin{theoremenum}
    \item the graph is connected.
    
    \item every vertex is an end of exactly $|S \union S^{-1}|$ edges
  \end{theoremenum}
\end{proposition}
\begin{proof}
  (i) Every element $x$ can be written as a product $x = x_{1}x_{2}\cdots x_{n}$
  of elements $x_{k} \in S \union S^{-1}$, so $e, x_{1}, x_{1}x_{2}, \ldots,
  x_{1}x_{2}\cdots x_{n-1}, x$ is a path from $e$ to $x$.  So the Cayley graph
  is connected.
  
  (ii) Given any $x \in G$, if we consider the set
  \[
    X = \{ xg : g \in S \union S^{-1} \},
  \]
  then we note that if $xg_{1} = xg_{2}$, the cancellation law says that
  $g_{1} = g_{2}$.  Therefore every edge $(x, xg)$ is distinct, and $|X| =
  |S \union S^{-1}|$.  Furthermore, if there is some $y$ such that $(y,x)$
  is an edge, then $x = yg$ for some $g \in S \union S^{-1}$, so $y =
  xg^{-1} \in X$.  So $X$ is exactly the set of all vertices connected to
  $x$ by one edge.  So $x$ is an end of exactly $|S \union S^{-1}|$ edges.
\end{proof}


\subsection*{Exercises}

\begin{exercises}
  \item Draw the Cayley graph of $C_{6}$ with generating set $\{a\}$.
  
    Draw the Cayley graph of $C_{6}$ with generating set $\{a^{2}, a^{3}\}$.

  \item Draw the Cayley graph of $D_{6}$ with generating set $\{a,b\}$.

  \item Draw the Cayley graph of $D_{8}$ with generating set $\{a,b\}$.

  \item Draw the Cayley graph of $C_{2} \cross C_{4}$ with generating set 
  $\{(a,1), (1,b)\}$.

  \item Draw the region near $0$ of the Cayley graph of $\integers$ with
    generating set $\{2, 3\}$.

  \item Draw the region near $0$ of the Cayley graph of $\integers^{2}$ with
    generating set $\{(1,0), (0,1)\}$.
    
    Draw the same region if the generating set is $\{(1,0), (0,1), 
    (1,1)\}$.
  
  \item Draw the region near $e$ of the Cayley graph of the free
    group\index{group!free} $F_{2}$ with generating set $\{a, b\}$.
\end{exercises}


\newpage
\section*{Assignment 4}

The following exercises are due Monday, April 19th.  Since this is a
long assignment, it will be worth twice as much as the other 3 assignments.

\begin{description}
  \item[3.1] Exercises 2, 4, 6.
  \item[3.2] Exercises 1, 3, 5.
  \item[3.3] Exercises 1, 3, 5.
  \item[3.4] Exercises 1, 4, 5.
  \item[3.5] Exercises 2.
  \item[3.6] Exercises 4.
  \item[3.7] Exercises 1, 2, 5, 6.
  \item[3.8] Exercises 2, 3, 6.
  \item[3.9] Exercises 1, 2.
  \item[3.10] Exercises 1, 2, 5, 6.
\end{description}

