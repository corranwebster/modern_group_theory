\chapter{Constructing Groups}

In Chapter 2 we saw a simple way that we could combine two groups to 
get a third group: the direct product.  In this chapter we look at 
other ways to construct new groups from already known groups.

\section{Quotient Groups}

Let $G$ be a group, and $N$ a normal subgroup of $G$.  We can (at least
potentially) define a binary operation $\ast$ on the set of cosets of $N$
by
\[
  Nx \ast Ny = Nxy.
\]
The difficulty with this definition is that there may be many
different choices for $x'$ and $y'$, so that $Nx' = Nx$
and $Ny = Ny'$, and it is not immediate why we should have $Nxy =
Nx'y'$.  However they are in fact equal since $x' = ux$ and $y' = vy$ for some
$u$ and $v \in N$, and since $N$ is normal, we have $z = xvx^{-1} \in N$, so
\[
  x'y' = uxvy = uxvx^{-1}xy = (uz)xy,
\]
so $Nx'y' = N(uz)xy = Nxy$.  So $\ast$ is a well-defined binary 
operation, and it is only well-defined if $N$ is normal.

\begin{proposition}
  Let $G$ be a group and $N$ a normal subgroup of $G$.  Let $G/N$ be the set
  of all cosets of $N$ in $G$. Then $(G/N, \ast, N)$ is a group.
\end{proposition}
\begin{proof}
  We first observe that $G/N$ is associative, since for any right cosets
  $Nx$, $Ny$ and $Nz \in G/N$, we have
  \[
    (Nx \ast Ny) \ast Nz = Nxy \ast Nz = Nxyz = Nx \ast Nyz = Nx 
    \ast (Ny \ast Nz).
  \]
  
  The set $N$ is an identity, since $N = Ne$, and so
  \[
    Ne \ast Nx = Nex = Nx \qquad \text{and} Nx \ast Ne = Nxe = Nx.
  \]
  
  Finally, $Nx^{-1}$ is the inverse of $Nx$, since
  \[
    Nx \ast Nx^{-1} = Nxx^{-1} = Ne = N \qquad \text{and} \qquad 
    Nx^{-1} \ast Nx = Nx^{-1}x = Ne = N.
  \]
\end{proof}

In fact, there is another way of looking at this product.  If we 
consider cosets $Nx$ and $Ny$, then we have that the product of the 
cosets as sets is
\[
  (Nx)(Ny) = (xN)(Ny) = (xN^{2})y = (xN)y = (Nx)y = Nxy,
\]
recalling that $Nx = xN$ since $N$ is normal, and $N^{2} = N$ since
$N$ is a subgroup.  In other words, $Nx \ast Ny$ is given by the
product of sets $(Nx)(Ny)$.  In some texts this is used as the 
definition of the product.

\begin{definition}
  If $G$ is a group, and $N$ a normal subgroup of $G$, then we call $G/N$ the 
  \defn{quotient group}{quotient group} of $N$ in $G$.  
\end{definition}

\begin{example}
  Let $D_{6} = \{1, a, a^{2}, b, ab, a^{2}b\}$ as usual.  If $N = 
  \{1, a, a^{2}\}$, then $N$ is normal, and the cosets are $N$ and 
  $Nb$.  Then the Cayley table of $D_{6}/N$ is simply
  \[
    \begin{array}{c|cc}
      \ast & N & Nb \\
      \hline
      N & N & Nb \\
      Nb & Nb & N
    \end{array}
  \]
  Clearly, $D_{6}/N \isom C_{2}$.
\end{example}

\begin{example}
  The additive group of integers is Abelian, so every subgroup is 
  normal.  If we have a subgroup of the form
  \[
    N = m\integers = \{mx : x \in \integers \},
  \]
  then we observed in Example~\ref{eg:equivmodm} that the cosets of 
  this subgroup are the sets of numbers which have the same remainer 
  modulo $m$, or more concretely,
  \[
    \integers/N = \{N, N + 1, N + 2, \ldots, N + (m-1)\}.
  \]
  The group operation on these cosets is just
  \[
    (N + x) + (N + y) = N + (x+y) = N + z,
  \]
  where $z = x + y \pmod{m}$.  In other words $\integers/m\integers 
  \isom Z_{m}$.  Indeed, this is a common way of defining addition modulo 
  $m$.
\end{example}

From the theory developed in the previous chapters, we can 
immediately conclude the following.

\begin{proposition}
  Let $G$ be a group, and $N$ a subgroup of $G$.  Then
  \begin{theoremenum}
    \item if $|G|$ is finite, then $|G/N| = |G|/|N|$,
    \item the function $\alpha(x) = Nx$ is a homomorphism from $G$ 
    to $G/N$, and $\ker \alpha = N$.
  \end{theoremenum}
\end{proposition}
\begin{proof}
  (i) This is immediate from the fact that $|G/N| = [G : N]$ and 
  Lagrange's Theorem.
  
  (ii) That $\alpha$ is a homomorphism follows immediately from the 
  fact that
  \[
    \alpha(x) \ast \alpha(y) = Nx \ast Ny = Nxy = \alpha(xy).
  \]
  It is immediate that $N \subseteq \ker \alpha$, since if $x \in 
  N$, $\alpha(x) = Nx = N$.  Similarly, if $\alpha(x) = N$, then 
  $Nx = N$, which only happens when $x \in N$. So $\ker \alpha = N$.
\end{proof}

\begin{corollary}
  A subgroup $N$ of a group $G$ is normal if and only if it is the 
  kernel of some homomorphism.
\end{corollary}

The relationship between quotient groups and homomorphisms is 
significantly deeper than this corollary, however.  These 
relationships are encapsulated in a trilogy of theorems called the 
Isomorphism Theorems.  Unfortunately there is little consensus about 
which of the three should be first, second and third.

\begin{theorem}[First Isomorphism Theorem]
  Let $G$ and $H$ be groups, and $\alpha: G \to H$ a homomorphism.  
  Then
  \[
    \alpha(G) \isom G/\ker \alpha.
  \]
\end{theorem}
\begin{proof}
  For simplicity of notation, let $N = \ker \alpha$.
  
  We would like to define a function $\beta : G/\ker \alpha \to
  \alpha(G)$ by $\beta(Nx) = \alpha(x)$, but it's not clear that this 
  is a well-defined function.  To verify that this definition is 
  good, we need to show that if $Nx = Ny$ then $\beta(Nx) = 
  \beta(Ny)$ so that the value of $\beta$ does not depend on the choice 
  of $x$.
  
  Now if $Nx = Ny$ we have that $xy^{-1} \in N$, so $\alpha(xy^{-1}) = 
  e$, and hence
  \[
    \alpha(y) = e\alpha(y) = \alpha(xy^{-1})\alpha(y) = 
    \alpha(xy^{-1}y) = \alpha(x).
  \]
  So we conclude that if $Nx = Ny$ then $beta(Nx) = \beta(Ny)$, and 
  so $\beta$ is well-defined.
  
  Furthermore, $\beta$ is a homomorphism, since
  \[
    \beta(Nx \ast Ny) = \beta(Nxy) = \alpha(xy) = 
    \alpha(x)\alpha(y) = \beta(Nx)\beta(Ny).
  \]
  
  We also have that $\beta$ is onto, since if $y \in \alpha(G)$, then 
  $y = \alpha(x)$ for some $x \in G$, but then $y = \beta(Nx)$.
  
  Finally, if $\beta(Nx) = \beta(Ny)$, then $\alpha(x) = \alpha(y)$, so
  \[
    \alpha(xy^{-1}) = \alpha(x)(\alpha(y))^{-1} = 
    \alpha(x)(\alpha(x))^{-1} = e.
  \]
  This means that $xy^{-1} \in \ker \alpha = N$, so $Nx = Ny$.  
  Hence $\beta$ is one-to-one.
  
  So $\beta$ is an isomorphism, and we conclude that $G/\ker \alpha 
  \isom \alpha(G)$.
\end{proof}

We will now turn to look at how the subgroup structure of $G$ and the
subgroup structure of $G/N$ are related.  Letting $\alpha: G \to G/N$
be given by $\alpha(x) = Nx$, we have that if $H \le G$, then
$\alpha(H) \le G/N$ and if $K \subseteq G/N$ then $\alpha^{-1}(K) \le
G$ from Propositions~\ref{prop:homsubgroup} and
\ref{prop:inversehomsubgroup}.  A deeper question is if there is any 
relationship between normal subgroups of $G$ and normal subgroups of 
$G/N$.

\begin{proposition}
  Let $G$ be a group and $N$ a normal subgroup of $G$.  Then every
  subgroup of $N$ is equal to $K/N$ where for some $K$ with $N \le K
  \le G$.  Furthermore $K/N$ is normal if and only if $K$ is normal.
\end{proposition}
\begin{proof}
  Let $\alpha: G \to G/N$ be given by $\alpha(x) = Nx$.

  Let $H$ be a subgroup of $G/N$, and let $K = \alpha^{-1}(H)$, so
  that $K$ is a subgroup of $G$, and since $N = \alpha^{-1}(\{e\}) 
  \subseteq \alpha^{-1}(H)$, so $N \le K$.  So $H = \alpha(K)$ and if we
  restrict $\alpha$ to $K$, the First Isomorphism Theorem tells us
  that
  \[
    \alpha(K) \isom K/N.
  \]
  Furthermore, recall from the proof of the First Isomorphism Theorem
  that this isomorphism is given by $\beta(Nx) = \alpha(x)$, and
  $\alpha(x) = Nx$, so $\beta$ is just the identity map, and so $H =
  K/N$.
  
  If $K$ is normal in $G$, then $x^{-1}Kx = K$ for each $x \in G$, and 
  so
  \[
    (Nx^{-1}) \ast \alpha(K) \ast (Nx) = \alpha(x^{-1}) \ast \alpha(K)
    \ast \alpha(x) = \alpha(x^{-1}Kx) = \alpha(K),
  \]
  so $H = \alpha(K)$ is normal in $G/N$.
  
  Conversely, if $H = K/N$ is normal in $G/N$, then
  Theorem~\ref{thm:inversenormal} tells us immediately that
  $\alpha^{-1}(H) = K$ is normal in $G$.
\end{proof}

Now if $N \lhd K \lhd G$, as in the last part of the Proposition, we
note that $N$ is normal when regarded as a subgroup of $K$ also, and
so we can take three quotients: $G/N$, $G/K$ and $K/N$.  The Second 
Isomorphism Theorem gives us a relationship between these three 
quotients.

\begin{theorem}[Second Isomorphism Theorem]
  Let $G$ be a group and $N$ and $K$ normal subgroups of $G$ with $N 
  \le K$.  Then
  \[
    (G/N)/(K/N) \isom G/K.
  \]
\end{theorem}
\begin{proof}
  We first note that since $K$ is normal, the previous proposition tells us 
  that $K/N$ is normal in $G/N$, and so $(G/N)/(K/N)$ is defined.
  
  We would like to define a function $\alpha: G/N \to G/K$ by 
  $\alpha(Nx) = Kx$, but once again we must be careful that this 
  well-defined, since there are multiple possible choices for $x$ 
  which give the same coset $Nx$.  If $Nx = Ny$, then we recall that 
  $xy^{-1} \in N$, and so $xy^{-1} \in K$ as well.  So $K = 
  K(xy^{-1})$,
  \[
    \alpha(Ny) = Ky = (Kxy^{-1})y = Kx = \alpha(Ny).
  \]
  So this is a well-defined function.
  
  Furthermore, $\alpha$ is a homomorphism, since
  \[
    \alpha(Nx \ast Ny) = \alpha(Nxy) = Kxy = Kx \ast Ky = \alpha(Nx) 
    \ast \alpha(Ny).
  \]
  We observe that $\alpha(Nx) = K$ if and only if $x \in K$, or
  equivalently, $Nx \in K/N$.  But this means that $\ker \alpha =
  K/N$.  We also have that since every coset of $K$ is of the form 
  $Kx$ for some $x \in G$, we have that $\alpha(G/N) = \{ \alpha(Nx) : 
  x \in G \} = \{ Kx : x \in G\} = G/K$, so $\alpha$ is onto.
  
  Now the First Isomorphism Theorem tells us that
  \[
    \alpha(G/N) \isom (G/N)/\ker \alpha,
  \]
  But we know that $\alpha(G/N) = G/K$ and $\ker \alpha = K/N$, so
  \[
    G/K \isom (G/N)/(K/N).
  \]
\end{proof}

Note that this theorem essentially says that the quotient operation
cancels in the way that you would expect a quotient to cancel: if $G$,
$K$ and $N$ were numbers you would expect the same equation to hold.

We can also ask what happens if $K$ is a general subgroup of $G$.  In
this case we can't talk about $K/N$, since we may not have $N
\subseteq K$.  However we do know that the meet of $K$ and $N$ is a
subgroup of $K$.  Indeed we have that for any $x \in K$, and $y \in K
\wedge N$ we have that $x^{-1}yx \in K$, since $K$ is a subgroup.  But
we also have that $x^{-1}yx \in N$, since $N$ is normal.  Hence $K
\wedge N$ is a normal subgroup of the subgroup $K$.  So we can
consider the quotient group $K/(K \wedge N)$.  Similarly, although $N$
is not a subgroup of $K$, we know that $N \vee K$ contains $N$, and
since $N$ is normal, we can consider the quotient group $(K \vee N)/N$.

\begin{theorem}[Third Isomorphism Theorem]
  Let $G$ be a group, $K$ a subgroup of $G$ and $N$ a normal subgroup 
  of $G$.  Then
  \[
    K/(K \wedge N) \isom (K \vee N)/N.
  \]
\end{theorem}
\begin{proof}
  Recall from Proposition~\ref{prop:normalproduct} that $K \vee N = 
  NK$ when $N$ is normal.
  
  We define a function $\alpha: K \to G/N$ by $\alpha(x) = Nx$.  This
  is a homomorphism since
  \[
    \alpha(xy) = Nxy = Nx \ast Ny = \alpha(x) \ast \alpha(y).
  \]
  The image of $\alpha$ is the set
  \[
    \alpha(K) = \{ Nx : x \in K \} = NK/N = (N \vee K)/N.
  \]
  Furthermore, the kernel of $\alpha$ is the set of $x$ such that 
  $\alpha(x) = N$, ie.~all $x \in K$ such that $Nx = N$.  But $Nx = N$ 
  if ans only if $x \in N$, so $x \in K \intersect N = K \wedge N$.
  
  So the First Isomorphism Theorem tells us that
  \[
    \alpha(K) = K/\ker \alpha,
  \]
  and so
  \[
    (N \vee K)/N \isom K/(K \wedge N).
  \]
\end{proof}


As you may expect, quotient groups can be used to shed some light on 
the structure of finite groups.

\begin{theorem}\label{thm:centrequotient}
  Let $G$ be a finite group which is not Abelian, and $Z(G)$ the
  centre of $G$.  Then $G/Z(G)$ cannot be cyclic.
\end{theorem}
\begin{proof}
  We first recall that $Z(G)$ is always normal, so $G/Z(G)$ is defined.
  If $G/Z(G)$ is cyclic, then we can find an element $t$ so that 
  $Z(G)t$ generates $G/Z(G)$, and so every coset of $Z(G)$ is of the 
  form $Z(G)t^{k}$ for some $k$. But then given arbitrary elements $x$ 
  and $y \in G$, we have that $x = ut^{k}$ and $y = vt^{l}$ for some 
  $u$ and $v \in Z(G)$.  So now, since $u$ and $v$ commute with all 
  elements of $G$,
  \[
    xy = ut^{k}vt^{l} = uvt^{k}t^{l} = uvt^{l}t^{k} = vt^{l}ut^{k} = 
    yx.
  \]
  So $G$ is Abelian, which is a contradiction.
\end{proof}

\begin{corollary}
  If $p$ is a prime number and $G$ is a group with order $p^{2}$, 
  then $G$ is Abelian.
\end{corollary}
\begin{proof}
  We know from Theorem~\ref{thm:centreprimepower} that the order of
  $Z(G)$ is either $p$ or $p^{2}$.  But if $|Z(G)| = p$, then $G$ is
  not Abelian, and we have that $|G/Z(G)| = |G|/|Z(G)| = p^{2}/p = p$,
  so $G/Z(G)$ must be a cyclic group of order $p$.  But the previous
  theorem showed that this cannot happen.
  
  Hence $|Z(G)| = p^{2}$, and so $G$ is Abelian.
\end{proof}

This corollary allows us to slightly simplify the proof of
Theorem~\ref{thm:groupsoforder9}, since the last paragraph can be replaced
by a reference to this corollary. Indeed, one can generalise that theorem to
all prime numbers.

\begin{theorem}
  If $p$ is a prime number and $G$ is a group with order $p^{2}$, 
  then $G$ is isomorphic to one of $C_{p^{2}}$ or $C_{p} \times C_{p}$.
\end{theorem}

\begin{proof}
  The previous corollary tells us that $G$ is Abelian.  If $G$ has an
  element of order $p^{2}$, then $G \isom C_{p^{2}}$.
  
  Otherwise every element of $G$ other then the identity $e$ has order $p$. 
  Let $a$ be such an element, and let $N = \langle a \rangle = \{1, a, a^{2},
  \ldots, a^{p-1}\}$. Lagrange's Theorem tells us that the quotient group $G/N$
  has order $p$, so $G/N \isom C_{p}$.  So there is some element $b$ with
  order $p$, such that $G/N$ is generated by the coset $Nb$, so $G/N = \{N, Nb,
  Nb^{2}, \ldots, Nb^{n-1}\}$.  Hence $G = \{a^{k}b^{l} : k, l = 0, 1, \ldots,
  p-1\}$ with $a^{p} = e$ and $b^{p} = e$.  Hence $G$ is isomorphic to
  $C_{p} \cross C_{p}$ via the isomorphism
  \[
    \alpha(u^{k},u^{l}) = a^{k}b^{l}.
  \]
\end{proof}



\subsection*{Exercises}

\begin{exercises}
  \item Consider the group $D_{8}$.  The subgroup $Z(D_{8}) = \{1,
    a^{2}\}$ is normal, since it is the centre of $D_{8}$.  Write down
    the Cayley table of $D_{8}/Z(D_{8})$.
    
    Explain why $D_{8}/Z(D_{8}) \isom V$.
  
  \item Consider the group $D_{12}$.  The subgroup $Z(D_{12}) = \{1,
    a^{3}\}$ is normal, since it is the centre of $D_{12}$.  Write down
    the Cayley table of $D_{12}/Z(D_{12})$.
    
    Explain why $D_{12}/Z(D_{12}) \isom D_{6}$?
  
  \item\label{ex:D2nquotient} Generalize the above two results, and show that if $n$ is 
    even, then $D_{2n}/Z(D_{2n}) \isom D_{n}$.
  
  \item Consider the normal subgroup $N = \{1, a^{2}, a^{4}\}$ of the 
    group $D_{12}$.  Write down the Cayley table of $D_{12}/N$.  Show 
    $D_{12}/N \isom V$.
  
  \item Let $G$ and $H$ be groups.  Show that the set
    \[
      K = \{ (g,e): g \in G \}
    \]
    is a normal subgroup of $G \cross H$.  Let $\pi: G \cross H \to H$
    be defined by $\pi(g,h) = h$ (this a homomorphism by
    Exercise~\ref{ex:directprodhoms}).  Use this homomorphism to show
    that
    \[
      (G \cross H) / K \isom H.
    \]
  
  \item Consider the additive group $\integers^{2}$.  Show that
    $\alpha(x,y) = 3x+2y$ is a homomorphism from $\integers^{2} \to
    \integers$, and $\ker \alpha = K = \{ (2k,-3k) : k \in integers \}$. 
    Show that
    \[
      \integers^{2} / K \isom \integers.
    \]
  
  \item Let $\alpha: G \to H$ be a homomorphism.  Use the first
    Isomorphism Theorem to show that $|\alpha(G)|$ divides both $|G|$ and
    $|H|$.  Show that if $|G|$ and $|H|$ have a greatest common
    divisor of $1$, then $\alpha(x) = e$ for all $x \in G$.
  
  \item Let $G$ be a group.  Recall the commutator subgroup $G'$
    defined in Exercise~\ref{ex:commutatorsubgroup} is normal.  Show
    that $G/G'$ is Abelian.
  
  \item Let $G$ be a group, and $N$ be a normal subgroup of $G$ such 
    that $G/N$ is Abelian.  Show that the commutator subgroup $G'$
    defined in Exercise~\ref{ex:commutatorsubgroup} is a normal subgroup
    of $N$.
\end{exercises}


\section{Automorphism Groups}

Recall that an automorphism of a group $G$ is an isomorphism from $G$ to
itself.  The set of all automorphisms of $G$ is denoted by $\Aut(G)$.  This
set is never empty since at the very least the identity map defined $\id(x)
= x$ is always an automorphism.

\begin{proposition}
  If $G$ is a group, then $(\Aut(G), \circ, \id)$ is a group, where $\circ$
  is function composition.
\end{proposition}
\begin{proof}
  Function composition of two automorphisms gives another automorphism, since
  if $\alpha$ and $\beta \in \Aut(G)$, then $\beta \circ \alpha: G \to G$ is
  an isomoprhism by Proposition~\ref{prop:homomorphismfacts}, so
  $\beta \circ \alpha \in \Aut(G)$.
  
  We already know that function composition is associative, so that group
  axiom holds.
  
  The identity map $\id$ is an identity under composition, since for any
  $x \in G$,
  \[
    (\id \circ \alpha)(x) = \id(\alpha(x)) = \alpha(x) \qquad \text{and}
    \qquad (\alpha \circ \id)(x) = \alpha(\id(x)) = \alpha(x),
  \]
  so we conclude that $\id \circ \alpha = \alpha \circ \id = \alpha$.
  
  Since $\alpha$ is an isomorphism, it has an inverse function which is also
  an isomorphism from $G$ to $G$ by Proposition~\ref{prop:homomorphismfacts}.
  We know that for any inverse function, $\alpha^{-1} \circ \alpha = \id =
  \alpha \circ \alpha^{-1}$, so $\alpha^{-1}$ is an inverse for $\alpha$ in
  the set of automorphisms.
\end{proof}

\begin{example}
  Consider the group $C_{4}$.  Any automorphism has to preserve the order of
  each of the elements, and since $a^{2}$ is the only element of order $2$,
  $\alpha(a^{2}) = a^{2}$ for every automorphism $\alpha$.  However, an
  automorphism could potentially swap $a$ and $a^{3}$.  Indeed, there are
  two automorphisms: $\id$ and the function $\alpha(x) = x^{-1}$, or more
  concretely,
  \[
    \begin{array}{c|c}
      x & \alpha(x) \\
      \hline
      1 & 1\\
      a & a^{3}\\
      a^{2} & a^{2}\\
      a^{3} & a
    \end{array}
  \]
  Verifying that this is an isomorphism is easy, since
  \[
    \alpha(a^{k}a^{l}) = \alpha(a^{k+l}) = a^{-k-l} = a^{-k}a^{-l}
    = \alpha(a^{k})\alpha(a^{l}),
  \]
  and it is clearly bijective.
  
  Since there are only two elements, $\Aut(C_{4}) \isom C_{2}$.
\end{example}

\begin{example}
  Consider the four-group $V$.  $V$ has $3$ elements of order $2$, $a$, $b$
  and $ab$, so an isomorphism could possibly interchange those elements.
  In fact any permutation of these three elements gives rise to a distinct
  automorphism.
  
  For example, the function $\alpha$ given by
  \[
    \begin{array}{c|c}
      x & \alpha(x) \\
      \hline
      1 & 1\\
      a & ab\\
      b & b\\
      ab & a
    \end{array}
  \]
  is an automorphism: it is clearly bijective, $\alpha(xx) = 1 =
  \alpha(x)\alpha(x)$ for any $x \in V$, since $x^{2} = 1$ for every element,
  $\alpha(1x) = \alpha(1)\alpha(x)$ for any $x \in V$, and the remaining cases
  are covered by
  \begin{align*}
      \alpha(a b) &= a = (ab)b = \alpha(a)\alpha(b)\\
      \alpha(a (ab)) &= b = (ab)a = \alpha(a)\alpha(ab)\\
      \alpha(b (ab)) &= ab = ba = \alpha(b)\alpha(ab).
  \end{align*}
  So $\alpha$ is indeed an automorphism.
  
  Given that every automorphism corresponds to a permutation of a set with
  3 elements, and function composition will correspond to composition of the
  permutations, we have that $\Aut(V) \isom S_{3}$.
\end{example}

As the previous example illustrates, finding all the automorphisms of a group
can be potentially difficult.  However, non-Abelian groups have a collection of
automorphisms which are easy to find.

\begin{proposition}
  If $G$ is a group, then the conjugation by x function
  \[
    \alpha_{x}(y) = x^{-1}yx
  \]
  is an automorphism.
\end{proposition}
\begin{proof}
  We first note that
  \[
    \alpha_{x}(yz) = x^{-1}yzx = x^{-1}yxx^{-1}zx = \alpha(y)\alpha(z),
  \]
  so $\alpha: G \to G$ is a homomorphism.  Now $\alpha_{x}(y) = e$ if and only if
  \[
    x^{-1}yx = e.
  \]
  But this implies that $y = xx^{-1} = e$.  So $\ker \alpha = \{e\}$, and
  so $\alpha_{x}$ is one-to-one.  Finally, $G$ is a normal subgroup of itself,
  so $x^{-1}Gx = G$, and so $\alpha_{x}(G) = G$.
  
  Hence $\alpha_{x}$ is an automorphism.
\end{proof}

We call such automorphisms \defn{inner automorphism}{automorphism!inner}, and
let $\Inn(G) = \{ \alpha_{x} : x \in G\}$ be the set of all inner automorphisms
of $G$.  Of course, not every one of these automorphisms need be distinct,
since if $x \in Z(G)$, then
\[
  \alpha_{x}(y) = x^{-1}yx = \id(y),
\]
for every $y$, and so $\alpha_{x} = y$.  So in particular, the collection of
inner automorphisms is just $\{\id\}$ if $G$ is Abelian.  In fact the set of
inner automorphisms is very closely related to the centre.

\begin{theorem}
  Let $G$ be a group.  Then $\Inn(G)$ is a subgroup of $\Aut(G)$, the function
  $\beta: x \mapsto \alpha_{x^{-1}}$ is a homomorphism from $G$ onto $\Inn(G)$, and
  \[
    \Inn(G) \isom G/Z(G).
  \]
\end{theorem}
\begin{proof}
  We start by observing that the function $\beta$ is a homomorphism:
  \[
    \beta(xy)(z) = \alpha_{(xy)^{-1}}(z) = xyzy^{-1}x = x(\alpha_{y^{-1}}(z))x
    = \alpha_{x^{-1}}(\alpha_{y^{-1}}(z)) = (\beta(x) \circ \beta(y))(z)
  \]
  for all $z \in G$, so $\beta(xy) = \beta(x) \circ \beta(y)$.
  
  Now the image of $G$ under $\beta$ is precisely $\Inn(G)$, so $\Inn(G)$ is
  a subgroup.  Furthermore, $\beta(x) = \id$ if and only if
  $\beta(x)(z) = z$ for all $z \in G$, or equivalently,
  \[
    xzx^{-1} = z
  \]
  for all $z \in G$.  So $x \in \ker \beta$ if and only if $x \in Z(G)$.
  Therefore $\ker \beta = Z(G)$.
  
  The First Isomorphism Theorem then tells us that
  \[
    \beta(G) \isom G/\ker \beta,
  \]
  so
  \[
    \Inn(G) \isom G/Z(G).
  \]
\end{proof}
\begin{corollary}
  If $G$ is a finite group which is not Abelian, then $\Inn(G)$ is never
  cyclic.
\end{corollary}
\begin{proof}
  This follows immediately from the above theorem and
  Theorem~\ref{thm:centrequotient}.
\end{proof}

\begin{example}
  The group $D_{6}$ has centre $Z(D_{6}) = \{1\}$, so $\Inn(D_{6}) \isom D_{6}$.
  In fact, since $D_{6}$ has $2$ elements of order $3$, and $3$ elements of
  order $2$, if $\alpha \in \Aut(D_{6})$ we must have
  \begin{align*}
    \alpha(a) &= a \text{ or } a^{2}\\
    \alpha(b) &= b, ab \text{ or } a^{2}b\\
  \end{align*}
  and since $\alpha(a^{k}b^{l}) = (\alpha(a))^{k}(\alpha(b))^{l}$, these
  choices completely determine the automorphism.  Therefore $|\Aut(D_{6})|
  \le 6$, and since $|\Inn(D_{6})| = |D_{6}| = 6$, every automorphism of
  $D_{6}$ is inner.
\end{example}

\begin{example}
  The group $D_{8}$ has centre $Z(D_{8}) = \{1, a^{2}\}$, so $|\Inn(G)| =
  |D_{8}|/|Z(D_{8})| = 8/2 = 4$.  But we know that $D_{8}/Z(D_{8})$ cannot be
  cyclic, so $\Inn(G) \isom V$.
\end{example}

It's worthwhile noting that if $H$ is a normal subgroup of $G$, then the
inner automorphisms of $G$ are automorphisms of $H$ when restricted to just $H$.
This follows because if $x \in G$, we have that
\[
  H = x^{-1}Hx = \alpha_{x}(H),
\]
so $\alpha_{x}$, regarded as a function defined on $H$, must be a bijective
homomorphism onto $H$, ie.~an element of $\Aut(H)$.

A \defn{characteristic subgroup}{subgroup!characteristic} $H$ of a group $G$
is a subgroup which is invariant under every automorphism of $G$, in other
words $\alpha(H) = H$ for all $\alpha \in \Aut(G)$.  Every characteristic
subgroup is automatically normal, since if $x \in G$, so $\alpha_{x} \in
\Inn(G)$, then
\[
  H = \alpha_{x}(H) = x^{-1}Hx.
\]
The centre of $G$ is always characteristic, since any isomorphism always
maps the centre to the centre.  The trivial subgroups $G$ and $\{e\}$ are also
always characteristic.

\begin{example}
  In the group $D_{6}$, the subgroup $H = \langle a \rangle$ is characteristic,
  since any automorphism must map $1$ to $1$ and elements of order $3$ to elements
  of order $3$, ie.~the set $\{a, a^{2}\}$ maps onto $\{a, a^{2}\}$.
\end{example}

\begin{proposition}
  If $G$ is a group, $N$ is a normal subgroup of $G$ and $H$ is a
  characteristic subgroup of $N$, then $H$ is a normal subgroup of $G$.
\end{proposition}
\begin{proof}
  Since $\alpha_{x} \in \Inn(G)$ is an automorphism of $N$, and $H$ is
  characteristic, then
  \[
    x^{-1}Hx = \alpha_{x}(H) = H.
  \]
\end{proof}

The inner automorphisms which come from characteristic subgroups are also
interesting.

\begin{theorem}
  Let $G$ be a group, and let $H$ be a characteristic subgroup of $G$.
  Then the set of inner automorphisms of the form $\{\alpha_{x} : x \in H \}$
  is a normal subgroup of $\Aut(G)$.
\end{theorem}
\begin{proof}
  We first note that since $\beta(x) = \alpha_{x^{-1}}$ is a homomorphism
  from $G$ to $\Inn(G)$, $\beta(H) = \{\alpha_{x} : x \in H \}$ is a subgroup
  of $\Aut(G)$.
  
  Let $x \in H$. Given any automorphism $\alpha$, we have that for any
  $z \in G$,
  \begin{align*}
    (\alpha^{-1} \circ \alpha_{x} \circ \alpha)(z)
      &= \alpha^{-1}(\alpha_{x}(\alpha(z)))\\
      &= \alpha^{-1}(x^{-1}\alpha(z)x)\\
      &= \alpha^{-1}(x^{-1})\alpha^{-1}(\alpha(z))\alpha^{-1}(x)\\
      &= (\alpha^{-1}(x))^{-1}z\alpha^{-1}(x)\\
      &= \alpha_{\alpha^{-1}(x)}(z).
  \end{align*}
  But since $H$ is characteristic, $\alpha^{-1}(x) \in H$, so $\alpha^{-1}
  \circ \alpha_{x} \circ \alpha \in \beta(H)$.  Hence $\beta(H)$ is a normal
  subgroup of $\Aut(G)$.
\end{proof}

\begin{corollary}
  If $G$ is a group, then $\Inn(G)$ is a normal subgroup of $\Aut(G)$.
\end{corollary}

\subsection*{Exercises}

\begin{exercises}
  \item Find the automorphism group of $C_{5}$.  Does $C_{5}$ have 
    any non-trivial inner automorphisms?
  
  \item Find the automorphism group of $C_{6}$.
  
  \item Find $\Aut(Q_{8})$.
  
  \item Show that the additive group of integers has only two
    automorphisms: $\id$ and $\iota(x) = -x$.  Conclude that
    $\Aut(\integers) \isom C_{2}$.
  
  \item Let $\integers_{m}$ be the additive group of integers modulo $m$,
    and let $\alpha : \integers_{m} \to \integers_{m}$ be a homomorphism.
    
    Show that if $\alpha(1) = k$, then $\alpha(x) = kx$.
    
    Show that $\alpha$ is an automorphism if and only if the greatest common
    divisor of $k$ and $m$ is $1$.
    
    Let $\alpha_{k}(x) = kx$ on $\integers_{m}$.  Show that $\alpha_{k} \circ
    \alpha_{j} = \alpha_{kj}$.
    
    Show that $\Aut(\integers_{m})$ is isomorphic to the multiplicative
    group $\integers_{m}^{*} = \{x : \gcd(x,m) = 1\}$.
    
    Show that $\Aut(\integers_{8}) \isom V$.
    
    Show that if $m = pq$ where $p$ and $q$ are distinct primes, then
    $|\Aut(\integers_{m})| = (p-1)(q-1)$.
  
  \item Show that if $(G, +, 0)$ is any finite Abelian group (written using
    additive notation), then the function
    \[
      \alpha_{k}(x) = \underbrace{x + x + x + \cdots + x}_{k\text{ times}}
    \]
    is a homomorphism from $G$ to $G$.  Show that if $k$ and $|G|$ have
    a greatest common divisor of $1$, then $\alpha_{k}(x) = 0$ if and only
    if $x = 0$.  Show that in this case, $\alpha_{k}$ is an automorphism.
    
    Show that $\alpha_{k} \circ \alpha_{j} = \alpha_{kj}$, so that the
    function $\beta : \integers_{|G|}^{*} \to \Aut(G)$ defined by
    $\beta(k) = \alpha_{k}$ is a homomorphism.
    
  \item Show that if $G$ is a group, the function 
    $\iota(x) = x^{-1}$ is an automorphism if and only if $G$ is Abelian.
  
  \item Show that if $G \isom H$, then $\Aut(G) \isom \Aut(H)$.
  
  \item Show that if $n$ is odd then $\Inn(D_{2n}) \isom D_{2n}$, 
    while if $n$ is even then $\Inn(D_{2n}) \isom D_{n}$ (Hint: you 
    may use Exercise~\ref{ex:D2nquotient} to prove this).
\end{exercises}


\section{Extension: Category Theory}

If you think about the basic outlines of the theory which we have developed
so far, you should notice some similarities between the theories of groups,
vector spaces, partial orders and lattices.  At a very abstract level we
have:

\begin{tabular}{l|l}
  Sets & Functions \\
  \hline
  Groups & Homomorphisms \\
  Vector Spaces & Linear Transformations \\
  Partially Ordered Sets & Order-preserving Functions \\
  Lattices & Lattice Homomorphisms
\end{tabular}

There are also similarities beyond this: in all cases there is the notion
of ``isomorphism'' between appropriate types of sets and the notion of a ``sub-''
object (like a subgroup or subspace), for example.

The model that we should keep in mind ofr what we are about to define
is simply a minimal set of axioms which sets and functions will satisfy:
\begin{enumerate}
  \item each function has a domain and codomain,
  
  \item if $\dom f = \cod g$ we can compose the functions,
  
  \item function composition is associative,
  
  \item for each set $X$, there is an identity function $\id_{A}: X
    \to X$, and this identity function has the property  $f \circ 
    \id_{A} = f$ and $\id_{A} \circ g = g$.
\end{enumerate}

Notice that group homomorphisms also satisfy all of these conditions.

\begin{definition}
  A \defn{category}{category} $\mathcal{C}$ consists of a set of
  \defn{objects}{objects}, $\mathcal{O}$; a set of
  \defn{arrows}{arrow} or \defn{morphisms}{morphism} $\mathcal{A}$,
  two functions
  \[
    \cod: \mathcal{A} \to \mathcal{O} \qquad \text{and} \qquad \dom:
    \mathcal{A} \to \mathcal{O}
  \]
  which assign to each arrow an object called, respectively, the
  \defn{domain}{domain} and \defn{codomain}{codomain} of the arrow; 
  a function
  \[
    \id: \mathcal{O} \to \mathcal{A},
  \]
  which assigns to each object $A$ an \defn{identity
  arrow}{arrow!identity} $\id_{A}$; and a
  \defn{composition}{composition} operation that assigns each to pair
  of arrows $(\alpha,\beta)$ with $\dom \alpha = \cod \beta$ an arrow
  $\gamma = \alpha \circ \beta$ with $\cod \gamma = \cod \alpha$ and
  $\dom \gamma = \dom \beta$.
  
  We will write $f : A \to B$ to denote that an arrow $f$ has domain 
  $A$ and codomain $B$, or diagramatically, write:
  \[
    A \to B
  \]
  
  These have to satisfy the following axioms:
  \begin{theoremenum}
    \item Associativity: if $f: B \to A$, $g: C \to B$, and $h: D \to
    C$, then $(f \circ g) \circ h = f \circ (g \circ h)$,
    
    \item Identity: for any $f : A \to B$, $f \circ \id_{A} = f$; and
    for any $g: B \to A$, $\id_{A} \circ g = g$,
  \end{theoremenum}
\end{definition}

Notice that these axioms are very similar to the definition of a 
group, but with added complexity because of the neccessity of dealing 
with the domains and codomains, and with no inverse axiom.

Categories are very closely related to directed graphs, and we can
often represent parts of a category graphically.  Many key facts in
category can be represented by succinctly by \defn{commuting 
diagrams}{commuting diagram}.  The key property of a commuting diagram 
is that any path following the arrows through a diagram that start 
from the same object and ends at the same object are equal.  For 
example, the associativity axiom can be represented by the following 
commuting diagram:

XXX Picture

Similarly, the following two diagrams represent the identity axioms:

XXX Picture

As is the case for associative binary operations, the associativity 
axiom for categories means that it doesn't matter where we put the 
parentheses in a composition of multiple arrows.  We can also show 
that for each $A$, $\id_{A}$ is unique.

\begin{example}
  The following are all categories:
  \begin{enumerate}
    \item $\mathbf{Set}$: the category with objects being all sets 
    contained in some universe $U$ and arrows being all functions 
    on those sets.
    
    \item $\mathbf{Grp}$: the category with objects being all groups
    contained in some universe $U$, and arrows being all group
    homomorphisms.
    
    \item $\mathbf{Abl}$: the category with objects being all Abelian
    groups contained in some universe $U$, and arrows being group
    homomorphisms between them.
    
    \item $\mathbf{Vec}(\field)$: the category with objects being all
    vector spaces over a field $\field$ (contained in some universe
    $U$), and arrows being linear transformations between them.
    
    \item $\mathbf{Lat}$: the category with objects being all lattices 
    contained in some universe $U$, and arrows being lattice 
    homomorphisms.
    
    \item $\mathbf{Set_{*}}$: the category whose objects are
    \defn{pointed sets}{set!pointed}: pairs $(X,x)$, where $X$ is a
    set contained in some universe $U$, and $x \in X$ is some
    distinguished point; and whose arrows are functions which map
    distingushed points to distinguished points: if $(X, x)$ and $(Y,
    y)$ are pointed sets, then $f:X \to Y$ is a morphism if and only
    if $f(x) = y$.
  \end{enumerate}
\end{example}

There are, of course, many, many other categories.  Indeed, whenever 
you encounter a new mathematical object, particularly in algebra, you 
should ask yourself ``what is the category that goes with this?''  If 
you can establish this, then you can immediately get a number of basic 
results for free.

For a good theory which encompasses the fundamentals of functions on
sets and group homomorphisms, we need to have more than just
composition and identity.  We also need to determine analogues of
injective functions (or group monomorphisms), surjective functions (or
group epimorphisms), and most importantly bijection (or group
isomorphisms).

\begin{definition}
  Let $\mathcal{C}$ be a category with objects $\mathcal{O}$ and
  arrows $\mathcal{A}$.  An arrow $\alpha: A \to B$ is
  \defn{invertible}{arrow!invertible} if there is an arrow
  $\alpha^{-1}: B \to A$ such that $\alpha^{-1} \circ \alpha =
  \id_{A}$ and $\alpha \circ \alpha^{-1} = \id_{B}$.  We say that two
  objects $A$ and $B$ are \defn{isomorphic}{isomorphic!objects} if there is an
  invertible arrow $\alpha: A \to B$.
  
  An arrow $\alpha: A \to B$ is \defn{monic}{arrow!monic} if whenever
  there are arrows $\beta_{1}$ and $\beta_{2}: C \to A$ such that
  $\alpha \circ \beta_{1} = \alpha \circ \beta_{2}$, then $\beta_{1} =
  \beta_{2}$ (ie.~we can cancel $\alpha$ on the left).  An arrow
  $\alpha: A \to B$ is \defn{epi}{arrow!epi} if whenever there are
  arrows $\beta_{1}$ and $\beta_{2}: B \to C$ such that $\beta_{1}
  \circ \alpha = \beta_{2} \circ \alpha$, then $\beta_{1} = \beta_{2}$
  (ie.~we can cancel $\alpha$ on the right).
  
  A \defn{right inverse}{arrow!inverse!right} of an arrow $\alpha: A
  \to B$ is an arrow $\rho: B \to A$ such that $\alpha \circ \rho =
  \id_{B}$.  A \defn{left inverse}{arrow!inverse!left} of an arrow
  $\alpha: A \to B$ is an arrow $\lambda: B \to A$ so that $\lambda
  \circ \alpha = \id_{A}$.  A right inverse of $\alpha$ is also called 
  a \defn{section}{section} of $\alpha$, while a left-inverse is 
  called a \defn{retraction}{retraction} of $\alpha$.
  
  If an object $A$ has the property that for any object $B$ we have a
  exactly one arrow $B \to A$, it is said to be
  \defn{terminal}{object!terminal}.  If instead it has the property
  that there is exactly one arrow from $A \to B$, then it is said to
  be \defn{initial}{object!initial}.
\end{definition}


\begin{example}
  In the categories of $\mathbf{Set}$ and $\mathbf{Grp}$, we have 
  that following correspondence:
  
  \begin{tabular}{l|ll}
     & $\mathbf{Set}$ & $\mathbf{Grp}$ \\
     \hline
     invertible arrow & bijective function & isomorphism \\
     monic arrow & injective function & monomorphism \\
     epi arrow & surjective function & epimorphism \\
     terminal object & any set with one element & any group with one 
     element \\
     initial object & the empty set & any group with one element
  \end{tabular}
  
\end{example}

We can prove a number of facts immediately:

\begin{proposition}
  If $\mathcal{C}$ is a category, then
  \begin{theoremenum}
    \item if an arrow $\alpha$ has a right inverse, then it is epi, 

    \item if an arrow $\alpha$ has a left inverse, then it is monic,
    
    \item an arrow $\alpha$ has both a left and right inverse if and
    only if it is invertible,
    
    \item if an arrow is invertible, it is both epi and monic.

  \end{theoremenum}
\end{proposition}

\begin{proof}
  (i) Let $\alpha: A \to B$ and let $\rho: B \to A$ be a right inverse
  of $\alpha$.  Then given any arrows $\beta_{1}$ and $\beta_{2}: B
  \to C$ such that $\beta_{1} \circ \alpha = \beta_{2} \circ \alpha$,
  we have that
  \begin{align*}
     \beta_{1} &= \beta_{1} \circ \id \\
     &= \beta_{1} \circ \alpha \circ \rho\\
     &= \beta_{2} \circ \alpha \circ \rho \\
     &= \beta_{2} \circ \id\\
     &= \beta_{2}.
  \end{align*}
  So $\alpha$ is epi.
  
  (ii) The proof of this is left as an exercise.
  
  (iii) If $\alpha$ is invertible, then the inverse is both a left 
  and right inverse, so $\alpha$ has both a left and right inverse.
  
  On the other hand, if $\alpha: A \to B$ and $\rho: B \to A$ and
  $\lambda: B \to A$ be right and left inverses of $\alpha$,
  respectively, then
  \begin{align*}
    \rho &= \id \circ \rho \\
         &= \lambda \circ \alpha \circ \rho \\
&= \lambda \circ \id \\
&= \lambda
  \end{align*}
  So $\rho = \lambda$ is an inverse for $\alpha$.
  
  (iv) This follows immediately from the first three parts.
  
\end{proof}

In the previous chapter, key results came from looking at an object in
another category which corresponded to the group, such as looking at
the subgroup lattice of a group in the category of lattices.  These
sorts of correspondences between categories were first recognised in
the study of algebraic topology, and are extremely powerful.  Indeed,
in some sense they are what justifies the study of categories as a
distinct topic.

\begin{definition}
  Let $\mathcal{C}_{1}$ and $\mathcal{C}_{2}$ be two categories with
  object sets $\mathcal{O}_{1}$ and $\mathcal{O}_{2}$, and arrow sets
  $\mathcal{A}_{1}$ and $\mathcal{A}_{2}$, respectively.  A 
  \defn{functor}{functor} $\mathcal{F} : \mathcal{C}_{1} \to 
  \mathcal{C}_{2}$ is a pair of functions $\mathcal{F_{O}} : 
  \mathcal{O}_{1} \to \mathcal{O}_{2}$ and $\mathcal{F_{A}} : 
  \mathcal{A}_{1} \to \mathcal{A}_{2}$ such that if $A \in 
  \mathcal{O_{1}}$, $\alpha$ and $\beta \in \mathcal{A_{1}}$, then
  \begin{alignat*}{2}
    \dom \mathcal{F_{A}}(\alpha) &= \mathcal{F_{O}}(\dom \alpha) & \cod
    \mathcal{F_{A}}(\alpha) &= \mathcal{F_{O}}(\cod \alpha)\\
    \mathcal{F_{A}}(\alpha \circ \beta) &= (\mathcal{F_{A}}(\alpha)
    \circ \mathcal{F_{A}}(\beta) \qquad & \mathcal{F_{A}}(\id_{A}) &=
    \id_{\mathcal{F_{O}}}(A)
  \end{alignat*}
  whenever $\alpha \circ \beta$ is defined.
\end{definition}

We usually don't distinguish between the functor, and the functions 
on the sets of objects and arrows, simply representing each of them 
by a single symbol $\mathcal{F}$.

\begin{example}\label{eg:funtorgrplat}
  Let $\mathbf{Grp}$ and $\mathbf{Lat}$ be the categories of groups
  and lattices defined earlier.  Then we know that the function
  $\mathcal{F_{O}}$ defined by $\mathcal{F_{O}}(G) = \Sub(G)$ is a
  function from the objects of $\mathbf{Grp}$ to the objects of
  $\mathbf{Lat}$.  Corollary~\ref{cor:grouphomtolathom} tells us that
  if we have a group homomorphism $\alpha: G \to H$, then we have a
  corresponding lattice homomorphism $\overline{\alpha} : \Sub(G) \to
  \Sub(H)$.  So we define $\mathcal{F_{A}}(\alpha) =
  \overline{\alpha}$.  This gives the first two of the four conditions
  that we need for $\mathcal{F}$ to be a functor.
  
  It is easily verified that $\mathcal{F_{A}}(\id_{G}) =
  \id_{\mathcal{F_{O}}(G)}$, while it is a little more work to check
  the remaining composition condition.  They do hold, however, so we 
  have a functor between the categories.
  
  This is not the only possible functor between these two categories,
  since we could also consider a functor which maps a group $G$ to the
  lattice consisting of the power set $\mathcal{P}(G)$ with meet and
  join being intersection and union.
\end{example}

\begin{example}
  The maps $\mathcal{F}(G) = G$ and $\mathcal{F}(\alpha) = \alpha$ 
  give a functor $\mathcal{F} : \mathbf{Grp} \to \mathbf{Set}$.
\end{example}

Functors such as the one in the last example are called
\defn{forgetful functors}{functor!forgetful} because we are forgetting
about all the extra structure that a group has and treating it just as
a set, and regarding homomoprhisms simply as functions.  Whenever we
have a category whose objects and arrows are specializations of
another categories objects and arrows we get a forgetful functor which
strips this additional structure away.

\begin{example}
  There is a forgetful functor $\mathcal{F} : \mathbf{Abl} \to
  \mathbf{Grp}$, since Abelian groups are simply groups with an
  additional requirement of commutativity, and homomorphisms between
  Abelian groups are still homomoprhisms.
\end{example}

When looking at the question of whether or not two objects within a
category are isomorphic or not, functors can help us say that that the two
objects are not isomorphic.

\begin{proposition}
  Let $\mathcal{C}$ be a category, and let $A$ and $B$ be two objects in
  $\mathcal{C}$.  If there is another category $\mathcal{D}$ and a functor
  $\mathcal{F}: \mathcal{C} \to \mathcal{D}$ such that $\mathcal{F}(A)$ and
  $\mathcal{F}(B)$ are not isomorphic in $\mathcal{D}$, then $A$ and $B$ are
  not isomorphic in $\mathcal{C}$.
\end{proposition}
\begin{proof}
  If $A$ and $B$ are isomorphic, then we have an invertible arrow $\alpha : A
  \to B$ and its inverse $\alpha^{-1} : B \to A$.  If $\mathcal{F}: \mathcal{C}
  \to \mathcal{D}$ is any functor, then
  \[
    \mathcal{F}(\alpha) \circ \mathcal{F}(\alpha^{-1}) = \mathcal{F}(\alpha
    \circ \alpha^{-1}) = \mathcal{F}(\id_{B}) = \id_{\mathcal{F}(B)}.
  \]
  Similarly $\mathcal{F}(\alpha^{-1}) \circ \mathcal{F}(\alpha) =
  \id_{\mathcal{F}(A)}$, and so $\mathcal{F}(\alpha) : \mathcal{F}(A) \to
  \mathcal{F}(B)$ has an inverse arrow $\mathcal{F}(\alpha^{-1})$, and hence
  $\mathcal{F}(A)$ is isomorphic to $\mathcal{F}(B)$.
  
  Hence if $\mathcal{F}(A)$ is not isomorphic to $\mathcal{F}(B)$, then $A$ and
  $B$ are not isomorphic.
\end{proof}

This very general result is at the core of many of the techniques we have
for distinguishing groups which are not isomorphic.  For example, the fact
that two groups with different subgroup lattices are not isomorphic is an
immediate corollary of this proposition, together with the functor of
Example~\ref{eg:funtorgrplat}.  The fact that two groups of different orders
are not isomorphic is an immediate corollary of this proposition, using the
forgetful functor from $\mathbf{Grp}$ to $\mathbf{Set}$.

Unfortunately, this doesn't help us in showing when two groups are
isomorphic, since there are many functors available, so checking every
possible functor is impossible.

The category of groups is not the only category of interest in mathematics,
of course, so it is useful to observe that in every category we find many
naturally occurring groups:

\begin{proposition}
  Let $\mathcal{C}$ be a category, and $A$ an object in $\mathcal{C}$.  Then
  the triple $(\Aut(A), \circ, \id_{A})$, where $\Aut(A)$ is the set of
  invertible arrows from $A$ to $A$, is a group.  We call this the
  \defn{automorphism group}{group!automorphism!of an object} of $A$.
\end{proposition}
\begin{proof}
  From the definition of a category, when $\circ$ is restricted to $\Aut(A)$,
  it is an associative binary operation, and $\id_{A}$ is an identity.  So the
  only thing that needs to be checked is that there is an inverse for every
  element, but this is guaranteed by the assumption that our arrows are all
  invertible.
\end{proof}

\begin{example}
  In the category $\mathrm{Grp}$, $\Aut(G)$ is precisely the automorphism
  group of $G$, as discussed earlier.
\end{example}

\begin{example}
  In the category of finite dimensional real vector spaces and linear
  transformations, we have that $\Aut(V)$ is the set of all invertible
  linear transformations from $V$ to $V$.  If $V$ has dimension $n$, then
  this group is isomorphic to $GL_{n}(\reals)$.
\end{example}

\begin{example}
  In the category $\mathrm{Set}$, the group $\Aut(X)$ consists of all
  bijections of $X$ onto itself, or in other words, the group of all
  permutations of $X$.  If $|X| = n$, then $\Aut(X) \isom S_{n}$.
\end{example}

\begin{example}\label{eg:symmetrycategory}
  One can consider a category whose objects are all subsets of $\reals^{n}$,
  and whose arrows are isometries which map one subset onto another.
  
  In this category, the automorphism group of an object is the set of all
  symmetries of the object.
\end{example}

As the above examples illustrate, the concept of an automorphism group
generalises the concept of symmetries that we first introduced in Chapter 1.
Whenever you see a new type of mathematical object being introduced, there
is usually a category associated with it, and hence there is some sort of
automorphism group associated with each object.  Given the wide variety of categories
that are of interest in mathematics, this underlines the importance of group
theory.

\begin{proposition}
  Let $\mathcal{C}$ be a category, and $A$ and $B$ two objects in
  $\mathcal{C}$.  If $A$ and $B$ are isomorphic, then $\Aut(A)$ and
  $\Aut(B)$ are isomorphic.
\end{proposition}
\begin{proof}
  Let $\alpha: A \to B$ be an invertible arrow.  Then we define a function
  $\overline{\alpha} : \Aut(A) \to \Aut(B)$ by
  \[
    \overline{\alpha}(\beta) = \alpha \circ \beta \circ \alpha^{-1}.
  \]
  One can easily verify that $\overline{\alpha}(\beta)$ is an
  invertible arrow, and hence the function is indeed into $\Aut(B)$, and
  furthermore
  \[
    \overline{\alpha}(\beta) \circ \overline{\alpha}(\gamma) = \alpha \circ \beta \circ
      \alpha^{-1} \circ \alpha \circ \gamma \circ \alpha^{-1} = \alpha \circ \beta
      \circ \gamma \circ \alpha^{-1} = \overline{\alpha}(\beta \circ \gamma),
  \]
  so $\overline{\alpha}$ is a homomorphism.  Similarly,
  $\overline{\alpha^{-1}} : \Aut(B) \to \Aut(A)$ is a homomorphism, and
  furthermore $\overline{\alpha} \circ \overline{\alpha^{-1}} =
  \id_{\Aut(B)}$ and $\overline{\alpha^{-1}} \circ \overline{\alpha} =
  \id_{\Aut(A)}$, so $\overline{\alpha}$ is an isomorphism.
\end{proof}

This means that in any time you have a category, you can use the automoprhism
groups to distinguish non-isomoprhic objects in the category via the
contrapositive of this result: if the automorphism groups are not
isomorphic, the objects are not isomorphic.

\begin{definition}
  If $\mathcal{C}$ is a category, and $A$ is an object in $\mathcal{C}$,
  then an \defn{action}{action} of a group $G$ on $A$ is a homomorphism
  $\alpha: G \to \Aut(A)$.
\end{definition}

For most categories of interest the objects of the category are
sets with additional structure and the arrows are functions with additional
conditions that they must satisfy. This means that $\alpha(g)$ is a
function of some sort from $A$ to $A$.  To avoid confusion, it is common to
write $\alpha(g) = \alpha_{g}$, so that we can use the less confusing
notation $\alpha_{g}(x)$ instead of $(\alpha(g))(x)$ to represent the image
of an element of $A$ under this automorphism.  In this case we have that
$\alpha_{gh} = \alpha_{g} \circ \alpha_{h}$.

\begin{example}\label{eg:inneraction}
  Given a group $G$, the map $\alpha: G \to \Aut(G)$ defined by $\alpha(g) =
  \alpha_{g}$, where $\alpha_{g}(x) = g^{-1}xg$ is an action of $G$ on
  itself.  The image of the homomorphism is $\Inn(G)$, and the kernel is
  $Z(G)$.
\end{example}

\begin{example}
  Given a group $G$ then for each $g \in G$ we have bijective functions
  $\lambda_{g}(x) = g^{-1}x$ and $\rho_{g}(x) = xg$.  Since $\lambda_{gh} =
  \lambda_{g} \circ \lambda_{h}$, and $\rho_{gh} = \rho_{g} \circ \rho_{h}$,
  the functions $\lambda: g \mapsto \lambda_{g}$ and $\rho: g \mapsto
  \rho_{g}$ are actions of $G$ on itself, when we consider it as an object
  in the category $\mathrm{Set}$.  We call these the
  \defn{left}{action!left} and \defn{right actions}{action!right} of $G$ on
  itself, respectively.
\end{example}

In a category in which the objects are sets with additional structure, and the
arrows are functions which satisfy some additional conditions, we define the
\defn{orbit}{orbit} of an element $x$ under an action of a group $G$ to be
the set
\[
  O(x) = \{ \alpha_{g}(x) : g \in G\}.
\]
We can define an equivalence relation using an action by $x \sim y$ if and
only if $y = \alpha_{g}(x)$ for some $g \in G$.  It is easy to verify that
this is indeed an equivalence relation, since the facts that $x =
\alpha_{e}(x)$; that if $x = \alpha_{g}(y)$ then $y = \alpha_{g^{-1}}(x)$; and
that if $x = \alpha_{g}(y)$ and $y = \alpha_{g}(z)$, then
\[
  x = \alpha_{g}(y) = \alpha_{g}(\alpha_{h}(z)) = \alpha_{gh}(z)
\]
give reflexivity, symmetry and transitivity, respectively.  We then have
that the orbit of $x$ is simply the equivalence class of $x$, ie.
\[
  O(x) = [x]_{\sim}.
\]

\begin{example}
  The orbit of an element $x \in G$ under the action of
  Example~\ref{eg:inneraction} is the conjugacy class of $x$, ie.~$O(x) =
  C(x)$.
\end{example}

\begin{example}
  The orbit of an element $x \in G$ under the left action of $G$ is $G$.
  
  However if we restrict the left action to some subgroup $H$ of $G$, so
  we only consider functions of the form $\lambda_{h}$ for $h \in H$, then
  $\lambda$ is an action of $H$ on $G$, and then $O(x)$ is the right coset
  $Hx$.
\end{example}

\begin{example}
  In the category of Example~\ref{eg:symmetrycategory}, consider closed disc
  of radius 1 with centre at the origin as the object $A$.  The
  automorphism group of this object consists of
  rotations $R_{\theta}$ and reflections $H_{\psi}$ of the disc, as discussed
  in Example~\ref{eg:circlesymmetry}.  There is an action of the additive
  group of real numbers $\alpha: \reals \to \Aut(A)$ given by $\alpha_{x} = R_{y}$,
  where $x = 2\pi k + y$, with $k \in \integers$ and $y \in [0,2\pi)$.
  
  In this case the orbit of the point $(1,0)$ is the unit circle, since we
  can find a rotation that maps that point to any other on the unit circle,
  and every rotation is in the image of the action.
  
  Indeed, the orbit of any point in the disc will be a circle.
\end{example}

The previous example should help you understand why an orbit is called an
orbit.

\section{Semidirect Products}

Group automorphisms and the actions of groups allow us to generalise the
notion of the direct product introduced in Chapter 2.  Let $G$ and $H$ be
groups, and let $\alpha$ be an action of $H$ on $G$.  We supply a binary
operation $\ast$ for the set
\[
  G \cross H = \{ (g,h) : g \in G, h \in H \}
\]
by
\[
  (g_{1}, h_{1}) \ast (g_{2}, h_{2}) = (\alpha_{h_{2}}(g_{1})g_{2}, h_{1}h_{2}).
\]

\begin{proposition}
  Let $G$ and $H$ be groups, and let $\alpha$ be an action of $H$ on $G$.
  The binary operation $\ast$ defined above makes $(G \cross H,
  \ast, (e,e))$ a group.  We call this group a \defn{semidirect product}{product!semidirect}
  of $G$ and $H$, and denote it symbolically by $G \rtimes_{\alpha} H$.
  If $H$ is a subgroup of $\Aut(G)$ acting in the obvious way, then we call
  this group the \defn{extension}{group extension} of $G$ by $H$.
  
  Moreover
  \begin{theoremenum}
    \item the subset $G' = \{(g,e): g \in G\}$ is a characteristic
    subgroup of the semidirect product which is isomorphic to $G$;
    
    \item $(G \rtimes_{\alpha} H) / G' \isom H$;
    
    \item the subset $H' = \{(e,h) : h \in H\}$ is a subgroup of the semidirect
    product which is isomorphic to $H$;
    
    \item the conjugate of an element $(g,e)$ of $G'$ by an element $(e,h)$
    of $H'$ is the element $(\alpha_{h}(g), e)$ of $G'$.  In other words,
    conjugation by elements of $H'$ is equivalent to the action $\alpha$ on $G$.
  \end{theoremenum}
\end{proposition}
\begin{proof}
  Showing that the semidirect product is a group is left as an easy exercise.
  It is also left as an exercise to show that $G' \isom G$ and $H' \isom H$.
    
  We note that the inverse of the element $(g,h)$ is the element
  $(\alpha_{h^{-1}}(g^{-1}), h^{-1})$.
  
  The function $\beta(g,h) = h$ from $G \rtimes_{\alpha} H$ to $H$ is a
  homomorphism, since
  \begin{align*}
    \beta((g_{1}, h_{1}) \ast (g_{2}, h_{2})) &= \beta(\alpha_{h_{2}}(g_{1})g_{2}, h_{1}h_{2}) \\
    &= h_{1}h_{2}\\
    &= \beta(g_{1}, h_{1})\beta(g_{2}, h_{2})).
  \end{align*}
  Now
  \[
    \ker \beta = \{ (g, h) : \beta(g,h) = h = e\} = \{ (g,e) : g \in G \} = G',
  \]
  so $G'$ is a normal subgroup, and furthermore the
  First Isomorphism Theorem says that since $\beta$ is onto,
  \[
    H \isom (G \rtimes_{\alpha} H)/\ker \beta = (G \rtimes_{\alpha} H)/G'
  \]
  
  Finally, by calculation,
  \begin{align*}
    (e,h)^{-1} \ast (g, e) \ast (e,h) &= (e,h^{-1}) \ast (g, e) \ast (e,h) \\
    &= (\alpha_{e}(e)g, h^{-1}e) \ast (e,h) \\
    &= (g, h^{-1}) \ast (e,h) \\
    &= (\alpha_{h}(g)e, h^{-1}h) \\
    &= (\alpha_{h}(g), e).
  \end{align*}
\end{proof}

\begin{example}
  For any $G$ and $H$, there is always the trivial action $\alpha_{h}(g) = g$.
  Under this action, the semidirect product $G \rtimes_{\alpha} H$ has product
  \[
    (g_{1}, h_{1}) \ast (g_{2}, h_{2}) =
      (\alpha_{h_{2}}(g_{1})g_{2}, h_{1}h_{2})
      = (g_{1}g_{2}, h_{1}h_{2}).
  \]
  In other words, this is simply the direct product of the two groups.
\end{example}

\begin{example}
  If $G$ is any Abelian group, there is an action of the multiplicative
  group $\{1, -1\} \isom C_{2}$ given by $\alpha_{1}(g) = g$ and
  $\alpha_{-1}(g) = g^{-1}$.  It is trivial that $\alpha_{1}$ is a
  homomorphism, and $\alpha_{-1}(gh) = (gh)^{-1} = h^{-1}g^{-1} =
  g^{-1}h^{-1} = \alpha_{-1}(g)\alpha_{-1}(h)$, since $G$ is Abelian.
  
  The semidirect product of $G$ and this group with this action is a group
  of order $2|G|$, and if $G$ has any element of order greater than $2$, this
  is distinct from the direct product.  Indeed, if it is distinct from
  the direct product, the semidirect product is not Abelian, since if $g_{1}$
  is an element of order greater than $2$, we have
  \[
    (g_{1}, 1) \ast (e, -1) = (g_{1}^{-1}, -1),
  \]
  but
  \[
    (e, -1) \ast (g_{1}, 1) = (g_{1}, -1),
  \]
  and these are only equal if $g_{1}^{-1} = g_{1}$, which implies that
  $g_{1}$ has order $2$.  So these elements do not commute.
  
  One can show that as a particular example of this action giving a semidirect
  product, we have $D_{2n} \isom C_{n} \rtimes_{\alpha} C_{2}$, where the
  generators $a$ and $b$ of $D_{2n}$ correspond to the elements $(u,1)$ and
  $(1,-1)$, respectively.
\end{example}


\newpage
\section*{Assignment 5}

The following exercises are due on day of final.

\begin{description}
  \item[4.1] Exercises 1, 2, 4, 5, 7.
  \item[4.2] Exercises 1, 4, 5, 8, 9.
\end{description}
